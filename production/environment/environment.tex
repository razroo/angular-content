\maketitle{}
\section{ Environment }

When getting ready for production, it is important that you have different
environment files for dev and prod. Luckily the CLI will offer two different
environment files out the box. So, it should be immediately inherit to any
developer experienced, or otherwise, that it is something that should be payed
attention to. However, the full scope(gamut if you will) might not be
immediately inherit, wherein certain elements migth not be looked into
appropriately.

\section{ Environment Architecture - Deep Dive}

\subsection{API URL's}
The most popular use of a environment file is to set up a url for prod and qa.
QA will most likely be used by dev as well. Therefore, you will have something
like the following in your UI:
\begin{lstlisting}
export const environment = {
  production: false,
  envName: 'qa'
};
\end{lstlisting}

Doing something as simple as the above, will tag on 'qa' to your url. Being that
in the reccomended architecture we are using GraphQL, we will have just one
place where we have to specify the url we are using.

\subsection{Cookie Names}
Something I've seen in some projects as well, is wherein the cookie name will
be different on dev than it will be on prod. This is a use for the environment
file as well. 
