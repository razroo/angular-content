\maketitle{}
\section{ Sockets }

When using caching withing an angular application, the first and only dillema
that will come to play is how update that cache. For most applications,
especially the nature of Angular, is that it will the data will need to be
updated frequently.

\subsection{ Understanding Internals of Sockets }
Sockets are that way of updating a cache. It sets up a hook
to the server, so that whenever the server is ready to respond it does. You
might be familiar with this, if you've had a network request that's taken a
while to return it's request. Sockets are like that. When ever the cache has
been updated, another socket goes out, to get it's data when needed. With
every request that is sent an id(requestId) is sent over, so that server knows
exactly where to send the data back, and it allows for a bi-directional flow
between the server and client.

\subsection{ Handshake Protocol }
When creating a websocket request, in an apollo  
