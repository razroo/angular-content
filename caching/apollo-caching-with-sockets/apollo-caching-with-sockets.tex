\maketitle{}
\section{ Apollo Caching with Sockets }

I would like to re-iterate, because I like you the reader, that the value of
being aware of architecture is twofold. One is that one has the option of
preventing future catastrophies. In addition, one has the ability to quickly
put together the correct solution, and have in available in a time intensive
setting. Caching with Apollo is exactly that scenario. One might include
caching, but realize that sockets is something that we would like included with
the app. As a result, the development team might decide to throw away caching,
so that data can be updated. I would like to say that it is very much possible.
I would also like to lay out, what in a couple of paragraphs that best strategy
to do so within Apollo.

\subsection{ Apollo Caching with Sockets }
\mybox{One side note, is that it will generally be a requirement that will
come from front end first, that they will need to use something like sockets.
However, backend will also need to setup websockets, in order to get everything
to work as expected.}

It should be noted that GraphQL offers three types of data queries:
\begin{itemize}
  \item Query
  \item Mutation
  \item Subscriptions
\end{itemize}

Subscriptions in particular is very similar to websockets. 

The package at this point which makes the most sense is
\href{https://github.com/apollographql/subscriptions-transport-ws}{subscriptions-transport-ws}
