\maketitle{}
\section{ State Management - Ngrx/store }

Ngrx/store is a layer on top of Redux. It is a state management tool that was
originally created, in order to solve two way binding performance issues within
Angular. \footnote{Need to further bring source for this one}. It then extended
as a way to bring redux natively to Angular, with the use of Observables.

Let's dive into integrating @ngrx/store into our app. \marginpar{This particular
component has been written in the fashion of TDD. However, another chapter
will be dedicated to TDD/BDD in order to specify this point specifically.}

\subsection{ Using nx ngrx to Generate State }

\subsubsection{ Create root state using nx ngrx }

First we are going to generate an empty root, for our StoreModule, as well as
our EffectsModule. Our StoreModule is responsible as a singular store object,
which will be holding all of store data. Our EffectsModule is a singular effects
object, which will be holding all of our effects. \footnote{We will discuss
effects in more detail later}

\begin{lstlisting}[language=Bash]
ng generate ngrx app --module=apps/angular-pixel-illustrator/src/app/app.module.ts --onlyEmptyRoot
\end{lstlisting}

\subsubsection{ Create component state using nx ngrx }

Next, we are going to create state for our choose-size component. This is done
with ease using nx ngrx \footnote{Trust me, I've been in situations where I
was not using a CLI. It is not good news}
