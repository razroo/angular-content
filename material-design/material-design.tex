\maketitle{}
\section{ Material Design }

I was debating writing this chapter. The reason primarily being, that depending
on the size of your company, you might up end writing your own design system. I
completely understand that, and it makes sense if you are a B2C
\footnote{Business to Consumer} enterprise.

However, I truly do not understand why a company using angular 5, or 6, would
not want to use material design. It is the most robust design framework that
exists within open source. In addition, the documentation for Angular components
is next to none. I personally have been in companies where they had a business
to business application and they decided not to use material design.

I really do not understand the reason for doing this. They could have saved
loads of resources not having to design and implement their own components. It
is out of the vast amount of use cases that I see Material Design being valuable,
that I have decided to go ahead and write about it.

\subsection{ Material Design - Talking to UX/UI }
This section right here, is perhaps why I like Material Design the most. Material
Design has documentation for how the UX should work. It also has an Angular
component library with demos, that I can show off to UX and show them, this is
how it works by default. It addition, theming for Material Design, is very easy.

Putting your own company specific spin on it, boils down to the following:
\begin{enumerate}
  \item Colors
  \item Font
  \item Spacing(Margin + Padding)
  \item Icons(not that this is anything particular)
  \item Buttons
\end{enumerate}

The above would be it for starters. As your designs go on, you will have components
that you will end up overridig.

\subsection{ Material Design - Create your own Confluence Doc }

It is important when working with UX/UI to document discrepencies. For
inspiraiton look at the \href{https://material.io/guidelines/components/sliders.html}{material design docs}.
The idea is to have a central place where UX can create Confluence doc
describing the differences they have made from DLS.

Engineers will generally have to be the one to begin the Confluence doc. From a
matter of ownership, engineering has a stronger discipline of documentation and
organization. Engineers should look to take ownership of the confluence doc.

\subsection{ Material Design - Use Invision }
It's interesting, because someone might not think of tooling as something which
is a part of engineering architecture. However, with regards to finding
discrepencies in DLS(Design Language System), Invision is integral. It will
make creating comments on particular components as something which will be fluid.

\subsection{ Material Design - Push Back }
The following will be worth alot of time for many different people within your
organization. Make sure that your component does not deviate from Material
Design. In addition, look into whether, or not it is pre-described for you to
go ahead, and create your own components. However, I can assure you designers,
product/business, and engineers will all be happy when you go with the default
components when possible. When building a product, unless it is beyond the MVP
go with what is available for you by default.

\subsection{ Material Design - Architecture Corner }
In a Material Design setting, there will be discrepencies in the design, which
we have mentioned above, two ways in order to address, and make sure that
engineering is in sync with Design.

However, how does Engineering make sure, that all engineers are adhering to the
principles layed out in the DLS. There are two methods which will help to a
great extent:
\begin{enumerate}
  \item Sass functions, with error reporting.
  \item Automating UI layer.
\end{enumerate}
