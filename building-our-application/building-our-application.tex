\maketitle{}
\section{ Building our Application }

In order to go through the full gamut of Angular, we are going to focus on as
lightweight of an applicaiton as possible. In order to go through entire Angular
architecture, we obviously do not want to over due it, nor do too little. It
goes without saying, that your classic todo app, will not suffice. Instead the
following is the application that we will be building.

We are going to call it a pixel to coordinate illustrator. The idea behind the
app, is that we should have a canvas, paint a pixel on that canvas. We then have
a coordinate, that will appear based on where pixels are currently located.

In our application we have:
\begin{enumerate}
  \item Form
    \begin{enumerate}
      \item Pixel Size
      \item Number of Rows
      \item Number of Columns
    \end{enumerate}
  \item Color Picker
    \begin{enumerate}
      \item Background Color Picker
      \item Pixel Color Picker
    \end{enumerate}
  \item Pixel Canvas
    \begin{enumerate}
      \item Pixel Grid
      \item Ability to Remove Pixel
      \item Ability to Add Pixel
      \item Ability to Change Pixel Color
    \end{enumerate}
  \item Coordinate Viewer
    \begin{enumerate}
      \item Show x Coordinate
      \item Show y Coordinate
      \item Show Pixel Size
      \item Show Pixel Color
    \end{enumerate}
\end{enumerate}

Application will be made responsive. Without further ado, let's begin!
