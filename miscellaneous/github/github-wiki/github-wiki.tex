
\chapter{ Github Wiki }

Using the wiki for Github, is really intuitive and there is no reason why a
chapter should have to be dedicated as to how to use it. However, knowing what
to use the wiki for within an interprise app, can be something that does need
explaining. The following are great uses for the github wiki.

\section{ Using Github Wiki for endpoints }
If you are using GraphQL, you will still have to setup data per each endpoint.
Graphiql client works by setting up data within the url, giving the option to
share a url. Once a developer will click on that link, they will be able to
test with the data for that GraphQL query without having to try it for
themselves.

Creating a wiki for github endpoints is similar. You will create a master page
that will specify this is where wiki pages go. You will then have child pages
per each endpoint that you can use for putting down links per each graphql
query, or mutation. These queries and mutations will include the full data
possible for these queries. This will help greatly alleviate overhead of team.
