
\chapter{ Versioning }

\section { Git }
It goes without saying that you should be using Git. Unless you have an
incredibly huge diffing tree in your repo, which is most of us, then you should
be using Git.

\section { Integration with JIRA }
In any JIRA ticket, if one has the proper webhooks setup with Github, or if one
is using BitBucket, then it will automatically hook up the commit into the
ticket. All that is neccesary, is for the the commit message to include the
ticket name. (Screenshot for example should go here).

\section { Analyzing a Git Branch }
In any git environment, the name of the git branch is important for the
following reasons:

\begin{enumerate}
  \item It can inform the developer about the type of branch
    \begin{enumerate}
      \item feature
      \item hotfix
      \item bugfix
      \item refactor
      \item cleanup
    \end{enumerate}
  \item It can inform the developer of ticket being worked on
  \item It can inform the developer of abstract of ticket.
\end{enumerate}

In addition, as we have mentioned above, it will be particularly advantageous to
having the branch name, when it comes to integrating with JIRA.

\section { Branching Name Convention }
Let's imagine our project is called PIXE. We have a ticket called PIXE-113,
which is responsible for adding a pixel color picker to our Pixel Illustrator
app. We would create our branch as following:

\begin{verbatim}
  feature/PIXE-113-details-and-actions
\end{verbatim}

\begin{itemize}
  \item feature is name of issue type all lowercase followed by forward slash
  \item PIXE-113 is name of ticket in all caps, followed by dash
  \item details-and-actions is abstract of ticket
\end{itemize} \marginpar{Todo: Create a graphic for this one}

\section { Git Client }
When it comes to using an IDE, I strongly believe that every team member should
have the freedom to use whatever they want. It is very important, in general,
that all team members feel free to use whatever they want. In this same vein
they should feel comfortable in using a client if they would like. I personally
prefer using the terminal.

\section { Fork and Pull Workflow }
It is reccomended to use a fork and pull workflow. There are a number of
benefits to adopting such a workflow. Most importantly, however, is the
realization, that setting up a workflow the sooner the better is for a project.
In addition, the realization, that it is a one time setup.

\subsection { Setting up a Fork and Pull Workflow with Github }
Similar to how in this book we reccomend the usage of particular architecture,
we also reccomend specific platforms to use. We will run through this workflow
with Github, however, it is very possible with any other client.

Click on the button which says fork: (Include image for fork here).
Then go ahead


\section { Squash and Merge }
Github, is our preferred client. It gives the option to squash and merge a
particular commit. Squashing your commits should be your preferred way of doing
commits. It allows for a clean git history. By having a clean git history, one
can create a very clean version log, which specifies which features were built
in the latest version.
