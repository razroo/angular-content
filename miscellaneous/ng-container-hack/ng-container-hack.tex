\chapter{ NG Container Hack for Structural Directives }

In Angular there are two quirks with regards to structural directives:
\begin{lstlisting}[caption=Quirk \#1 - Requiring Extra Element to Implement Structural Directive]
<div *ngIf="someVariable">
  <p>The show goes on.</p>
  <p>and on and on and on.</p>
</div>
\end{lstlisting}

\begin{lstlisting}[caption=Quirk \#2 - Inability to apply more than two 
structural directives on the same element]
<ul>
  <li *ngFor="let box of boxes" *ngIf="box.item === 'food'">{{ box.name }}</li>
</ul>
\end{lstlisting}

The quirk probably naturally arises, because for loops, and if statements
weren't meant to be used within a template. So if we think about it that way,
wherein we can use some sort of Angular functionality, to bring it out of the 
DOM, it brings us to ng-container. 

\section{Understanding ng-container}
In the \href{https://angular.io/guide/structural-directives#group-sibling-elements-with-ng-container}{Angular documentation}, ng 
\lstinline{ng-container} is specified as a way to group elements without introducing a new 
html element. Some of the examples mentioned include a \lstinline{<span>} element that 
might introduce some accidental styling, or attempting to put a \lstinline{<span>} inside 
of a \lstinline{select} element. \lstinline{ng-container} will allow for to side step those 
issues, by not introducing a new element to the actual DOM. We can use the 
\lstinline{ng-container} to solve the quirks we mentioned earlier: 

\begin{lstlisting}[caption=Solution to Quirk \#1 - Requiring Extra Element 
  to Implement Structural Directive]
<ng-container *ngIf="someVariable">
  <p>The show goes on.</p>
  <p>and on and on and on.</p>
</ng-container>
\end{lstlisting}

Here, by introducing an ng-container, we no longer have to introduce a new div, 
if we want the content to show conditionally. Likewise, to solve the issue we 
had before of being able to use two structural directives, we can do the 
following: 
\begin{lstlisting}[caption=Solution to Quirk \#2 - Inability to apply more 
  than two structural directives on the same element]
<ul>
  <ng-container *ngFor="let box of boxes">
    <ng-container *ngIf="box.item === 'food'">
      <li>{{ box.name }}</li>
    </ng-container>
  </ng-container>
</ul>  
\end{lstlisting} 

We could technically apply the \lstinline{*ngIf} on the \lstinline{li} element 
itself, for consistency sake. If I had a team member that preferred otherwise, 
I would be more than happy with that, just my preference. 

\textit{Thank you to Austin Spivey for being the person where I saw this 
approach from.}