\maketitle{}
\section{ Constants }

\subsection{ What is a Constant? }
In Javascript the idea of having a constant would be assigning a variable, to a
particular value. Whenever we would like that value, instead of typing out the
value, we would use the variable. At first, however, it might seem
counter-intuitive. Why use the constant value, if it is literally named the same
thing as the actual value?

\begin{lstlisting}[caption=Example of a Constant]
// Update last updated value to have latest payload data
const LAST_UPDATED = "LAST_UPDATED";
new updateValue(payload, LAST_UPDATED);
\end{lstlisting}

\subsection{ Benefits of a Constant }

\subsubsection { Creates a Table of Contents }
When one creates a series of constants in particular file for a certain
component, one can peruse through the constant file, being able to see all
actionable items in one. For instance, the following:
\begin{lstlisting}[caption=Example of a Constant]
// imagine these constants, are in a folder called ValueActionTypes
const UPDATE_VALUE = "UPDATE_VALUE";
const ADD_VALUE = "ADD_VALUE";
const DELETE_VALUE = "DELETE_VALUE";

//imagine the following code is in a new folder called valueActions
import * as types from "../ValueActionTypes";
import {BuyerValue} from './buyer-filter.interfaces';

export class UpdateValue implements Action {
  readonly type = UPDATE_VALUE;

  constructor(public payload?: BuyerValue, public keyName?) {};
}

export class AddValue implements Action {
  readonly type = ADD_VALUE;

  constructor(public payload?: BuyerValue, public keyName?) {};
}

export class DeleteValue implements Action {
  readonly type = DELETE_VALUE;

  constructor(public payload?: BuyerValue, public keyName?) {};
}
\end{lstlisting}
