\maketitle{}
\section{ Creating a Second Component }
Creating a second component within your app is a monumentous occasion. It lays
down the groundwork for how you are going to integrate numerous components
together within your app. Let's lay down at a high level what this means with
regards to a Mono Repo, in an enterprise Angular setting:

\begin{enumerate}
  \item Lib Folder
  \item Angular CLI Command
  \item Ngrx/nx ngrx considerations
  \item Routing Considerations
  \item Service Considerations
  \item Pipe Considerations
  \item Responsive Considerations
\end{enumerate}

\subsection{ Angular Pixel Illustrator - Example Use Case }
Let's take the above considerations into our app. The next component we are
going to create is a color picker. It will contain two sub components, a
background color picker, and a pixel color picker, re-used by a singular
component.

It will most definitely be going to into the lib folder. We will be using the
angular cli. We will be creating state, called color. With regards to a route,
we do want to create a generic route, that will switch out from the pixel grid
chooser, over to a a color picker view. We are also going to want a
consideration for mobile as this is a PWA. There aren't any pipes we will be
using.

\subsection{ Dissecting Business Requirements for Color Picker }
With regards to the color picker. There are three unique aspects of the business
logic:
\begin{enumerate}
  \item Color picker + Background Color Picker - Shared Logic
    \begin{enumerate}
      \item RGB
      \item HEX
      \item Convert RGB to HEX and vice versa
      \item Have color bar below pixel picker, change based on color value.
    \end{enumerate}
  \item Background Color picker
    \begin{enumerate}
      \item Change grid background, based on background color picker.
    \end{enumerate}
  \item Color picker
    \begin{enumerate}
      \item Change pixel color, based on pixel color picker.
    \end{enumerate}
\end{enumerate}

\subsection{ Creating a Second Component - Putting it all Together }
As we mentioned in the beginning in this chapter, this process is going to
be created each time we create a new component. Therefore, it would make sense
to sum up this process so we can repeat the process in every app. In addition,
for redundancy sake, we will go back to this checklist, and repeat the process.

\begin{tabular}{@{} l *4c @{}}
\toprule
 \multicolumn{1}{c}{Color Picker} & Need  & Do Not Need \\
\midrule
 Model $X$ & X1 & X2 & X3 & X4 \\
 Model $Y$ & Y1 & Y2 & Y3 & Y4 \\bottomrule
\end{tabular}
