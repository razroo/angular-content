\maketitle{}
\section{px schematics}

Now that we have deep dived into creating a schematic. We have the particulars
with regards to our application. We would like to architect a folder structure
that might include some repeat files, such as our app logo.

\mybox{
\subsection{A Word to the Wise}
First off, there is a need to run npm run build within the repo, everytime
that you go ahead and create a schematic. Otherwise, it will not work as
expected.
}

\subsection{Create a Files Directory}
Within our px-schematics, let's create a files directory:
\\
\\
\begin{forest}
  [libs
    [common
      [animations
      ]
      [assets
      ]
      [core
       [auth]
       [guards]
       [pipes]
       [validators]
      ]
      [models
      ]
      [testing
      ]
      [ui
      ]
      [utils
      ]
      [styles
      ]
      [vendor
      ]
    ]
  ]
\end{forest}

Due to how Angular Schematics works, we can go ahead and create a folder
directory as is, and supplant it within our app. We are going to create the
following folder directory with placeholder files for our app.
