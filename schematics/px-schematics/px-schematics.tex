
\chapter{px schematics}

Now that we have deep dived into creating a schematic. We have the particulars
with regards to our application. We would like to architect a folder structure
that might include some repeat files, such as our app logo.

\mybox{
\section{A Word to the Wise}
First off, there is a need to run npm run build within the repo, everytime
that you go ahead and create a schematic. Otherwise, it will not work as
expected.
}

\section{Analyzing a File Directory}
Within our px-schematics, let's create a files directory:


\begin{forest}
  [libs
    [common
      [animations
      ]
      [assets
      ]
      [core
       [auth]
       [guards]
       [pipes]
       [validators]
      ]
      [models
      ]
      [testing
      ]
      [ui
      ]
      [utils
      ]
      [styles
      ]
      [vendor
      ]
    ]
  ]
\end{forest}

Due to how Angular Schematics works, we can go ahead and create a folder
directory as is, and supplant it within our app. We are going to create the
following folder directory with placeholder files for our app.

\section{Creating a Files Directory}
We are going to follow a really simply approach when it comes to creating our
files directory and follow a functional approach. We are going to create a
.gitkeep file for each of our directories so that they can be committed to the
app and kept there. It looks something like this:

\begin{lstlisting}
import {
  apply, branchAndMerge, chain, mergeWith,
  Rule,
  SchematicContext, template,
  Tree, url,
} from '@angular-devkit/schematics';
import { strings } from '@angular-devkit/core';
import { libVersions } from "@nrwl/schematics/src/lib-versions";
import { DEFAULT_NRWL_PRETTIER_CONFIG } from "@nrwl/schematics/src/utils/common";

// You don't have to export the function as default. You can also have more than one rule factory
// per file.
export function pxSchematics(options: any): Rule {
  return (host: Tree, context: SchematicContext) => {
    const templateSource = applyTemplateSource(options);
    return chain([branchAndMerge(chain([mergeWith(templateSource)]))])(
      host,
      context
    );
  };
}

let applyTemplateSource = (options: any) => {
  const npmScope = options.npmScope ? options.npmScope : options.name;

  return apply(url('./files'), [
    template({
      utils: strings,
      dot: '.',
      tmpl: '',
      ...libVersions,
      ...(options as object),
      npmScope,
      defaultNrwlPrettierConfig: JSON.stringify(
        DEFAULT_NRWL_PRETTIER_CONFIG,
        null,
        2
      )
    })
  ]);
};
\end{lstlisting}

\section{ Briefly Discussing Code Base}
What we have done in our code, is that we have created a virtual file system,
which has been read from our ./files directory. It is then combined with our
actual file system. This function is called branchAndMerge. This is a really
simple rinse and repeat for generating files within your app.

The documenation for schematics is ever evolving. The project might be
specifically under the angular-cli. However, there is no reason as to why this
should be limited to one framework in particular. I highly reccomend using it
for other frameworks as well.
