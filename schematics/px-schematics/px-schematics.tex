\maketitle{}
\section{px schematics}

Now that we have deep dived into creating a schematic. We have the particulars
with regards to our application. We would like to architect a folder structure
that might include some repeat files, such as our app logo.

\mybox{
\subsection{A Word to the Wise}
First off, there is a need to run npm run build within the repo, everytime
that you go ahead and create a schematic. Otherwise, it will not work as
expected.
}

\subsection{Analyzing a File Directory}
Within our px-schematics, let's create a files directory:
\\
\\
\begin{forest}
  [libs
    [common
      [animations
      ]
      [assets
      ]
      [core
       [auth]
       [guards]
       [pipes]
       [validators]
      ]
      [models
      ]
      [testing
      ]
      [ui
      ]
      [utils
      ]
      [styles
      ]
      [vendor
      ]
    ]
  ]
\end{forest}

Due to how Angular Schematics works, we can go ahead and create a folder
directory as is, and supplant it within our app. We are going to create the
following folder directory with placeholder files for our app.

\subsection{Creating a Files Directory}
We are going to follow a really simply approach when it comes to creating our
files directory and follow a functional approach. We are going to create a
.gitkeep file for each of our directories so that they can be committed to the
app and kept there. It looks something like this:

\begin{lstlisting}
import {
Rule,
SchematicContext,
Tree,
} from '@angular-devkit/schematics';
import {normalize} from "@angular-devkit/core";

// You don't have to export the function as default. You can also have more than one rule factory
// per file.
export function pxSchematics(): Rule {
return (tree: Tree, _context: SchematicContext) => {
  generateAppFolders(tree);
  return tree;
};
}

function generateAppFolders(tree: Tree) {
tree.create(normalize('common/animations/.gitkeep'), '');
tree.create(normalize('common/assets/.gitkeep'), '');

tree.create(normalize('common/core/.gitkeep'), '');
tree.create(normalize('common/core/auth/.gitkeep'), '');
tree.create(normalize('common/core/guards/.gitkeep'), '');
tree.create(normalize('common/core/pipes/.gitkeep'), '');
tree.create(normalize('common/core/validators/.gitkeep'), '');

tree.create(normalize('common/data-access/.gitkeep'), '');
tree.create(normalize('common/data-models/.gitkeep'), '');
tree.create(normalize('common/testing/.gitkeep'), '');
tree.create(normalize('common/ui/.gitkeep'), '');
tree.create(normalize('common/utils/.gitkeep'), '');
tree.create(normalize('common/styles/.gitkeep'), '');
tree.create(normalize('common/vendor/.gitkeep'), '');
}
\end{lstlisting}

Just one quick note, is that the normalize function is the only way of turning
a string into an actual path.
