\maketitle{}
\section{ Schematics Deep Dive }

In the previous chapter we have created a px-schematics schematics. The first
thing that we are going to want to modify is the collection.json file for
px-schematics. Let's add an alias for 'px', as well as create a description.

\subsection{ Creating an Alias and Description }
\begin{lstlisting}
  "px-schematics": {
    "description": "Schematic for generating app folder structure",
    "aliases": ["app"],
    "factory": "./px-schematics/index#pxSchematics"
  }
\end{lstlisting}

\subsection{ Using NPM Link for development }
Being that we do not have an npm module yet, there is a super easy way to hook
up our custom schematic, to the actual Angular Schematics. Go into the root
of your project. For instance, for me that would be:
\begin{verbatim}
/Users/charlie/angularPixelillustrator
\end{verbatim}

Ok, so let's run:
\begin{verbatim}
npm link px-schematics
\end{verbatim}

\subsubsection{ What NPM Link Actually Does? }
When we ran npm link px-schematics, it automatically targeted our px-schematics
folder, and
\begin{lstlisting}
  /Users/charlie/angularpixelillustrator/node_modules/px-schematics ->
  /Users/charlie/.npm-global/lib/node_modules/px-schematics ->
  /Users/charlie/angularpixelillustrator/libs/px-schematics
\end{lstlisting}
