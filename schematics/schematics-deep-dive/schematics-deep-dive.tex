
\chapter{ Schematics Deep Dive }

In the previous chapter we have created a px-schematics schematics. The first
thing that we are going to want to modify is the collection.json file for
px-schematics. Let's add an alias for 'px', as well as create a description.

\section{ Creating an Alias and Description }
\begin{lstlisting}
  "px-schematics": {
    "description": "Schematic for generating app folder structure",
    "aliases": ["app"],
    "factory": "./px-schematics/index#pxSchematics"
  }
\end{lstlisting}

\section{ Using NPM Link for development }
Being that we do not have an npm module yet, there is a super easy way to hook
up our custom schematic, to the actual Angular Schematics. Go into the root
of your px-schematics project. For instance, for me that would be:
\begin{verbatim}
cd /Users/charlie/angularPixelillustrator/libs/px-schematics
npm link;
\end{verbatim}

Now go back into your app root
\begin{verbatim}
cd /Users/charlie/angularPixelillustrator/libs/px-schematics;
npm link px-schematics;
\end{verbatim}

\subsection{ What NPM Link Actually Does? }
When we ran npm link px-schematics, it automatically targeted our px-schematics
folder, and
\begin{lstlisting}
  /Users/charlie/angularpixelillustrator/node_modules/px-schematics ->
  /Users/charlie/.npm-global/lib/node_modules/px-schematics ->
  /Users/charlie/angularpixelillustrator/libs/px-schematics
\end{lstlisting}

\section{ Creating a schema.json for px-schematics }
The schema.json works like a regular schema, telling the CLI(command line
interface) what options can be used. In order to add a schema that is specific
for a collection create a schema.json file in you collection root. E.g.:
\begin{verbatim}
cd libs/px-schematics/src/px-schematics; touch schema.json;
\end{verbatim}

\begin{lstlisting}
  {
    "$schema": "http://json-schema.org/schema",
    "id": "px-schematics",
    "title": "Add px-schematics support to a app directory",
    "type": "object",
    "properties": {
      "name": {
        "type": "string",
        "description": "Name of the folder to contain files",
        "$default": {
          "$source": "argv",
          "index": 0
        }
      }
    }
  }
\end{lstlisting}
If we look at our schema.json file, we will notice, that it contains a couple
of important points. One is that a type field, string in our instance. Second,
that it has the familiar description key/value. In addition, we have created
name as a default. Therefore if we were to pass our schema an option, without
giving it a flag \footnote{For instance: \mybox{px angularPixelillustrator}}
then it will execute it by default.

\section{ Change Folder Architecture }
We are going to want to brace for future collections. So let's create a
collections folder to put our schematics into.
\begin{verbatim}
mdkir collections; mv px-schematics collections
\end{verbatim}

We are also going to modify our collections.json config to let schematics know
that we have our own schematics file. Our collections.json for px-schematics
should look like the following:

\begin{lstlisting}
{
  "$schema": "../node_modules/@angular-devkit/schematics/collection-schema.json",
  "schematics": {
    "px-schematics": {
      "description": "Schematic for generating app folder structure",
      "aliases": ["px"],
      "schema": "./collections/px-schematics/schema.json",
      "factory": "./collections/px-schematics"
    }
  }
}
\end{lstlisting}

If we were to run
\begin{verbatim}
ng g px-schematics:px world
\end{verbatim}

It will output test as is appropriate.
