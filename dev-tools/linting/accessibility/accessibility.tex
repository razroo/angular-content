\chapter{ Accessibility with Codelyzer}    
Codelyzer if not familiar already, is a static code analyzer built on top of Tslint. It's a layer over Tslint with pre-made rules. Some of which include 

More important, is the fact that Codelyzer will be default include rules into Tslint. Some of the rules pre-included are: 
\begin{enumerate}
  \item Component selectors are kebab-case
  \item Directive selectors are camelCased
  \item no-host-metadata-property - Disallows use of \lstinline{host} within components
\end{enumerate}

I love Codelyzer, because it's a great example of how simply thinking ahead, and finding the right tool, can make all the difference in your application. True, you could implement these \lstinline{a11y} compliant linting rules later on. However, having it initially will make you cognizant of everything to begin with. Front loading things ahead of time, while developing new features, will always make things easier long term. 

\section{Angular CLI and Codelyzer}
Angular CLI will actually inlude codelyzer out of the box. However, Codelyzer will not include accessibility rules by default, as these are all "experimental" aka not enough people care about enough by default. Which is a shame, because once again, empathy. So, if you want to go ahead and include these into your application, you will have to physically add them to the tslint.json file. 

\subsection{Add rules to tslint.json}
We are assuming you have codelyzer installed already due to the fact that Angular CLI/NX by default installs Codelyzer. If you do not have it installed, feel free to run the following: 
\begin{verbatim}
npm i codelyzer --save-dev;
\end{verbatim}

\begin{lstlisting}
{
  "rulesDirectory": [
    "codelyzer"
  ],
  "rules": {
    ...,
    "template-accessibility-alt-text": true,
    "template-accessibility-elements-content": true,
    "template-accessibility-label-for": true,
    "template-accessibility-tabindex-no-positive": true,
    "template-accessibility-table-scope": true,
    "template-accessibility-valid-aria": true,
    "template-click-events-have-key-events": true,
    "template-mouse-events-have-key-events": true,
    "template-no-autofocus": true,
    "template-no-distracting-elements": true
  }
}  
\end{lstlisting}

Granted you have tslint enabled within your code editor, there will be two scenarios in which you can take advantage of these rules. 
\begin{enumerate}
  \item From within your Typescript files.
  \item Whenever you run \lstinline{ng lint} over entire project.
\end{enumerate}