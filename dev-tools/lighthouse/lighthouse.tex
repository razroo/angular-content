\chapter{ Lighthouse }    

Lighthouse is a framework agnostic that allows you to audit web pages. Although it's created to be framework agnostic, it's an integral part of the Angular ecosystem.

It is one of the most powerful tool for auditing websites. 

In particular, it is especially powerful for auditing the following:
\begin{enumerate}
  \item SEO 
  \item Performance
  \item accessibility
  \item progressive web apps
\end{enumerate}

Here is an example of what lighthouse looks like:

[insert image here] 

\section{My Concern with Lighthouse}

Lighthouse is open-sourced and automated for improving the quality of web pages. However, it works on periodic checks. However, while continuous integration (CI) is available, it's not pushed heavily by the community. Despite this, Lighthouse should still be used for analysis and performance bench marking. 

\section{Using Lighthouse CI}
The Lighthouse CI is split into two different parts: 
\begin{enumerate}
  \item Lighthouse Node CLI - \say{Runs Lighthouse, asserts results, and uploads reports.}
  \item Lighthouse Node Server - \say{stores results, compares reports, and displays historical results with a dashboard UI.}
\end{enumerate}

\section{Using the Lighthouse Node CLI}
\subsection{Install the Lighthouse Node CLI}
\begin{verbatim}
npm install -g @lhci/cli
\end{verbatim}

