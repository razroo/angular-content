\chapter{ Lighthouse }    

Lighthouse is a framework agnostic piece of technology for auditing web pages. While framework agnostic, it is an integral part of the angular ecosystem. It is the singular most powerful tool for auditing websites. In particular, it is especially powerful for auditing the following:
\begin{enumerate}
  \item SEO 
  \item Perforance.
\end{enumerate}

Here is an example of what lighthouse looks like: 

\section{My Concern with Lighthouse}
While Lighthouse is a fantastic tool, it is more so marketed as a one off tool, whereing you can check periodically you website. This is great, and fantastic for the web developer-journeyman. However, for the average software engineer working on an enterprise level application, it can seem like a bit frustrating. Lighthouse is such a powerful tool, yet included as CI does not seem to push heavily. That being said, we recommend that you use Lighthouse. In addition, we want to make the CI approach more approachabe, so that this becomes the default approach. I for one am not a fan of how TravisCI is the only CI tool introduced alongside Lighthouse CI. From what my devops friends tell me, whether, or not they are right, is that 

\section{Lighthouse Features}
Before we jump into the nitty gritty of setting up Lighthouse CI within our application, I would like to do a deep dive of Lighthouse features. Speaking for myself, when I definitively know that a tool is incredibly useful, I tend to be more motivated to use it. I believe Lighthouse is incredibly useful, and I would liek to show you the same. 

\section{Using Lighthouse CI}
The Lighthouse CI is split into two different sections: 
\begin{enumerate}
  \item Lighthouse Node CLI - \say{Runs Lighthouse, asserts results, and uploads reports.}
  \item Lighthouse Node Server - \say{stores results, compares reports, and displays historical results with a dashboard UI.}
\end{enumerate}

\section{Using the Lighthouse Node CLI}
\subsection{Install the Lighthouse Node CLI}
\begin{verbatim}
npm install -g @lhci/cli
\end{verbatim}

