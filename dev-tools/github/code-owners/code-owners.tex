
\chapter{ Creating Code Owners }

GitHub is one of the most popular repository clients for pull requests. As the team grows, it becomes impoerative that a default code review workflow is set up. Creating a \codeowners{} file lets you do this.


\section{ What is a \codeowners{} file? }
A \codeowners{} file is a config file that's used with GitHub. It will allow you to specify which GitHub user are considered as a  code owner for a specific project.

This person will be the default code reviewer for files that are pushed into a particular branch. You can also set a target code owner for a specific file type.

For example, if you have a dynamic team, you'll most likely end up with a dynamic code base that can vary in language implementations. Part of the team develops in Python, and the other half develops in JavaScript. You can set the \codeowners{} to target each language based on file extensions. 

\section{ How to create a \codeowners{} file? }
There are two places that you can create a \codeowners{} file. The first is at the root of your app. The second is at the .github folder at the root of your repo. It's recommended that you set your \codeowners{} file here. This allows you to create webhooks if required. 

Do create a \codeowners{} file, use the following command:

\begin{verbatim}
  mkdir .github; cd .github; touch \codeowners{}
\end{verbatim}

When you make a pull request within your GitHub app, GitHub will automatically pick up on this file.

\section{ Creating \codeowners{} }

Here's what your \codeowners{} file can potentially look like: 

\begin{lstlisting}
# Lines starting with '#' are comments.
# Each line is a file pattern followed by one or more owners.

# These owners will be the default owners for everything in the repo.
*        @CharlieGreenman

\end{lstlisting}

\section{ Creating \codeowners{} Based on File Type }
\begin{lstlisting}
.js @CharlieGreenman
\end{lstlisting}

Now Charlie Greenman will be a code owner whenever a js file is a part of a pull request.

\mybox{
\section{ Final Note }
When using a \codeowners{} file for the first time, the \codeowners{} will only be created once an actual push is made to your repo.  
}
