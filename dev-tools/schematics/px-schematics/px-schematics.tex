
\chapter{px schematics}

Now it's time to architect a folder structure for containing repeated versions of a file, such as the app logo. 

\mybox{
\section{Sidenote}
You need to run npm run build within the repo every time that you want to create a schematic. Otherwise, it won't work as expected. 
}

\section{Analyzing a File Directory}
Within our px-schematics, let's create a files directory:


\begin{forest}
  [libs
    [common
      [animations
      ]
      [assets
      ]
      [core
       [auth]
       [guards]
       [pipes]
       [validators]
      ]
      [models
      ]
      [testing
      ]
      [ui
      ]
      [utils
      ]
      [styles
      ]
      [vendor
      ]
    ]
  ]
\end{forest}

Due to how Angular Schematics works, we can create a folder directory as is, and supplant it within our app. We are going to create the following folder directory with placeholder files for our app.

\section{Creating a Files Directory}
We are going to follow a functional approach for constructing our file directory. 

Start with a .gitkeep file for each of our directories so that they can be committed to the
app and kept there. It looks something like this:

\begin{lstlisting}
import {
  apply, branchAndMerge, chain, mergeWith,
  Rule,
  SchematicContext, template,
  Tree, url,
} from '@angular-devkit/schematics';
import { strings } from '@angular-devkit/core';
import { libVersions } from "@nrwl/schematics/src/lib-versions";
import { DEFAULT_NRWL_PRETTIER_CONFIG } from "@nrwl/schematics/src/utils/common";

// You don't have to export the function as default. You can also have more than one rule factory
// per file.
export function pxSchematics(options: any): Rule {
  return (host: Tree, context: SchematicContext) => {
    const templateSource = applyTemplateSource(options);
    return chain([branchAndMerge(chain([mergeWith(templateSource)]))])(
      host,
      context
    );
  };
}

let applyTemplateSource = (options: any) => {
  const npmScope = options.npmScope ? options.npmScope : options.name;

  return apply(url('./files'), [
    template({
      utils: strings,
      dot: '.',
      tmpl: '',
      ...libVersions,
      ...(options as object),
      npmScope,
      defaultNrwlPrettierConfig: JSON.stringify(
        DEFAULT_NRWL_PRETTIER_CONFIG,
        null,
        2
      )
    })
  ]);
};
\end{lstlisting}

\section{ Briefly Discussing Code Base}
We've created a virtual file system that's being read from our ./files directory. When we generate files, it is implemented with our actual file system. This function is called branchAndMerge. 

The documentation for schematics is evolving and can be extended to other frameworks and libraries. 
