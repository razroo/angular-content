\maketitle{}
\section{ Using Angular CLI in an Nx Workspace }

Now that we have created an nx workspace, let's create our app. Run
\begin{verbatim}
  ng g app angularPixelIllustrator --routing
\end{verbatim}

This will create an app called angularPixelIllustrator with routing capabilities
using the Angular CLI \footnote{If you will recall, we discussed the Angular
CLI folder/file directory in the Angular CLI Chapter}.

We can now serve\footnote{I.e. run on a server for development reasons} our app,
by running:
\begin{verbatim}
  ng serve --app=angularPixelIllustrator
\end{verbatim} \footnote{It's important to note, that ng serve angularPixelIllustrator
, without app will work by default. }

At this time, if you were to go to localhost:8080 you will see our app, is
ready to go.

Let's now create our first component. For our Pixel Illustrator, we want a form.
We will name the component choose size.
