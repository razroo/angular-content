\chapter{ Scully: Static Site Generation for Angular }
Traditionally, static sites are just that - static. They didn’t change and what you coded is essentially what you’d get. However, with the growth of APIs, the idea of JAMstack - that is, JavaScript, APIs, and Markup - grew in popularity.

By 2017, enterprise-level JAMstack projects started to make an appearance, with the first JAMstack conference commencing in 2018.

The idea behind JAMstack is that you can run entire websites without the need for server side code. This leaves you with the ability to focus on your front end experiences and reduce server response times for pages significantly. Your data needs are covered by APIs, with JavaScript dealing with the necessary connections between APIs and what users get to experience.

Scully sits in the sweet spot of being Angular’s first static site generator that is helping fuel the growth of Angular based JAMstack sites. 

Under normal Angular circumstances, we’d lazy load each component as they’re needed. When it comes to JAMstack, pre-generated static pages are required. This means that if JavaScript is disabled for whatever reason, the site will continue to work.

When using Scully, you’re essentially adding an additional step to the build process that compiles and builds the required static pages, in addition to the ability to generate the required code with the help of ‘plugins’ to get started.

The fun part of Scully is that the code generated can be unit tested. This means that Scully can be implemented and integrated into your projects with end to end testing capabilities.

In part, this is because the code generated fits neatly into the Angular code seamlessly. 

\section{Getting started with Scully - the static part}
Scully can sit on top of any existing Angular project and can easily be added via the CLI.

To add Scully to a project, simply use the following commands while in the root folder of your project.

\begin{verbatim}
ng add @scullyio/init
\end{verbatim}

To build and run with Scully, use the following commands:

\begin{verbatim}
ng build
npm run scully
\end{verbatim}

Once you’ve done this, you’d find static files inside your dist folder. A folder called static will appear alongside your application folder. 
Viola! You’ve just converted your existing Angular app into a JAMstack, static (but still data dynamic capable) site.

\section{Getting started with Scully - the generator part}
The perk with Scully is that it allows us to generate a blog using Angular’s generation schematics. What this means is that you can use ng generate to create Scully templates.

To generate a simple blog, use the following command:

\begin{verbatim}
ng g @scullyio/init:blog
\end{verbatim}

This command will create a blog module with the required routes. You’ll also get a blog folder with some markdown files. Every time Scully builds, the markdown files will be rendered into HTML.

To create your first post, use the following command:

\begin{verbatim}
ng g @scullyio/init:post --name=”the-first-post”
\end{verbatim}

If you look at your blog folder, you’d find a file called the-first-post.md with some markdown content inside.

When you want to run the blog, you’ll need to build it because Angular can’t read markdown and the contents of the post will need to be compiled into HTML.

Every time you build, the markdown will be converted into the necessary code and cut down on server response time because it’s serving from generated static files.

To serve your Scully site, run the following command:

\begin{verbatim}
npm run scully serve
\end{verbatim}

It’s time we talked about Scully services
Creating a Scully service is similar to creating a typical Angular service. What a Scully service enables your app with the ability to tap into the pre-rendered pages and display the content as needed.

To do this, you need to import ScullyRoute and ScullyRoutesService into your component and then initialize it as an Observable. ScullyRouteService is a pre-written service by Scully and comes packaged with the library.

\begin{lstlisting}
import{ ScullyRoute, ScullyRoutesService } from ‘@scullyio/ng-lib’
…
…
export class AppComponent implements OnInit{
   posts$: Observable<ScullyRoute[]>;
   constructor(private srs: ScullyRoutesService){ }
}

To connect and display it in your view:

<ul>
   <li *ngFor=”let post o posts$ | async”>
      <a [routerLink]=”post.route”>{{post.title}}</a>
   </li>
</ul>
\end{lstlisting}

\section{What about Angular Universal?}
Angular Universal has been around for a while and is the natural solution to single-page apps SEO related rendering woes. 

The similarities between Angular Universal and Scully lies in their pre-rendering aspect. However, Angular Universal and Scully diverge on how pre-rendering is done, thus marking the end of their main comparison point.

When it comes to Angular Universal, there is server-side rendering involved. This means that pages are built on the fly and then sent over to the client-side. On a technicality, Angular Universal is not JAMstack because it involves server-side rendering and is not a truly statically generated site. 

Scully however, comes pre-rendered at deployment. This works well if the content deployed is not expected to change over time - making it perfect for blogs and blog posts.

While Angular Universal is pretty simple in approach - generate a pre-render on the server-side to give to client-side - Scully’s methodology runs at two levels.

The first level is that the pre-rendered page is given to the user when requested. The second level is the actual ‘real’ Angular app that sits in the background on top of the pre-rendered view. So what you’re essentially getting is the pre-render for a particular view (if available) and the actual app - rather than the app and then the page.

This cuts down on the initial delay and the need to call your database for data. As a result, Scully can help reduce the read loads on databases in the long term, allowing for more cost-effective ways to scale. This is because reading static files is much lower in cost than reading from a database.  

\section{Final thoughts}
It may not seem like a big deal but static site generation is a rendering concept that reduces overall server loads and increases the potential to scale in a cost-efficient manner. Scully is an exciting community development because it automates the pre-rendering process and contributes greatly to the JAMstack ecosystem. 

Another perk of pre-rendered single-paged apps that are not widely discussed enough is how it impacts on SEO and the ability to rank effectively on search engines. 

The issue that many new developers face with SPAs is that their app may be working but fails to get picked up by major search engines because crawlers and bots are simple in design and work best on content that is not dynamically generated.

This problem is fixed when you use Scully as all pages are pre-rendered and load data only when you navigate away and into a space that requires dynamic content. 

Integrating Scully into your existing development workflow to leverage the ability to generate static content is something that can be easily done. It is just a matter of importing Scully as a module, building the project and then running it. This entire process can be scripted inside your package.json scripts section. 

As Scully grows, the ecosystem surrounding Scully is also expected to grow in the same way it did with Angular Universal. While there is debate if Scully will replace Angular Universal completely, it’s still too early to make any dramatic conclusions.

Scully, after all, is still quite new. 

Scully does, however, provide a methodology that’s highly needed but unsupported effectively for blog related activities. The Internet is built on content and Scully is a tool that helps Angular developers get a leg up on supporting parts of the app that’s traditionally linked with content that’s unlikely to change often. 








