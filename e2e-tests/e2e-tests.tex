\maketitle{}
\section{ E2E Tests + TDD/BDD }

One of the tricky things with regards to E2E tests, is how it fits into a
TDD/BDD environment. Writing unit tests before we can see anything in our UI
already takes quite a bit of discipline. Adding in an E2E test to the workflow
seems like a bit much? Ok, so let's get into the thick of it. I think we'll all
have a good time!

\subsection{ What is TDD? }

\begin{enumerate}
  \item Start by writing a test
  \item Run the test and any other tests. At this point, your newly added test
   should fail. If it doesn’t fail here, it might not be testing the right
   thing and thus has a bug in it.
  \item Write the minimum amount of code required to make the test pass
  \item Run the tests to check the new test passes
  \item Optionally refactor your code
  \item Repeat from 1
\end{enumerate}

\subsection{ The Benefits of TDD? }
\begin{enumerate}
  \item Higher test coverage. \footnote{Emphasis is put on testing}
  \item Focus
    \begin{enumerate}
      \item Focus one part of issue one of a time.
      \item Allows one to realize when to stop coding.
    \end{enumerate}
\end{enumerate}


First things, first. Which comes first, unit test, or E2E tests? E2E tests will
end up going first, however, we will end up satisfying the unit tests first.
