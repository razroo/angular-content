\maketitle{}
\section{ PWA Toolset }

\subsection{ Physical Devices }
There is currently a formula with which devices to use. Latest regular sized
Iphone, and Iphone Plus. Latest Google Pixel non plus
\footnote{That's right, skip the Samsung.}. Latest Ipad + Ipad Mini. That
is it. These are mostly used as a way to see web in real time, as you are
developing your application.

Therefore, the reccomended Physical mobile devices are as follows:
\begin{enumerate}
  \item Iphone 8
  \item Iphone 8 Plus
  \item Google Pixel 2
  \item Ipad (2018)
  \item Ipad Mini 4
\end{enumerate}

\subsection{ Testing Local Server on Physical Device }

Now that we have our physical devices that we would like to work on, let's set
up a way that we can test on these mobile devices. Ideally the following three
criteria should be solved:
\begin{enumerate}
  \item Url that remains the same for dev - to be used on mobile device
  \item When edit is made, it should update all mobile devides simultaneously
  \item Have all mobile devices in a central location, so that we can visibly
  see all changes that are being made
  \item synchronized interactions \footnote{Clicking on a button in one place
  change it in all other places.}
\end{enumerate}

\subsection{ Ghost Labs }

First, our winner for responsive testing is Ghost Labs. Ghost labs fulfills all
of the above criteria mentioned above. Going into short why we chose it over
all other contendors:
\begin{enumerate}
  \item Very easy to setup, and therefore removes overhead for initial setup
  \item There isn't anything required to install on different devices. It is
  simply a url that is used, and shared across device.
  \item Screenshots on remote mobile devices
  \item Ghostlab has a built in inspector for debugging
  \item One click workspace, in order to start up all devices once again.
  \item Presentation mode, allowing users to present web app.
\end{enumerate}
