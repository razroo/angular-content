\maketitle{}
\section{ Color Picker }

Let's go through the steps again, being that we are creating our second
component. Part of learning is not only discovery, but maintenance. As discussed
in the preface, whenever we have the chance to repeat steps we have discussed
once before, we will re-iterate them on a high level for memory sake.

\subsection{ Outline of what needs to be done }
\begin{enumerate}
  \item Create a UI dumb component for color picker in Lib folder
    \begin{enumerate}
      \item Should be generated in app folder in UI folder.
      \item Use ClI to generate module
      \item Use ClI to generate component
    \end{enumerate}
  \item Import module into pixel-grid page
  \item Add component to pixel-grid page html
  \item Add proper styling to the illColorPicker
\end{enumerate}

\subsection{ CLI - Creation of Module and Component with Routing }
First, let's create our component in the lib folder of our app.

\begin{lstlisting}
ng g lib color-picker --routing --directory="dealworks/ui"
ng g component color-picker -a=color-picker --export
\end{lstlisting}

\subsection{ Add Color Picker to Pixel Grid Page Component }
Let's import our ill color picker into our Pixel Grid page.

\begin{lstlisting}
// pixel-grid-page.module.ts
+ import { IllColorPickerModule } from "@ill/ill-color-picker";
// in imports array
+ IllColorPickerModule
\end{lstlisting}

In addition, let's add the ill-color-picker component to our html.
\begin{lstlisting}
+ <ill-color-picker></ill-color-picker>
\end{lstlisting}

\subsection{ Add Styling Element to Create Basic  }
Let's add a basic width and height for out element.
\begin{lstlisting}
.IllColorPicker {
  display: flex;
  flex-direction: column;
  width: 200px;
  height: 100%;
}
\end{lstlisting}
