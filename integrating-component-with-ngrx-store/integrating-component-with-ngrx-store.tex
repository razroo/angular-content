\maketitle{}
\section{ Integrating a Component with @ngrx/store }

Another chapter has been dedicated solely to integrating a component
with @ngrx/store. This is because it is a cookie cutter process. Observables
are notoriously known, for abstracting events, and therefore allowing the same
code to be repeated time, and time again. The following is what can be expected
to be repeated time and time again in your application, after initially
generating files.

Just for redundancy sake, there are six steps, that go into setting up state
with any given component, that are handled by the nx ngrx cli:
\begin{enumerate}
  \item Store
  \item Action
  \item Store
  \item Reducer
  \item Initial State/Enums\footnote{In some other framework it might be constants}
  \item Effects
\end{enumerate}

All that is left, is for us to now to do things
