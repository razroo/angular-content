\maketitle{}
\section{ Dialogs }

There are certain components that are expected in any architecture. Dialogs are
one of those components. However, their architecture can be so complex, that it
is very important to come up with a strategy that works with state management.
This architecture was originally adopted from Thomas Burleson who I have a
tremendous amount of respect for.

A couple of issues that dialogs have, is that they are:
\begin{enumerate}
  \item Defined in Templates
  \item Created at App Startup
  \item Managed in UI Components
  \item They complicate UI Components
    \begin{enumerate}
      \item Open, close, cancel
      \item Services
      \item Store Actions
      \item Pending State
    \end{enumerate}
  \item Dialogs may interupt NgRx Flows
\end{enumerate}

\subsection{ Centralizing Dialogs + Best Practices}
As with the rest of our architecture, we are trying to centralize the way we
use our components. Therefore, the following are dialogs best practices:
Section 1
\begin{enumerate}
  \item Use @angular/material Dialogs
  \item Use Dialog Service to open and and manage custom dialog components
  \item Use Dialog configurations to customize dialog instance
\end{enumerate}
\begin{enumerate}
  \item Use Ngrx Effects to Manage Dialogs
  \item Inject Store/Facade into Dialog Components
\end{enumerate}

\begin{enumerate}
  \item Use Ngrx Dialog Effects
  \item Disconnect Dialog from Feature Effects
\end{enumerate}

Hmm, I really feel that this point onwards in this part of the editor is really
out of place with regards to what we are trying to achieve in this chapter. 
