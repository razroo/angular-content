\section{ Services }
\maketitle{}

Services in Angular can be a very confusing topic. I know that for myself it's
been very confusing. A very common core concept in programming is DRY. That is
do not repeat yourself. If there is an option to do something once, and re-use
it throughout one's application, then that is what someone should do. Services
very much so fill in this gap. That is, any functionality that is contained
within a component should be put into a service. That service now serves two
purposes:
\begin{enumerate}
  \item Keeps the component logic-less. Any logic, is now transferred over to
  the service.
  \item It potentially allows the service to be re-usable.
\end{enumerate}

The above is the base role of a service. It is what is a result of the common
sense principle of DRY.

\subsection{ Where Services Get Complicated }
At this point, services start to get complicated. The following is a great
example. Let's say we have an overview page. Any logic on the overview page is
going to be unique to the overview page, as it's logic is unique. So we create
service that is unique for the overview page. Now that isn't neccesarily dry is
it? I mean, the component could have very much so done what the service is doing.
In addition, when it comes to unit testing, it probably would be easier to keep
it all within the component.

There are other dillemas, such as what happens when we are using a service to
fetch data. What about a service that is used for graphics? What about a
service that we use for handling errors? They end up becoming abstracted to the
point, where it is very much so more beyond being DRY. It is pure logic, that
is really independent of any component. It beckons a more proper definition.

\subsection{ Services - Creating a more Robust Definition }
Service is a broad category encompassing any value, function,
or feature that an app needs. \footnote{https://angular.io/guide/architecture-services}
Services fill any gaps left by any other part of any application. In a perfect
world, services will solely interact with data, and should be doing so 80\% of
the time.
