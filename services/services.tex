\section{ Services }
\maketitle{}

Services in Angular can be a very confusing topic. I know that for myself it's
been very confusing. A very common core concept in programming is DRY. That is
do not repeat yourself. If there is an option to do something once, and re-use
it throughout one's application, then that is what someone should do. Services
very much so fill in this gap. That is, any functionality that is contained
within a component should be put into a service. That service now serves two
purposes:
\begin{enumerate}
  \item Keeps the component logic-less. Any logic, is now transferred over to
  the service.
  \item It potentially allows the service to be re-usable.
\end{enumerate}

The above is the base role of a service. It is what is a result of the common
sense principle of DRY. 
