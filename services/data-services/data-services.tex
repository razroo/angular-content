
\section{ Data Services - Directory Structure }

In an @ngrx/store setting, a very large part of an app's services will be
directed towards handling data directly. For instance, let's say you have a
GraphQL user query that get's a current user's data. The service would look
something like this:
\begin{lstlisting}
import { Injectable } from '@angular/core';

import { Observable, from } from 'rxjs';
import { pluck } from 'rxjs/operators';
import { Apollo } from 'apollo-angular';

@Injectable({ providedIn: 'root' })
export class UserService {
  getUser(): Observable<User> {
    const user$ = this.apollo.query({ query: GetCurrentUser });

    return from(user$).pipe(pluck('data', 'getCurrentUser'));
  }
  constructor(private apollo: Apollo) {}
}
\end{lstlisting}

It's a rough guess, but odds are that 80\% of services within any
Angular app, will be built to directly work with data. These services will be
used by a number of effects. In addition, they will contain their own respective
unit tests.

\subsection{ Setting the Landscape }
As of Angular 6, Angular offers a metadata option called providedIn. In short,
if providedIn is set to 'root', it allows us to bypass the need of using a
module, and creates a tree-shakable version \footnote{https://angular.io/guide/dependency-injection-providers\#creating-tree-shakable-providers}.

Therefore, the bundler will be able to tell if this service is being used in a
particular module. If it is not, service will not be bundled with particular
module. This allows us to be very liberal with our services and put them in a
central location. (Of course we would be able to do this with a module, but
it make's it more architecturally appealing without the need of using a
module)

\subsection{ Data Services Folder/File Structure }
\begin{forest}
  [libs
    [px-illustrator
      [data-services
        [src
          [lib
            [services
              [user
                [\_user.service.ts,file]
                [\_user.service.spec.ts,file]
              ]
            ]
          ]
          [\_index.ts,file]
          [\_test.ts,file]
        ]
        [\_karma.conf,file]
        [\_README.md,file]
        [\_tsconfig.lib,file]
        [\_tsconfig.lib.json,file]
        [\_tsconfig.spec.json,file]
        [\_tslint.json,file]
      ]
    ]
  ]
\end{forest}


Just as simple as that. We have a central location called data-services where
we put bulk of services that deal with retrieving data, and returning it
as an observable. If you want to be really technical about it, you can create
every data-service as it's own lib. However, I think this is a bit much, as
services tend to not touch the DOM and run really fast as unit tests.
