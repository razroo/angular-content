\maketitle{}
\section{ Linting Sass }

As we mentioned in the previous chapter, there is a greate matter of importance
when it comes to linting. In particular when it comes to sass, the Angular CLI,
nor the Nrwl Nx cli will offer sass linting out of the box. In particular,
we will be choosing the package sass-lint.

\subsection{Installing Sass Lint}
\begin{verbatim}
  npm install sass-lint --save-dev
\end{verbatim}

\subsection{Adding a Lint Config File}
For sass-lint, it will hook into by default a file that is in the root of
folder called .sass-lint.yml. It's quite long, and you can see the rest of the
file in the actual github repo. However, you would create a sass-lint.yml file.
\begin{verbatim}
  touch sass-lint.yml
\end{verbatim}
Inside of the sass-lint.yml file, it will look something like this:
\begin{lstlisting}
options:
  formatter: stylish
files:
  include:
    - '{apps,libs}/**/*.scss'
  ignore:
    - 'libs/font-awesome/**/*.scss'
rules:
  # Extends
  extends-before-mixins: 1
  extends-before-declarations: 1
  placeholder-in-extend: 1

  # Mixins
  mixins-before-declarations: 1

  # Line Spacing
  one-declaration-per-line: 1
  empty-line-between-blocks: 1
  single-line-per-selector: 1

  # Disallows
  no-attribute-selectors: 0
  no-color-hex: 0
  no-color-keywords: 1
  no-color-literals: 1
  no-combinators: 0
  no-css-comments: 1
  no-debug: 1
\end{lstlisting}

The list for what sass-lint disallows goes on and on. Some of the linting rules
I really like is no color words, empty line between blocks, bem-depth, for
starters. From an architects perspective, and from a developers perspective, it
has made the code review process much easier. When this linting process is
combined with the functional sass paradigms we will mention, it becomes
self managing architecture for styling. Let me say that again, SELF MANAGING. I
thought that might be worth repeating.

\subsection{Adding an NPM Script in your package.json}
\begin{verbatim}
  "lint-scss": "sass-lint -v -q",
\end{verbatim}

\subsection{The Final Touch}

Adding the sass-lint npm script as part of your CI/CD architecture is what
really brings it all together. This is twofold of course. One so that when you
make a pr for your Github repo, it will check to make sure there are no Sass
linting errors. In addition, when pr is actually merged, pipeline runs as well,
to make sure there are no errors.
