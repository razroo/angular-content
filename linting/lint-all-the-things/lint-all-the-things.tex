
\chapter{ Lint all the Things }

\section{ Lint all the Things }

Linting is incredibly important. Think of it this way. Someone's favorite food
item might be brocolli, another person's favorite food item, might be steak.
Now the two might work well together(on a number of levels, mmm hungry), but
forcing one person to adopt the other food item as their favorite, is unjust!

Ok, bad example, but the concept is the same. Things such as how many spaces per
indentation, double quotes vs. single quotes, trailing commas, empty interfaces,
these are all important points. To be honest, I've only found my peers arguing
on more minute points, such as double vs. single quotes, and how many spaces
per indentation. Therefore, linting allows:
\begin{enumerate}
  \item Agreement on code formatting rules.
  \item Automated way of keeping track of things that need to be changed.
  \item Self documentation through cli, on what needs to be changed.
\end{enumerate}

Another important point, is that we would ideally like a formatter, that
automatically formats these things for us. So, from an architectural perspective
when we are looking for a linter, we are also looking for a formatter to go along
with it.

\section{ What are we trying to lint? }
We are trying to lint HTML, SASS, and Typescipt. Angular CLI, offers Tslint out
of the box. A Tslint has been created out of the box for the Angular CLI.
However, the tslint.json that comes out of the box, will not work well with the
prettier file produced by the Nrwl NX CLI. The best method as of this time, is to
work through with the prettier CLI and work through differences. There is also
a gist which you can copy for your convenience, put here.

Ideally, we would like to create a linter that is agnostic to IDE, so that we
can allow the team to use whatever it is available.

That being said, the Angular CLI only offers a linter outside of the box for
Typescript. However, it should, in my humble opinion it should also offer a
linter for HTML and SASS. We will therefore take the default linting task, and
add three npm scripts on top of it:

\begin{lstlisting}
  "lint-html": "htmlhint --rulesdir './rules/' '{apps,libs}/**/*.html'",
  "lint-scss": "sass-lint -v -q",
  "lint-ts": "ng lint --format=stylish",
\end{lstlisting}

We will get to the specifics of the above three scripts right now.

\section{ Linting Typescript }
As we mentioned Typescript will offer an ng lint command out of the box for
Typescript. This should be moved over to it's own lint-ts command to make room
for linting for html and scss. In addition, the tslint.json should be updated
accordingly to work with prettier.

\section{ Linting Sass }
For linting SASS, we are going to be using the sass-lint package.
\begin{lstlisting}
npm install sass-lint --save-dev
\end{lstlisting}

We are also going to want to create a .sass-lint.yml file. Make sure to put the
dot before the .sass-lint.yml file, so that the sass-lint linter will pick up
on the file by default.

In addition, there is now prettier that it now available for sass. We will have
to follow the same formula that we did for Typescript. Worst comes to worst, we
will have to modify the yml according to what the linter tells us we can, and
cannot do. \footnote{The appropriate commands for prettier will be added in the
following chapter}

Note: Put a link to the appropriate gist here.

\section{ Linting HTML }
For HTML, we are going to be using HTML Hint. It can be argued that it hasn't
been maintained for a while. Two things: One we are banking on the fact that
prettier is going have formatting for HTML soon. Second, how much has html really
changed structurally over the last couple of years. We expect this solution to
change over the next couple of years, but for now, we are using ts-lint.

\begin{lstlisting}
npm install htmlhint --save-dev
\end{lstlisting}

We are also going to create a .htmlhintrc file.

There is no prettier yet, so don't worry about this one yet. We'll update the
book and let you know when it happens.
