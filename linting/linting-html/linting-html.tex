\maketitle{}
\section{ Linting HTML }

Linting html is not as mature as it is with regards to Sass and Javscript in my
opinion. Howvever, that makes sense as html is essentially strucutred xml that
is used within the context of a web app.

That being said, once again, there is no linter that is offered out of the box
through the Angular CLI, or Nrwl Nx either. It is, however, beneficial.

\mybox{
\subsection{ Side Bar }
Why not use something like Pug(once called Jade) for html templating. The main
benefit of something like Pug I have found is that it tells me where the html
begins and ends. In a complex html element, where there are many levels of
nesting this can be beneficial. However, in a templating engines, there tends to
be many quirks, in particular when it comes to using various frameworks. I have
rarely been on a team where after selling Pug, engineers have been enthusiastic
in using it.
}

\subsection{Why We Chose HTML Hint}
The honest truth is that the landscape for html hasn't drastically changed at
it's core level over the past couple of years. Of course, if there was a more
mature linter, the linter would complain that the engineer should use certain
html element as opposed to another. That being said, we have found html hint
to be the most robust html linter, even though development has been lack luster
over the past couple of years.

\subsection{Installing HTML Hint}
\begin{verbatim}
  npm install htmlhint --save-dev
\end{verbatim}

\subsection{Create an .htmlhintrc config file}
In the root of your app, create an .htmlhintrc file. The .htmlhintrc is set up
to be the default config name for html hint. A sample config for html hint will
just be a simple JSON object containing key values. For instance:
\begin{lstlisting}
{
  "attr-value-double-quotes": true,
  "src-not-empty": true,
  "alt-require": true,
}
\end{lstlisting}

\subsection{Adding an NPM Script in your package.json}
\begin{verbatim}
  "lint-html": "htmlhint --rulesdir './rules/' '{apps,libs}/**/*.html'",
\end{verbatim}
