\maketitle{}
\section{ Charts }

Having charts in one's application might not seem like an architectural
decision. In particular, there is a very popular library library out there
called d3. This might not be the right place to bring it up as it is Angular
agnostic. However, data can be something that can be quite abstract. It might
be tempting to tag it all into a singular service, and load it in the
appropriate component. However, using the power on Angular, it would allow for
the compartementalization of graphics, and makes Angular a very powerful tool
for the job.

\subsection{Install D3}
\begin{verbatim}
  npm install --save d3
  npm install --save-dev @types/d3
\end{verbatim}

\subsection{Interfacing D3}
When using any framework in general with a graphics library, the proper approach
is to interface through that component. So, for instance, instead of keeping
the framework really abstract, we can tie it to the DOM of the component. When
we add an arc component, it adds an arc to the component. When we add a tooltip
component, to the app, it adds a donut chart. I would just like to touch lightly
on how this would be done. It is also important to note, that much of what will
be discussed in this chapter, such as models, or visuals are native to Angular.
However, being that we are bringing over something which is not in an Angular
setting, I will be discussing everything, to remove any confusion.

\subsection{ Simplifying an Interface in the Context of Angular }
We've discussed previously the concept of dumb and smart components. That is,
having a component purely for visual purposes, and progating the need for
retrieving data, and handling events to the smart component. This is a very
similar phenomenon, with d3. Only the smart componet is going to be the
donut-chart, and child compoents, will be arcs, and legends. The second
cornerstone to understanding interfacing d3, is to create a service, that will
handle any non-feature specific d3 logic. For instance, let's say we would like
to click on an arc, or hover over, and have a specific function, we can turn
that into a core service. Therefore, moving forward, we would be able to attach
this logic over to another component if need be.

\subsection{ Re-building Pixel Grid, interfacing d3 using Angular }
Let's create a d3 rect component. Instead, we are going to call it the pixel
component. Inside of our pixel component we are going to interface it with
width and height.
