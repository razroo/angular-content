\maketitle{}
\section{ Charts }

Having charts in one's application might not seem like an architectural
decision. In particular, there is a very popular library library out there
called d3. This might not be the right place to bring it up as it is Angular
agnostic. However, data can be something that can be quite abstract. It might
be tempting to tag it all into a singular service, and load it in the
appropriate component. However, using the power on Angular, it would allow for
the compartementalization of graphics, and makes Angular a very powerful tool
for the job.

\subsection{Install D3}
\begin{verbatim}
  npm install --save d3
  npm install --save-dev @types/d3
\end{verbatim}

\subsection{Interfacing D3}
When using any framework in general with a graphics library, the proper approach
is to interface through that component. So, for instance, instead of keeping
the framework really abstract, 
