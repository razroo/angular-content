\section{ Directives }
\maketitle{}

Assuming you are already familiar with Directives, at first glance having
components and directives can seem a bit superfluous. They both seem to do the
same thing. However, in summary, the following is a very clear cut definition
between directives and components.
\mybox{
 A @Component requires a DOM whereas a @Directive does not
}

\subsection{ What Definition Translates To }
What this definition translates to is that if there is something that translates
as a piece of functionality that you woul consider as an add-on, make that a
directive. Something that would be considered as it's own self contained UI
element, it should be considered as a component.

\subsubsection{A Great Example}
A great example of this, is adding drag and drop functionality to a component.
That in it's self is not component worthy. However, adding a dropzone directive
to the element, would make any potential component drag and droppable worthy.
A great example of a component, would be a data table. A large part of it's
functionality is strickly tied to actual UI element.

\mybox{
  A side note with regards to our architecture, directives will very be used. If
  you find yourself writing, with what feels like too many components, don't
  worry, it's too be expected.
}
