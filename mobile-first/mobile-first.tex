\maketitle{}
\section{ Mobile First - Building a Progressive Web App }

\subsection{ Why Build a Progressive Web App? }
When building a an enterprise application, think about building a Progressive
Web App. It will allow your web experience to be built to feel as if it is a
native app experience. Not only will it make it progressive, but it will make
your users feel as if they are a part of an experience that is all encompassing.
It will give them the overencompassing feeling that they are getting the best
experience possible \footnote{We will discuss moving the app over to a native
app soon using NativeScript.}. Swipe right on Progressive Web Apps \footnote{
That is a millenial joke, but also a darn good PWA pun.}.

\subsection{ The Technical Benefits of a PWA }

\begin{enumerate}
  \item Progressive - Work for every user, regardless of browser choice because
  they’re built with progressive enhancement as a core tenet.
  \item Responsive - Fit any form factor, desktop, mobile, tablet, or whatever
  is next.
  \item Connectivity independent - Enhanced with service workers to work offline
  or on low quality networks.
  \item App-like - Use the app-shell model to provide app-style navigations and
  interactions.
  \item Fresh - Always up-to-date thanks to the service worker update process.
  \item Safe - Served via TLS to prevent snooping and ensure content hasn’t been
  tampered with.
  \item Discoverable - Are identifiable as “applications” thanks to W3C
  manifests and service worker registration scope allowing search engines to find them.
  \item Re-engageable - Make re-engagement easy through features like push
  notifications.
  \item Installable - Allow users to “keep” apps they find most useful on their
  home screen without the hassle of an app store.
  \item Linkable - Easily share via URL and not require complex installation.
\end{enumerate}
Kudos to Addy Somani for this List \footnote{https://addyosmani.com/blog/getting-started-with-progressive-web-apps/}
