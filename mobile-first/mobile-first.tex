\maketitle{}
\section{ Mobile First - Building a Progressive Web App }

\subsection{ Why Build a Progressive Web App? }
When building a an enterprise application, think about building a Progressive
Web App. It will allow your web experience to be built to feel as if it is a
native app experience. Not only will it make it progressive, but it will make
your users feel as if they are a part of an experience that is all encompassing.
It will give them the overencompassing feeling that they are getting the best
experience possible \footnote{We will discuss moving the app over to a native
app soon using NativeScript.}. Swipe right on Progressive Web Apps \footnote{
That is a millenial joke, but also a darn good PWA pun.}.

\subsection{ The Technical Benefits of a PWA }

\begin{enumerate}
  \item \textbf{Progressive} - Work for every user, regardless of browser choice
  because they’re built with progressive enhancement as a core tenet.
  \item \textbf{Responsive} - Fit any form factor, desktop, mobile, tablet, or
  whatever is next.
  \item \textbf{Connectivity} independent - Enhanced with service workers to work
  offline or on low quality networks.
  \item \textbf{App-like} - Use the app-shell model to provide app-style
  navigations and interactions.
  \item \textbf{Fresh} - Always up-to-date thanks to the service worker update
  process.
  \item \textbf{Safe} - Served via TLS to prevent snooping and ensure content
  hasn’t been tampered with.
  \item \textbf{Discoverable} - Are identifiable as “applications” thanks to W3C
  manifests and service worker registration scope allowing search engines to find them.
  \item \textbf{Re-engageable} - Make re-engagement easy through features like
  push notifications.
  \item \textbf{Installable} - Allow users to “keep” apps they find most useful
  on their home screen without the hassle of an app store.
  \item \textbf{Linkable} - Easily share via URL and not require complex
  installation.
\end{enumerate}
\textit{Kudos to Addy Somani for this List}
\footnote{https://addyosmani.com/blog/getting-started-with-progressive-web-apps/}

We will go into detail in the following chapters, into detail
\subsection{ Developing a PWA - The Toolset - An Overview}
When Developing Mobile First, there are three tools, which will be particularly
advantageous:
\begin{enumerate}
  \item Physical Mobile Devices \footnote{A moment on which one's it is, that
  you should work on}
  \item Sauce Labs \footnote{Section on why we chose Sauce Labs, over Browser
  Stack}
  \item Chrome Dev Tools \footnote{Section on why we chose Chrome Dev Tools,
  over Firefox}
\end{enumerate}

\subsection{ Developing a PWA - Software - An Overview}
When developing a PWA, it is important for us to keep in mind, that there will
be specific pieces of software to develop. Of course, there is an official PWA
checklist \footnote{https://developers.google.com/web/progressive-web-apps/checklist}
, but specific technologies, would be as follows: 
\begin{enumerate}
  \item Service Worker
  \item Manifest
  \item Lighthouse
\end{enumerate}
