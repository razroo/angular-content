\maketitle{}
\section{ Creating Feature Flags }

Feature flags are an important part of Angular Architecture. What it means is,
that a feature can be hidden, disabled, change code flow \footnote{Changing code flow means having an if statement turned off for instance}
or have routing being prevented from going to component.

\subsection{ Why are Feature Flags Important? }
Feature flags are important, because they can allow the UI Engineering team to
work independently from the back end and Automation/QA engineering. Some sample
situations to illustrate this point:

\mybox{Product asks for a new cc + bcc field to be added to the email client.
We do not have the backend available yet, but would like to integrate it with the
backend at a later date it is ready. Sure, we can put the UI on hold until back
end is ready. However, what we can do, is test and hide ahead of time from the
UI side of things. We can then hide the feature and integrate with backend
when the time is right.}

\mybox{Another great example might be if this is a feature we only want to
introduce to certain members. In addition, we might have admins within our app.
These users when they log in, are given the option to go to the admin interface.
Those who are not admins, are not allowed to see it. A feature flag might be
something else that would be great.}

\mybox{Another use case of when a feature flag might be useful is if we have a
new feature that we only want to show to our closest clients and see what they
think about it, and give them the chance to sample it first. A feature flag
would something incredibly useful in this situation as well.}

\mybox{Let's say there are features that haven't been QA'd yet, and they are
holding back new features that are ready from making their way to stage. This
feature can have a flag put on it, so that it only works for dev and not stage.
That way, QA can test it, and when it is ready to go, we can go ahead and
remove the feature flag. This would stop it from inhibiting stage from having
the push being made. }

\subsection{ Creating a config using an API and server }
Ideally a feature flag system is created through the backend. There would be
some sort of management system, where a product owner can go in, and turn off
features specific to a certain part of the application. There would then be a
report that would go out monthly, let's say, that would tell everyone within the
company which features have been turned off, to make sure there isn't dead code
laying around for features that aren't used. I will not go into detail here, but
this is ideally how this should be built out. [TODO discuss this option more in
detail].

\subsection{ Creating a config }
If your app currently does not have an api server that can be used in order to
pull in the config, then you can create your own config. You can refer to the
chapter on creating a config, on how someone would solve this. With regards to
using the config with app, a service would need to be created as follows:

\lstinputlisting[language=JavaScript]{./configs/feature-flags/feature-flag.service.ts}
