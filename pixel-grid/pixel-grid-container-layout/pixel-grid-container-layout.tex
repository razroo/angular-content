\maketitle{}
\section{ Pixel Grid Container Layout }

Now that we have introduced Flex Layout and our designs call for three elements:
\begin{enumerate}
  \item Code Viewer
  \item Pixel Grid
  \item Color Picker
\end{enumerate}

\subsection{ Anticipate for Future Components }

We know that we will have three components that will be set up side by side.
On our main page component, which will contain all three, we will set the
following three fxLayout directives:
\begin{verbatim}
  fxLayout="row"
  fxLayout.xs="column"
  fxFlexFill
\end{verbatim}

The above is pretty straight forward. On screens not extra small, the layout
will be flex row. When the screen is extra small, the layout will be column.
With regards to fxFlexFill, it will populate the host element with the following:


\begin{tabular}{@{} l *4c @{}}
\toprule
 \multicolumn{1}{c}{Key} & Value \\
\midrule
 margin & 0         \\
 width  & 100\%     \\
 height & 100\%     \\
 min-width & 100\%  \\
 min-height & 100\% \\
\end{tabular}
\footnote{Taken from documentation here:
https://github.com/angular/flex-layout/wiki/fxFlexFill-API}

\subsection{ Adding FxLayout to Pixel Grid Page }

\subsubsection{ Add Flex Layout to App }
\begin{verbatim}
  npm i --save @angular/flex-layout
\end{verbatim}

\subsubsection{ Add Flex Layout to Pixel Grid Page Module }
\begin{lstlisting}
+import { FlexLayoutModule } from '@angular/flex-layout';

+    FlexLayoutModule,
\end{lstlisting}
