\chapter{ Create a Custom Type Definition }
Typescript by default actually includes type definitions for all of core
Javascript. Any time you are using a core javascript method, typescript
within it's core framework has a typing for it. However, any time you
plan on using something not a part of core Typescript, you will have to
create a type definition. 

\section{Example Scenario}
For instance, in another chapter, we reccomend using the native Network 
Information API. Because it is experimental(atthe time of this writing),
as is the case in all scenarios, the Typescript team has not included 
type definitions for it. So, we need to go ahead, and create our own 
custom type definitions. 

In addition, there will be times within your Angular application, wherein 
due to the requirements of your organization, you might need to create your 
own custom type definition. Whether is be new technology, or simply security
concerns. 

The following is the cookie cutter process for creating a custom type 
definition within Typescript in Angular. 

\section{Create a Custom Type Definition File}
A custom type definition file, is a file that Typescript will automatically 
pull out. By default adding the type definition \lstinline{*.d.ts} suffix to
the app, will cause Typescript to know to use this type definition.

It is also important to note that as of Typescript 2.* and greater, the
tsconfig.json has two properties available: 
\begin{enumerate}
  \item \lstinline{typeRoots}
  \item \lstinline{types}
\end{enumerate}

