\section{ Data Models }
\maketitle{}

A data model is an abstract model, that organizes different sorts of data,
and how they relate to each other. Interfaces in Typescript, very much so cater
to this cause. Coupling various functionality together is arguably Typescript's
greatest strength. Imagine 40 functions using the same data model, which has
required properties. If we add a single required property, or change/update
a single required property, the compiler will complain on the 40 functions that
now to be updated. This is incredibly valuable. There are many more benefits
which are beyond the scope of this chapter. However, introducing data-models
does introduce an interesting dillema.

\subsection{Dillema with interfaces in Typescript}
In Typescript, in any enterprise data heavy app, the majority of interfaces will
be created to be used in unison with data being retrieved from the backend.
In an Angular setting, this means that we will be using a service to make the
request. In addition, we will be feeding the data through the entire ngrx/store
pipiline, for http requests. Effect > Action > Action > Reducer. In addition,
inside of the component that will be consuming this data, we will also need
to use the interface. Not to mention, if we want to tightly couple our unit
tests to our interface. We wantto make sure that there aren't any use cases
that we do not test properly. Data types can be unique. We might be missing as a result of data type(array, dictionary, object), some particular use case.
Integrating data type with component as well, is important. Keeping the
following scenarios in mind:
\begin{enumerate}
  \item Data Services
  \item ngrx/store pipeline (Data Access)
  \item Component (Consuming business logic)
  \item Unit Tests
\end{enumerate}

What would the ideal place for us to place our interfaces?
