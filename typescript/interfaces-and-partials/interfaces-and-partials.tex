\maketitle{}
\section{ Interfaces and Partials }

As we discussed in the previous chapter, interfaces are a large part of
data-access architecture. They are the glue that makes sure that all parts of
data-access are using the same data schema. However, within the interface, we
have the option to use either an optional, real value. The question is, do we
want to turn the interface into a real representation of the data in the real
world, or do we want to turn all parts of the data as neccesary? The value in
having all parts of the interface as neccesary, is that it will force the mock
to include all values. However, on the flip side, if we test all services within
out graphql application, it will complain the fragments do not match. However,
having the interfaces directly complain as a result of the data of the mock
not matching is ideal.

\subsection{Solution to the Above Dillema}
In Typescript there is the option to use:
\begin{verbatim}
<Partial>
\end{verbatim}
Partial will treat an interface that will require every data type, as if
all values are optional.

\subsection{Where We Might Want a Partial}
So that being said, within our data-access architecture we have:
\begin{enumerate}
  \item GraphQL
  \item Models
  \item Service/Facade
  \item State
\end{enumerate}

Our actual services, facades, and state can take full interface without an
issue. They are retrieving data from the back end. In addition, in our mocks.ts
file, we want to include the interface fully, to make sure that all potential
data is included in mock. The mock for the most part is going to be used across
the spec for services and effects.

\subsection{Word to the Wise}
Ideally interfaces should be a direct one to one correlation between GraphQL
queries within app. If any of these items are used, it should be updated, and
appropriately updated in the interface. It makes considerable sense to create a
tool around this.

TODO asses situations in which a partial would need to be created, if ever.
