\maketitle{}
\section{ Tsconfig }

\subsection{ What is the Tsconfig? }
The tsconfig file corresponds to the configuration of the Typescript compiler.

\subsection{ What the Default Tsconfig Looks Like. }
Angular CLI will generate a tsconfig out the box:
\begin{lstlisting}
  {
    "compileOnSave": false,
    "compilerOptions": {
      "sourceMap": true,
      "declaration": false,
      "moduleResolution": "node",
      "emitDecoratorMetadata": true,
      "experimentalDecorators": true,
      "target": "es5",
      "typeRoots": [
        "node_modules/@types"
      ],
      "lib": [
        "es2017",
        "esnext.asynciterable",
        "dom"
      ],
      "baseUrl": ".",
      "paths": {
        "@angularPixelIllustrator/*": [
          "libs/*"
        ]
      }
    },
    "exclude": [
      "node_modules",
      "tmp"
    ]
  }
\end{lstlisting}


\subsection{ Some Notable Mentions }

\subsubsection{ sourceMap }
Source map true, will help debugging while using the console in the browser.
However, when debugging unit tests, it will cause somewhat of an issue.
Switching this to false at times can be helpful. Alternatively, one can hijack
the npm script for npm run test using. 
\begin{verbatim}
  --source-maps=false
\end{verbatim}
