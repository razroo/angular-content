
\chapter{ Component Inheritance }

JavaScript is built on the idea of prototypes. In short, a prototype is an object that can be inherited by other prototypes. This allows you to use classes in JavaScript without explicitly using the traditional class syntax.  

When it comes to inheritance, JavaScript only has one construct: objects.
Each object has an internal link to another object called its prototype. That
prototype object has a prototype of its own, and so on until an object is
reached with null as its prototype. null, by definition, has no prototype, and
acts as the final link in this prototype chain.

\section{ Extending Classes }
In Typescript land, we have the ability to use a class to extend a parent class.
A classic example of this is let's say we have the following parent component:

\lstinputlisting{./typescript/component-inheritance/base.component.ts}

If we were to try and inherit it, using the following child component:

\lstinputlisting{./typescript/component-inheritance/child.component.ts}

We would unfortunately have to pass all parent providers into the child
component to extend it.

\section{ Creating a Class to Store Injector }
What we can do, is create a class to store our store injector. For instance:
\lstinputlisting{./typescript/component-inheritance/app-injector.service.ts}

We are then able to inject this into the AppInjector:
\lstinputlisting{./typescript/component-inheritance/main.ts}

\section{ New and Improved Base Component }
Our base component now uses the injector service in order to retrieve all
dependencies.
\lstinputlisting{./typescript/component-inheritance/updated-base.component.ts}

Now in our child component, we can simply do the following:

\lstinputlisting{./typescript/component-inheritance/updated-child.component.ts}

\section{ Where Component Inheritance Really Shines }
Component Inheritance is of course, very valuable. However, having a dumb and
smart component seems to suffice in many scenarios. However, what is very
valuable is when using forms. Inheritance really shines in those scenarios. 
