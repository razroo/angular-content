\maketitle{}
\section{ Data Access Interface }
In the previous chapter we discussed having a data access folder to manage
the various parts of our data. One of the primary benefits of such an
architecture, is that it can be enforced that there is to be one interface. This
interace can be used for service, component, state, and respective files.

\subsection{ The Beauty of a Singular Data Access Interface }
By having a singular data access interface, we can ensure that the data is
being formatted in the same way across the service, component, and state. In
addition, and perhaps more valuable, it allows us to mock all of specs. By
having a singular interface used across the app, it forces the mocked
functioanlity to be in sync with our specs.

\subsection{ Singleton Interface - An Example }
Just for clarity sake, let's run through singleton interface.
\begin{lstlisting}
export interface User {
  id: string;
  name: string;
  location: string;
}

//user.service.ts file
getAllUsers(sort: object): Observable<AllBuyers> {
    const query = AllUsersQuery;
    const variables = {
      projectId: this.projectFacade.projectId,
      sort,
    };

    const allUserss$ = this.apollo
      .query<any>({ query, variables })
      .pipe(pluck('data'), pluck('allBuyers'));

    return combineLatest(allBuyers$, (allUsers: User[]) => allUsers);
  }

//user.reducer.ts
  case UserTypes.UserLoaded: {
      const user = <User>action.payload;

      return {
        ...state,
        user,
      };
    }

// user.effect.ts
@Effect()
loadUser$ = this.dataPersistence.fetch(UserTypes.LoadUser, {
  run: (action: LoadUser, state: UserStateModelState) => {
    const userId: string = action.payload;
    return this.service
      .getAllUsers(userId)
      .pipe(map((users: User[]) => new UserLoaded(user)));
  },

  onError: (action: LoadUser, error) => {
    console.error('Error', error);
  },
});

// user-page.component.ts
loadUser() {
  this.userSevice.allUsers();
}
\end{lstlisting}
