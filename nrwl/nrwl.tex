\maketitle{}
\section{Introducing Nrwl Nx}

\subsection{Arguably–The Greatest Strength of Angular}

Arguably one of the greatest strengths of the Angular(2+) ecosystem is
consistency(a point we've discussed before). As a result, Angular has the most
advanced Front End Framework CLI.

\subsection{Angular CLI Shortcoming with Regards to State Management}

However, the Angular CLI does not deal with state management. State management
in Angular2, for the most part should be done using NGRX. State management in
it's self, however, can be unwieldy. For every new reducer, a new constant file,
as well as action, effect, and proper unit testing files are needed.
In addition, the Angular Framework is not built around state management. It is
easy for it to fall through the cracks, and for an app to be done without it in
the first place. Introducing the Nrwl nx CLI.

\subsection{NX CLI}
NX is built by a team called Nrwl. They are perhaps thought leaders in the
Angular space, as they well should be, being that a large part of them came
from the core Angular team. They decided to build a cli around state management.
However, using their cli comes the following pre-conditions, include the NX
Workspace.

\subsection{Introducing the NX Workspace}
One of the things that the Nrwk team really tries to push, that isn't mainstream
yet, is the concept of a workspace. Perhaps you will remember an article
floating around a while back, about
\href{https://cacm.acm.org/magazines/2016/7/204032-why-google-stores-billions-of-lines-of-code-in-a-single-repository/fulltext}
{Google's Mono Repo}. [Worth noting, this idea
has been popularized at
\href{https://code.facebook.com/posts/218678814984400/scaling-mercurial-at-facebook/}{Facebook}
as well]. The idea is that there is a singular repository for everything.
The benefits of such are
(\href{https://nrwl.io/nx/why-a-workspace}{Taken from Nrwl site}):

\begin{enumerate}
  \item Unified versioning
    \begin{enumerate}
      \item Everything at that current commit works together
      \item A label or branch can capture the same
    \end{enumerate}
  \item Promotes code sharing and reuse
    \begin{enumerate}
      \item Easy to split code into lib modules
      \item Easy to consume implement that code and the latest changes to it
    \end{enumerate}
  \item Easier dependency management
    \begin{enumerate}
      \item One node\_modules for all code
      \item One build setup (like the AngularCLI)
    \end{enumerate}
  \item Refactoring benefits
    \begin{enumerate}
      \item Code editors and IDEs are "workspace" aware
      \item Can have a single commit for a refactor that spans applications in the domain
    \end{enumerate}
  \item Consistent developer experience
    \begin{enumerate}
      \item Ensures all necessary dependent code is available
    \end{enumerate}
\end{enumerate}

Some of the biggest disadvantages include:
\begin{enumerate}
  \item Taking time to limit access to part of workspace.
  \item One upgrade in a lib, changes all areas.
  \item Make it overkill to work on a small feature.
\end{enumerate}

\subsection{Why We Love the Iea of a Workspace}

\paragraph{Code Re-use}
In any large application, code re-use is key. Any corporation not using a
workspace, will have to create a a separate npm repo, for libraries and code
shared. A shared workspace makes it easy to share code, and have it available
for all to see. A separate app can be created for users to see code created if
need be.

\paragraph{It Emphasizes Smaller Modules}
Being that in a workspace environment, things can become unwieldy by using the
tree directory, there is more of an emphasis on keeping all of the code,
specific to that feature, service, or component, in the aforementioned folder.
It makes it so everything is in the same place, and that is a perfect place to
move on the to the next point. 
