\maketitle{}
\section{Introduction}

The current landscape of UI development is in a very interesting place, for many
reasons. For starters, the mere capacity of web to do many tasks previously
unavailable(Add examples here), is growing by the year. In addition, frameworks
which allow for scalability, DRY development, and consistency is ever going. In
your average web application, it generally includes type checking, unit testing,
integration testing, as well as state management, and observables/effects. These
are all things that 3 years ago were not common place in the enterprise, and it
is only becoming a greater landscape as time goes on.

Angular in my opinion, having worked with other frameworks such as Elm, Vue,
React, and Cycle, is in a very unique place. I will admit, there are many
reasons as to why other to use other front end frameworks(, or if you prefer to
not call them frameworks, I understand that as well). For instance, Vue, is very
simple to use. React, is always on the cusp of cutting edge. Reason in
particular as of this writing is simply incredible. Cycle, in it's approach
towards functional programming, is refreshing. However, Angular specializes in
consistency, therefore productivity, as well as safety.

It does not surprise me that the most robust command line, by a long shot, is
with Angular. Everything, from how to style a component, folder structure,
routing, observables, state management, type checking, is agreed upon by the
entire Angular community. It is a very safe bet when building out an enterprise
application. It makes architectural decisions very easy, and it does have the
ability to layer on newer technologies, if need be, albeit harder than some
frameworks such as React. It does not surprise me, that the first full gamet
book will be written on Angular. It towers over the rest in consistency, and
that deserves a head nod.

In this the full gamut series, we simultaneously want to introduce chapter by
chapter a new piece of technology. However, undoubtably, as we build our
application, we might discuss pieces from another chapter, that perhaps weren't
discussed before. This is done so that the reader can see how different parts
of the book interact, as well as learning, while doing at the same time. Which
is really the format in which most developers learn.
