\maketitle{}
\section{Introduction}

The current landscape of UI development is in a very interesting place, for many
reasons. For starters, the mere capacity of web to do many tasks previously
unavailable(Add examples here), is growing by the year. In addition, frameworks
% The grammer here needs to be fixed 'is ever going'
which allow for scalability, DRY development, and consistency is ever going. In
your average web application, it generally includes type checking, unit testing,
integration testing, as well as state management, and observables/effects. These
are all things that 3 years ago were not common place in the enterprise, and it
is only becoming a greater landscape as time goes on.

Angular in my opinion, having worked with other frameworks such as Elm, Vue,
React, and Cycle, is in a very unique place. I will admit, there are many
reasons as to why to use other front end frameworks (, or if you prefer to
not call them frameworks, I understand that as well). For instance, Vue, is very
% Are these sentences perhaps sentence fragments?
simple to use. React, is always on the cusp of cutting edge. Reason in
particular as of this writing is simply incredible. Cycle, in it's approach
towards functional programming, is refreshing. However, Angular specializes in
consistency, therefore productivity, as well as safety.

It does not surprise me that the most robust command line, by a long shot, is
with Angular. Everything, from how to style a component, folder structure,
routing, observables, state management, type checking, is agreed upon by the
entire Angular community. It is a very safe bet when building out an enterprise
application. It makes architectural decisions very easy, and it does have the
ability to layer on newer technologies, if need be, albeit harder than with some
frameworks such as React. It does not surprise me, that the first full gamet
book will be written on Angular. It towers over the rest in consistency, and
that deserves a head nod.

The issue with many technical books, is that as a reader I will walk away from
it feeling like I scratched the surface. That is, I am not confident I have
covered the full gamut of that topic. So that if I plan on building it myself,
I still have to do research on my own. In addition, when I do plan on building
an app, I do not have an example app to work against. Essentially making the
book useless. Like seriously, it has me wondering the benefit of many of the
technical books I've reading lately!

As of now, it pretty much helps for discovery, and for maintaining new material.
However, learning it does not help. This book on the other hand will help you,
really one of the first of it's kind and sort of revolutionary. By offering a
QR code, with the latest commit, linked to a Stackblitz, you will simultaneously
be able to follow along while reading the book, seamlessly. In addition, when
you have read the entire book, and need to reference bits here, and there,
you can look at a specific commit for a certain part of the book, and remind
yourself how to do it once again.

% Does the reader know about your plans to release a series?
In this the full gamut series, we will build a sample application. The
application will be a pixel illustrator. With the ability to draw a pixel on an
artboard, and plot the coordinates on the left side. In addition, it will have
% This is perhaps a bit not clear what it means to change colors on right side
the ability to change colors on the right side.

The intent of the book, is that it will be built in the most cutting edge
% Not sure if my change is the best wording
fashion. The reader should be walking away knowing, that what they have learnt
from this book is best practices, as well the Full Gamut of Angular. Being
confident, that they are aware of the full scope of the Angular ecosystem at
this time.

In addition, conventions will be casually sprinkled from time to time.
% Next two sentences need to be re-written
Conventions are on wide spectrum of impact. Some conventions beckon being
chosen, while others are chosen without the thought of doing so. From an
architectural perspective, they are equally as important, because choosing them
ahead of time, will save days of unnecessary re-factoring. This book as a result
will also consider conventions as part of it's architecture, and will put them
% What is this blue box?
in a blue box.

In addition, this book acknowledges that there are three parts to learning:
\begin{enumerate}
  \item Discovery
  % Should this be Maintence perhaps?
  \item Maintaining
  \item Learning
\end{enumerate}

Some parts of this book are discovery, and some parts are learning. This are
parts that every book has, but this one would like to also add maintenance to
the mix. As we build out our app throughout this book, we will also repeat steps
when possible, that we have already learnt.

Part of the value of this book as well, is if this is something that you would
like to implement, the time can be minimized by using the content mentioned
in this book.

In addition, for the core part of this book. I tried, to make the core
documentation as similar to that, that can me found in the documentation, to make
it as least confusing as possible for the reader.

Some might consider this book opinion. However, after working at 5 different
companies, I can say with confidence that they all could have used the same
% Sentence needs rewritting I think
architecture, and component library. That is, if you have web application, that
is data heavy, trying to allow user to interact with said data, creating,
removing, updating, and deleting, this architecture will work for you. It is
therefore my opinion that this is the singular best architecture for Angular,
and I feel 100\% comfortable calling it the Definitive Guide.

Also, the idea of a good architecture, isn't necessarily so that it goes through
everything. The idea of good architecture, is so that if something new comes up,
you will be able to drastically change your codebase. % and not what?

% What do you mean its going to work 'against the app'?
This book is going to be app agnostic. However, it is going to work against the
app in order to show how technology should be integrated in real time. In
addition, there are going to be commits made in the repo alongside the book,
so that it can give a better idea of how this architecture will work in real
time.

Each section is meant to be looked through thoroughly before one goes ahead and
one actually does actual work on them. For instance, for the section on state,
it goes through the architecture for all sections on the ngrx/store. Only once
it does so, will it go through the folder/file architecture for state.

\subsection{Enterprise examples in Core}
One other thing that this book does, is that contrary to many other books of
this nature, it will include enterprise examples in the core section. This is
because we are assuming that those who are reading this book mean business, and 
at the very least, want to be on a level where they can develop enterprise
applications.
