
\chapter{ Lazy Loading Routes }
One of the more initally overlooked pieces of UI Architecture, is with regards to 
lazy loading. Lazy loading, if not already familiar, is the concept of loading
something when it is required, rather than all at once on page load. Similar to 
how lazy people only do things when it is required of them, lazy loading will 
only load when it is required for the page it is navigating to. With regards to Angular,
lazy loading is a routing/module architecture, and heavily tied to routing. 

The main benefit of lazy loading, is so that on initial load of the web page, we
drastically decrease the bundle size. This improves user experience. Thankfully, Angular 
makes it relatively easy to include a lazy loaded module into the app. The Angular CLI 
even has a command, for easily setting up a lazy loaded route. However, before we go 
ahead and show that command, let's discuss how to add a lazy loaded route. 

\subsection{Example of What Lazy Loading Looks Like}
Let's imagine that we have an about page, that we are creating using Angular.
An example of such a lazy route, would be as follows:
\begin{lstlisting}
{
  path: 'about',
  loadChildren: () =>
    import('@razroo/razroo/about').then(
      module => module.RazrooAboutModule
    )
},
\end{lstlisting}

Test

You will notice the above syntax is using the Angular 8 syntax for lazy loaded
modules. In addition, being that we are using Nrwl Nx, which is a reccomended 
technology throughout the book, we are using the shortened path, in our import 
path. 

As mentioned in the actual docs:
\mybox{ The router will use registered NgModuleFactoryLoader to fetch an
NgModule associated with 'about'. Then it will extract the set of routes defined
in that NgModule, and will transparently add those routes to the main
configuration.}

\section{Adding a tsconfig.app.json config}
Therefore, we need to set up a config inside the main tsconfig.app.json in the
include imports:
\begin{verbatim}
  "include": [
    "**/*.ts",
    "../../../libs/users/users/index.ts"
  ],
\end{verbatim}

\section{Adding forChild inside of Actual Route}
\begin{lstlisting}
RouterModule.forChild([
  { path: '', component: UserComponent },
]),
\end{lstlisting}

The above is the cookie cutter process involved with create a component with a
lazy route. It is architecture that is worth implementing early on in the app.
Especially being some future components such as side navigations and modals
might be complicated.
