
\chapter{ Lazy Loading Routes }
One of the more initally overlooked pieces of UI Architecture, is with regards to 
lazy loading. With regards to Angular, lazy loading is a routing architecture.
Similar to how lazy people only do things when it is required of them, lazy loading 
will only load when it is required for the page it is navigating to. 

The main benefit of lazy loading, is so that on initial load of the web application, 
or Angular site, we drastically decrease the bundle size. Thankfully, Angular 
makes it relatively easy to include a lazy loaded module into the app. The Angular CLI 
even has a command, for easily setting up a lazy loaded route. However, before we go 
ahead and discuss that command, let's discuss the internal of what happens when we
add a lazy loaded route. 

\subsection{Example of What Lazy Loading Looks Like}
An example of a lazy route is as follows:
\begin{lstlisting}
{
  path: 'customers',
  loadChildren: 'lib/users/users.module#UsersModule'
}
\end{lstlisting}

As mentioned in the actual docs:
\mybox{ The router will use registered NgModuleFactoryLoader to fetch an
NgModule associated with 'team'. Then it will extract the set of routes defined
in that NgModule, and will transparently add those routes to the main
configuration.}

\section{Adding a tsconfig.app.json config}
Therefore, we need to set up a config inside the main tsconfig.app.json in the
include imports:
\begin{verbatim}
  "include": [
    "**/*.ts",
    "../../../libs/users/users/index.ts"
  ],
\end{verbatim}

\section{Adding forChild inside of Actual Route}
\begin{lstlisting}
RouterModule.forChild([
  { path: '', component: UserComponent },
]),
\end{lstlisting}

The above is the cookie cutter process involved with create a component with a
lazy route. It is architecture that is worth implementing early on in the app.
Especially being some future components such as side navigations and modals
might be complicated.
