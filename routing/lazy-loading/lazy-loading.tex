\maketitle{}
\section{ Lazy Loading Routes }
One of the more initally overlook routing architectures when it comes to routing
is lazy loading. That would simply mean that a module is only loaded once
the page goes to a specific route. This will cause large initial loading times
if all of the pages are initially bundled into the root app.

An example of a lazy route is as follows:
\begin{lstlisting}
{
  path: 'customers',
  loadChildren: 'lib/users/users.module#UsersModule'
}
\end{lstlisting}

As mentioned in the actual docs:
\mybox{ The router will use registered NgModuleFactoryLoader to fetch an
NgModule associated with 'team'. Then it will extract the set of routes defined
in that NgModule, and will transparently add those routes to the main
configuration.}

\subsection{Adding a tsconfig.app.json config}
Therefore, we need to set up a config inside the main tsconfig.app.json in the
include imports:
\begin{verbatim}
  "include": [
    "**/*.ts",
    "../../../libs/users/users/index.ts"
  ],
\end{verbatim}

\subsection{Adding forChild inside of Actual Route}
\begin{lstlisting}
RouterModule.forChild([
  { path: '', component: UserComponent },
]),
\end{lstlisting}

The above is the cookie cutter process involved with create a component with a
lazy route. It is architecture that is worth implementing early on in the app.
Especially being some future components such as side navigations and modals
might be complicated.
