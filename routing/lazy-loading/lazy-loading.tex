
\chapter{ Lazy Loading Modules }
One of the more initally overlooked pieces of UI Architecture, is with regards to 
lazy loading. Lazy loading, if not already familiar, is the concept of loading
something when it is required, rather than all at once on page load. Similar to 
how lazy people only do things when it is required of them, lazy loading will 
only load when it is required for the page it is navigating to. With regards to Angular,
lazy loading is a routing/module architecture, and heavily tied to routing. 

The main benefit of lazy loading, is so that on initial load of the web page, we
drastically decrease the bundle size. This improves user experience. Thankfully, Angular 
makes it relatively easy to include a lazy loaded module into the app. The Angular CLI 
even has a command, for easily setting up a lazy loaded route. However, before we go 
ahead and show the command, that automatically scaffolds lazy loading for us, let's discuss
how to add a lazy loaded route if we were to do that process manually. 

\section{Adding a Lazy Loaded Module Using Angular CLI}
\begin{verbatim}
  ng g lib about --routing --lazy --directory=razroo
\end{verbatim}

This command will automatically add a module to our lib. In addition, will 
modify the route within the about.module.ts, so that it can be used as a
lazy loaded route. 

\section{What We Should Edit Post Generation}
Now that we have generated a route for our "about" page, let's make the two 
edits required post CLI generation. 

\subsection{Editing app.module.routing.ts File}
Edit one, will be in our main \lstinline{app.module.routing.ts} file:
\begin{lstlisting}
{
  path: 'about',
  loadChildren: () =>  
    import('@razroo/razroo/about').then(
      module => module.RazrooAboutModule
    )
},
\end{lstlisting}
\footnote{Just in case you are familiar with a different syntax, this is the 
latest syntax for Angular 8+.}

You will notice two things in the above code: 
\begin{enumerate}
  \item A \lstinline{path} key, standard for Angular routing, to specify what 
  module should be loaded when navigating to a specific route. 
  \item A \lstinline{loadChildren} key, which calls a function followed by the 
  standard syntax for importing a module. 
\end{enumerate}

In addition, being that we are using Nrwl Nx (which this book is littered with)
the import path is using our Nx workspaces shortened path. Here that would be the 
\lstinline{razroo-workspace/razroo-lib/lib-name}.

The second edit for us to make, will be in the actual module for our about page: 
\begin{lstlisting}
@NgModule({
  imports: [
    //...
    RouterModule.forChild([
      {path: '', pathMatch: 'full', component: AboutComponent}
    ])
  ],
  declarations: [AboutComponent]
})
export class RazrooAboutModule {}
\end{lstlisting}

\mybox{You will notice that in the above code we are actually using an AboutComponent. 
This component will need to be generated in addition to module we've already created, and 
can be done simply by navigating to our libs folder, and runnning 
\lstinline{ng g component about}.
}


\section{Adding a tsconfig.app.json config}
Therefore, we need to set up a config inside the main tsconfig.app.json in the
include imports:
\begin{verbatim}
  "include": [
    "**/*.ts",
    "../../../libs/users/users/index.ts"
  ],
\end{verbatim}

The above is the cookie cutter process involved with creating a lazy loaded module 
within an Angular application. It is architecture that is worth implementing early on 
in the app. Especially being, architectural decisions, such as implementing modals, and 
navigations might be affected from lazy loading. 