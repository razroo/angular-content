\maketitle{}
\section{ What is ngrx/router-store? }

NgRx is an Angular module that leverages RxJS -  a reactive programming library
that deals with the management of data streams and propagation of change. It is
time bound, meaning that data tracking and histories are stored for referencing
and tracing.

Under the traditional model, data is decentralized and sits on each routing
state. This means that if the route changes, history of that data is lost. There
is no past and future states, only the present and navigating away discards any
memory of such occurrences. This can become quite a challenge to keep track of
everything in medium to larger sized applications, especially when navigation
is expected to occur in high frequency.

NgRx solves this issue and provides a solution by creating a central storage
space. It keeps all your current application’s data in one space – turning parts
of your browser’s memory into a  storage bucket for all your data where all
mutations occur only through explicit dispatch actions known as reducers and
becomes the application’s single source of truth. It allows for your
application’s events to exist in a unified manner rather than decentralized
across different parts, children, siblings, partials, factories and routes.

router-store, specifically, is the portion of NgRx module that allows for
listeners to be used for routing actions, meaning that data is allowed to be
stored, shared, consumed and mutated based on the routing status from
a single source. \texttt{@ngrx/router-store}, in a way, is like an in-memory database
for your application’s route related data.

\subsection{Why do we need it?}
When data is decentralized and exists on the fly, it becomes prone to errors due
to a lack of history tracking and mutations can occur from different directions.
Duplications can accidentally happen as we try to replicate certain data in
different states and parts of the application.

When relying on Angular’s routing system, we rely on data persistence through
params from navigation/router state. If a child or sibling component requires
that data, it becomes coupled with the parent and data needs to be presented
again in order to be consumed. While factory patterns may solve this issue, it
can quickly get messy if external entry is granted without explicit knowledge.

In larger teams, factory patterns may not be enough to control the flow and
history of data and human error may introduce inconsistencies in the code.

ngRx solves this, along with the reduction of time and code overheads needed to
create factory patterns and singular storage spaces. The library comes ready to
be plugged into any Angular application with its own set of Redux inspired
approach to centralized state storage. Each cycle in a router-store captures a
snapshot of the route’s state and its associated data. When data is decoupled
from routing, it allows your application to become more agile and less dependent
on data states through route params.

\subsection{How to install router store}
Once you have your Angular app, you can use npm to install router-store by using the following commands:

\begin{verbatim}
Npm install @ngrx/router-store –save
\end{verbatim}

If you’re using yarn:
\begin{verbatim}
Yarn add @ngrx/router-store
\end{verbatim}

If you’re project is created with Angular CLI version 6+, you can use the
following command:

\begin{verbatim}
ng add @ngrx/router-store
\end{verbatim}

To check your Angular CLI version, use the following command:
\begin{verbatim}
ng --version
\end{verbatim}

To use inside your application, you’ll need to import the StoreRouterConectingModule and routerReducer from @ngrx/router-store like so:

\begin{lstlisting}
import { StoreRouterConnectingModule, routerReducer } from '@ngrx/router-store';
\end{lstlisting}

\subsection{Router actions and why they might be useful}
An action is anything that you can do to your application. A router action is an
event that can occur against a specific route on your Angular app. When it comes
to routing, are 5 specific actions that can occur and they are the request, the
action of navigation, the aftermath (also known as navigated), cancellation and
navigation error.

Being able to access and track these actions allows you to control the flow of
route state storage management, access to data and the life cycle process. When
used in conjunction with route guards – a feature that allows you to protect
your views from rendering when there isn’t enough information or the right
access permissions – router actions can help with the resolutions of states and
its consumption.

A request action always kickstart the process which then runs the navigation
action that determines if the dispatch should occur or not. If guards are valid,
a successful navigation will occur and result in a router navigated action. If
something went wrong due to exceptions or lack of user permissions, the router
action will return a \lstinline{ROUTER\_CANCEL} action and nullify any attempts to
access the route.

A \lstinline{ROUTER\_ERROR} action may occur during the navigation life cycle and
returns the stored state before navigation occurred. This is particularly useful
as it allows the application to back track its action and restore its former
data – a sort of back button without the need for extra configuration or call
to the router state bucket.

\subsection{How to use a custom serializer}

A custom serializer prevents the mutation of snapshot data during the dispatch
process. As data during the navigation cycle is prone to mutability, a custom
serializer returns only what you need to be added to the payload and store. So
in essence, it tracks the difference and change of a particular state without
modifying the entire stored state snapshot.

A custom serializer can be implemented through the abstract class
RouterStateSerializer. It is, in a way, a middleman class that processes the
difference between what the current state is, what is to be changed and updates
only what is necessary.

To create a custom serializer, you will need to import Params and
RouterStateSnapshot from @angular/router, along with RouterStateSerializer from
@ngrx/router-store
\begin{verbatim}
import { Params, RouterStateSnapshot } from '@angular/router';
import { RouterStateSerializer } from '@ngrx/router-store';
\end{verbatim}

To create a custom serializer, export a class that implements
RouterStateSerializer with an interface to ensure object uniformity.

Using the serialize() method to convert the state object into a unified format
that conforms to your application’s requirements. This often comes in the form
of mapping router state values to a predefined interface that may look something
like this:

\begin{lstlisting}
//to be used by the serialization process
export interface RouterStateUrl {
  url: string;
  params: Params;
  queryParams: Params;
}
\end{lstlisting}

The CustomSerializer class that implements the imported RouterStateSerializer
using the RouterStateUrl interface created.

\begin{lstlisting}
export class CustomSerializer implements RouterStateSerializer<RouterStateUrl> {
   //serialization code here
}
\end{lstlisting}

Set up the serialization method to return a uniformed set of parameters based
on the template set in the RouterStateUrl interface.

\begin{lstlisting}
serialize(routerState: RouterStateSnapshot): RouterStateUrl {
    let route = routerState.root;

    while (route.firstChild) {
      route = route.firstChild;
    }

    const {
      url,
      root: { queryParams },
    } = routerState;
    const { params } = route;

    // returning the object based on the RouterStateUrl interface
    return { url, params, queryParams };
  }

\end{lstlisting}

To use the custom serializer, implement it inside your @NgModule and call your
exported CustomSerializer class inside the StoreRouterConnectingModule.

\begin{lstlisting}
@NgModule({
  imports: [
    StoreModule.forRoot(reducers),
    RouterModule.forRoot([
      // routes
    ]),
    StoreRouterConnectingModule.forRoot({
      serializer: CustomSerializer,
    }),
  ],
})
\end{lstlisting}

\subsection{Benefits of using router-store with ngrx/store-freeze}
\texttt{ngrx/store-freeze} is a dev tool that can be used during the development
phase of an Angular application to prevent state mutation when using
\texttt{router-store}. It sits on top of \texttt{router-store} as meta data that
acts as an insurance against changes in state data during the process of
transfer to state storage.

As \texttt{router-store} provides snapshots of the \texttt{RouterState} during
the navigation life cycle, it is vital that snapshots passed do not change
during the process of dispatch as this will result the store cycle’s truthiness
breaking due to inaccurate snapshot data.

While serialization already prevents this, \texttt{store-freeze} acts as an
additional safeguard with exceptions thrown when mutations do occur at runtime.
It automatically \emph{`deep freezes'} the entire store state object and
dispatch actions, resulting in a read only effect before it gets passed to the
serializer. This allows errors to be caught before it gets dispatched,
serialized and passed into storage.
