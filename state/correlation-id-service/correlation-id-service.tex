\section{ Correlation ID Service }
\maketitle{}

We have addressed in a previous chapter whether, or not when to use state. One
of the more peculiar situations within an Angular application is file upload.
Generally, it is for the following three reasons:
\begin{enumerate}
  \item There is a before, and after state. What does the file look like before
  the upload, and what does it look like before the download.
  \item Depending on scenario, we might have multiple components on the page,
  and therefore need to make sure, that the state of one, does not affect the
  state of the other.
  \item State is contained within a single component, albeit there might be a
  number of different components on the same page. It would seem state is
  superfluous in this scenario.
  \item Following dumb/smart component architecture, in order to keep
  application dry, we are going to need to inroduce our state from some outside
  source, using event emitters.
\end{enumerate}

The scenario that commonly applies to file upload can happen in many other
situations as well. While @ngrx/store is something which is a good idea for the
majority of any enterprise application, it is not neccesarily the right choice
in this scenario.

\subsection{Identifying Bloat of @ngrx/store}
@ngrx/store bloat is real. To put into perspective, it requires an action, a
reducer, a facade, in better architectured solutions, ngrx/entity, and
selectors. In addition, state isn't naturally re-usable, as it is an object.
Lastly, and most definitely, not least, special nomenclature must be put in
place to make it, so state can be re-used in the future.

\subsection{Architectural Danger of Using a Service}
Naturally using a service in this scenario makes the most sense. Services can
stay true to a particular guideline, and be manageble. One is to keep in mind 
