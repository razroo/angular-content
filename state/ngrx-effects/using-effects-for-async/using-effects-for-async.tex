
\chapter{ Using Effects for Async }

Effects in an ngrx setting are a secret weapon when is comes to creating async
events. For instance, let's imagine that after a piece of data is loaded, we
would like to display a toolbar, and then navigate to a specific page. Using an
effect, we can make this process quite simple. We can download the data using
an effect, and then once complete, we can cause another effect to laod, that
would cause the page to navigate to another. This is a very common use case that
could be used within our situation.

\section{ Understanding Architecture in totality }

As mentioned before, effects are a way to handle asynchronous data in ngrx. As
a result, it can be used to show really anything within the app. For instance,
let's say you would like to show a toolbar once a certain something is
downloaded within app. When we download that data within the app, we would like
to pass it into our store, so that our data can be used across the app.
Intuitively, one might decide that one can do it all within the component.
For instance, make the data request, and then
