\maketitle{}
\section{ History of State Management }

I wanted to write this, because having a history of state management put's into
perspective why we need state management. It also put's into perspective how
much so things change, and how important having a foundation in software is.
In particular, to be aware of alternatives, and to help with learning new
concepts. Jumping right in, Jquery was created a very long time ago, already
back in \href{https://en.wikipedia.org/wiki/JQuery}{2006}. Show, hide, remove,
add, as well as element selectors, were already present in
\href{http://api.jquery.com/category/version/1.0/}{Version 1}. Javascript had
this capability as well if need be. However, no one really thought of it as
state management.

\subsection{ State Management with jQuery }
A classic component, I remember that was always created with Jquery, would be
image sliders. In the more elequent apps, they would use singleton classes,
perhaps \href{https://www.w3schools.com/js/js\_object\_prototypes.asp}{prototypes}
if they knew what they were really doing. Variables would be cached by
initializing once. Functions would be kept small, and everything including css,
would have very unique nomenclature(\href{http://getbem.com/introduction/}{BEMCSS}
for instance). Folder/file structure was important, but there wasn't really
anything like state management. Ironically, many smaller websites at this time
were more performant in many ways. Why? Because, many intentionally kept them
small, in order to do more. 2015-2016 was a great year of performance, due to a
growth spurt in \href{https://chromereleases.googleblog.com/2015/03/stable-channel-update.html}{browser capabilities}.
The change log for 2015, is the last time you will see google chrome mentioning
performance in it's logs.

\subsubsection{Jquery and Javascript example}
The following is a great example of how Jquery and Javascript 
"state management" would work(updated to use es6). A file which would
contain values, would be used to create/add/delete/update across the app:

\begin{lstlisting}
// _elem.js file
storeValues: [],
storeColors: [],
sassColorVariables: [],
lessColorVariables: []

// _grid.js file
updateGridColor: () => {
  for(let x = 0; x < elem.s.columnCount; x++) {
    for(let y = 0; y < elem.s.rowCount; y++) {
      ctx.strokeStyle = `${elem.el.backgroundRed.value + 44}. ${elem.el.backgroundGreen.value + 44}. ${elem.el.backgroundBlue.value + 44}`;
      ctx.strokeRect(x * elem.s.pixSize, y * elem.s.pixSize, elem.s.pixSize, elem.s.pixSize);
      ctx.fillStyle = elem.el.backgroundHexColor.value;
      ctx.fillRect(x * elem.s.pixSize + 1, y * elem.s.pixSize + 1, elem.s.pixSize - 2, elem.s.pixSize - 2);
    }
  }

  for(let x = 0; x < elem.s.storeValues.length; x++){
    ctx.fillStyle = elem.s.storeValues[x][2];
    ctx.fillRect(parseFloat(elem.s.storeValues[x][0]) + 1, parseFloat(elem.s.storeValues[x][1]) + 1, elem.s.pixSize - 2, elem.s.pixSize - 2);
  }
}

\end{lstlisting}
Above code taken from the \href{codeIllustrator}{https://github.com/CharlieGreenman/codeIllustrator} repo.

\subsection{ State Management with Backbone }
Backbone applications to me were so funny, and still are. It literally looked
like a well architected Jquery app minus routing. Which now that I think about
it, isn't funny. Backbone was a big step up. Routing was a very nice touch that
offered something like that out of the box. Ultimately, there really was no
concept of state management with backbone either. However, I remember apps being
performant, and unmanageable in many cases due to the bad architecture. A step
up, of course from badly engineered Jquery applications. So, no state management
at this point yet, still! However, using model, there was somewhat a way to do
this, that was baked into best practices:

\begin{lstlisting}
  // note_model.js
"use strict";
APP.NoteModel = Backbone.Model.extend({
  // you can set any defaults you would like here
  defaults: {
    title: "",
    description: "",
    author: "",
    // just setting random number for id would set as primary key from server
    id: _.random(0, 10000)
  },
  //...
  // note_edit.js
  save: function (event) {
      event.stopPropagation();
      event.preventDefault();

      // update our model with values from the form
      this.model.set({
        title: this.$el.find('input[name=title]').val(),
        author: this.$el.find('input[name=author]').val(),
        description: this.$el.find('textarea[name=description]').val()
      });
  //...

\end{lstlisting}

This model would global, or per each component, and could be updated using the
above syntax.

\subsection{ State Management with AngularJS }
AngularJS was fantastic because it offered two way binding out of the box. Alot
of web applications need that. It also came hand in hand with Jasmine unit
testing, and event handling. Completely irrelevant to state management. However,
because it introduced services, it really was the first framework to start boxing
applications into, "this is what front end architecture should look like",
paving the way for state management.

Services, while mainly used for data, were also used different parts of the
applciation to interact with each other. Being that Angular applications were
Single Page Applications by default, this worked. State was synonymous with
services. If you wanted different components to know about the data of service,
you would have a setter and getter for that service. The issue with this
approach, is that there would be 4, or 5 services that would interact with each
other, and it would cause serious issues. In addition, in many applications, old
code/bad practices would use \$scope in the code base, causing some serious
perfomance issues. The following is what a sample servic using a factory would
look like:

\begin{lstlisting}
  var myApp = angular.module('myApp',[]);
  myApp.factory('myService', function() {
      var test = 5;
      var obj = {
          test : 5
      }

      return{
        setTestVal: function(val){
          test = val;
          obj.test = val;
        },
        getTestVal: function(){
          return test;
        },
        data : obj
      }


  });

  function MyCtrl($scope, myService) {
      $scope.test = myService.getTestVal();
      $scope.data = myService.data;
  }

  function SetCtrl($scope, myService){
      $scope.newTestVal = '';
      $scope.setTestVal = function(val){
        myService.setTestVal(val)
      }
  }
\end{lstlisting}


\subsection{ State Management with React }
React came around, and interested me atleast for two reasons. It offered
flexibility being a library and not a framework. Second, it was fast in
comparison to AngularJS. Flux came out, and was my first introduction to a
state management system. Redux came out 6 months after Flux already, so
admittedly, I only had a chance to work with Flux for a month, before we already
started moving to Redux. Flux was a bit difficult, and during that month time,
I remember my code reviews being rampant, with don't do this, do that etc.
Shortly after Flux, Redux came around, and for the first time it felt like a
mature state management system came around.

\begin{lstlisting}
// pixel-color-picker.js component
handlePixelColorChange(e){
    const {dispatch} = this.props;
    this.setState({pixelHex: e.target.value}, function(){
        dispatch(PixelColor(this.state.pixelHex));
        dispatch(PixelColorRGB(hexToRgb(this.state.pixelHex).r,
         hexToRgb(this.state.pixelHex).g, hexToRgb(this.state.pixelHex).b));
    });
// control-panel.js actions
export function PixelColor(color){
  return{
    type: types.PIXEL\_COLOR,
    pixelHex: color
  }
}

// colorPicker.js reducer
case types.PIXEL\_COLOR:
  return Object.assign({}, state, {
    pixelHex: action.pixelHex || state.pixelHex
  });
\end{lstlisting}
\footnote{code take from \href{https://github.com/CharlieGreenman/pixelLight}{here}}

\subsection{ Reactive State Management with React and Angular }
Around this time @ngrx/store came out, reactive programming became more popular.
Within the context of state, this meant redux-observable for React, and
@ngrx/store for Angular. For Angular, this meant that state is now cookie
cutter. For React and Angular, it meant that users have the ability to tie state into the
rest of their application.

\begin{lstlisting}
Observable.merge
  // Create observable map for  when background hex changes, and use that
  // value to update store for backgroundColor
  this.changePixelColor$.map((value: any) => (
    PixelColor(value)
  )),
  this.changePixelColorRGB$.map((value: any) => (
    PixelColorRGB(value.pixelRed, value.pixelGreen,
      value.pixelBlue)
  ))
)
.subscribe((action)=>{
  store.dispatch(action)
})
\end{lstlisting}

// code take from \href{https://github.com/CharlieGreenman/angularPixel_illustrator}{here}

\subsection{ Hooks and Context with React + Vue }
Where we are at currently, is that new waves are being made with regards to
state management. State is being baked into frameworks in ways that make it
more lightweight, and easier to deal with. Vue and React now have a feature
called hooks, and context. These allow an app to have state out of the box.
Redux + Redux Observable still have they're place. There are times where state
is neccesary to allow components on different pages interact with each other.
Other times, it can be a way of managing the data, to make sure the app is
maintanable. If you see your app heading in the direction of the latter. Redux +
Redux Observable is still reccomended.

\begin{lstlisting}
// theme-context.js

// Make sure the shape of the default value passed to
// createContext matches the shape that the consumers expect!
export const ThemeContext = React.createContext({
  theme: themes.dark,
  toggleTheme: () => {},
});

// theme-toggler-button.js

import {ThemeContext} from './theme-context';

function ThemeTogglerButton() {
  // The Theme Toggler Button receives not only the theme
  // but also a toggleTheme function from the context
  return (
    <ThemeContext.Consumer>
      {({theme, toggleTheme}) => (
        <button
          onClick={toggleTheme}
          style={{backgroundColor: theme.background}}>
          Toggle Theme
        </button>
      )}
    </ThemeContext.Consumer>
  );
}

export default ThemeTogglerButton;
\end{lstlisting}
code take from \href{https://reactjs.org/docs/context.html}{here}

\subsection{ Final Words on State Management }
One point I would like to end off on. I remember 5 years when all of the
framworks were coming out, there was a developer who told me that if you know
what you are doing, really none of the frameworks are neccesary. That being said,
no one in their right mind, is going to create their own framework when they have
it readily availalble. That is unless the company has the agenda to make one.
However, what is important, is to understand the internals, so that you can
that much more valuable when it comes to performance, and mainatanability of
your project. I think that is obvious, but just wanted to bring it up.
