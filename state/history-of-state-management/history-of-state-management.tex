\maketitle{}
\section{ History of State Management }

I wanted to write this, because having a history of state management put's into
perspective, the use of state management. In addition, as someone who might be
considering being an architect of front end, it can be very advantageous to
know the history to be aware of alternatives. First and foremost, Jquery was
created a very long time ago, already back in \href{https://en.wikipedia.org/wiki/JQuery}{2006}.
Show, hide, remove, add, as well as element selectors, were already present in
\href{http://api.jquery.com/category/version/1.0/}{V1}. Javascript had this capability
as well if need be. However, no one really thought of it as state management.

\subsection{ State Management with Jquery }
A classic component, I remember that was always created with Jquery, would be
photo sliders. In the more elequent apps, they would use singleton classes,
perhaps \href{https://www.w3schools.com/js/js\_object\_prototypes.asp}{prototypes}
if they knew what they were really doing. Variables would be cached by
initializing once. Functions would be kept small, and everything including css,
would have very unique nomenclature(\href{http://getbem.com/introduction/}{BEMCSS}
for instance). Folder/file structure was important, but there wasn't really
anything like state management. Ironically one thing I do remember about this
time, is that websites were more performant in many ways. Why? Because, we
intentionally kept them small, in to do more.

The following is a great example of how Jquery and Javascript state management
would work(updated to use es6):

\begin{lstlisting}
toggleView: (e) => {
  let x = e.keyCode;
  if(x === 71) {
    if(c.classList.contains("allow-handle-click")) {
      cntrlView.stopHandleClick();
      cntrlView.removeTiles();
      cntrlView.addBackTiles();
    }
    else {
      cntrlView.allowHandleClick();
      cntrlView.removeTiles();
      grid.createGridIllustrator();
      cntrlView.redoGrid();
    }
  }
}
};
\end{lstlisting}
\footnote{The above code is taken from the repo \href{https://github.com/CharlieGreenman/codeIllustrator/blob/master/app/js/\_control\-view.js}{here}}
\subsection{ State Management with Backbone }
Backbone applications to me were so funny, and still are. It literally looked
like a well architected Jquery app. Routing was probably the nicest touch, that
it offered something like that out of the box. Ultimately, there really was no
concept of state management with backbone either. However, I remember apps being
performant, and unmanageable in many cases due to the bad architecture. A step
up, of course from badly engineered Jquery applications. So, no state management
at this point yet, still! However, using model, there was somewhat a way to do
this, that was baked into best pracices:

\begin{lstlisting}
  // note_model.js
"use strict";
APP.NoteModel = Backbone.Model.extend({
  // you can set any defaults you would like here
  defaults: {
    title: "",
    description: "",
    author: "",
    // just setting random number for id would set as primary key from server
    id: _.random(0, 10000)
  },
  //...
  // note_edit.js
  save: function (event) {
      event.stopPropagation();
      event.preventDefault();

      // update our model with values from the form
      this.model.set({
        title: this.$el.find('input[name=title]').val(),
        author: this.$el.find('input[name=author]').val(),
        description: this.$el.find('textarea[name=description]').val()
      });
  //...

\end{lstlisting}

\subsection{ State Management with AngularJS }
AngularJS was fantastic because it offered two way binding out of the box. Alot
of web applications did that. It also came hand in hand with Jasmine unit testing,
and event handling. Completely irrelevant to state management. However, becuase
it introduced services, it really was the first framework to start boxing
applications into, "this is what front end architecture should look like".
Services were used for applciation to interact with each other. State was sort
of a thing from the perspective of services. If you wanted different components
to know about the data of service, you would have a setter and getter for that
service. The issue with this approach, is that there would be 4, or 5 services
that would interact with each other, and it would cause serious issues. In
addition, in many applications, old code that used \$scope would be in the code
base, causing some serious perfomance issues.
\begin{lstlisting}
  var myApp = angular.module('myApp',[]);
  myApp.factory('myService', function() {
      var test = 5;
      var obj = {
          test : 5
      }

      return{
        setTestVal: function(val){
          test = val;
          obj.test = val;
        },
        getTestVal: function(){
          return test;
        },
        data : obj
      }


  });

  function MyCtrl($scope, myService) {
      $scope.test = myService.getTestVal();
      $scope.data = myService.data;
  }

  function SetCtrl($scope, myService){
      $scope.newTestVal = '';
      $scope.setTestVal = function(val){
        myService.setTestVal(val)
      }
  }
\end{lstlisting}


\subsection{ State Management with React }
React came around, and interested me atleast for two reasons. It offered
flexibility being a library and not a framework. Second, it was fast in
comparison to AngularJS. Flux came out, and was my first introduction to a
state management system. Redux came out 6 months after Flux already, so
admittedly, I only had a chance to work with Flux for a month, before we already
started moving to Redux. Flux was a bit difficult, and during that month time,
I remember my code reviews being rampant, with don't do this, do that etc.
Shortly after Flux, Redux came around, and for the first time it felt like a
mature state management system came around.

\subsection{ Reactive State Management with React and Angular }
Around this time @ngrx/store came out, reactive programming became more popular.
Within the context of state, this meant redux-observable for React, and
@ngrx/store for Angular. It made state more cookie cutter, and allowed for

\subsection{ Hooks and Context with React + Vue }
Where we are at currently, is that new waves are being made with regards to
state management. State is being baked into frameworks in ways that make it
more lightweight, and easier to deal with. Vue and React now have a feature
called hooks, and context. It makes it so that using something such as redux
with redux-observable isn't neccesarily so important.

\subsection{ Final Words on State Management }
One point I would like to end off on. I remember 5 years when all of the
framworks were coming out, there was a developer who told me that
