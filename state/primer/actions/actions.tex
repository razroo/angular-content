\maketitle{}
\section{ Primer - Actions }
At this is a definitive guide, and covers all aspects of the framework, one
would be remiss without going through the entire lifecycle of the store, and
mentioning it here.

An action is a function which contains two very important pieces of information:
\begin{enumerate}
  \item What the name(type) of the action is. \footnote{Why this is important we
  will get to soon}
  \item The payload of the action. Which is a single object, containing all the
  data passed into the store.
\end{enumerate}

\subsection{An example of an Action in an Angular setting}
An Action is pretty generic across different frameworks. \footnote{For certain
frameworks, this is less relevant, such as React + Vue.}. However, within an
Angular setting, it means that we are using @ngrx/store for controlling state
across our app and Typescript. Within an Angular setting, an action will look
like the following:
\begin{lstlisting}
export class LoadCodeBox extends Action {
  readonly type = CodeBoxTypes.LoadCode;

  constructor(public payload: {id: string}) { }
}

export class CodeBoxLoaded extends Action {
  readonly type = CodeBoxTypes.LoadCode;

  constructor(public payload: CodeBox) { }
}
\end{lstlisting}

In the above code, is a perfect representation of your classic action. It
contains the type of action. So that in you reducer, or effect(which we will
go into momentarily if not familiar), we can trigger reduce(combine) data passed
in, with data already present.

\mybox{In an Angular/Typescript setting, you will commonly find a general export
type, for the entire app:
  export type Actions = LoadCodeBox | CodeBoxLoaded;

This is usually used within the reducer, and is called a union interface type.
It allows us to combine the type for numerous actions within a single type. We
can then pass this as a type to a reducer, and make sure that we do not
introduce any action beyond that for the particular feature state. It's a great
way to make the develop think twice, if they introduce an action outside of the
feature state into the app.
}

\subsection{An example of using an Action in an Angular setting}
As a matter of philosophy, this book will only introduce how to use something
as it is in the real world. Generally, an action will only be used in a variety
of situations such as an effect, facade, or guard. \footnote{If not familiar, no
need to worry, we will go through this as time goes on.}. As a simple example,
let's use it in a facade \footnote{Something we will discuss, a dedicate an
entire chapter}.

\begin{lstlisting}
  loadCodeBox(id) {
    new LoadCodeBox({id})
  }
\end{lstlisting}

Here we are calling the class by using new. It will call the action whenever the
facade is called.

This would be as simple as a primer can get for actions. It might seem like
there are more questions that came up as a result (Go through potential
questions one might have).
