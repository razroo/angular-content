
\section{ The Case for Using Ngrx/Entity by Default}

\subsection{ A Synopsis of Normalization}
The idea of normalization with regards to relational databases(a database
structured to recognise relations among stored items of information) is to
eliminate data redundancy, and anomalies with regards to Insertion, Delete, and
updating. Within a front end setting, it is mostly the backend that deals with
a large portion of eliminating data redudancy, and CRUD anomalies. However,
there are many use cases still with regards to Insertion, Delete, and Updating
anomolies, that can be solved by using normalization within ngrx/platform.

\subsection{Why Ngrx/entity is Exciting}
@ngrx/entity for me personally, is the most exciting part of ngrx/platform for
two reasons:
\begin{enumerate}
  \item It is a library based on computer science fundamentals of building
  relational databases.
  \item It actually introduces something novel to the Redux ecosystem (something
  I found that the Context and Hooks API didn't even give that sort of
  freshness).
\end{enumerate}

\subsection{Dis-secting a Real World Example of ngrx/entity}

\subsubsection{CodeBox Interface}
Let's imagine that we are building a Pixel Illustrator application. Something
similar to the \href{http://charliegreenman.github.io/codeILL/}{following}. One
of our interfaces for the state contained within the codeBox, would look
something like this:
\begin{lstlisting}
  interface CodeBox {
    id: string;
    color: string;
    xPosition: number;
    yPosition: number;
  }
\end{lstlisting}

Let's imagine we have an array of CodeBox interface(AKA CodeBox[]) that we would
like to use. If we want to select a specific slice of data within our database
table, things can be a little bit tricky. We would need to loop through all of
the data, and then find the appropriate set of data with the id/value that we
want.

Instead, what would make our data table more efficient, is if we have an array
of ids that corresponds directly to our dictionary of data. In our store we can
add some extra key values such as selectedId. This way, we can simply say the
data that we would like is based on one, or two parameters passed to a function.
This can be re-used in other stores, as data is structured the same way, and
selected in the same fashion.

\subsection{ngrx/entity follows this pattern}
Ngrx/entity is a library which follows this pattern, so that data used within
ngrx/store is normalized. This means, within the context of using JSON data,
that we have a dictionary, combined with an array of ids. The index of the ids
directly correlates with the key/value in the dictionary. This is very powerful,
and I will prove it by bringing up a couple of situations.

\subsubsection{The Power of Ngrx/Entity Deep Dive}
Scenario One:
Updating one dictionary data set. Let's imagine that we were to load our
Pixel Illustrator a set of coordinates to populate the pixel illustrator. However,
let's say we wanted to update one particular piece of that data set, by only
a smaller portion of data, when loading entire data set, and then a larger portion
when expanding details for particular data set. Ngrx/entity allows us to maintain
the same store, but simply update the once piece of data in the dictionary.

Scenario Two: 
Let's say we have two independent components pulling data from the same store,
but we do not want them to interact with each other. We would need to create a
unique id for both that would allow us to follow this pattern. ngrx/entity
naturally allows us to pass in a unique id, that allows us to accomplish this
correlation id pattern. This is such a powerful pattern.

Scenario Three:
Let's say we have a data table, and we would like to know which set's of data
we have selected. We have the ability to pass in id, into another section of
data store, called selectedIds. So we can have a checkbox, select multiple
checkboxes, and know which data sets have been selected.

Scenario Four:
Let's say we want to delete, update, or upsert a certain subset of data. Being
able to pass in the id only, makes this process infinitely easier, and the api
more manipilating data, more re-usable.

These are just four real worlds scenarios, and the list goes on.

\subsection{Choosing Ngrx/entity as the Default}
This leads to the point this chapter is trying to make. In any situation with
data, odds are that you are going to update, delete, upsert, or pull in more
data for a specific subset of data. This means, that anytime that you are going
to use more data, using ngrx/entity as the default makes sense. The Nrwl CLI
for ngrx/store follows this pattern, and it is highly reccomended.
