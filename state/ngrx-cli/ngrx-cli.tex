\maketitle{}
\section{ Ngrx CLI }

Of one of the better reccomendations I can give with regards to using Angular
within you app, is using the Nrwl Nx cli. If you are using Nrwl Nx already,
which you should, and have a mono repo setup within your organization, then
it should be readily available. If you haven't already, please refer to the
chapter on Nrwl Nx setting it up.

\subsection{ Why Use a CLI? }
One of the benefits of using a CLI, is that there it subtly enforces the entire
team to use a particular convention. With the Nx CLI for ngrx, this is doubly
true, as it strongly enforces conventions to be used as to how ngrx works. In
addition, within the ngrx arena, it strongly enforces how certain files should
be built. 

\subsection{ Why use a CLI for ngrx }
Without a doubt the most fustrating thing about Angular before the CLI came
around, is the amount of boilerplate that would be required in order to work
with Angular. The Nrwl Nx for the most part solves this. In addition, there is
much more than can be done on top of this in the future. In our pixel app, in
particular, when generating state for the pixel-form, it will generate the
following files:
\begin{enumerate}
  \item libs/<libname>/src/+state/products.actions.ts
  \item libs/<libname>/src/+state/products.effects.ts
  \item libs/<libname>/src/+state/products.effects.spec.ts
  \item libs/<libname>/src/+state/products.reducer.ts
  \item libs/<libname>/src/+state/products.reducer.spec.ts
\end{enumerate}

There is also the option to add a facade, which is highly reccomended. In
addition, creating a seperate selector file is extremely valuable.
