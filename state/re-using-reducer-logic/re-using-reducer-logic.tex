\maketitle{}
\section{ Re-using Reducer Logic }

In any state management setting, many times an individual will come across
business logic that is repeated many times throughout one's feature state. For
instance, let's say that we hava fileUpload service that is going to be used in
our app. The purpose of this service, is to upload a file, and return a
FileMetadataId to be used in another api.

The logic for this state manangement workflow will be the same for actions,
facade, reducer, and effects. The question is how can we make our state
re-usable so that it we can use it in multiple reducers without having to
re-write logic?

\subsection{ Strategy }

The strategy for how we would create a re-usable state is as follows. We would
have a facade/action that would pass in an actionTypeName along with the
regular payload. The effect would be passed this unique actionTypeNamem which
would then call the appropriate facade/action for when data is uploaded, using
re-usable service. We would then pass in uploaded data from the action.payload
into the reducer. However, to make sure that reducer logic can be re-used, we
pass a special param for the reducer function called actionTypeName. We then
compare the actionNameIdentifier passed in from the action.paylod to the
actionTypeName passed in from the reducer. If they are equal to each other, then
we allow for the reducer logic to happen. Let's deep dive on this approach.

\subsection{Creating a re-usable Facade and Action}
Our UploadFile action will pass the following:
\begin{itemize}
  \item actionNameIdentifier - For allowing us to re-use reducer
  \footnote{Will Discuss this in more detail}
  \item entityId - Primarily for allowing us to use the ngrx/entity api
  \footnote{Will discuss this in more detail}
  \item file - Using the name html input api, File is what will be passed in
  through from the input.
\end{itemize}
\begin{lstlisting}
export class UploadFile implements Action {
  readonly type = UploadActionTypes.UploadAttachment;
  constructor(public payload: { actionNameIdentifier: string, entityId: string;
     file: File; name: string}) {}
}
\end{lstlisting}
\footnote{In a real world application, normally I would include a param for name
of the file. However, for the simplicity of this chapter, I decided to leave
it out.}

In our Facade, would have the following:

\begin{lstlisting}
uploadFile(actionType: string, file: File, entityId: string): void {
    this.store.dispatch(
      new UploadFile({ actionNameIdentifier, entityId, file })
    );
  }
\end{lstlisting}

Most notable, we pass in a parameter called actionNameIdentifier. This will be
compared against created by the reducer to make sure we are targeting the
featureState that we would like to work with.

\subsection{Higher Order Reducers}

A higher order reducer borrows it's name from a higher order function.

\mybox{
A higher order function is:
Functions that operate on other functions, either by taking them as arguments or
by returning them, are called higher-order functions.
}

A higher order reducer is very similar. It allows us to create a reducer, that
can be passed into another reducer, to create single combined reducer.

\begin{lstlisting}
export function attachmentReducer(
  state: UploadState = initialState,
  action: UploadAction,
  actionTypeName: string,
): UploadState {
  const { actionNameIdentifier } = action;
  if (actionNameIdentifier !== actionTypeName) return state

  switch (action.type) {
    case DraftActionTypes.UploadAttachments: {
      return attachmentsAdapter.addAll(action.payload, {
        ...state,
        loaded: true,
      });
    }
  }

  return state;
}
\end{lstlisting}

You will notice that in the above code, we have created a param for
actionTypeName in our reducer. In our action, we are passing in a similar unique
identifier called actionNameIdentifier. In our reducer we compare the
actionTypeName created in the reducer, against the actionTypeName in the action,
in order to re-use reducer logic, for any feature state we would like to use it.

\subsection{Combining Higher Order Reducer with Feature State Reducer }

Higher order reducers are fantastic. However, many times, the higher order
reducer only solves one specific problem of state with regards to a particular
component. So, without us being able to combine multiple higher order reducers
within the app, a higher order reducer would be worthless. That being said, we
would be able combine reducers by doing the following:

\begin{lstlisting}
import { reducers } from './reducers';
@NgModule({
  imports: [
    ...
    StoreModule.forFeature('userModule', {
      userReducer,
      attachments: attachmentReducer('userReducer'),

    })
  ],
  ...
})
export default class UserModule { }
\end{lstlisting}

Now, any action that is made that matches the name for feature state we are
working on, will update the state for the respective reducer.

\subsection{Re-Using Effects}
Effects are really a bridge from an action, to service, back to an action again.
We are therefore able to re-use effect by simply passing in the
actionNameIdentifier and making sure that it makes it's way back to reducer
once again.
\begin{lstlisting}
@Effect()
  uploadAttachment$ = this.dataPersistence.fetch(
    DraftActionTypes.UploadAttachment,
    {
      run: (action: UploadAttachment, state: UploadState) => {
        const { actionNameIdentifier } = action;
        return this.filesService
          .uploadFile(action.payload, FileLocation.ATTACHMENTS)
          .pipe(
            map((file: FileMetadata) => {
              return new AttachmentUploaded({
                id: action.payload.id,
                file,
                actionNameIdentifier,
              });
            })
          );
      },

      onError: (action: UploadDraftAttachment, error) => {
        console.error('Error', error);
      },
    }
  );
\end{lstlisting}

\subsection{ Ending Notes }
There is some discussion along the lines of whether, or not this pattern makes
sense. Arguably some scenarios where one might be tempted to use a pattern like
this, might not be ideal. Therefore, it is worth noting to re-assess the reason
for which you would like to re-use state. Perfectly valid reasons for which to
use a re-usable reducer includes pagination, sorting, and potentially filtering. 
