\maketitle{}
\section{ Aggregation Pattern }

Many times within any web setting, numerous apis will be feeding into a singular
request. An example would be in an e-commerce setting wherein data for a
particular item, might come from numerous locations. The data for the pants
might come from one api, the analytics api might update, and then a third api
along the lines of data persistence will be called at that time as well.

In a backend setting, usually we try and keep a singular api request for the
business logic of a particular use case. However, there are many times wherein
this will be beyond the power of the developer. In can be in antiquated apis,
or in unique use cases.

\subsection{ The Unique Challenge with Ngrx/effects}
In an ngrx/effects use case, when a user is hitting a single service, and
returning a single action, it is relatively straight forward. This will look
something like the following:
\begin{lstlisting}
@Effect()
getProductInformation$ = this.dataPersistence.fetch(
  ProductTypes.getProductInformation, {
  run: (action: GetProductInformation, state: ProductModelState) => {
    const { userId, productId } = action.payload;

    return this.service
      .getProductInformation(userId, productId)
      .pipe(
        map((product: Product) => new ProductLoaded(product))
      );
  },

  onError: (action: GetProductInformation, error) => {
    console.error('Error', error);
  },
});
\end{lstlisting}

In this example we fire off a service, and return the data from that one
service. We might even have the option to turn the map into a switchMap, that
can be used for numerous actions, originating from a singular service. However,
it immediatly becomes a problem once we start to have a singular effect call
numerous services, wherein one action is expected to have all of it's data.

\subsection{Using the Aggregator Pattern}

\mybox{Before we introduce the aggregator pattern, it is important to make one
note. In any situation with numerous actions, we are going to want to include a
correlation id within our app. It is important to do this, so that as we have
numerous actions coming in at a time, the request is being done for the proper
item at that time.}

\subsubsection{Defining Types for Actions}
\begin{lstlisting}
export type AggregatableAction = Action & { correlationParams?: CorrelationParams };

export type FailActionForAggregation = Action & { error?: Error, correlationParams?: CorrelationParams };
\end{lstlisting}

First we define the types for our different actions. Both will be a standard
action with the addition of correlationParams, which we will use to make sure
that we are indded calling the correct action.

\subsubsection{Passing in Params for Functions}
\begin{lstlisting}
export function aggregate<T extends AggregatableAction,
    TAction1 extends AggregatableAction,
    TAction2 extends AggregatableAction,
    TFailAction extends FailActionForAggregation>
(
    action1$: Observable<TAction1>,
    action2$: Observable<TAction2>,
    failAction$: Observable<TFailAction>
): OperatorFunction<T, [TAction1, TAction2]> {
\end{lstlisting}
Now we go ahead and specify the params that we need to use for the aggregator
pattern. We are going to assume that there will never be a situation wherein
we will ever use more than two actions.

\subsubsection{ Creating a Filter Action }
\begin{lstlisting}
    const filterAction = (sourceAction: AggregatableAction, t: AggregatableAction) =>
        t.correlationParams && sourceAction.correlationParams &&
        t.correlationParams.correlationId === sourceAction.correlationParams.correlationId &&
        t.correlationParams.parentActionType === sourceAction.type;
\end{lstlisting}

First:
\begin{enumerate}
  \item We confirm that the function passed in correlation params, and that our
source action has correlation params.
  \item We check to make sure that the correlationId of the correlation params
  matches that of the correlationId of the correlationParams of the soruceAction
  \item The correlationParams action type, is equal to that of the
  sourceAction's type.
\end{enumerate}

\subsubsection{ Creating Aggregated Actions }
\begin{lstlisting}

    const getAggregatedActions = (sourceAction: AggregatableAction): Observable<[TAction1, TAction2]> => {
        let a1$ = action1$
            .pipe(
                filter(a => {
                    return filterAction(sourceAction, a);
                }),
                first()
            );
        let a2$ = action2$
            .pipe(
                filter(a => {
                    return filterAction(sourceAction, a);
                }),
                first()
            );

        let f$ = failAction$
            .pipe(
                filter(a => {
                    return filterAction(sourceAction, a);
                }),
                first(),
                switchMap(b => {
                    return Observable.throw(b.error);
                })
            );

        return race(forkJoin([a1$, a2$]), f$);
    };
\end{lstlisting}

We hvae to re-visit this and see if there is a use case to where we would
actually use an aggregator pattern.
\begin{lstlisting}
    return (source: Observable<AggregatableAction>) => source.pipe(
        switchMap(sourceAction => getAggregatedActions(sourceAction))
    );
}
\end{lstlisting}
