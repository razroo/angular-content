\maketitle{}
\section{ State Directory Structure }

When one begins an Angular project for the first time, it can be increasingly
difficult to manage ngrx/store. State, while it should ideally be tied to a
feature, as the app moves on, might not neccesarily be tied to a specific
feature, or page. In addition, +state by nature, as a single giant object,
is global by nature. Also, it takes up a large portion of any app. It makes
sense to put all state in a single repository, so that state within the app can
be transparent(go into this a bit more). Finally, for testing purposes, state is
a very large chunk of business logic for app, and deserves it's own module for
bundling, and testing purposes.

\subsection{Data Access Folder/File Structure }
State being a way of accessing data, an appropriate name for the folder/file
tree for state makes sense to be called data-access. It will look something
like the following:

% Tree structure bleeding outside of page. Need to comment out for now.
% \begin{forest}
  [libs
    [px-illustrator
      [data-access
        [code-box
          [src
            [lib
              [\_+state
                [\_code-box.actions.ts,file]
                [\_code-box.adapter.ts,file]
                [\_code-box.effects.spec.ts,file]
                [\_code-box.effects.ts,file]
                [\_code-box.facade.mock.ts,file]
                [\_code-box.facade.spec.ts,file]
                [\_code-box.facade.ts,file]
                [\_code-box.reducer.spec.ts,file]
                [\_code-box.reducer.ts,file]
                [\_code-box.selectors.ts,file]
              ]
              [\_px-illustrator-data-access-code-box.module.ts,file]
              [\_px-illustrator-data-access-code-box.module.spec.ts,file]
              [\_px-illustrator-data-access-code-box-testing.module.spec.ts,file]
            ]
            [\_index.ts,file]
            [\_test.ts,file]
          ]
          [\_karma.conf,file]
          [\_README.md,file]
          [\_tsconfig.lib,file]
          [\_tsconfig.lib.json,file]
          [\_tsconfig.spec.json,file]
          [\_tslint.json,file]
        ]
      ]
    ]
  ]
\end{forest}


Most notably, all of the state related code is contained within a single
folder. By doing so, it solves all of the above three issues:
\begin{enumerate}
  \item State is global, and therefore can now be used by multiple features
  \item We have the ability to run
  ng test --project=px-illustrator-data-access-code-box and it will run code
  specifically for this data-access feature state
  \item By globalizing naming convention, we can streamline naming convention
  of all global files intended towards working towards the same purpose. Namely,
  our data-services, and features.
  \item In addition, it alleviates the potential issue of circular dependencies.
  If, for instance, we have feature folder A, and feature folder B. B might need
  state from A, and A might need feature state from B. By keeping all of our
  state global, it helps us circumvent this circular dependency problem.
\end{enumerate}