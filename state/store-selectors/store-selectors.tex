\maketitle{}
\section{ Store Selectors }

Selectors are pure functions /footnote{pure function in case you are not aware
of already, are functions which always return the same result, given a certain
parameter} that take slices of state as arguments and return some state data
that we can pass to our components.

\mybox{Selectors can become nitty gritty, especially as we start to get into
the finer detail of pure functions. The following covers about 90\% of all
use cases.}

\subsection{Object That We Will be Working With}
\begin{verbatim}
{
  settings: {
    column: '20',
    row: '20',
    pixel: '20',
    draw: true,
    false: false,
  },
  colorPicker: {
    backgroundHex: '#191919',
    backgroundRed: '25',
    backgroundGreen: '25',
    backgroundBlue: '25',
    pixelHex: '#000000',
    red: '25',
    green: '25',
    blue: '25'
  },
  codeBox: {
    css: [
    {'123': {color: blue, x: 20, y: 17}},
    {'246': {color: orange, x: 20, y: 18}},
    {'246': {color: orange, x: 20, y: 19}},
    ],
    sass: false,
    less: false,
    js: false
  }
}
\end{verbatim}

\subsection{Basics of Select using ngrx/store}

The simplest method is to grab state by the using store select method:
\begin{lstlisting}
export const settings$ = this.store.select('settings');
\end{lstlisting}

This will grab the store data for settings and return an observable. So for
instance, let's say inside of our component, doing:
\begin{lstlisting}
  this.gridFacade.settings$.subscribe((data) => {
   console.log(data)
  });
\end{lstlisting}
 will produce JSON for:
\begin{verbatim}
  {
    column: '20',
    row: '20',
    pixel: '20',
    draw: true,
    false: false,
  }
\end{verbatim}

\subsection{ Feature State in NGRX }
Previously we discussed being able to select a certain slice of state being
using store.select. However, there are many scenarios wherein it is not
as simple as selecting the featureState and passing it to the component. For
instance:
\begin{enumerate}
  \item When using ngrx/entity, creating a dictionary of values, and an array
  of ids. In order to grab ids, entities, ids and entities, a specific entity,
  it's important to be able to have the a base feature selector to work off of.
  \item When having numerous nested peices of data, under one specific feature
  state slice. We can keep our code DRY \footnote{Don't Repeat Yourself}, by
  want be able to build off of nested data on a number of levels.
\end{enumerate}

\mybox{It's important to note, that in any enterprise setting, where the data
in the store is being populated using backend(GraphQL preferably), we strongly
recommend the use of ngrx/entity. It will simplify any given situation if you
do anything beyond pulling in data as is. Therefore the following
createFeatureSelector will always be used, and in the enterprise applications
that I work on, they are consistently used for every data-access feature I
create.}

\subsection{ Feature State in NGRX - An Example using Ngrx/entity}

Ngrx/entity if not familiar already, will use what is called a normalized
database. It is a way of having the way of accessing something independent of
the actual data. Therefore, it easily allows for an item to be created, removed,
updated, or deleted. Ngrx/entity is ngrx/store's solution to creating a
noramlized database. Using feature state with ngrx/entity is imperative in order
to streamline all the various ways of selecting ngrx/entity data. For instance,
let's say we had entities for our codebox, and we wanted to select data in a
number of ways:

\begin{enumerate}
  \item Select all entities
  \item Select all ids
  \item Select the current selectedId by using an Entity
\end{enumerate}

In order to streamline this process, we would do the following:
\begin{lstlisting}
const { selectAll, selectEntities, selectIds } = codeBoxAdapter.getSelectors();
const getCodeBoxState = createFeatureSelector<CodeBoxState>(
  'codeBoxStateModel'
);

const getCodeBoxEntities = createSelector(
  getCodeBoxState,
  selectEntities
);

const getCodeBoxIds = createSelector(
  getCodeBoxState,
  selectIds
);

const selectedBuyerId = createSelector(
  getCodeBoxState,
  state => state.selectedBuyerId
);

const getSelectCodeboxId = createSelector(
  getCodeBoxEntities,
  selectedBuyerId,
  (codeBoxEntities, id) => codeBoxEntities[selectedBuyerId]
);
\end{lstlisting}

\mybox{
\subsection{How this would look like in a real world application}
Just to douse curiousity of someone who might be wondering how this would work
in the real world. Normally one would wire this into a facade file. The facade
line of code would look something like the following:

codeBoxEntities = this.store.select(getCodeBoxEntities);

}
