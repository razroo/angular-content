\maketitle{}
\section{ Store Selectors }

Selectors are pure functions /footnote{pure function in case you are not aware
of already, are functions which always return the same result, given a certain
parameter} that take slices of state as arguments and return some state data
that we can pass to our components.

\mybox{Selectors can become nitty gritty, especially as we start to get into
the finer detail of pure functions. The following covers about 90\% of all
use cases.}

\subsection{Object That We Will be Working With}
\begin{verbatim}
{
  settings: {
    column: '20',
    row: '20',
    pixel: '20',
    draw: true,
    false: false,
  },
  colorPicker: {
    backgroundHex: '#191919',
    backgroundRed: '25',
    backgroundGreen: '25',
    backgroundBlue: '25',
    pixelHex: '#000000',
    red: '25',
    green: '25',
    blue: '25'
  },
  codeBox: {
    css: [{id: '123', color: blue, x: 20, y: 17, id: }],
    sass: false,
    less: false,
    js: false
  }
}
\end{verbatim}

\subsection{Basics of Select using ngrx/store}

The simplest method is to grab state by the using store select method:
\begin{lstlisting}
export const settings$ = this.store.select('settings');
\end{lstlisting}

This will grab the store data for settings and return an observable. So for
instance, let's say inside of our component, doing:
\begin{lstlisting}
  this.gridFacade.settings$.subscribe((data) => {
   console.log(data)
  });
\end{lstlisting}
 will produce JSON for:
\begin{verbatim}
  {
    column: '20',
    row: '20',
    pixel: '20',
    draw: true,
    false: false,
  }
\end{verbatim}

\subsection{ Feature State in NGRX }
