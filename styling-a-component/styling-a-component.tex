\maketitle{}
\section{ Styling a Component }

Styling a component, of course, is a very complex topic. With styling, as an
architect, there are 5 things that you will have to keep in mind:
\begin{enumerate}
  \item Responsive design(even if you have a mobile/tablet app)
  \item Design system
  \item Pre-processor of choice(Scss, Less, PostCss, etc.)
  \item Naming convention of CSS classes
    \begin{enumerate}
      \item Material Design(Google)
      \item Fluent Design(Microsoft)
      \item Flat Design(Apple)
    \end{enumerate}
  \item Dynamic Classes
\end{enumerate}

\subsection{ Responsive Design }
With regards to responsive design, there are a number of frameworks, one can
choose with regards to creating a web app. The following are quite popular:
\begin{enumerate}
  \item Foundation
  \item Bootstrap
  \item Semantic UI
\end{enumerate}

However, the above for a grid system tend to be overkill in my opinion.
Specifically, the direction many UI web app tend to head, is that it will only
be used on Desktop. For mobile and tablet, there will be a separate Android and
IOS app created. In addition, due to the nature of angulars component
architecture, the use of ready made components, containers for apps, are the
only thing which will actually have specific media queries.

It is strongly suggested that your own super lightweight grid is created, or
used. I think it is important to keep in mind that most grid systems can be
limited to 500 lines of code, or less. I personally prefer to use Skeleton
\footnote{http://getskeleton.com/\#grid}. However, I am in the process of
creating my own grid system using css-grid.
