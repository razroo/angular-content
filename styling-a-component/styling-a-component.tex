
\chapter{ Styling a Component }

Styling a component, of course, is a very complex topic. With styling, as an
architect in an Angular setting, there are 4 things that you will have to keep
in mind:
\begin{enumerate}
  \item Pre-processor of choice(Scss, Less, PostCss, etc.)
  \item Design system
    \begin{enumerate}
      \item Material Design(Google)
      \item Fluent Design(Microsoft)
      \item Flat Design(Apple)
    \end{enumerate}
  \item Responsive design(even if you have a mobile/tablet app)
  \item Naming convention of CSS classes
\end{enumerate}

\section{ Pre-processor of choice }
For our preprocessor, we have chosen Sass. \footnote{Incude link for a
discussion of why that is}

\section{ Naming Convention }
For our naming convention, we will go with BEM. It is an extremely easy way
of setting a part a specific component from an html and css side of things. A
quick primer on BEM.
Block is a component. We will be using pascal casing for ours \footnote{Link to
airbnb style guide}
Element is a child of block. It uses an underscore. For instance:
\begin{verbatim}
<div class = 'ChooseSize__input'></div>
\end{verbatim}
M stands for modifier. A modifier is an element, which modifies an already
existing element.
\begin{verbatim}

\end{verbatim}

\section{ Design System }
In an Angular setting, the component library which seems to make most sense is
Material Components
\footnote{https://material.angular.io/components/categories}. For starters, it
is a complete design system. All component's design will be synonymous with
each other. In addition, it is in the process of creating a cdk, which makes
all of these components customizable. In addition, it is a really nice design,
and feels native to the way Angular works. I have used it versus other libraries
and I can really say the documentation is just fantastic. I have used it in more
complex settings(e.g. the data-table), and adding on new functionality has been
just a joy.

\section{ Adding Material Design to Our App }
First install Angular Material components and Angular Animations to our app.
\begin{verbatim}
  npm install --save @angular/material @angular/cdk
  npm install --save @angular/animations
\end{verbatim}

In addition, we will need to add default styling to our app, in order for
styling to be applied to our Angular Material component. Inside of our
styles.scss file, import the following.
\begin{lstlisting}
@import '~@angular/material/prebuilt-themes/deeppurple-amber.css';
\end{lstlisting}

\section{ Our first component }
In our app, we are going to create our first component. It is essentially a form
with three fields:
\begin{itemize}
  \item Columns
  \item Rows
  \item Pixel Size
\end{itemize}

In addition, there will be a button which will say, 'Create Grid'. We are also
going to wrap our component, with the <mat-card> component, add a width of
300, margin-top and center.

\subsection{ Notable Mention - @HostBinding }
In an angular app, many times, we will want to add a specific class to our
parent container. In our situation, we will be using BEM, and creating a
ChooseSize class. It will implement flex, and use justify content, in order to
center the <mat-card> component.

\begin{lstlisting}[caption=My Javascript Example]
import { Component, HostBinding, OnInit } from '@angular/core';

@Component({
  selector: 'app-choose-size',
  templateUrl: './choose-size.component.html',
  styleUrls: ['./choose-size.component.scss']
})
export class ChooseSizeComponent implements OnInit {
  @HostBinding('class') class = 'ChooseSize';
  constructor() {}

  ngOnInit() {}
}
\end{lstlisting}

By putting @HostBinding as a decorator \footnote{If not familiar with decorator
, it is a function that is run when particular class is called} within our app,
it causes the host class to have the ChooseSize class. We are then able to
target our host element our scss:
\begin{verbatim}
  :host.ChooseSize {
    display: flex;
    justify-content: center;
  }
\end{verbatim}

\section{ CSS Naming Convention }
In your modern day front end framework, such as Angular, generally, we do not
have to worry about clashing namespaces. \footnote{Historal footnote, the
turning point for me was with this article \href{https://glenmaddern.com/articles/css-modules}{here}}.
Many other issues with css at scale, have been solved as well, have been 
solved by the general ecosystem, scss included.

However, reccomended architecture is that one still use something like BEM. I
would like to argue for using BEM in an Angular setting:
\begin{enumerate}
  \item It allows for easy grep in code base, when inspecting element first
  within chrome.
  \item It documents the type of element that it is.
  \item It will give structure to html, without need of using pug, or some other
  html pre-processor.
  \item Ease's creation of classes for integration testing\footnote{Use a modifer
  for BEM}
\end{enumerate}

It should be noted, that within our app, the form has been made a particular
width, which will work on all screen sizes, without the need of adjusting width.
As we move along in our app, we will have the option to look into sitations
wherein we can use actual media queries.
