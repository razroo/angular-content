\maketitle{}
\section{ Styling a Component }

Styling a component, of course, is a very complex topic. With styling, as an
architect, there are 5 things that you will have to keep in mind:
\begin{enumerate}
  \item Design system
    \begin{enumerate}
      \item Material Design(Google)
      \item Fluent Design(Microsoft)
      \item Flat Design(Apple)
    \end{enumerate}
  \item Responsive design(even if you have a mobile/tablet app)
  \item Pre-processor of choice(Scss, Less, PostCss, etc.)
  \item Naming convention of CSS classes
  \item Dynamic Classes
\end{enumerate}

\subsection{ Design System }
In an Angular setting, the component library which seems to make most sense is
Material Components
\footnote{https://material.angular.io/components/categories}. For starters, it
is a complete design system. All component's design will be synonymous with
each other. In addition, it is in the process of creating a cdk, which makes
all of these components customizable. In addition, it is a really nice design,
and feels native to the way Angular works. I have used it versus other libraries
and I can really say the documentation is just fantastic. I have used it in more
complex settings(e.g. the data-table), and adding on new functionality has been
just a joy.

\subsection{ Adding Material Design to Our App }
First install Angular Material components and Angular Animations to our app.
\begin{verbatim}
  npm install --save @angular/material @angular/cdk
  npm install --save @angular/animations
\end{verbatim}

In addition, we will need to add default styling to our app, in order for
styling to be applied to our Angular Material component. Inside of our
styles.scss file, import the following.
\begin{verbatim}
@import '~@angular/material/prebuilt-themes/deeppurple-amber.css';
\end{verbatim}

\subsection{ Our first component }
In our app, we are going to create our first component. It is essentially a form
with three fields:
\begin{itemize}
  \item Columns
  \item Rows
  \item Pixel Size
\end{itemize}

In addition, there will be a button which will say, 'Create Grid'.


\subsection{ Responsive Design }
With regards to responsive design, there are a number of frameworks, one can
choose with regards to creating a web app. The following are quite popular:
\begin{enumerate}
  \item Foundation
  \item Bootstrap
  \item Semantic UI
\end{enumerate}

However, the above for a grid system tend to be overkill in my opinion.
Specifically, the direction many UI web app tend to head, is that it will only
be used on Desktop. For mobile and tablet, there will be a separate Android and
IOS app created. In addition, due to the nature of angulars component
architecture, the use of ready made components, containers for apps, are the
only thing which will actually have specific media queries.

It is strongly suggested that your own super lightweight grid is created, or
used. I think it is important to keep in mind that most grid systems can be
limited to 500 lines of code, or less. I personally prefer to use Skeleton
\footnote{http://getskeleton.com/\#grid}. However, I am in the process of
creating my own grid system using css-grid. (need to get back to this one).
