\chapter{ Using React with Angular }

One of the singular most fustrating parts of front end development, is the fact
that frameworks change quite often as time goes on. In addition, different frameworks have different usages. Two of the currently most profitable front end "ecosystems" are Angular + React. Sometimes, having something only available in Angular can be cumbersome. Being able to break into the React ecosystem, for instance, can be extremely valuable. 

Microsoft came across the use case of needing to have React within their Angular applications. They created the open source library called \lstinline{angular-react}. It allows for the ability to use React in Angular. Let's discuss that.

\section{ Reasons to use React in Angular? }

\begin{enumerate}
  \item Internally one team built a very large component in React. Your Angular team would now like to use this React component, with limited overhead.  
  \item Industry trends push your team to migrate towards React. For instance, one time I was approached by a CTO of a company who mentioned he is coming across more React developers in the market. They might not architecting their React applications as they need, but that might be the benefit of React, for a startup. Wherein it doesn't require as much overhead. For their business, a lesser architected application might be beneficial, because their getting to the market and earning a profit is integral given the funding they have. 
  \item A very large component library is only availabe in React, and not in Angular. For instance, in the case of Microsoft, who actually created the open source library \lstinline{angular-react}, they needed to use Office UI Fabric, which was only available in React. 
\end{enumerate}

\subsection{Sidebar - When Not to Use React in Angular}
Intuitively, one can look at the

\section{Performance Concerns of React in Angular }
