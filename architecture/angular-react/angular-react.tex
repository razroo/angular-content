\chapter{ Using React with Angular }

One of the singular most fustrating parts of front end development, is the fact
that frameworks change quite often as time goes on. In addition, different frameworks have different usages. Two of the currently most profitable front end "ecosystems" are Angular + React. Sometimes, having something only available in Angular can be cumbersome. Being able to break into the React ecosystem, for instance, can be extremely valuable. 

Microsoft came across the use case of needing to have React within their Angular applications. They created the open source library called \href{github.com/microsoft/angular-react}{angular-react}. It allows for the ability to use React in Angular. Let's discuss that.

\section{ Reasons to use React in Angular? }

\begin{enumerate}
  \item Internally one team built a very large component in React. Your Angular team would now like to use this React component, with limited overhead.  
  \item Industry trends push your team to migrate towards React. For instance, one time I was approached by a CTO of a company who mentioned he is coming across more React developers in the market. They might not architecting their React applications as they need, but that might be the benefit of React, for a startup. Wherein it doesn't require as much overhead. For their business, a lesser architected application might be beneficial, because their getting to the market and earning a profit is integral given the funding they have. 
  \item A very large component library is only availabe in React, and not in Angular. For instance, in the case of Microsoft, who actually created the open source library \lstinline{angular-react}, they needed to use Office UI Fabric, which was only available in React. 
\end{enumerate}

\subsection{Sidebar - When Not to Use React in Angular}
Intuitively, one can look at the ability to use React in Angular, and think great! I'll use the best of both frameworks!!! Ok, well hold on their Cowboy, or Cowgirl. It might not be the best idea to create a Frankenstein of an Angular + React application. The two ecosystems can clash with each other. Between the way data is passed around, rendering, libraries used for state management etc. However, if you find yourself gravitating towards one of the three reasons above, using React in Angular, can be very beneficial.

\section{Performance Concerns of React in Angular}
Obviously rendering React elements inside of Angular seems like a little bit too go to be true, right? Well this is a partially correct assumption. Let's dissect the logic within the \lstinline{angular-react}. It creates a layer around React methods, so that they are understandable by Angular. It then will call React within Angular. The main fault with this approach, is there will be two separate rendering engines at play. So, if you attempt to have them both render at the same time on the same component, from the browser's side of things, it can cause them to trip over each other. The \lstinline{angular-react team has created a demo proving this point \href{https://microsoft.github.io/angular-react/performance/mixed}{here}.

\section{Nature of the Library}
I always try to do my background research on a library before I go ahead and use it within my application. In this one, there are two red flags: 
\begin{enumerate}
  \item They created this library to use Office UI Fabric. I.e. rendering of singular components within an Angular setting. 
  \item It is a product team on Microsoft, that are the main maintainers for this library. Given the limited use case of this library, expect it to not be optimized for maximum performance, or when multiple components are needed. Honest, a React Elements approach makes more sense in this scenario. 
\end{enumerate}

In most applications I've worked on, the use case for \lstinline{angular-react} is very limited. Proceed to the next chapter on custom web components for a better long lasting solution.