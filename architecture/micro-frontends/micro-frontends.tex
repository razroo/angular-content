\chapter{ Micro Frontends }
Micro frontends is definitely a "buzzword". I consider it most synonymous with the buzzword "the cloud". Simply it means breaking up the application into smaller more manageable parts. So, you might be asking yourself, what makes this different than simply architecting an application well? The answer to that,is that it's not so much folder/file directory, and how you dissect your components, but rather flexibility in bundling and deployment. By allowing for flexible bundling and deployment within app. This accomplishes three things: 
\begin{enumerate}
  \item Increases testability
  \item Increases re-usability
  \item Ability to select multiple frameworks
\end{enumerate}

\section{Micro Frontends are actually already in the wild!}

The truth is that micro-frontends have been around for a really long time. Doing something like this is possible in vanilla Javacript is possible, by simply separating micro applications into different JS bundles. When the page goes to a specific route, it will call this unique JS bundle. Incredible simple in practice. I've also seen application get somewhat fancy, wherein routing was generated by the backend, via devops. The js bundle would be given a different id everytime, and therefore made sure prior js version would be removed. It was a relatively painless way for the JS team to create micro front ends. 

However, I'm not sure if anyone on the team has ever coined this as "micro frontends". In addition, doing something like this was a real pain, and the flexibility was really only there on a single route. 

So, why is there is now a push for micro frontends architecture, and why is it now coming into the forefront? For this, we have to understand a bit more of the history of web applications.

\section{History of Web Applications}
Let's dive into that really quickly.

\subsection{Monolith Applications}
In the earlier days, when front ends were relatively new, applications were generally built in a monolith fashion. Applications would be built with a single Java WAR file. Your backend code, and and front end code, were contained in a single file. 

\subsection{Separating Backend and Frontend}
As time went on, backend and frontend would be separated. Html files, and javascript would be separated. The appropriate data would be inserted using php, or perl tokens, or the like. Every single api was in the php application, and it would be used, by simply inserting the php data needed for that site. However, the whole of the backend would be in a single php application. 

\subsection{Creating Micoservices}
As time went on, and mobile devices became more popular(i.e. apps), it became more cost efficient for companies to create micro services. Primarily, so that multiple teams could work on it. It's important to note that a micro service is not a silver bullet, and won't neccesarily increase the efficiency, and maintainability of your application. However, it has it's place, and greatly resolves scaling issues. So mutliple backend applications would be built by either different teams, or singular teams. These would then be able to be deployed in a REST Service fashion, allowing different clients, such as mobile, or desktop, to consule. 

\subsection{Applying Microservice Architecture to Microfrontends}
However, as front ends are growing very large in size, and it is becoming increasinly easy to do so, due to the frameworks readily available, large teams are required to work on a single application. It's much easier to break up those teams, give them a singular component library to work off of, and then have them work in their own clusters. 

However, in my personal opinion, micro-frontends are not ready yet for prime time. There is no open source framework that I can point to, that would allow for this to be easily done. Or rather, this can be done, but having a development tool that would allow for this to be done, 

Within frameworks yes, and that is a large step. However, it can be a bit awkward to try and shove on bundle of a particular framework into another. In particular, because state management, should ideally be global. Doing this across multiple frameworks, is very difficult at this point in time. 

\section{Recap on Microservices Design Principles}
\begin{enumerate}
  \item High Cohesion 
  \item Autonomous 
Deployed independently 
  \item Business Centric
References a specific area within business 
  \item Resiliency 
If one area of application faults, allows us to degrade just one section of the application. 
  \item Observable 
Centralized logging, and monitoring, we can see what our individual micro front ends are doing.
  \item Automation 
Each micro frontend are independently testable and Deployable. 
\end{enumerate}

\section{Micro Frontends}
As we have expressed above, micro frontends work very similar to micro services above. However, what we attempt to do, is co-ordinate with the micro service, and try to create a micro frontend slice that is exactly vertical to the micro service slice. So let's say we have a product api, we would have a micro service product slice. If we have a cart api, then we would create a cart slice. 

\section{Autonomous Features}
So when something is autonomous in programming it means that everything can be changed independently, without breaking anything else within our app. So that would mean that code is isolated, so that variables are not shared from one to another etc. In addition, we should have a base app, that houses all of our micro-fronteds into a singular application. All micro frontends must communicate independently, in a similar way, micro frontends should do that as well. 

We will therefore use our base app/DOM, or our root ngrx/store in order to get all micro frontends to talk to each other. 

\section{Team Ownership}
This part can be a bit controversial. The idea here, is that now that we have a micro application, now members of the team are 

\section{Illustration in Code}
A simple way of illustrating how we can create micoservices within Angular, would be as follows. Let's say that we have an application. One of them is a pdf viewer application. The other is the application for analytics. We want our teams to be able to build applications independent of each other. 

\subsection{Adding The Tools We Need}
\begin{verbatim}
ng add @angular/elements
ng add ngx-build-plus
\end{verbatim}

This adds Angular Elements to our app. In addition, we are going to be adding a tool called \lstinline{ngx-build-plus} which allows us to extend the default Angular CLI without injecting. That way, we can continue adding default Angular CLI builds to our app. 

\subsection{General Strategy}
The general strategy is like this. We are going to create miniature using Angular Elements. We are going to then insert these bundled applications into our main Angular application. 

We then will go ahead and route to those specific applications within our application. Ideally using the Angular router. Particularly, because we want it to be robust enough, so that we can go ahead and create router-outlets. 

\section{Difference Of Approach Using Custom Web Components}
The difference between the approach we are using here, and how we approached using custom web components is as follows. When creating micro-frontends, we don't just need access to our custom web components, however, we actually want to build them separately. Once again we do this so that: 
\begin{enumerate}
  \item Increase testability
  \item Increase re-usability
  \item Ability to select multiple frameworks for these specific apps
\end{enumerate}

\section{Router Dillema}
The main dillema within our Angular applications, is the fact that angular routers consist of two parts: 
\begin{enumerate}
  \item forRoot
  \item forChild
\end{enumerate}

Within our web components, or our Angular Elements rather, our forRoot will be reserved for our root component specifically. If we want to use web components within our app, forChild won't work either, because these are all micro front ends. In fact using something like Angular Router for our application, is a bit of an anti-pattern, because the whole point is to create a framework agnostic architecture. So, what should we use within our Angular application. 
