\chapter{ Internationalization and Localization}

If not familiar with already, internationalization is the process of making sure
your app can be used by a worldwide audience. Localization, not to be confused 
with internatilization, is the process of changing app's text, to represent 
locale of user.

Internationalization is one of those things that is generally done farther down
the life cycle of an app. This makes complete sense, as when building a product, 
a team will generally want to build the infrastructure first, based on a 
particular language. Then, it will want to make it so that the app can be based 
on numerous different demographics. However, being aware of the following will 
make internationalization more of a seamless process. 

\section{General Concerns of Internationalization }

\begin{enumerate}
  \item What type of dates, numbers, perxentages and currencies do you want. 
  \item Setting aside text in components, so can be swapped out to the 
  appropriate language. 
  \item Plural words are different in different languages. So it's important 
  have something baked within the framework that does this. 
  \item Alternate text based on scenario(e.g. if person in New York display 
  such and such event, if person in San Francisco display message such and 
  such.)
\end{enumerate}

\section{General Concerns of Localization }
\begin{enumerate}
  \item Creating multiple language versions of app(will get to how that is done)
  \item Extracting Localizable text(not sure exactly what that is).
  \item Building and servimg an app for a given locale. 
\end{enumerate}

\section{Default Locale}
Localization for years, well before Angular, have been using ids for the 
applicaton to understand what the language should be. Angular, as well, has an
extensive list of different languages within it's framework. 

\mybox{Angular's internal framework uses the BCP47 norm. It's very important
to note that if you plan on building on top of Angular's localization framework,
that you use this as your standard within the framework.}

Angular by default will use 

\section{ Architecting Internationalization }
The reason why architecting internationalization can be something that is
something for us to keep in mind, is that it can have an affect on how we might
view the architecture of our application. A data table, for instance, is a great
example. We might want to propose a rigid re-usable data-table. However, once
we introduce internationalization, having data tables that are unique per each
language will be important. Having a re-usable data table that takes care of
information through passing in props can be cumbsersome.

\section{ Smart and Dumb Components }
Creating smart and dumb components, is your most valuable resource when it comes
to internatilization. It will allow you to set up different components based on
certain languages. 
