\section{ Http Interceptors }
\maketitle{}

An http interceptor, is a way to intercept an HTTP request, before passing them
along. Some of the more popular reasons of doing something like this, include 
Authentication, and adding default Headers to requests. I would like to add 
scenarios for an Angular dev to be aware of, wherein they should apply 
interceptors. In addition, I would like to reccomend the use of using Apollo 
Client middleware over Angular's Http Interceptors, as that is Razroo's 
reccomendation. Third, I would like to bring in code examples, so that it will
be easier for the dev reading this to get up and running.

\subsection{Dissecting Two Ways of Creating Interceptors}
Within our architecture we assume that your backend team will be using GraphQL. 
However, this is something that is beyond you as a UI Engineer. As a result,
we give a head nod towards there being multiple apis, including those using a 
regular RestSvc. This translates to: 
\begin{enumerate}
  \item Angular's native \lstinline{HttpInterceptor}
  \item Apollo Client 
\end{enumerate}

\subsection{Understanding Interceptors In General}
An Interceptor will generally take in the current outgoing request, and pass 
in the next interceptor. Alternatively, it can transform the response stream 
it's self. 

\subsection{Example of Interceptor using HttpInterceptor}

The following is a great example of how interceptors work. Let's say that we
want to console out an error whenever it happens. In addition, we want to 
set up different errors based on how they happened. Something simple like 
the above can setup error handling notifications across the site for all users.  
\begin{lstlisting}
  return next.handle(req).pipe(
    tap((event: HttpEvent<any>) => {
    if (event instanceof HttpResponse && event.status === 400) {
      this.dialog.error("There was an error trying to make your request. 
      If this continues to persists, please e-mail 401@razroo.com");
    }
  })
  );
\end{lstlisting}

\subsection{Example of Interceptor using Apollo Client}
Apollo Client out of the box offers out of the box middleware to intercept an
http request. 

\subsection{Popular Scenarios For Using Interceptors}
Just to make it so, so that someone embarking on interceptors on the first 
time can be aware, as well, as those who have already worked on it. In
particular, so they can reference this chapter when working on them again. 

\subsubsection{ Authentication }
Starting from what will be time, and time again, the most single most important
piece of your application. Authentication, if not already aware, is the process 
of identifying an individual usually based on a username and a password. There
are certain situations wherein intercepting the http request before it is made,
is required for authentication:
\begin{enumerate}
\item Add Bearer Token - The word "Bearer" in "Bearer Token" is to be 
understood as "give access to the bearer of this token". It is a token 
generated by the server on initial login, and is to be used for every protected
request within the app. 
\item Refresh Token - Refresh tokens are used when the original Bearer Token
expires, and new ones are to be issued.
\footnote{Look into this, we might want Refresh tokens removed} 
\item Redirect To Login Page - This will be needed if the Bearer Token expired,
and we need the user to go back to the login page to retrieve a brand new 
bearer token. 
\end{enumerate}

An old co-worker of mine, Sam Severance, once told a funny story, how we was
once working on the couch and his wife stopped by and asked him what he was 
working on. He told her, "Nothing too much, just authentication". She said, "oh
what's that?". He told her, "Oh, it's just something that makes programmers run
around like chickens with their heads cut off". I always thought it was a funny
one. 

I guess the point behind that story, is that authentication is such an 
important piece of an application, and end's up being used in every request. It
therefore becomes: 
\begin{enumerate}
  \item Relatively difficult to manage.
  \item High cost to application if software is mis-managed. 
\end{enumerate}

In addition, every time authentication is worked on for the first time in an 
application, it tends to be specific to framework, and technology within 
framework you are working on. 

