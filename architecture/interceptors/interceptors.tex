\chapter{ Http Interceptors }


An http interceptor, is a way to intercept an HTTP request, before passing them
along. Some of the more popular reasons of doing something like this, include 
Authentication, and adding default Headers to requests. I would like to add 
scenarios for an Angular dev to be aware of, wherein they should apply 
interceptors. In addition, I would like to reccomend the use of using Apollo 
Client middleware over Angular's Http Interceptors, as that is Razroo's 
reccomendation. Third, I would like to bring in code examples, so that it will
be easier for the dev reading this to get up and running.

\section{Dissecting Two Ways of Creating Interceptors}
Within our architecture we assume that your backend team will be using GraphQL. 
However, this is something that is beyond you as a UI Engineer. As a result,
we give a head nod towards there being multiple apis, including those using a 
regular RestSvc. This translates to: 
\begin{enumerate}
  \item Angular's native \lstinline{HttpInterceptor}
  \item Apollo Client / Apollo Link
\end{enumerate}

\section{Should We Even Use Apollo Link At All?}
One of the popular questions people ask, is should even be using apollo link at 
all. HTTP Interceptor is used under the hood for Apollo Client. The truth is, 
this is a fundemental question that can be applied in many similar scenarios. 
We are using a framework. One piece of it that it offers, is not required by 
our application. Should we use it, or should we not? 
\begin{enumerate}
  \item Are other teams using our framework, and if not will they have a chance
  to use our tech? 
  \item Do we ever expect to use any framework beyond the one we are using(in 
  our scenario Apollo Client)? 
  \item If unexpectedly, which it always is. If unexpectedly, we end up 
  changing over our framework, how difficult will it be to create this code 
  once again? 
  \item Ease of use for the framework, vs using native code. 
  \begin{enumerate}
    \item Documentation better for one, over the other? 
    \item Integration for this particular aspect of framework easier granted 
    already using other pieces of framework? 
  \end{enumerate}
\end{enumerate}

Using the above back and forth, it would make sense to use Apollo Client's 
native functionality for middleware. To use HTTP Interceptors, and then 
move over to Apollo Client for everything else, such as cache, makes 
maintenance more cumbersome. For that reason Razroo reccomends using Apollo
Link for middleware.

\section{Understanding Interceptors In General}
An Interceptor will generally take in the current outgoing request, and pass 
in the next interceptor. Alternatively, it can transform the response stream 
it's self. 

\section{Example of Interceptor using HttpInterceptor}

The following is a great example of how interceptors work. Let's say that we
want to console out an error whenever it happens. In addition, we want to 
set up different errors based on how they happened. Something simple like 
the above can setup error handling notifications across the site for all users.  
\begin{lstlisting}
  return next.handle(req).pipe(
    tap((event: HttpEvent<any>) => {
    if (event instanceof HttpResponse && event.status === 400) {
      this.dialog.error("There was an error trying to make your request. 
      If this continues to persists, please e-mail 401@razroo.com");
    }
  })
  );
\end{lstlisting}

\section{Example of Interceptor using Apollo Client}
Apollo Client out of the box offers out of the box middleware to intercept an
http request. Generally that this means, that there will be a series of 
apollo functions in your lib folder. 

\begin{lstlisting}
// authContext to set Authorization token for every request sent from client
const authContext = setContext(async (request, previousContext) => {
  // Getting the token from the session service
  const token = await this.session.getToken();

  // return {} if token is not set yet
  if(!token) {
    return {}
  }

  // Set Authorization headers with token
  return {
    headers: {Authorization: `Bearer ${token}`}
  }
});
\end{lstlisting}

Using apolloLink, we will combine all of the middleware we have created 
throughout the \lstinline{module.ts} file. 

\begin{lstlisting}
// creating the conditional link for http and ws requests
const link = split(({query}) => {
  const { kind, operation } = getMainDefinition(query);
  return kind === 'OperationDefinition' && operation === 'subscription';
}, ws, ApolloLink.from([authContext, error, afterwareLink, http]));
\end{lstlisting}

While in our code, for purposes of demonstration, we have only included the
chapter that includes the link. Other links are bundled together. We can 
then include this general link in our apollo \lstinline{app.module.ts} file.

\begin{lstlisting}
// creating the final Apollo client link with all the parameters
apollo.create({
  link: link,
  cache: new InMemoryCache(),
  defaultOptions: {
    query: {
      fetchPolicy: 'network-only'
    }
  }
});
\end{lstlisting}

In the above, we are including our link in the apollo create. This makes it, so
that all our middleware intercepts our apollo requests before they happen. 

\section{Folder/File Structure for HTTP Interceptors}
Regardless, if using Apollo, or HttpInterceptors within your application, it 
would make sense to create a special lib folder for apollo within common. 
Within the special folder for apollo, we can create a series of link functions
and import them within the \lstinline{app.module.ts} file.

\subsection{ Authentication }
Starting from what will be time, and time again, the most single most important
piece of your application. Authentication, if not already aware, is the process 
of identifying an individual usually based on a username and a password. There
are certain situations wherein intercepting the http request before it is made,
is required for authentication:
\begin{enumerate}
\item Add Bearer Token - The word "Bearer" in "Bearer Token" is to be 
understood as "give access to the bearer of this token". It is a token 
generated by the server on initial login, and is to be used for every protected
request within the app. 
\item Refresh Token - Refresh tokens are used when the original Bearer Token
expires, and new ones are to be issued.
\footnote{Look into this, we might want Refresh tokens removed} 
\item Redirect To Login Page - This will be needed if the Bearer Token expired,
and we need the user to go back to the login page to retrieve a brand new 
bearer token. 
\end{enumerate}

An old co-worker of mine, Sam Severance, once told a funny story, how we was
once working on the couch and his wife stopped by and asked him what he was 
working on. He told her, "Nothing too much, just authentication". She said, "oh
what's that?". He told her, "Oh, it's just something that makes programmers run
around like chickens with their heads cut off". I always thought it was a funny
one. 

I guess the point behind that story, is that authentication is such an 
important piece of an application, and end's up being used in every request. It
therefore becomes: 
\begin{enumerate}
  \item Relatively difficult to manage.
  \item High cost to application if software is mis-managed. 
\end{enumerate}

In addition, every time authentication is worked on for the first time in an 
application, it tends to be specific to framework, and technology within 
framework you are working on. 

\section{Other Interceptors to Be Aware Of}
There are many other interceptors to be aware of. It is important to note, that
Authentication is the main one. It is also possible to set up global errors 
across the site based on different errors codes.

