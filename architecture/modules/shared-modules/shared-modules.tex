\chapter{ Shared Modules }
Within an Angular architecture, we have modules. Modules very commonly use the same 
sort of imports time and time again. Coming up with a sort of shared modules 
architecture, can help in two regards: 

\begin{enumerate}
  \item Prevent circular dependency issues.
  \item Allow for easier imports within app.
\end{enumerate}

\section{Shared Modules in Practice}
So in thoery, a shared module is simple. If there are a series of modules that
need to be used time and time again, these are put into the shared module. However,
in practice, it can be a very confusing thing, because what goes into a shared module. 
In addition, it can be concerning, because one has to know how it might affect 
performance. 

\subsection{Performance Impact of Unused Modules}
As we mentioned earlier, the one potential issue with shared modules, is that
they might affect performance. So the question is, that if we implement shared 
modules, and we include mutliple modules, that are not actually used by the 
component, will there be any performance issues to be concerned off. In that 
regard, let's dissect the code base.

As a test, I imported the \lstinline{MatButtonModule} into the razroo website 
page module. The following code get's added to the general bundle as a 
result: 

\begin{lstlisting}[caption=Extra code for module]
_angular_material__WEBPACK_IMPORTED_MODULE_11__["MatButtonModule"],
\end{lstlisting}

If using AOT, the above will look a bit different. However, the resulting 
code is the same. It will call 
\begin{verbatim}

\end{verbatim}

Which will then call: 
\begin{lstlisting}
  _angular_material__WEBPACK_IMPORTED_MODULE_11__  = 
  __webpack_require__(/*! @angular/material */ "../../node_modules/@angular/material/__ivy_ngcc__/esm2015/material.js");
\end{lstlisting}

\begin{lstlisting}[caption=webpack require source code]
  // bundle.js
  /******/ (function(modules) { // webpackBootstrap
  /******/         // The module cache
  /******/         var installedModules = {};
  /******/
  /******/         // The require function
  /******/         function __webpack_require__(moduleId) {
  /******/
  /******/                 // Check if module is in cache
  /******/                 if(installedModules[moduleId]) {
  /******/                         return installedModules[moduleId].exports;
  /******/                 }
  /******/                 // Create a new module (and put it into the cache)
  /******/                 var module = installedModules[moduleId] = {
  /******/                         i: moduleId,
  /******/                         l: false,
  /******/                         exports: {}
  /******/                 };
  /******/
  /******/                 // Execute the module function
  /******/                 modules[moduleId].call(module.exports, module, module.exports, __webpack_require__);
  /******/
  /******/                 // Flag the module as loaded
  /******/                 module.l = true;
  /******/
  /******/                 // Return the exports of the module
  /******/                 return module.exports;
  /******/         }
  /******/
  /******/
  /******/         // expose the modules object (__webpack_modules__)
  /******/         __webpack_require__.m = modules;
  /******/
  /******/         // expose the module cache
  /******/         __webpack_require__.c = installedModules;
  /******/
  /******/         // define getter function for harmony exports
  /******/         __webpack_require__.d = function(exports, name, getter) {
  /******/                 if(!__webpack_require__.o(exports, name)) {
  /******/                         Object.defineProperty(exports, name, {
  /******/                                 configurable: false,
  /******/                                 enumerable: true,
  /******/                                 get: getter
  /******/                         });
  /******/                 }
  /******/         };
  /******/
  /******/         // getDefaultExport function for compatibility with non-harmony modules
  /******/         __webpack_require__.n = function(module) {
  /******/                 var getter = module && module.__esModule ?
  /******/                         function getDefault() { return module['default']; } :
  /******/                         function getModuleExports() { return module; };
  /******/                 __webpack_require__.d(getter, 'a', getter);
  /******/                 return getter;
  /******/         };
  /******/
  /******/         // Object.prototype.hasOwnProperty.call
  /******/         __webpack_require__.o = function(object, property) { return Object.prototype.hasOwnProperty.call(object, property); };
  /******/
  /******/         // __webpack_public_path__
  /******/         __webpack_require__.p = "";
  /******/
  /******/         // Load entry module and return exports
  /******/         return __webpack_require__(__webpack_require__.s = 0);
  /******/ })
  /************************************************************************/
  /******/ ([
  /* 0 */
  /***/ (function(module, exports) {
  /***/ })
  /******/ ]);  
\end{lstlisting}

Webpack require takes something like less than a millisecond to perform. 
However, too many of them there in your application, and this can add a 1ms
here, or there. Shared modules can potentially cause the following issues: 
\begin{enumerate}
  \item Performance issues. Shared modules only work within the context of 
services wherein one has the ability to use \lstinline{providedIn: root}. 
However, for modules, they will be bundled regardless. So using shared 
modules within your app, can cause bloat in your application when it is 
not needed. 
  \item Maintanability. Modules can be difficult to know what exactly is 
  contained within. The only module that consistently get's used time and time
  again, is \lstinline{CommonModules}. \lstinline{CommonModules} is used 
  across all modules. However, if it's an arbitrary module, not used
  consistently across the app, it will end up being confusing for all team
  members. 
\end{enumerate}

For the above reasons Razroo feels that the general concept of Shared Modules
does not make sense. It is something to be perhaps be re-factored at a later
time if unique to your app. 
