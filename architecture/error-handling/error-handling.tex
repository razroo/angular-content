
\chapter{Error Handling}

This chapter is a bit different then the rest of the chapters in the book. Many
technologies by simply knowing about them, you are heads and shoulders above the
rest of them, simply by knowing about them. For instance, if you were to know
ngrx/store, ngrx/entities, apollo/graphql, that is more than enough to prod you
in the right direction with regards to fantatic architecture. However, one thing
that I've noticed by working in many different apps, that tends to fly under
the radar, is error handling.

\section{ No Error Handling is Not Disatrious }
Angular is a very robust ecosystem. By that, I mean that many of the proper
technologies to use within one's app already have error handling baked into it.
So, if one does not use error handling, then it is not completely disastrious.
Many of the technologies that you are using will have error handling.

\section{ The Benefits of Error Handling }
\begin{enumerate}
  \item Send User Errors to Server
  \item Allow Errors to be more specific
\end{enumerate}

\section{ Server Specific Angular Errors }
In any Angular app that is data centric, a large part of your errors are due to
server side errors. Many times the server will return the proper error. However,
the argument can be made for server side error reporting to be more explicit.

\section{ Client Side Error Reporting }
Go into depth as to how we can potentially do error reporting client side. 
