
\chapter{ Angular Router Guards }

\section{ Router Guards - A Primer }
A Router Guard is a simply a way to guard someone from going to a page if they
aren't allowed to go there(Also known as authentication).

Without going into detail, it is important to recognize that there is such a
thing as Route Guards. Most likely, you will be using them within you app
for authentication reasons. In particular, your back end will provide you a JWT
token. The following are the find fundementals of Angular Route Guards:
\begin{enumerate}
  \item CanActivate
  \item CanActivateChild
  \item CanDeactivate
  \item CanLoad
  \item Resolve
\end{enumerate}

\subsection{ CanActivate }
It is used to determine if a certain route can be activated. An example would
be as follows:
\begin{lstlisting}
export class userGuard implements CanActivate {
  constructor(
    private router: Router,
    private userService: UserService,
    private userFacade: UserFacade,
    private projectFacade: ProjectFacade
  ) {}

  canActivate(
    route: ActivatedRouteSnapshot,
    state: RouterStateSnapshot
  ): Observable<boolean> {
    const userId = route.paramMap.get('userId');

    return this.projectFacade.projectId$.pipe(
      switchMap(projectId =>
        this.userService.getuser(userId, projectId).pipe(
          tap(user => {
            if (user && user.id) {
              this.userFacade.userLoaded(user);
            }
          }),
          switchMap(user => {
            return user && user.id
              ? of(true)
              : _throw('Unable to retrieve user');
          }),
          catchError(error => {
            this.router.navigateByUrl(this.getPrimaryOutletUrl(state.url));
            return of(false);
          })
        )
      )
    );
  }

}
\end{lstlisting}
Without going into detail the above snippet is a great example of how to tap
into the power of canActivate. Here we have two things going on. One if the user
actually has an id, and is actually a legitimate user, we will pass along the
userFacade which will populate our store, thus populating the page with actual
data.

\subsection{ CanActivateChild }
CanActivateChild is very similar to canActivate only it is for the childRoute.
Will not go into detail on this one, use documentation.

\subsection{ CanDeactivate }
TODO

\subsection{ CanLoad }
TODO

\subsection{ Resolve }
TODO
