\chapter{ Lazy Loading Images }

Even though we discussed lazy loading modules, we might also want to lazy load 
the content inside of the lazily loaded modules. There is a very popular 
article from \href{https://www.thinkwithgoogle.com/marketing-resources/data-measurement/mobile-page-speed-new-industry-benchmarks/}{Google} 
that is generally spread around with regards to performance for webpages. 
In short, it presents the following very persuasive set of data: 

\begin{center}
  \begin{tabular}{@{} l *4c @}
    \toprule
    \multicolumn{1}{c}{\color{red}As page load time increases}\\
    \toprule
    \multicolumn{1}{c}{\color{red}Seconds} & Probability of Bounce \\
    \midrule
    1s to 3s       & Increases by 32\% \\
    1s to 5s       & Increases by 90\% \\
    1s to 6s       & Increases by 106\% \\
    1s to 10s       & Increases by 123\% \\
  \end{tabular}
\end{center}  

As we can see with the data presented above, performance of our webpages are 
very important. More so it presents the data very clearly that the faster 
a webpage is, the higher probability there is to retain our user base. 

\section{The Idea of Lazy Loading Images}
There are, of course many different ways to increase performance. The intent 
of this chapter, however, is just to discuss the one performance boost that 
is gained by lazy loading images. The idea of lazy loading images, similar 
to lazy loading modules in general, is to load an image only when a user 
get's to that image. 

\subsection{Side bar - User Experience and Lazy Loaded Images}
It is important to note, that we do not want to lazy load all images
that are present on our page. For instance, imagine that we were creating a 
blog for our website. On each single blog page, we have a feature image 
that shows up first for our blog. In addition, we have more images that 
show up throughout the remainder of the blog. It can be strongly argued 
that lazy loading should not be applied to the feature image. Because, it 
would potentially cause an awkward experience for the user, to load the page 
and then wait an additional second to see what the feature image looks like. 

Therefore, when creating lazy loaded images, it is important to not create a 
blanket rule that will apply to all images. Rather, those images which are not 
primary to the page, those should be lazy loaded. 