\maketitle{}
\section{ Material Overrides }

Based on Razroo best practices, we have established the easiest way to get 
your app up and running, is to use material design within your Angular 
application. Generally, an organization will want to roll it's own theme into 
Angular Material. When that happens the developer will have to go ahead and 
customize the build for material design. Let's talk about how we can go ahead 
and do that. 

\subsection{Understanding Colors in Material}
First and foremost it is important to understand something called a color 
palette. I know some of you might be aware of what it is, but personally, I was
not aware. A color palette in the digital world, refers to the full range of 
colors that can be displayed on a device. Within a material design application 
it refers to the range of colors that can be used within the application. 
Material's design language makes use of two main colors for it's color palette: 

\subsubsection{Primary and Secondary Values}
\begin{enumerate}
  \item Primary - Color displayed most frequently across your app's screens and
  components. 
  \item Secondary - "Provides more ways to accent and distinguish your product."
  \begin{enumerate}
    \item Floating Action Button(Literally buttons that float over main content)
    \item Selection controls(sliders, switches etc.)
    \item Highlighting selected text
    \item Progress bars
    \item Links and headlines
  \end{enumerate}
\end{enumerate}

\subsubsection{Material Color Maps}

It will then create a series of light and dark variants based on the primary 
and secondary values. The primary and second values must be a map of colors 
going from lightest(50) to darkest(900). 

Material has already created a series of 16 color maps, for it's design system. 
An example of material green color map, for instance, will look something like 
this: 
\begin{lstlisting}
$mat-green: (
  50: #e8f5e9,
  100: #c8e6c9,
  200: #a5d6a7,
  300: #81c784,
  400: #66bb6a,
  500: #4caf50,
  600: #43a047,
  700: #388e3c,
  800: #2e7d32,
  900: #1b5e20,
  A100: #b9f6ca,
  A200: #69f0ae,
  A400: #00e676,
  A700: #00c853,
  contrast: (
    50: $dark-primary-text,
    100: $dark-primary-text,
    200: $dark-primary-text,
    300: $dark-primary-text,
    400: $dark-primary-text,
    500: $dark-primary-text,
    600: $light-primary-text,
    700: $light-primary-text,
    800: $light-primary-text,
    900: $light-primary-text,
    A100: $dark-primary-text,
    A200: $dark-primary-text,
    A400: $dark-primary-text,
    A700: $dark-primary-text,
  )
);
\end{lstlisting}

For primary values, the regular value will start at 500. For secondary values, 
the regular value will start at 200. The significagance of these values is that 
Material Design will follow a Hierarchical system. The darker the color is, the 
more of an emphasis we are placing on that button. The lighter it is, the less 
emphasis we are placing on that element. 

\mybox{You might be wondering about two things. For starters, why it is that 
the values progress by 100's. Second what is up with the values that have an 
"A" attached to the left side. Values progress by 100's is merely a convention
used by Material Desing. Other design frameworks progress by 10's(IBM Design)
instead of 100's, for instance, or even by 1's(Open Color). It is merely a
convention used to show that values are progressing.}

\subsection{Material Design and Sass}
First and foremost, we have already established that we will be working within
a Sass environment. Material Design offers Sass out of the box, and makes it 
incredibly easy to customize your environment based on sass overrrides. 

\subsection{Npm Install Material Theme}
First and foremost, let's make sure that we have properly installed and Angular
Material in our Angular application. 
\begin{lstlisting}
npm install --save @angular/material @angular/cdk @angular/animations
\end{lstlisting}

You package.json will now include packages needed to use Angular Material 
within the application in general. In addition, the package
(\lstinline{@angular/material}) to make the Sass changes we so dearly need. 

\subsection{Import Material Design and Call Core Styles}
The next step, is for us to go ahead and import Material Design in our 
\lstinline{styles.scss} file. The \lstinline{styles.scss} file can be found
in the root Angular application \lstinline{src} folder.

\begin{lstlisting}[caption=styles.scss]
@import '~@angular/material/theming';
// always include only once per project
@include mat-core();
\end{lstlisting}

\mybox{You will notice that we are adding a tilda\lstinline{\~} next to the
node module folder, containing the sass file we need. This tell the sass that
file we would like to import is located inside of the \lstinline{node\_modules}
folder.}

What the above does is import the \lstinline{theming.scss} file that contains
all of the theming variables for material design. We are also calling the
\lstinline{mat-core()} function, which is a, \begin{quote}
\say{Mixin that renders all of the core styles that are not theme-dependent.}
\end{quote} \footnote{This quote can be found in the \lstinline{_theming.scss}
file}

\subsection{Material Light + Dark Theme}
Angular offers out of the box in the \lstinline{_theming.scss} file a light and
dark theme function. The function looks as follows: 
\begin{lstlisting}
@function mat-light-theme($primary, $accent, $warn: mat-palette($mat-red)) {
  @return (
    primary: $primary,
    accent: $accent,
    warn: $warn,
    is-dark: false,
    foreground: $mat-light-theme-foreground,
    background: $mat-light-theme-background,
  );
}  
\end{lstlisting}

It takes in two required parameters: 
\begin{enumerate}
  \item primary - Primary color
  \item accent - Accent color 
\end{enumerate}
and one optional parameter called warn, which by default will be red. Material 
will also by default specify warn as being red. 





