
\section{ Sass Error Reporting }

I have decided to include the chapter on Sass Error Reporting in the chapter
for Design Language System. The main reason for this, is that any core style is
going to be considered as part of the core Design Language System. However, it
can be very difficult to maintain a core design, without some safe guards in
place, to make sure it is consistent across the app.

\subsection{ When to use Sass Error Reporting }
One should use Sass Error reporting if it is a core style. It is a core style if
it is used in more than one page, as a foundational piece of styling, non
unique to specific component.

\subsection{ What We Are Looking For With Using Sass Functions }
\begin{enumerate}
  \item No values other than these are used
  \item When a Pr comes our way, and we say to use the above, we have a function which is self documenting.
  \item Have a UI of sorts that also trains developers on how the internal of
  the DLS works, so that they should be aware if anything is wrong.
\end{enumerate}

The following is a great example of what a core design function would look like.
\begin{lstlisting}
// Gutter variables, for padding + margin
// function to take in multiplier(8), which must emit of one of values within ill-space-amounts
@function ill-space-multiplier($n) {
  $ill-space-amounts: (0, 4, 8, 16, 24, 32, 40, 48, 56, 64);
  $ill-space-multiplier: 8;

  @if(index($ill-space-amounts, ($n * $ill-space-multiplier))) {
    @return #{$n * $ill-space-multiplier}px;
  }
  @else {
    @error "Must contain one of the following numbers: #{$ill-space-amounts}.";
  }
}
\end{lstlisting}

If we have any component that is going to go ahead and use this function, we
can simply go ahead and use it:

\begin{lstlisting}
@import 'src/styles/variables';

:host {
  padding: ill-space-multiplier(2);
}
\end{lstlisting}

In this particular situation this helps, so that if the input passed to the
multiplier is not a number, it will complain. In addition, if the result is not
one of the multipliers, it will complain as well. So for instance, if the number
passed in, is 1.5, it will cause the function to error out, being that there is
no number 12, that is one of the ill space amounts.

\subsection{ Applying Architecture to Design Language System as a whole? }
The truth is that this pattern only applicable to padding, and spacing. For
instance, we technically could create a function for breakpoints:
\begin{lstlisting}
// breakpoints to be used in conjunction with media queries across app
@function ill-breakpoint($breakpoint) {
  $breakpoints: (
    'small': 400,
    'medium': 720,
    'large': 1024,
    'extra-large': 1424
  );
  @if (map-get($breakpoints, $breakpoint)) {
    @return #{map-get($breakpoints, $breakpoint)}px;
  } @else {
    @error 'Must contain one of the following strings: #{$breakpoints}.';
  }
}
\end{lstlisting}

However, we reccomend the use of flex-layout. Flex layout currently does not
have the ability to use this pattern. Nonetheless, for padding, and spacing
alone, this singular piece of sass functionality will probably be used on a
daily basis, and is very much so worth it. 
