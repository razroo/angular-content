\chapter{ Angular Material Typography }

Typography at a design level, is a way of presenting text in an attractive 
fashion. In particular, this is in three areas: 
\begin{enumerate}
  \item Decipher difference between letters.
  \item Make blocks of elements, such as paragraphs, or headers, easy to 
  distinguish between each other. 
  \item Have text that draws you in, or speaks to you as a reader. It 
  should ideally be unique to your brand.
\end{enumerate}

Typography, at a functionality level is a paradox. A it's core, it is 
not that complex. There maybe are 10-20 different html elements to keep in
mind. However, because those 10-20 different elements get used literally 
everywhere, it makes typography the single most used item in your site. Typography,
while it by nature design, and therefore involves less functionality, it should 
still be treated with the same gravity as UI architecture. 

This book, as well as the creators (i.e. Razroo), are strong believers of 
using Material Design as part of your MVP. After that, moving onto some sort 
of other design language system as soon as your product is validated. So 
we recommend the use of Material Design with Angular, and do indeed use it
throughout the book. Understanding how to override the Material typography,
so that you yourself can do it, and let your message, and brand bleed 
through the text is important!

\section{Understanding Different Levels of Typography in Angular Material}
First and foremost, let us begin at ground zero. Let's discuss the different 
levels of Angular Material Typography. The easiest way do this, is to dissect 
the core sass function for the Angular Material typography config. Razroo has 
also added comments, to make it appropriate for the context of this book.

\begin{lstlisting}[caption=@angular/material/\_theming.scss]
// Represents a collection of typography levels.
// Defaults come from https://material.io/guidelines/style/typography.html
// Note: The spec doesn't mention letter spacing. The values here come from
// eyeballing it until it looked exactly like the spec examples.
@function mat-typography-config(
  $font-family:   'Roboto, "Helvetica Neue", sans-serif',
  // Large, one-off header, usually at the top of the page (e.g. a hero header).
  $display-4:     mat-typography-level(112px, 112px, 300, $letter-spacing: -0.05em),
  // Large, one-off header, usually at the top of the page (e.g. a hero header).
  $display-3:     mat-typography-level(56px, 56px, 400, $letter-spacing: -0.02em),
  // Large, one-off header, usually at the top of the page (e.g. a hero header).
  $display-2:     mat-typography-level(45px, 48px, 400, $letter-spacing: -0.005em),
  // Large, one-off header, usually at the top of the page (e.g. a hero header).
  $display-1:     mat-typography-level(34px, 40px, 400),
  // Section heading corresponding to the <h1> tag.
  $headline:      mat-typography-level(24px, 32px, 400),
  // Section heading corresponding to the <h2> tag.
  $title:         mat-typography-level(20px, 32px, 500),
  // Section heading corresponding to the <h3> tag.
  $subheading-2:  mat-typography-level(16px, 28px, 400),
  // Section heading corresponding to the <h4> tag.
  $subheading-1:  mat-typography-level(15px, 24px, 400),
  // Bolder body text.
  $body-2:        mat-typography-level(14px, 24px, 500),
  // Base body text.
  $body-1:        mat-typography-level(14px, 20px, 400),
  // Smaller body and hint text.
  $caption:       mat-typography-level(12px, 20px, 400),
  // Buttons and anchors.
  $button:        mat-typography-level(14px, 14px, 500),
  // Line-height must be unit-less fraction of the font-size.
  // Form input fields.
  $input:         mat-typography-level(inherit, 1.125, 400)
) {
  // ..rest of function goes here
}  
\end{lstlisting}

There are a total of 13 typography items, which we have the ability to override. 
To understand how these feed into general components throughout the site, 
let's take two a look at two examples. This is so that we can get an intuitive 
sense as to how we can create our own typography based on Angular Material 
Design. 

\subsection{Angular Material Cards}
Let's dissect the typography Sass mixin the Angular team uses for the material card
component. 
\begin{lstlisting}[caption=@angular/material/\_theming.scss]
@mixin mat-card-typography($config) {
  .mat-card {
    font-family: mat-font-family($config);
  }

  .mat-card-title {
    font: {
      size: mat-font-size($config, headline);
      weight: mat-font-weight($config, title);
    }
  }

  .mat-card-header .mat-card-title {
    font-size: mat-font-size($config, title);
  }

  .mat-card-subtitle,
  .mat-card-content {
    font-size: mat-font-size($config, body-1);
  }
}
\end{lstlisting}

\begin{enumerate}
  \item Title - Uses the config relating to \lstinline{h2} for size
  (\lstinline{$headline}), and weight of equivalent config for \lstinline{h3}.
  \item Title within Header - Uses the config equivalent for (\lstinline{h3}) 
  through out.
  \item Subtitle and Content - Uses smaller body config (\lstinline{$body-1}).
\end{enumerate}

This gives us a bit of an idea. The config, as expected, directly correlates to 
the purpose of the mat-card to give extra importance. It also brings home, that
mat-card isn't meant to directly encompass main content of the page. (However,
once again, it's ultimately you who get's decide how you want to use something.)

\section{Dynamics of Applying Angular Material Typography Globally}
It is important to note, that the Angular Material theme will not by default 
change native global HTML elements. HTML Elements such as headers
(\lstinline{<h1>, <h2>, <h3>}), list items(\lstinline{<li>}), and \lstinline{<p>} 
tags will not be styled by default. However, Angular Material Design design 
does have it's own internal typography system, that can be seen here: 
\begin{lstlisting}[caption=@angular/material/\_theming.scss]
@mixin mat-base-typography($config, $selector: '.mat-typography') {
  .mat-h1, .mat-headline, #{$selector} h1 {
    @include mat-typography-level-to-styles($config, headline);
    margin: 0 0 16px;
  }

  .mat-h2, .mat-title, #{$selector} h2 {
    @include mat-typography-level-to-styles($config, title);
    margin: 0 0 16px;
  }

  .mat-h3, .mat-subheading-2, #{$selector} h3 {
    @include mat-typography-level-to-styles($config, subheading-2);
    margin: 0 0 16px;
  }

  .mat-h4, .mat-subheading-1, #{$selector} h4 {
    @include mat-typography-level-to-styles($config, subheading-1);
    margin: 0 0 16px;
  }  
  // ...
}
\end{lstlisting}

In the above, we can see that Angular Material stays true to the Material
spec, and applies the respective style. I.e. 
\begin{enumerate}
  \item .mat-h1 - headline
  \item .mat-h2 - title
  \item .mat-h3 - subheading-2
  \item .mat-h4 - subheading-1
\end{enumerate}

This is true for all of other global typography HTML elements as well. 

\subsection{Code Example}
There is a class that one can use called \lstinline{.mat-typography}
that the Angular Material library provides. To apply it globally, you 
can apply it on the div wrapper for your content: 
\begin{lstlisting}[caption=app.component.html]
<div class="page-wrap">
  <razroo-header></razroo-header>
  <div class="content mat-typography">
    <router-outlet></router-outlet>
  </div>
  <razroo-footer></razroo-footer>
</div>
\end{lstlisting}

\section{Realizing That Material Specs Do Not Cover Everything}
It is important to realize, that the Material specs will not cover
every use case. In particular, this is exemplified by code inside of 
the \lstinline{\_theming.scss} file. 

\begin{lstlisting}
// Note: the spec doesn't have anything that would correspond to h5 and h6, but we add these for
// consistency. The font sizes come from the Chrome user agent styles which have h5 at 0.83em
// and h6 at 0.67em.
.mat-h5, #{$selector} h5 {
  @include mat-typography-font-shorthand(
      // calc is used here to support css variables
    calc(#{mat-font-size($config, body-1)} * 0.83),
    mat-font-weight($config, body-1),
    mat-line-height($config, body-1),
    mat-font-family($config, body-1)
  );

  margin: 0 0 12px;
}

.mat-h6, #{$selector} h6 {
  @include mat-typography-font-shorthand(
      // calc is used here to support css variables
    calc(#{mat-font-size($config, body-1)} * 0.67),
    mat-font-weight($config, body-1),
    mat-line-height($config, body-1),
    mat-font-family($config, body-1)
  );

  margin: 0 0 12px;
}
\end{lstlisting}

As you can see in the above example, core Angular Engineers have had to comment 
inside of their core code base, that the spec doesn't have anything that 
correlates, but they have created styling for consistencty sake. In the same vein,
it is important to realize that Material Design will not cover all use cases 
that you need. You will have to use your general sense of what Material Design 
covers in order to fill in the gaps where there is nothing mentioned specifically 
in the specs. 

\section{Mat Typography Customization}
Within Angular Material we have the ability to customize internally the 
material typography config: 

\begin{lstlisting}
@import '~@angular/material/theming';

// Define a custom typography config that overrides the font-family as well as the
// `headlines` and `body-1` levels.
$custom-typography: mat-typography-config(
  $font-family: 'Roboto, monospace',
  $headline: mat-typography-level(32px, 48px, 700),
  $body-1: mat-typography-level(16px, 24px, 500)
);
\end{lstlisting}

\mybox{It should be noted, that we have the option to override any of the 
mat-typography-config variables. To see them, navigate to the Angular Material 
theming Sass file, and search for mat-typography-config.}

Once we have created our config we can use it in one of four ways: 
\begin{enumerate}
\item Override Typography CSS classes(e.g. \lstinline{mat-h1, mat-display-1, etc.})
\begin{verbatim}
@include mat-base-typography($custom-typography);
\end{verbatim}

\item Override typography for a specific Angular Material component.
\begin{verbatim}
@include mat-checkbox-typography($custom-typography);
\end{verbatim}

\item Override typography for all Angular Material
\begin{verbatim}
@include angular-material-typography($custom-typography);
\end{verbatim}

\item Override typography in the core CSS
\begin{verbatim}
@include mat-core($custom-typography);
\end{verbatim}
\end{enumerate}

My personal favorite is to use \lstinline{mat-core}, just because it anyways  
has to be included atleast once per project. 



