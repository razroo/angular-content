\chapter{ Typography }

Typography at a design level, is a way of presenting text in an attractive 
fashion. In particular, this is at three levels: 
\begin{enumerate}
  \item Decipher different letters.
  \item Make blocks of elements, such as paragraphs, or headers, easy to 
  distinguish between each other. 
  \item Have text that draws you in, or speaks to you as a reader. 
\end{enumerate}

Typography in this regard is very unique. Because at it's core, it's 
not that complex. There maybe are 10-20 different elements to keep in
mind. However, because those 10-20 different elements get used literally 
everywhere, it makes typography the single most used item in your site. 

This book, as well as the creators (i.e. Razroo), are strong believers of 
using Material Design as part of your MVP. After that, moving onto some sort 
of other design language system as soon as your product is validated. So 
we reccomend the use of Material Design with Angular, and do indeed use it
throughout the book. Understanding how to override the typography, so that
you yourself can do it, and let your message, and brand bleed through the 
text is important. 

\section{Understanding Different Levels of Typography in Angular Material}
First and foremost, let's start at ground zero. Let's discuss the different 
levels of Angular Material Typography. The easiest way do this, is to copy 
and paste the sass function the material typography config. I have also 
added comments, to make it appropriate for the context of this book.

\begin{lstlisting}[caption=@angular/material/\_theming.scss]
// Represents a collection of typography levels.
// Defaults come from https://material.io/guidelines/style/typography.html
// Note: The spec doesn't mention letter spacing. The values here come from
// eyeballing it until it looked exactly like the spec examples.
@function mat-typography-config(
  $font-family:   'Roboto, "Helvetica Neue", sans-serif',
  // Large, one-off header, usually at the top of the page (e.g. a hero header).
  $display-4:     mat-typography-level(112px, 112px, 300, $letter-spacing: -0.05em),
  // Large, one-off header, usually at the top of the page (e.g. a hero header).
  $display-3:     mat-typography-level(56px, 56px, 400, $letter-spacing: -0.02em),
  // Large, one-off header, usually at the top of the page (e.g. a hero header).
  $display-2:     mat-typography-level(45px, 48px, 400, $letter-spacing: -0.005em),
  // Large, one-off header, usually at the top of the page (e.g. a hero header).
  $display-1:     mat-typography-level(34px, 40px, 400),
  // Section heading corresponding to the <h1> tag.
  $headline:      mat-typography-level(24px, 32px, 400),
  // Section heading corresponding to the <h2> tag.
  $title:         mat-typography-level(20px, 32px, 500),
  // Section heading corresponding to the <h3> tag.
  $subheading-2:  mat-typography-level(16px, 28px, 400),
  // Section heading corresponding to the <h4> tag.
  $subheading-1:  mat-typography-level(15px, 24px, 400),
  // Bolder body text.
  $body-2:        mat-typography-level(14px, 24px, 500),
  // Base body text.
  $body-1:        mat-typography-level(14px, 20px, 400),
  // Smaller body and hint text.
  $caption:       mat-typography-level(12px, 20px, 400),
  // Buttons and anchors.
  $button:        mat-typography-level(14px, 14px, 500),
  // Line-height must be unit-less fraction of the font-size.
  // Form input fields.
  $input:         mat-typography-level(inherit, 1.125, 400)
) {
  // ..rest of function goes here
}  
\end{lstlisting}

There are a total of 13 items, which we have the ability to override. Just an 
example, as to how these feed into general components throughout the site, and 
also in our actual app. Let's take two a look at two examples, to get an intuitive 
sense as to how we can implement Angular Material typography ourselves. 

\subsection{Angular Material Cards}
Let's dissect the Sass Mixin the Angular team uses for typography,
for the material card. 
\begin{lstlisting}[caption=@angular/material/\_theming.scss]
@mixin mat-card-typography($config) {
  .mat-card {
    font-family: mat-font-family($config);
  }

  .mat-card-title {
    font: {
      size: mat-font-size($config, headline);
      weight: mat-font-weight($config, title);
    }
  }

  .mat-card-header .mat-card-title {
    font-size: mat-font-size($config, title);
  }

  .mat-card-subtitle,
  .mat-card-content {
    font-size: mat-font-size($config, body-1);
  }
}
\end{lstlisting}

\subsubsection{Title}
\subsubsection{Subtitle and Content}
\subsubsection{Subtitle and Content}