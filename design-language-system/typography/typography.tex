\chapter{ Typography }

Typography at a design level, is a way of presenting text in an attractive 
fashion. In particular, this is at three levels: 
\begin{enumerate}
  \item Decipher different letters.
  \item Make blocks of elements, such as paragraphs, or headers, easy to 
  distinguish between each other. 
  \item Have text that draws you in, or speaks to you as a reader. 
\end{enumerate}

Typography in this regard is very unique. Because at it's core, it's 
not that complex. There maybe are 10-20 different elements to keep in
mind. However, because those 10-20 different elements get used literally 
everywhere, it makes typography the single most used item in your site. 

This book, as well as the creators (i.e. Razroo), are strong believers of 
using Material Design as part of your MVP. After that, moving onto some sort 
of other design language system as soon as your product is validated. So 
we reccomend the use of Material Design with Angular, and do indeed use it
throughout the book. Understanding how to override the typography, so that
you yourself can do it, and let your message, and brand bleed through the 
text is important. 

\section{Understanding Different Levels of Typography in Angular Material}
First and foremost, let's start at ground zero. Let's discuss the different 
levels of Angular Material Typography. The easiest way do this, is to copy 
and paste the sass function the material typography config. I have also 
added comments, to make it appropriate for the context of this book.

\begin{lstlisting}[caption=@angular/material/\_theming.scss]
// Represents a collection of typography levels.
// Defaults come from https://material.io/guidelines/style/typography.html
// Note: The spec doesn't mention letter spacing. The values here come from
// eyeballing it until it looked exactly like the spec examples.
@function mat-typography-config(
  $font-family:   'Roboto, "Helvetica Neue", sans-serif',
  // Large, one-off header, usually at the top of the page (e.g. a hero header).
  $display-4:     mat-typography-level(112px, 112px, 300, $letter-spacing: -0.05em),
  // Large, one-off header, usually at the top of the page (e.g. a hero header).
  $display-3:     mat-typography-level(56px, 56px, 400, $letter-spacing: -0.02em),
  // Large, one-off header, usually at the top of the page (e.g. a hero header).
  $display-2:     mat-typography-level(45px, 48px, 400, $letter-spacing: -0.005em),
  // Large, one-off header, usually at the top of the page (e.g. a hero header).
  $display-1:     mat-typography-level(34px, 40px, 400),
  // Section heading corresponding to the <h1> tag.
  $headline:      mat-typography-level(24px, 32px, 400),
  // Section heading corresponding to the <h2> tag.
  $title:         mat-typography-level(20px, 32px, 500),
  // Section heading corresponding to the <h3> tag.
  $subheading-2:  mat-typography-level(16px, 28px, 400),
  // Section heading corresponding to the <h4> tag.
  $subheading-1:  mat-typography-level(15px, 24px, 400),
  // Bolder body text.
  $body-2:        mat-typography-level(14px, 24px, 500),
  // Base body text.
  $body-1:        mat-typography-level(14px, 20px, 400),
  // Smaller body and hint text.
  $caption:       mat-typography-level(12px, 20px, 400),
  // Buttons and anchors.
  $button:        mat-typography-level(14px, 14px, 500),
  // Line-height must be unit-less fraction of the font-size.
  // Form input fields.
  $input:         mat-typography-level(inherit, 1.125, 400)
) {
  // ..rest of function goes here
}  
\end{lstlisting}

There are a total of 13 items, which we have the ability to override. Just an 
example, as to how these feed into general components throughout the site, and 
also in our actual app. Let's take two a look at two examples, to get an intuitive 
sense as to how we can implement Angular Material typography ourselves. 

\subsection{Angular Material Cards}
Let's dissect the Sass Mixin the Angular team uses for typography,
for the material card component. 
\begin{lstlisting}[caption=@angular/material/\_theming.scss]
@mixin mat-card-typography($config) {
  .mat-card {
    font-family: mat-font-family($config);
  }

  .mat-card-title {
    font: {
      size: mat-font-size($config, headline);
      weight: mat-font-weight($config, title);
    }
  }

  .mat-card-header .mat-card-title {
    font-size: mat-font-size($config, title);
  }

  .mat-card-subtitle,
  .mat-card-content {
    font-size: mat-font-size($config, body-1);
  }
}
\end{lstlisting}

\begin{enumerate}
  \item Title - Uses the config relating to h2 for size(\$headline), and 
  weight of equivalent for 
  \item Title within Header - Uses the config for title(h3) through and through.
  \item Subtitle and Content - Uses smaller body(\$body-1).
\end{enumerate}

This gives us a bit of an idea. The config, as expected, directly correlates to 
the purpose of the mat-card. It also brings home, that mat-card isn't meant to 
directly encompass content of a page, and rather for minor pieces of content 
here and there. 

\subsection{Angular Material Headers}
It is important to note, that the Angular Material theme will not by default 
change native elements. HTML Elements such as headers(\lstinline{<h1>, <h2>, <h3>}), 
list items(\lstinline{<li>}), and \lstinline{<p>} tagswill have not have default 
material styling. However, material design does have it's own internal typography 
system, that can be seen here: 
\begin{lstlisting}[caption=@angular/material/\_theming.scss]
@mixin mat-base-typography($config, $selector: '.mat-typography') {
  .mat-h1, .mat-headline, #{$selector} h1 {
    @include mat-typography-level-to-styles($config, headline);
    margin: 0 0 16px;
  }

  .mat-h2, .mat-title, #{$selector} h2 {
    @include mat-typography-level-to-styles($config, title);
    margin: 0 0 16px;
  }

  .mat-h3, .mat-subheading-2, #{$selector} h3 {
    @include mat-typography-level-to-styles($config, subheading-2);
    margin: 0 0 16px;
  }

  .mat-h4, .mat-subheading-1, #{$selector} h4 {
    @include mat-typography-level-to-styles($config, subheading-1);
    margin: 0 0 16px;
  }  
  // ...
}
\end{lstlisting}

In the above, we can see that Angular Material stays true to the Material
spec, and applies the respective style. I.e. 
\begin{enumerate}
  \item .mat-h1 - headline
  \item .mat-h2 - title
  \item .mat-h3 - subheading-2
  \item .mat-h4 - subheading-1
\end{enumerate}

\section{Realizing That Material Specs Do Not Cover Everything}
It is important to realize, that the Material specs will not cover
every use case. In particular, this is exemplified by code inside of 
the \lstinline{\_theming.scss} file. 

\begin{lstlisting}
// Note: the spec doesn't have anything that would correspond to h5 and h6, but we add these for
// consistency. The font sizes come from the Chrome user agent styles which have h5 at 0.83em
// and h6 at 0.67em.
.mat-h5, #{$selector} h5 {
  @include mat-typography-font-shorthand(
      // calc is used here to support css variables
    calc(#{mat-font-size($config, body-1)} * 0.83),
    mat-font-weight($config, body-1),
    mat-line-height($config, body-1),
    mat-font-family($config, body-1)
  );

  margin: 0 0 12px;
}

.mat-h6, #{$selector} h6 {
  @include mat-typography-font-shorthand(
      // calc is used here to support css variables
    calc(#{mat-font-size($config, body-1)} * 0.67),
    mat-font-weight($config, body-1),
    mat-line-height($config, body-1),
    mat-font-family($config, body-1)
  );

  margin: 0 0 12px;
}
\end{lstlisting}

As you can see in the above example, core Angular Engineers have had to comment 
inside of their core code base, that the spec doesn't have anything that 
correlates, but they have created styling for consistencty sake. In the same vein,
it is important to realize that Material Design will not cover all use cases 
that you need. You will have to use your general sense of what Material Design 
covers in order to fill in the gaps where there is nothing mentioned specifically 
in the specs. 

\section{Apply Mat Typography Globally}
As we have mentioned before, material design will not apply to global css by 
default. There is a class that one can use called \lstinline{.mat-typography}
that the Angular Material library provides. To apply it globally, one can do 
something such as the following: 
\begin{lstlisting}[caption=app.component.html]
<div class="page-wrap">
  <razroo-header></razroo-header>
  <div class="content mat-typography">
    <router-outlet></router-outlet>
  </div>
  <razroo-footer></razroo-footer>
</div>
\end{lstlisting}

It's a viable option, but it requires implementing it on a case per case basis.

\section{Mat Typography Customization}
A more 
sophisticated approach, is being able to customize internally the typography 
and apply it to your app. We have the ability to customize internally the 
material typography config, using the internal material theming: 

\begin{lstlisting}
@import '~@angular/material/theming';

// Define a custom typography config that overrides the font-family as well as the
// `headlines` and `body-1` levels.
$custom-typography: mat-typography-config(
  $font-family: 'Roboto, monospace',
  $headline: mat-typography-level(32px, 48px, 700),
  $body-1: mat-typography-level(16px, 24px, 500)
);
\end{lstlisting}

\mybox{It should be noted, that we have the option to override any of the 
mat-typography-config variables. To see them, navigate to the material 
theming sass file, and search for mat-typography-config.}

Once we have created our config we can use it in one of four ways: 
\begin{enumerate}
\item Override Typography CSS classes(e.g. \lstinline{mat-h1, mat-display-1, etc.})
\begin{lstlisting}
@include mat-base-typography($custom-typography);
\end{lstlisting}

\item Override typography for a specific Angular Material component.
\begin{lstlisting}
@include mat-checkbox-typography($custom-typography);
\end{lstlisting}

\item Override typography for all Angular Material
\begin{lstlisting}
  @include angular-material-typography($custom-typography);
\end{lstlisting}

\item Override typography in the core CSS
\begin{lstlisting}
@include mat-core($custom-typography);
\end{lstlisting} 
\end{enumerate}



