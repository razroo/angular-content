\maketitle{}
\section{ Material Design }

I was debating writing this chapter. The reason primarily being, that depending
on the size of your company, you might up end writing your own design system. I
completely understand that, and it makes sense if you are a B2C
\footnote{Business to Consumer} application, or a B2B application. If you are a
Business to Business application, then using an out of the box design system
probably makes sense. If you are building a cosumer application, I can see how
you would want your experience to be unique to that of other websites(granted
not working on an MVP).

However, I truly do not understand why a company using Angular, would not want
to use material design. It is the most robust design framework that exists
within open source. In addition, the documentation for Angular components is
next to none. I personally have been in companies where they had a business
to business applications and they decided not to use material design. It was an
absolute mess! I'll never forget the conversation we had 6 months in, wherein I
asked if it was possible for us to get design closer to the internal design
system we agreed on, so we can create the custom component! The designers
response is a classic! ``I thought the developers were doing that on their own!
''. Avoiding a scenario like this, is very difficult, and in my opinion, not
worth it for many team.

Companies choosing to use Material Design could have saved loads of resources
not having to design and implement their own components. It is out of the vast
amount of use cases that I see Material Design being valuable, that I have
decided to go ahead and write about it.

\subsection{ Material Design - Talking to UX/UI }
This section right here, is perhaps why I like Material Design the most.
Material Design has documentation for how the UX\footnote{User Experience}
should work. It also has an Angular Component Library with demos, that I can
show off to UX and show them, this is how it works by default. It addition,
theming for Material Design, is very easy.

\subsubsection{Theming your Material Design}
Putting your own company specific theme on it is generally very easy. In
addition, it can help alleviate any concerns those might have of using Angular
Material Components, due to it being possible to move over into a different
library. From professional experience, I have found the following to be the
cornerstone of what your team can expect to customize:
\begin{enumerate}
  \item Colors
  \item Font
  \item Spacing(Margin + Padding)
  \item Icons(not that this is anything particular)
  \item Buttons
\end{enumerate}

The above would be it for starters. As your designs go on, you will have
components that you will end up overriding. These will go in a partial sass
file, something that we will go into more detail as time goes on.

\subsection{ Material Design - Create your own Confluence Doc }
It is important when working with UX/UI to document discrepencies. For
inspiraiton look at the \href{https://material.io/guidelines/components/sliders.html}{material design docs}. The idea is to have a central place where UX can document the
differences they have made from the general Material DLS. Something like a
Confluence doc(if you are familiar with Atlassian), is a bit excessive. I have
found that it's too difficult for developers who spend the majority of their
time in code tools to document on confluence. In addition, for designers to
spend their time outside of the design tools(e.g. Sketch and Invision).

From a matter of ownership, engineering has a stronger discipline of
documentation and organization, due to code being very abstract at times.
Engineers should look to take ownership of the confluence doc. However, an
Invision doc, seems to be more efficient. Design should look to create an
Invision doc, that spans maybe 5 - 15 pages, on the DLS deviations they have
from actual Material Design.

\subsection{ Material Design - Use Invision }
It's interesting, because someone might not think of tooling as something which
is a part of engineering architecture. However, with regards to finding
discrepencies in DLS(Design Language System), Invision is integral. It will
make creating comments on particular components as something which will be fluid.

\subsection{ Material Design - Push Back }
The following will be worth alot of time for many different people within your
organization. Make sure that your component does not deviate from Material
Design. In addition, look into whether, or not it is pre-described for you to
go ahead, and create your own components. However, I can assure you designers,
product/business, and engineers will all be happy when you go with the default
components when possible. When building a product, unless it is beyond the MVP
go with what is available for you by default.

\subsection{ Material Design - Architecture Corner }
In a Material Design setting, there will be discrepencies in the design, which
we have mentioned above, two ways in order to address, and make sure that
engineering is in sync with Design.

However, how does Engineering make sure, that all engineers are adhering to the
principles layed out in the DLS. There are two methods which will help to a
great extent:
\begin{enumerate}
  \item Sass functions, with error reporting.
  \item Automating UI layer. I.e. overrides for material components across the
  app.
\end{enumerate}
