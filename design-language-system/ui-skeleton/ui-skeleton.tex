\section{ UI Skeleton }
\maketitle{}

Within a design language system, a UI Skeleton, or Ghost Elements, as it should
probably be called, is a good idea. Most commonly, it is gray box depiction of
UI component awaiting action, that will be available in the future.

\subsection{ One True Way of Implementing Ghost Views }
With regards to implementing Ghost elements, there is one battle tested true
way of implementing them. That would be creating inline ghost elements with
async loads. This will then use a data wrapper to allow data items to track
state. The state will then then be reflected in the view components.

\mybox{There are
another two options:
\begin{enumerate}
  \item Creating an Overlay Ghost Element
  \item Creating an Inline Ghost Element
\end{enumerate}

The only qualm with creating an overlay ghost element, is that it needs a
direct representation of what the entire component will look like. However,
many times, a larger piece of a component, is comprised of many smaller pieces.
At times the process seems synthetic, and odds are, one will frequently have to
change the ghost element per the look and feel of the actual updated component.

With regards to creating an inline component, it doesn't base it's self on
loading data etc.
}

\subsection{Use Case for not using Ghost Element}
A common scenario, where one might think of using a ghost element, and that the
ghost element is not a good solution, is where we are involving pure text, that
is already available to us. In particular, with regards to file uploads, a
ghost element will not be ideal.
