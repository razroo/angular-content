\maketitle{}
\section{ Design Language System }

Creating a design language system is imperative to any architecture. Within
Angular the Full Gamut, we are going to assume that you are using Material
Design as your DLS. Reason as we discussed before, is that it is the most robust
library for creating Angular components. However, there are particulars of
Angular that one is going to want to modify. This is where having a light
design language system coming in can be very important.

\subsection{ Identifying Key Points of DLS }
The following are the 10 points that are a part of DLS:

\begin{enumerate}
  \item Colors
  \item Styles
  \item Icons
  \item Grid and Spacing
  \item Typography
  \item Buttons
  \item Form Controls
  \item Navigation
  \item Cards and Portlets
  \item Data Tables
\end{enumerate}

These are arguably the 10 parts of any material application that will be used
the most.

\subsection{ Identifying Proper Architecture }
With regards to overriding material design, is the part where architecture kicks
in. This is a very important part of the application and I will go through one
by one, the parts of the application that have similar architecture with regards
to overrides.

\subsubsection{ Colors }
Material design has the ability to be overriden in a sass file. It is important
to note, that the material theme allows for overrides using Sass Variables. So,
one would do something like this:
\begin{lstlisting}
@import 'src/styles/themes/blue-orange';
@import 'src/styles/material-overrides/material-overrides';
\end{lstlisting}

Doing something like the following:
\begin{lstlisting}
$gray-50: #fafafa;
$gray-200: #dbe1ea;
$gray-300: #e0e0e0;
$gray-400: #cccccc;
$gray-500: #bdbdbd;
$gray-600: #9b9b9b;
$gray-700: #757575;
$gray-800: #444444;
$gray-900: #212121;
\end{lstlisting}

Now the colors you have are specific to your app.

\subsubsection{ Grid and Spacing }
This one etc.
