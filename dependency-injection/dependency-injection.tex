\maketitle{}
\section{ Dependency Injection }

Dependency Injection is one of the sticking points of angular. It makes it
distinctively different than other frameworks in this regard. That being said,
it is very important to understand the nuances of injection when it comes to
architecture when it comes to unit tests, performance, and re-usability.

\subsection{ Understanding Providers }
As per the angular documentation, "Dependencies are services or objects that a
class needs to perform its function. DI is a coding pattern in which a class
asks for dependencies from external sources rather than creating them itself."
If you are reading this book, the understanding is that you have already have
an idea of what dependency injection is, and how services might work. However,
because dependencies are a large part of files that Angular will be using,
properly understanding the specifics can have a large impact on performance.

\subsection{ providedIn metaData }
The @Injectable decorator has the providedIn metaData option. It will allow you
to specificy the root, or a specific module for the service to be injected in.
To clarify, what this means, is that if root is specified as the providedIn
metaData, for instance:
\begin{verbatim}
@Injectable({
  providedIn: 'root'
})
export class UserService {}
\end{verbatim}

It will be considered as a singleton in your application, meaning that you can
use this service anywhere you want in your application, without having to
specify it as a provider.
\subsubsection{Make Unit Testing Easier}
This makes unit testing easier, as it is no longer neccesary to specify
service in in unit testing module. However, being that it is good practice
to mock services, it is still reccomended to do the following:
\begin{lstlisting}
// let's assume we have a users mock elsewhere, that we are importing in
class MockUserService{
  getUsers() {
    return users;
  }
}
beforeEach(() => TestBed.configureTestingModule({
  providers: {provider: UserService, useClass: MockUserService}
}));
\end{lstlisting}

In fact, one can argue, that the fact that it will not error out in one's unit
test, might make unit testing even more difficult. Nonethecreating-a-config.texless, performance is
where providedIn: Root is very valuable.

\subsubsection{ Increase Performance }
Without using providedIn, and adding a service to the provider of a module, the
service will be included in the final bundle size regardless of whether, or not
it is used. However, when using providedIn: 'root', if the service is never
used within the app, it is not included in the final bundle size.

\subsubsection{When to use providedIn and When Not To}
Using providedIn as a matter of architecture, I would reccomend the following:
For any services within the actual application, do not use providedIn. These
services are always going to be used, and if they are not going to be used,
then the service should be considered as dead code and removed.

Any services that are going to be used within a library, should have
providedIn: 'root' added to them. A service that is included in a module, for
instance, might not neccesarily be sued within app.

\subsection{ProvidedIn in our Data Access Architecture}
In our data-access architecture, we have a number of services that are
considered as common within our app. In particular, in two places:
\begin{enumerate}
  \item ui/common
  \item data-services
\end{enumerate}

In these two places, providedIn should be used regardless. 
