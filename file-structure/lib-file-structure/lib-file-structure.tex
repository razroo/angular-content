\maketitle{}
\section{ Lib File Structure }

When working in a monospace repo, file archticture is very important, as
different parts of an application tend to be abstracted. First let's outline
the different potential parts of an Angular application.


\begin{forest}
  [libs
    [common
      [animations
      ]
      [assets
      ]
      [core
       [auth]
       [guards]
       [pipes]
       [validators]
      ]
      [models
      ]
      [testing
      ]
      [ui
      ]
      [utils
      ]
      [styles
      ]
      [vendor
      ]
    ]
  ]
\end{forest}

\subsection{ Lib File Structure in Detail }

\subsubsection{Animations}
Animations are where any common animations might go. Things such as ripple
effects, ghost elements, etc. Animations are really a whole different science
when it comes to development, and therefore makes sense for them to have their
own folder.

\subsubsection{ Assets }
This is where commonly re-used assets, such as icons, or logos are used.

\subsubsection{ Core }
Any piece of functionality that is re-used the app is used here.

\subsubsection{ Models }
This is where interfaces used across the app are used. This will simply type
annotations across the app. In addition, when using a place to reference the
full capacity of data requests within app, this model will be a life saver.

\subsubsection{ Testing }
All data mocks that are common, can go here as well. Data tends to be re-used
many different times within different parts of apps. So ideally, all mocks
should go here.

\subsubsection{ UI }
All presententational components go here. Some examples of might go into an
example app is a header, footer, loading spinner, or charts.

\subsubsection{ Utils }
Shared services used across different apps. For instance, dates.

\subsubsection{ Styles }
Any shared styles used across app, for instance spacing, media queries etc.

\subsubsection{ Vendor }
The vendor folder is used to customized 3rd party libraries. Some libraries
customized might include Angular Materal.
