\maketitle{}
\section{ Creating a component }

For re-iteration purposes, the definition of a component is something consituting
of a larger whole. Ideally anything we can turn into a component in an Angular
environment, will help us. In addition, anything which we can re-use across the
app, is beneficial as well.

When creating components, Angular also makes use of modules. Once again, just to
re-iterate, a module is an independent unit, which is used to construct a larger
interallated construct.

Angular stays true to these two definitions. A component can only be declared by
one component. If it is used by two, or more Angular will complain, saying that it
is already used by another component. Which by definition only constitues a larger whole.
A module on the other hand, is simply an independent unit. If we ever want to use
our component with two components, we will need to include it as part of a module.

Therefore, it is reccomended as general good practice, whenever creating a
component (unless that particular component has children), to always create it
with a module. For other reasons as well, it is smart idea. We will get into
those later \footnote{If you can't wait, and want the full list now, go here to
find it}.

We have already created a component as needed for our router, but for redundancy
sake here are the steps again.

Also, because we will be using sass, let's make sure that our cli is using sass.
Open up the .angular-cli.json file, and change two areas. One:
\begin{verbatim}
  ng set defaults.styleExt scss
\end{verbatim}
This will make it, so that whenever we set up our components using the cli,
again it will be in sass. Second, change your existing styles.css file to
styles.scss.

Let's use the cli to create our first module called choose size.
\begin{verbatim}
  ng g module choose-size
  ng g component choose-size --exports
\end{verbatim}

(The file at this time is included in our app as a route. Let's remove the
default nrwl text from app, so that all we have is choose-size works.)
