\maketitle{}
\section{ Sass Unit Testing }

\subsection{ When Does Sass Unit Testing Make Sense? }
One of the concerns with any architecture, is over engineering. With regards to
unit testing Sass, to what extent should one unit test? Should it be for every
class, for every function, any core class used within a framework?

After much back and forth the answer should be, a core class, which is used
across the app. However, the actual class that is re-used should not be unit
testing, because it can be re-used in the wrong context. In addition, the
concept of combining two classes is closer to OOP than it is to functional
programming.

The ideal approach would be to unit test functional scss that is used as a
core style. The convention should be that when using a core style, such as
padding, or a breakpoint.

\subsection{ The Benefits of using Functional Sass as a Convention }
What would be the benefit of using Functional Sass as a convention? For starters
within an app, skimming through css, it all looks the same. Even within our
architecture using BEM, it can be hard to determine the difference between a
core class, a component specific style, a media query. In addition, the
importance, as well as degree of impact with regards to functional progamming.

While this book will not offer a complete paradigm, for a structured design
pattern within scss, in order to design discernable sass. We do have a way to
discern what is a core style, and what is not. This is by using functional Sass.
The convention, is that for a specific app, all core style use functional sass.
Similr to when we use sass error reporting. If a pr goes out, and it is not
function used for styling, then comment is made that, "Per convention, this is
a core style, and should be using the appropriate function".

\subsection{ Within a Design Language System, Choosing Core Functions }
With a core framework, specifically built on top of a design language system,
the following is what a functional DLS core framework would look like:
\begin{enumerate}
  \item Buttons
  \item Chips
  \item Colors
  \item Data Table
  \item Font
  \item Grid, Padding, and Border
  \item Icons
  \item Input, Date Picker, Checks, Toggle, Radio \& Tabs
  \item Line Style, Elevation, Element Styles
  \item Side Nav
  \item Toolbar
\end{enumerate}

\subsection{ Unit Testing Within Our Specific App }
Being that we are going to be creating sass functions for our core theming, it
would make sense to unit test them as well. If they are going to be use in 10,
or more places per each app, then we would like to make sure, that they are
indeed working in the fashion that they should be.

\subsection{ Using Sass True }
Sass True, is a set of Sass unit tests, written in Sass, so that they can mimic
the usual describe, it, assert, and expects, one can expect from a usual
unit test. The following is an example of a unit test one might make for a
typical mixin.

\begin{lstlisting}
// Test CSS output from mixins
@include it('Outputs a font size and line height based on keyword') {
  @include assert {
    @include output {
      @include font-size('large');
    }

    @include expect {
      font-size: 2rem;
      line-height: 3rem;
    }
  }
}
\end{lstlisting}

It should all be very familiar with your classic Mocha test.

\subsection{ Installing Sass True }
\begin{verbatim}
npm install sass-true --save-dev
\end{verbatim}

\subsection{ Setting up a scss.spec.ts }
We are going to set up our own jasmine sass-test runner, that will pick up on
all sass unit tests within directory. It will look like the following:
\begin{lstlisting}
const path = require('path')
const sassTrue = require('sass-true')
const glob = require('glob')

describe('Sass', () => {
  // Find all of the Sass files that end in `*.spec.scss` in any directory in this project.
  // I use path.resolve because True requires absolute paths to compile test files.
  const sassTestFiles = glob.sync(path.resolve(__dirname, '**/*.spec.scss'))

  // Run True on every file found with the describe and it methods provided
  sassTestFiles.forEach(file =>
    sassTrue.runSass({ file }, describe, it)
  )
})
\end{lstlisting}

\subsection{ Run Sass True Directly, without using CLI }
As of this time Sass True uses node-sass, which does not play well with the CLI.
Therefore, we will need to run the sass spec runner directly. As a result,
let's set up a specific npm script for running sass-true.
\begin{verbatim}
"sass-test": "jasmine libs/scss.spec.ts",
\end{verbatim}

\subsection{ Wrapping it up - Reccomended Folder Structure }
