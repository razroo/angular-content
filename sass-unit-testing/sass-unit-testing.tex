\maketitle{}
\section{ Sass Unit Testing }

\subsection{ When Does Sass Unit Testing Make Sense? }
One of the concerns with any architecture, is over engineering. With regards to
unit testing Sass, to what extent should one unit test? Should it be for every
class, for every function, any core class used within a framework?

After much back and forth the answer should be, a core class, which is used
across the app. However, the actual class that is re-used should not be unit
testing, because it can be re-used in the wrong context. In addition, the
concept of combining two classes is closer to OOP than it is to functional
programming.

The ideal approach would be to unit test functional scss that is used as a
core style. The convention should be that when using a core style, such as
padding, or a breakpoint.

\subsection{ The Benefits of using Functional Sass as a Convention }
What would be the benefit of using Functional Sass as a convention? For starters
within an app, skimming through css, it all looks the same. Even within our
architecture using BEM, it can be hard to determine the difference between a
core class, a component specific style, a media query. In addition, the
importance, as well as degree of impact with regards to functional progamming.

While this book will not offer a complete paradigm, for a structured design
pattern within scss, in order to design discernable sass. We do have a way to
discern what is a core style, and what is not. This is by using functional Sass.
The convention, is that for a specific app, all core style use functional sass.
Similr to when we use sass error reporting. If a pr goes out, and it is not
function used for styling, then comment is made that, "Per convention, this is
a core style, and should be using the appropriate function".

Being that we are going to be creating sass functions for our core theming, it
would make sense to unit test them as well. If they are going to be use in 10,
or more places per each app, then we would like to make sure, that they are
indeed working in the fashion that they should be.
