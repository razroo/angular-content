\section{ Integrating ngrx/store with Apollo Client }
In many architectures, it is most likely going to make sense that it is
microservice based. That is, your data will be served over a slew of different
apis. In which case the intent is that in a regular application, you will be
using GraphQL. It is reccomended that you use Apollo Client as well. We will
cover real quick, how to integrate GraphQL with Apollo Client.

\subsubsection{ What is GraphQL }
GraphQL is a backend data query language. It will allow you to use a query to
make a request, as opposed to having to supply, url, endpoint, and type of
request.

\subsubsection{ The Benefit of Apollo }
Apollo is a GraphQL client, used to help ease use of GraphQL http requests
within app. In particular:
\begin{enumerate}
  \item Allows the application developer to easily execute GraphQL queries, and
  configure transport-specific features like headers.
  \item Ensure that all GraphQL results currently being displayed in an app, are
   consistent with one another.
  \item Provide flexible ways to update the cache with results from the server
  when using mutations, pagination, subscriptions, and more.
\end{enumerate}

\subsubsection{ Dilemma When Using Apollo Client with @ngrx/store }
Apollo will create it’s own inMemoryCache, without dependency on Redux. However,
this creates two separate stores within the app when using @ngrx/store.
One for ngrx/store, and another for the Apollo client. It would be much easier,
obviously, if we had a singular cache/store, for the app.

\subsubsection{ Enter apollo-angular-cache-ngrx }
apollo-angular is a series of packages for the integration of the Apollo client
with Angular. apollo-angular-cache-ngrx is a package officially a part of one of
the apollo-angular packages. It solves this exact problem, and allows one to use
@ngrx/store as one’s Apollo Cache. (The following can be seen in the github
README.md for the apollo-angular-cache-ngrx repo, but going to put here for
convenience reasons).

\subsubsection{ Installation }
\begin{lstlisting}
npm install apollo-angular-cache-ngrx —-save
\end{lstlisting}

\subsubsection{ Usage }
\begin{lstlisting}
import {StoreModule} from '@ngrx/store';
import {
  NgrxCacheModule,
  NgrxCache,
  apolloReducer,
} from 'apollo-angular-cache-ngrx';

@NgModule({
  imports: [
    StoreModule.forRoot({
      apollo: apolloReducer,
    }),
    NgrxCacheModule,
  ],
})
class AppModule {
  constructor(ngrxCache: NgrxCache) {
    const cache = ngrxCache.create({});
  }
}
\end{lstlisting}

If one were to now make an Apollo GraphQL query in your Typescript component,
your current ngrx/store will be populated with the appropriate Apollo data.
You will be able to subscribe to it as usual. For instance:
\begin{lstlisting}
  constructor(store: Store<any>, private apollo: Apollo) {
    this.store = store;
      apollo
        .query({
          query: gql`
            {
              users {
                status
                id
                name
              }
            }
          `
        })
        .subscribe((initialData: any) => {
          console.log(this.initialData);
        });
  }
\end{lstlisting}

At this time however, we are not using GraphQL within the above fashion.
Nonetheless, your average app, will be using this sort of architecture, and it
is very good to be aware of it, as well as how to use it.
