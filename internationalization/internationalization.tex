\maketitle{}
\section{ Internationalization }

Internationalization is one of those things that is generally done farther down
the life cycle of an app. "We would like a data table put on every page with
specific data", she says. Only later on i nthe app will having it be
translatable to French and Mandarin, something that we really want to integrate
within the app.

\subsection{ Architecting Internationalization }
The reason why architecting internationalization can be something that is
something for us to keep in mind, is that it can have an affect on how we might
view the architecture of our application. A data table, for instance, is a great
example. We might want to propose a rigid re-usable data-table. However, once
we introduce internationalization, having data tables that are unique per each
language will be important. Having a re-usable data table that takes care of
information through passing in props can be cumbsersome.

\subsection{ Internationalization in Angular - A Primer }
