
\section{ Code Reviews }
Code reviews are a very intuitive process. It can potentially be looked at as
something that I would do. If the pull request isn't looking at the code the
same way I would, then I should comment. However, it's the part of commenting
and accepting criticism, that makes this is entire process very tricky.

\subsection{ Code Reviews - The Golden Rule }
There are multiple ways of doing something. If the code reviewer leaves a
comment for doing something in an alternate way, and the person recieving the
code resists, then the code reviewer has no right to insist on her/his way. It
is then important, however, from that point onwards, that the team agrees on a
convention.

\mybox{\subsection{Story Time}
  The following is a great example of how this golden rule can manifest in real
  life. Once upon a time, my team was working on a component, wherein every tab
  was to be in uppercase. The person submitting the code felt that explicitly
  typing out every word in uppercase: NAME, STATUS, TIME; Made more sense. I
  expressed that adding a css class with text-transform: uppercase;
  would make more sense. The pr submitter expressed that they felt explicitly
  typing out everything made more sense. I mentioned that ok, I can see your
  point of view, and I will remove my comment.

  The truth is that if this was a B2C\footnote{Business to Consumer}
  application, then I would have been adament about my approach. However, this
  was a B2B\footnote{Business to Business} application, and allowing this person
  to code the way they feel comfortable and be happy, because life really is too
  short, is ultimately what is important.
}

\subsection{Setting Conventions}
Another important part of the code review process is to set conventions. In my
humble opinion, the reason conflict happens with regards to code, is due to
uncertainty. When there are conventions set up before work on code happens it is
to point to code guidelines and say this is what we do. In addition, somone has
the ability to challenge code guidelines, and it is more challenging code
guidelines. This allows those involved in discussion to save face.

\subsection{ A Time to Learn }
It is also very important for others to learn. It is a way for me as the code
reviewer to ask what a certain piece of code does. A good convention is that if
it is something new that you haven't done before, then you can ask about it
and learn about it. This is one of those points that is obvious yet it is passed
on more than most.

\subsection{ A Time to Mentor }
It is also a time to setup \codeowners{} across the app, and specifically make
junior developers code owners. Thereby mentoring them and bringing them up to
speed on what needs to be done. 
