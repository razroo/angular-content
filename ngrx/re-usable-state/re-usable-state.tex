\maketitle{}
\section{ Re-Usable State }

Creating re-usable state in an Angular application might be one of the most
singular important architectural decisions you might also make. In addition, it
will probably be the longest lasting architectural decision you might make, as
redux is something which is pretty agnostic across many different frameworks.

\subsection{Why Create Re-usable State?}
If there is a singular component that is going to be used across a different
page, having re-usable state, will greatly simplify the architecture. The
logic for reducers can be created once. That logic can then be re-used numerous
times within your app. However, I would like to express how having re-usable
state in @ngrx/store is an anti-pattern.

\subsection{Re-usable State - An Anti-Pattern}
Let's imagine that you have a re-usable data-table, that you would like to use
on numerous pages. There are certain pieces of logic that you want to use with
your state. For instance, you want to create a reducer to determine which
rows have been selected, and if all have been selected. If all has been
selected, then it is moved over to the selected key/value. This logic you have
decided should be mover over to state, so that it can re-used within the data
table, so that it can be passed around and re-used within the app time and time
again.

The only issue with re-usable state, is that as soon as you are using re-usable
state, you are recognizing that the component has to be re-usable. As soon as
you are saying the component is going to be re-usable, you are recongnizing
that there is need for there to be a dumb component and a smart component. As
soon as you are saying there is going to be a dumb component, then any logic
relating to the interface should remain within the components logic it's self.
Therefore, the only state that you will be needing, is the data that is loaded.
That part of state is simplified to a great extent, to where it makes sense to
have state unique per each page. 
