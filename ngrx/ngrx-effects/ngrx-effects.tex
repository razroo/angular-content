\maketitle{}
\section{ Ngrx Effects }

\subsection{Ngrx Effects - A Primer}
Ngrx effects can be one of the more ambigious parts of the ngrx stack. They are
by definitition something that is supposed to happen when something else has
happened. It will listen for a particular action, and true to Ngrx, return an
observable. In this observable one will have the option to do whatever they want
as well as publish(return) to the action stream.

The line, however, can be blurred, however, as to what the difference is between
an ngrx/effect and a ngrx/store. It is therefore important to distinguish for
architectural reasons. In addition, it can be difficult to determine the
different use cases wherein someone would use an effect. It can indeed be a
slippery slope wherein when to use an effect.

\subsubsection{ Code Example }
\begin{lstlisting}
@Effect({ dispatch: false })
 userDeleted$ = this.dataPersistence.fetch(
   UserActivitiesTypes.UserDeleted,
   {
     run: (action: UserDeleted, state: UserStateModelState) => {
       this.snackBar.open('User Deleted', 'Ok', {
         duration: 2000,
         verticalPosition: 'top',
       });

       return null;
     },

     onError: (action: ActivityDeleted, error) => {
       console.error('Error', error);
     },
   }
 );
\end{lstlisting}

This code example, is a great example as to when someone might use an effect.
\subsection{Difference between @ngrx/store and @ngrx/effect}
I rarely quote articles within this book, but there is one article by a one
Tanya Gray entitled, "Understanding NgRx Effects and the Action Stream". It is
perhaps my number one article of all times, for it's ability to take an
otherwise very complicated subject, and making it very easy for anyone to
digest. I will be using her manner of describing effects in this chapter, and
credit is owed where due.

The article is very useful in determining the different use cases for creating
an effect. How why to use an effect is something which is not nex
