\maketitle{}
\section{ Ngrx Entity }

The Ngrx repo until recently had many similar functionalities to your regular
redux app. It included actions, reducers, selectors. However, there has been
efforts to go ahead and create libraries for aspects of ngrx that can perhaps
be re-usable. One of these is ngrx entities.

\subsection{ Ngrx Entity at a High Level }
At it's core, ngrx entity is an API for manipulating and querying entity
collections. In particular:
\begin{enumerate}
  \item Reduce boilerplate for creating reducers that manage a collection of
  models.
  \item Providing performant CRUD operations for managing entity collections.
  \item Extensible type-safe adapters for selecting entity information.
\end{enumerate}

This architecture works really well when creating data as a single source of
truth. For instance, let's say in your application, you have a data table on
every page that pulls in data. Throughout every page, you have a way of
manipulating this data. Using ngrx/entity will allow for this architecture to
be fluid, and have all manipulation of data be within a singular area.

\subsection{ Example of Ngrx Entity }
Within our app we the ability to illustrate a pixelated charachter using pixels.
Every time that a pixel within the grid is selected, we are going to add it to
our store. This store is going to be used to display the code version of the
app. In addition, we are going to have to remove the pixel when clicked on
within our store. In addition, if we have selected a new color, and we select a
new pixel with that color, that pixel should be updated with the proper color.
What we have just described is a perfect CRUD app.

\subsection{ Installing ngrx/enity }
\begin{verbatim}
  npm install @ngrx/entity --save
\end{verbatim}
