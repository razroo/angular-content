\chapter{ Angular Universal }

\section{What is Server Side Rendering?}
\begin{quote}
  \say{Angular Universal is Angular's way of rendering something server side.}
\end{quote}

By default, Angular will render HTML on the page using Javascript within the browser. Server side rendering, however, will produce all of the HTML off the browser and within the server. Normally, people come across using Angular Universal for two reasons: 
\begin{enumerate}
  \item Mobile app - Javscript engines on mobile phones, while much stronger, are still lacking. In addition, while having the server generate the HTML instead of the client, can potentially boost battery performance of application.
  \item SEO reasons - Angular Universal will make your site static. 
\end{enumerate}

Another added benefit of Angular Universal, is that it has a very fast FCP(first contentful paint). When you use Angular Universal for the first time, this will immediately be apparent. It should be noted however, the TTI(Time to interactive), wherein user will be able to interact with the page, will still be loading in the background. 

\section{Angular Universal Actually Requires A Sever}
One thing that people using Angular Universal will be surprised, when using it for the first time, is that it actually requires a server! Looking back it will make sense, because obviously it is server side rendering. However, the name "Angular Universal" doesn't exactly lend to that assumption. Anywho, I digress, any web server will work with Angular Universal. 

\section{A Couple of Points to Keep in Mind}

\subsection{How Angular Universal Works}
\begin{enumerate}
  \item Angular Universal uses the \lstinline{platform-server} package under the hood. This provides low level features that don't rely on a browser(\lstinline{XMLHttpRequest}, etc.)
  \item Server passes the client request (for application pages) to the \lstinline{ngExpressEngine} which calls the Universal \lstinline{renderModuleFactory}. This function inputs: 
  \begin{enumerate}
    \item Template HTML page.
    \item Angular \lstinline{module} containing components.
    \item \lstinline{route} determining which components to display. 
  \end{enumerate}
  \item The \lstinline{renderModuleFactory()} function, renders view within the \lstinline{<app>} tag of template. 
  \item The server will then return rendered page to the client. 
\end{enumerate}

\subsection{Dynamic of Browser APIs with Angular Universal}
A universal app works on the server side. Because of that, it needs to create abstractions over classic APis such as \lstinline{window}, or \lstinline{location}. There some bugs that might show as a result, if your app does anything fancy. 

Another really important point, is that a Universal app cannot interact with a mouse, or keyboard event. Without some hack/workaround, it is important to make your entire app routable. 

\subsection{Using Absolute URLs - Serving on Browser vs. Serving on Server }
There is an interesting dynamic between serving on the browser, and serving on the server. When serving on the browser, the paths for a url are relative. When serving on a server, the paths must be absolute. So why is it that this is indeed the dynamic? 

Well, think of it this way. When generation is happening within the browser, it is fully aware of everything going on from the client side of things. The actual url is irrelevant. There could potentially be a way of approaching this problem. However, after looking into the github issues, for the angular github, one will find that this is actually a surprisingly complicated issue. Given the way the package works out. 

From a server side, it does not have access to the client side, so using relative paths is not an option. You will need to create an interceptor to pass a full url to your server, based on your client data. 

\subsection{Test}