\chapter{Bundle Size}

Bundle sizes ironically, are one of those things that tend to get overlooked. 
It's ironic, because it can make such a large difference to a user, using the
app for the first time, while simaltaneously being one of the easier things to 
tackle. I think for that reason it tends to not get the attention it needs,
because it doesn't exactly fit into the box of computer science.

\section{Being Aware of Bundle Size}
When you run \lstinline{ng build --prod}, Angular will generate 4 files: 
\begin{enumerate}
  \item runtime.*.js
  \item main.*.js
  \item polyfills.*.js
  \item styles.*.css
\end{enumerate}

Of the above, the largest will be main.js files. If we dissect our bundle size, 
it's important to keep in mind that this bundle size will be contributed to by 
a number of different things. This chapter won't go into how to decrease the 
bundle size before it is bundled, but rather after it has been bundles. 

\section{Gzip}
gzipping for those that are not familiar, is the process of taking a chunk of 
data and making it smaller. The original data can be restored by un-zipping the
compressed file. Within the context of HTTP protocol, it has the ability to
unzip a file. There is a bit of a cost from the side of the browser to unzip a 
file. However, generally the benefit of a lower bundle size, outweighs the fact
that the browser will have to unzip the file. All that is needed, is for your 
devops person, to set the gzip setting on your server.

As a general rule of thumb, gzipped files are about 20\% the size of the 
original file, which of course, will drastically decrease the initial load time
of your app. If you would like to check and see whether, or not your files are 
gzipeed, you can simply open up your console, and check the "Content-Encoding" 
under the Response Headers. If it says "Content-Encoding: "gzip", then you are 
in good hands, otherwise, you might be in trouble.

\subsection{ How to Gzip }
Gzip'ing is something that should be controlled by your CDN, which will also 
deal with a slew of other things to make sure your files are served. However, 
your backend team will be able to deal with this one relatively quick, if they 
are not using a CDN system, due to many packages out of the box dealing with 
this. 

\section{Analyze Your Angular Bundle}
Webpack has a built in tool that one can use to analyze build. For starters, 
it has an incredible visualization of what your entire build looks like.
\footnote{Insert photo here on snapshot.} In addition, it will tell you about 
things such as: 
\begin{enumerate}
  \item You forgot to remove some packages that you aren't using anymore.
  \item Some packages are larger than expected, and can be replaced with
  another.
  \item You have improperly imported libraries.
\end{enumerate}

In order to get the above data: 
\begin{enumerate}
\item 
\begin{verbatim}
npm install -g webpack-bundle-analyzer 
\end{verbatim}
Then, in you Angular app, run: 
\item 
\begin{verbatim}
ng build --stats-json
\end{verbatim}
We are going to be running stats on the original non minified build, as we 
want to make sure we do not muddy the results coming back from out stats. 
\item 
Finally we run: 
\begin{verbatim}
webpack-bundle-analyzer path/to/your/stats.json  
\end{verbatim}
and navigate to \lstinline{localhost:8888} wherein we will see our results.
\end{enumerate}

\section{Monitoring Bundle Size}
In Angular 7, and later, angular adds a config in your \lstinline{angular.json}
file. It has a property called "budgets", and it looks like this: 
\begin{lstlisting}
"budgets": [
  {
    "type": "initial",
    "maximumWarning": "2mb",
    "maximumError": "5mb"
  }
]   
\end{lstlisting}

This will throw an error if bundle size exceeds 2mb, and throw an error if 
bundle size exceeds 5mb. It is very much so possible to use this feature in
your CI/CD pipeline. Feel free to go over to your devops person, and ask them 
to integrate it. 
