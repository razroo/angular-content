\chapter{ Track By }
In Angular, there are performance enhancements that are valuable here and there.
It can be easy to miss them, because they are performance ehancements. So they are not integral towards the core documentation. However, for anyone building an application, these are things that one should be aware of. 

In particular, when it comes to for loops for Angular, the way that change detection works, is that if any part of the array is changed, Angular will agressively change the entire DOM. This is of course can lead to performance issues, if all we need is a particular piece of data to be changed. The Angular framework has an internal \lstinline{trackBy} function to combat these performance leaks. 

\section{ Using Track By in Angular }
Using \lstinline{*ngFors} in Angular is actually quite common. I cannot think of one application that doesn't use a data table of some sort. Suffice to say,an ngFor that parses through alot of data, as well as optimizing it for
performance reasons, is very important. Even it's an application that has pagination through the backend, and we only allow a maximum of 50 rows at a time.

\subsection{ So What is trackBy?}
What trackBy does is check the set as a whole for order changes (sorting,
insertion, deletion). When the order changes, instead of removing all elements
from the DOM and creating new ones, the trackBy function is used to identify
which elements do not need to be removed from the DOM. This reduces the number
of DOM calls, and also reduces the number of angular digest cycles.

Under the hood \lstinline{trackBy} tracks an order change. It does this by taking in index of each item, and a unique identifier of that item. If item does not exists, from prior id's and indexes, then it knows to insert that one particular item into the DOM. 

\subsection{ Track By Within Data Tables }
Specifically, within our architecture, we call for using Material. By default,
if a trackBy function is not given, Material Table will deeply compare the
elements in the set. So, a trackBy function is really used to reduce the amount
of checks necessary to compare elements in a set. Instead of a deep copy, you
can check for a single unique property.

\section{ Track By in Practice }
\begin{lstlisting}[caption=data-table.component.html]
  <mat-table
  [trackBy]="trackByBuyerId">
  <mat-table/>
\end{lstlisting}

\begin{lstlisting}[caption=data-table.component.ts]
//.. this code is inside of our component class  
trackByBuyerId(index, item) {    
  return item.id; // unique id corresponding to the item
}
\end{lstlisting}

By doing the above, track by will in a material setting make sure that
objects in the data table are not deeply compared, comparing objects directly to
each other. 

\mybox{This, of course, is equally as effective in a non Angular Material setting. However, I just wanted to bring up an enterprise setting, and this is most definitely one.}
