\section{ Track By }
\maketitle{}

In Angular, there are performance enhancements that are valuable here and there.
That being said, it can be hard to miss them. This is one of the few chapters
written with regards to performance. In particular when it comes to for loops
for angular. The way that change detection works, is that if any part of the
array is changed, Angular will agressively change the entire DOM. This is of
course can lead to performance issues.

\subsection{ Using Track By in Angular }
Using *ngFors in Angular is actually quite common. I'm sure many of us listen
to music. I cannot think of one application that doesn't use a data table of
some sort when it comes to displaying the music you are listening to. Suffice
to say, an ngFor that parses through alot of data, as well as optimizing it for
performance reasons, is very important.

\subsubsection{ What Track By Actually Does }

What trackBy does is check the set as a whole for order changes (sorting,
insertion, deletion). When the order changes, instead of removing all elements
from the DOM and creating new ones, the trackBy function is used to identify
which elements do not need to be removed from the DOM. This reduces the number
of DOM calls, and also reduces the number of angular digest cycles.

\mybox{
It is important to distinguish and clarify, once again, TrackBy is not about
tracking changes to a particular element in a set. It is about tracking an order
change.
}

\subsubsection{ Track By Within Data Tables }
Specifically, within our architecture, we call for using Material. By default,
if a trackBy function is not given, Material Table will deeply compare the
elements in the set. So, a trackBy function is really used to reduce the amount
of checks necessary to compare elements in a set. Instead of a deep copy, you
can check for a single unique property.
