\chapter{ Image Performance in Angular }

Image optimization is a concept which is ubiquitous to all of web development.
Nonetheless, it is a bit unique of how to work it into an Angular setting. It 
is also worth noting, that in many applications, they are more data heavy, and 
less image heavy. Having this conversation might only be neccesary when you 
are finally ready to move over to prod. However, this is something that can be 
done ahead of time, granted you are: 
\begin{enumerate}
  \item Aware of it. 
  \item Implementing it is made easy to implement in less than 3 steps
\end{enumerate}

That is what we are going to attempt to do. 

\section{Lazy Loading Images}
Lazy loading, if not already familiar, is the idea of loading an image only
when neccesary. It is powerful primarily, because, non-critical images are 
loaded when needed, allowing for a quicker load of content on page load. It 
is also useful for the additional following reasons: 
\begin{enumerate}
  \item It reduces data consumption, i.e. less work for the server side.
  \item Less workload for the browser. 
  \item Improve webpage loading time.
\end{enumerate}

great way to see what images might need optimization, is to use Lighthouse 
in Chrome DevTools to see what might need changing. 

\section{ Using lazysizes for Loading Images }
\lstinline{lazysizes}, in my opinion is the most robust library for making
use of lazy loading libraries. Some of the benefits of using it as a library 
include: 
\begin{enumerate}
  \item Requires no configuration.
  \item Fantastic Performance.
  \item Optional integration with \href{https://developer.mozilla.org/en-US/docs/Web/API/Intersection_Observer_API}{Intersection Observer}. 
  \footnote{It's a relatively modern api that provides a way to asynchronously observe 
  changes in the intersection of a target element.
  \item Supports Plugins
\end{enumerate}




