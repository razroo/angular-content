\chapter{ Ahead Of Time Compilation }
Angular as a framework uses Typescript. In addition, it uses directives, components, 
\lstinline{@Input()}'s and \lstinline{@Output}'s. So, when one thinks about it, you 
start to realize, "hey! Is Angular compiling it to browser ready HTML and Javascript 
for me"? The answer to this question, is yes. How Angular does that, cannot be entirely 
controlled, as that is handled by the internals of the framework. However, the 
real question that we should be asking ourselves, is when does this happen. 

\section{Exploring Ahead of Time Compilation}
Without using the ahead-of-time compilation flag \lstinline{--aot} i.e. 
\begin{verbatim}
ng serve
\end{verbatim}

Angular will convert the code from Angular code, to browser readable 
code at run time. By using the \lstinline{--aot} compiler i.e.
\begin{verbatim}
ng serve --aot
\end{verbatim}

The Angular compiler will compile the code at build time. The benefits 
of compiling the code at build time are as follows: 
\begin{enumerate}
\item Faster rendering - Browser doesn't have to compile code first
\item Fewer Requests - The compiler will inline external html and css.
\item Smaller bundle sizes - Angular Compiler not included in bundle size. 
\item Template errors - They happen inside of compiler, instead of at runtime.
\item Better security - HTML and Components(css included), are compiled into 
Javascript before the webpage loads. Preventing against many different types
of injections. 
\end{enumerate}

The internals are somewhat complex, and the Angular documentation does a really
good job at discussing what that is. From a practical perspective, let's discuss 
the expression limitations. I will admit, at this time, why there are these 
limitations is beyond me. 

\section{Expression Syntax Limitations}
The AOT collector only understands a subset of Javascript. It's quite a long 
list. Therefore, when using the \lstinline{--aot} flag, and the AOT compiler 
comes across something it doesn't understand, it will throw the error into 
the \lstinline{.metadata.json}. Later on, if it needs that piece of code 
to generate the application code, the compiler will complain. However, I have 
personally found that there are scenarios wherein the AOT compiler will say
that things are working as expected, but it won't actually work. 

The \lstinline{ng serve} will go through as 
usual, but it will fail silently in the form of it not appearing on the site. 
The reason that the AOT compiler works this way, is 

The easiest way to be aware of the limitations is to change the config 
in your \lstinline{angularCompilerOptions}. 

\begin{lstlisting}
"angularCompilerOptions": {
  ...
  "strictMetadataEmit" : true
}  
\end{lstlisting}

By changing the \lstinline{strictMetadataEmit} to \lstinline{true}, it will
emit an error to the console directly, when using the \lstinline{--aot} flag.

