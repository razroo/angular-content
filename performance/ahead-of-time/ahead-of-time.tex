\chapter{ Ahead Of Time Compilation }
Angular as a framework uses Typescript. In addition, it uses directives, components, 
\lstinline{@Input()}'s and \lstinline{@Output}'s. So, when one thinks about it, you 
start to realize, "hey! Is Angular compiling it to browser ready HTML and Javascript 
for me"? The answer to this question, is yes. How Angular does that, cannot be entirely 
controlled, as that is handled by the internals of the framework. However, the 
real question that we should be asking ourselves, is when does this happen. 

\section{Exploring Ahead of Time Compilation}
Without using the ahead-of-time compilation flag \lstinline{--aot} i.e. 
\begin{verbatim}
ng serve
\end{verbatim}

Angular will convert the code from Angular code, to browser readable 
code at run time. By using the \lstinline{--aot} compiler i.e.
\begin{verbatim}
ng serve --aot
\end{verbatim}

The Angular compiler will compile the code at build time. The benefits 
of compiling the code at build time are as follows: 
\begin{enumerate}
\item Faster rendering - Browser doesn't have to compile code first
\item Fewer Requests - The compiler will inline external html and css.
\item Smaller bundle sizes - Angular Compiler not included in bundle size. 
\item Template errors - They happen inside of compiler, instead of at runtime.
\item Better security - HTML and Components(css included), are compiled into 
Javascript before the webpage loads. Preventing against many different types
of injections. 
\end{enumerate}

The internals are somewhat complex, and the Angular documentation does a really
good job at discussing what that is. From a practical perspective, let's discuss 
the expression limitations. I will admit, at this time, why there are these 
limiations is beyond me. 

\seciton{Expression Syntax Limitations }

