
\section{ Unit Testing - Mocking Providers }
Another important part of Angular Architecture, is that one has an option to
import a service as is, or to mock it out. This will apply, as per our
architecture to facades as well. Not doing so, will leave your unit tests at
the mercy of your services, and might have you unit testing your service as
well as your specific component. In addition, and perhaps, more dangerous,
that if you are not careful, you might end having your data being retrieved
from the back end while you are unit testing.

\subsection{ Re-iterating Previous Point }
We have already discussed a previous point with regards to unit testing and
interfaces. The point is that we can use a singular data mock to keep all of
unit tests in sync. This is indeed a very important point that works in tandem
with mocking providers, and we will touch on this in this chapter.

\subsection{ When to Mock Providers within App }
The point of mocking service dependencies is in order to test the component in
an isolated environment. This should includes pipes, services, and in our
architecture especially facades. If it is a module, which is completely focuses
UI, then there is no need to worry about an isolated environment, because there
is no logic wherein to afffect the component.

\subsection{ Mocking Providers - Setting the Landscape }
Just setting the landscape for what an example situation might be like with
regards to mocking providers. In our Angular - The Full Gamut architecture, we
have a facade which is always going to be responsible for bridging the data
retrieved by our service, with our component. It is going to be service that
will ultimately be used in our component to retrieve data.
\begin{lstlisting}
gridForm$: Observable<GridForm[]> = this.store.pipe(select(getGridForm));
\end{lstlisting}

The above is an example snippet of our getGridForm.facade.ts file, that will
be responsible for pulling data from our store, using the getGridForm selector
already specified else where. If we were to pull in this facade as is, it would
end up actually pulling data from the server while doing unit tests! That would
be a cardinal sin. I
\begin{verbatim}
<form>
<div>{{ (gridForm | async).row }}</div>
<div>{{ (gridForm | async).column }}</div>
<div>{{ (gridForm | async).size }}</div>
</form
\end{verbatim}

\subsection{ Mocking Providers Within Unit Test - A Primer }
In our unit test in order to mock the above Observable gridForm\$ stream, we
can very simply use the data mock we have specified in our mocks.ts file.
\begin{lstlisting}
providers: [
  {
    provide: GridFormFacade,
    useValue: {
      gridForm$: of(generateMockGridForm()),
    },
  },
],
\end{lstlisting}

In this very simple scenario, we have just made it so that the data returned
within our component for our unit test, is using the central mock being used
elsewhere in our data-access architecture. This gives us more control over what
we want to do. We can make the data more complex, to test different use cases
per it block that we want to use.

\mybox{
\subsection{A Final Note}
The architecture with providers, given the already written chapter on interfaces
and mocking data, is really just a skip and a jump away. The reason I decided
to write this chapter, is because while simple in practice, it tends to be
overlooked. This chapter properly informs a software engineer when that
situation might be appropriate and the proper way to do so.
}
