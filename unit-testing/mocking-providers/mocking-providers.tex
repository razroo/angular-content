\maketitle{}
\section{ Unit Testing - Mocking Providers }
Another important part of Angular Architecture, is that one has an option to
import a service as is, or to mock it out. This will apply, as per our
architecture to facades as well. Not doing so, will leave your unit tests at
the mercy of your services, and might have you unit testing your service as
well as your specific component.

\subsection{ Re-iterating Previous Point }
We have already discussed a previous point with regards to unit testing and
interfaces with regards to unit testing. This is indeed a very important point
that works in tandem with mocking providers, and we will touch on this in this
chapter.

\subsection{ Mocking Providers - Setting the Landscape }
Just setting the landscape for what an example situation might be like with
regards to mocking providers. In our Angular - The Full Gamut architecture, we
have a facade which is always going to be responsible for bridging the data
retrieved by our service, with our component:
\begin{lstlisting}
gridForm$: Observable<GridForm[]> = this.store.pipe(select(getGridForm));
\end{lstlisting}

The above ^ is an example snippet of our getGridForm.facade.ts file, that will
be responsible for pulling data from our store, using the getGridForm selector
already specified else where. If we were to pull in this facade as is, it would
end up actually pulling data from the server while doing unit tests! That would
be a cardinal sin.

\subsection{ Mocking Providers Within Unit Test - A Primer }
In out unit test in order to mock the above Observable gridForm\$ stream, we
would do the following:
\begin{lstlisting}
generateMockGridForm(): GridForm {
  return {
    row: '20',
    column: '20',
    size: '20',
  }
}

providers: [
  {
    provide: GridFormFacade,
    useValue: {
      gridForm$: of(generateMockGridForm()),
    },
  },
],
\end{lstlisting}

In this very simple scenario, we have just made it so
