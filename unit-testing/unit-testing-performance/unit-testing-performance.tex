\maketitle{}
\section{ Unit Testing Performance }

Unit Testing isn't neccesarily one of those things that we tend to equate with
performance. There are tools that allow one to a unit test by a specific folder.
In addition, there is parallel unit testing, for the pipeline, so that unit
tests can be sped up. However, as apps get larger, so do their unit tests. From
a developer perspective, it is valuable to run all unit tests, to make sure
that when larger impacting edits are made, that none of the unit tests are
failing. Unit tests will be run when a pr is being made. In addition, they are
going to be run when deploying. So there is enough going on, to say that being
conscious of performance boost is something which is important.

\subsection{ Component and Integration Testing }
It is quite common that many enterprise apps will take advantage of integration
testing within their unit tests. However, creating the component within the unit
test is intuitively, and is the most expensive task one can do within a unit
test. Avoiding creating a component unless needed, is the ideal. For instance,
the ideal scenario when creating a component, looks something like this:
\begin{lstlisting}
describe('BannerComponent', () => {
  let component: BannerComponent;
  let fixture: ComponentFixture<BannerComponent>;

  beforeEach(async(() => {
    TestBed.configureTestingModule({
      declarations: [ BannerComponent ]
    })
    .compileComponents();
  }));

  beforeEach(() => {
    fixture = TestBed.createComponent(BannerComponent);
    component = fixture.componentInstance;
    fixture.detectChanges();
  });

  it('should create', () => {
    expect(component).toBeDefined();
  });
})
\end{lstlisting}

Here we are setting a TestBed, configuring the test module, and then compiling
the component. While, in practice, there is nothing wrong with this, but for
every run, this will re-compile our components. So, if there is a way we can
get around re-compiling our component, this would obviously go a bit quicker.
There are a couple of approaches. One of them involves using something called
ng-bullet. I am bit skeptical of using something like this, as I can see it
causing some issues amongst the build. The second approach, is to only compile
the component if you need it. It would be a good to suggest Jest, which is a
faster way of running unit tests. In addition, for current karma/jasmine
unit tests to be mindful of how you are writing unit tests.

\subsection{ Component Testing }
There are three different ways of unit testing a component:
\begin{enumerate}
  \item Isolated Unit tests
  \item Shallow Unit Tests
  \item Integrated Unit Tests
\end{enumerate}

Of the three, Isolated Unit Tests, are the most performant. It will not cause
the component to re-render itself. However, by using an isolated unit tests,
we are simply testing the logic of the component, without testing how it
interacts without actual html. This usually only works in certain scenarios.

In most scenarios, there will be a requirement of running the TestBed, which
can cause performance issues when running a large amount of tests, and can be
dealt with.

\subsection{ Running tests in Parallel }
The next item in the checklist, is to separate your tests into separate modules,
so that they can be run separately. You will then be able to run the tests in
parallel. This will allow for the tests to run quicker.

\subsection{ Karma Parallel }
Karma Parallel is an npm package that be used to run unit tests in parallel. It
would require for the karma config to be updated accordingly. The one downfall
of the karma confi. Tjose

\subsection{ Ng-Bullet }
There is a fantastic library that has been written to accomodate for some
performance boosts with regards to Angular Unit Tests. In short, there has been
discussion around increasing the performance of Angular Unit Tests for quite
some time. The way that the new Angular compiler will work, will greatly
decrease time it will take for Karma/Jasmine unit tests regardless. I would
imagine this is the reason why those working on Angular, have decided not to
comment. However, Ng-bullet is a possibility to be used for those that seriously
need the performance boost in their Angular App.

\subsection{ Style Cleanup }
The biggest improvement with regards to performance boosts comes with improving
memory leaks. The largest memory leak is caused by CSS. If you do not clear
your css in your unit tests, your karma tests will consistently append hundreds
if not thousands of <style> tages to your body. This will incredibly slow down
how fast your unit tests run. This can be alleviated by adding an afterAll to
your unit tests:
\begin{lstlisting}
export function cleanStylesFromDOM(): void {
  const head: HTMLHeadElement = document.getElementByTagName('head')[0];
  const styles: HTMLCollectionOf<HTMLStyleElement>
  | [] = head.getElementsByTageName('style');

  for (let i: number = 0; i < styles.length; i++) {
    head.removeChild(styles[i]);
  }
}
afterAll(() => {
  clearStylesFromDOM();
});
\end{lstlisting}

The above functionality will improve performance of unit tests by five fold.
