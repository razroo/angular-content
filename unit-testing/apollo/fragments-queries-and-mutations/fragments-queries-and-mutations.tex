
\chapter{ Fragments Queries and Mutations }

An important part of architecture with regards to using GraphQL, is splitting
up data requests into fragments + queries/mutations. It greatly simpifies
the architecture. The closest resemblance of how to describe using fragments
in GraphQL, is functional composition.

\mybox{
\section{ Primer on Queries and Mutations }
query = SELECT
mutation = INSERT, UPDATE, DELETE
}


\section{ Example Fragment }
\begin{verbatim}
export const GridFragment = `
  fragment gridFields on GridSize {
    rowSize
    columnSize
    pixelSize
  }
`;
\end{verbatim}

\section{ Using Fragment in Query }
Now in our query, we have the ability to include our fragment.
\begin{verbatim}
export const GridSizeQuery = gql`
  query gridSize($projectId: ID!, $id: ID!) {
    gridSize(projectId: $projectId, id: $id) {
      ...gridFields
    }
  }
  ${GridFragment}
`;
\end{verbatim}

\subsection{ Using Fragment as interface }
It also helps when creating an interface. We have the ability to keep a one to
one relationship between our fragments and interfaces as well. If we want to
make sure that our interfaces are up to date for a specific data type, we can
just look at our fragments file, and make sure that they are both up to date
with each other. For instance:
\begin{lstlisting}
export interface GridSize {
  rowSize: string;
  columnSize: string;
  pixelSize: string;
}
\end{lstlisting}

\section{ Using Fragment in Mutation }
We can now use the same fragment that we used for the GridSizeQuery, for the
AddGridSizeMutation.

\begin{lstlisting}
export const AddGridSizeMutation = gql`
  mutation addGridSize($projectId: ID!, $gridSize: GridSizeInput!) {
    addGridSize(projectId: $projectId, gridSize: $gridSize) {
      ...gridFields
    }
  }
  ${GridFragment}
`;
\end{lstlisting}
