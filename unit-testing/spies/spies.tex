\maketitle{}
\section{ Spies }

Spy's are an integral part of any unit testing suite. For those familiar, you
might think of it as something that doesn't neccesarily deserve it's own
chapter. However, spies help simplify the unit test suite to such a great
extent, that not discussing the finer detais of how it should work, almost
seems like a crime.

\subsection{ A Primer }
First and foremost, spies are actually one of the finer points of unit testing.
Personally, when I started writing them professionally for Verizon, spies were one of the more difficult things for me to understand. I found the name
particularly confusing. A spy?! You mean I am spying on a function that I
already know about? Wait, so you mean I am giving my unit tests the ability to
tell if function has run, or what it has been called with? Also, it gives me the
ability to hijack it with a different function, or call through as is? Wow, I
can't think of any name to call that? Ok, now I understand why it was called a
spy.

\subsubsection{ Spy User Example }

\begin{lstlisting}
describe('', () => {
  const userId = '123';
  it('should call the userFacade.getUsers function, when getusers' +
  'is called', () => {
    spyOn(userFacade, 'getUsers');
    component.getUsers(userId);
    expect(userFacade.getUsers).toHaveBeenCalledWith(userId);
  });
});
\end{lstlisting}

The above is a classic example of what a real world spy would do. Here, we just
want to just make sure that the facade is indeed being called when the
appropriate compoennt function is run.
