\maketitle{}
\section{ No Errors Schema }

In Angular, when using the testBed, any providers used in the Angular component
will cause an error, unless provided for within the Angular TestBed. There is a
way, however, to turn off these errors, by using the no errors schema within
Angular. It's syntax look like this:
\begin{lstlisting}
beforeEach(() => {
  Testbed.configureTestingModule({
    declarations: [pxComponent],
    schemas: [NO_ERRORS_SCHEMA]
  });
  fixture = Testbed.createComponent(pxComponent);
});
\end{lstlisting}

Adding NO_ERRORS_SCHEMA to the schemas key, will make the unit test ignore
errors if you encounter an unknown attribute, or unknown element.

\subsection{ Concerns Around Using No Errors Schema }
One of the concerns around using NO_ERRORS_SCHEMA, is that you are disabling
any errors that might potentially be dangerous. Being that we are turning off
any template errors, we might potentially be testing something wrong. What
exactly, might be tested wrong? Well, just because Angular doesn't error out
for the template it's self, that does not mean that an unassuming developer
will not go ahead and attempt to develop a template related item, such as an
input value. It is therefore best practice to avoid using the NO_ERRORS_SCHEMA
whenever possible.

\subsubsection{A Better Alternate Solution}
A better alternate solution is to stick with the reccomended SMAG architecture,
and provide the appropriate mocks alongside your services, pipes, and
interfaces. 
