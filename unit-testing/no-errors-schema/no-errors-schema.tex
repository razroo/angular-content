\maketitle{}
\section{ No Errors Schema }

In Angular, when using the testBed, any providers used in the Angular component
will cause an error, unless provided for within the Angular TestBed. There is a
way, however, to turn off these errors, by using the no errors schema within
Angular. It's syntax look like this:
\begin{lstlisting}
beforeEach(() => {
  Testbed.configureTestingModule({
    declarations: [pxComponent],
    schemas: [NO_ERRORS_SCHEMA]
  });
  fixture = Testbed.createComponent(pxComponent);
});
\end{lstlisting}

Adding NO_ERRORS_SCHEMA to the schemas key, will make the unit test ignore
errors if you encounter an unknown attribute, or unknown element.

\subsection{ Concerns Around Using No Errors Schema }
One of the concerns around using NO_ERRORS_SCHEMA, is that it might cause 
