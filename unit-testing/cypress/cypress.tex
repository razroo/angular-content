\maketitle{}
\section{ Visual Unit Tests with Cypress }

Unit testing is all great until something changes. The current issue with
writing traditional unit tests is that you miss out on the point of front end
development - to create experiences for the end user. While unit tests may do
very well in determining if an input produces an expected output, it only
captures moment in time rather than a complete flow of actual events. This is
where Cypress excels, especially in the end-to-end space.

e2e is easier to validate a user's experience and their interaction with an
application. This makes it a good supplement to existing unit tests. While
larger enterprises may have their own QA teams, e2e tests at the developer
level makes coding driven by design rather than an additional checkpoint at the
end.

Performance wise, it is easy to implement and has the potential to run tests in
parallel - meaning, that multiple scenarios and outcomes can be tested
simultaneously with auto reloading and snapshots for developers to debug if
such an event should arise.

Cypress has solid documentation with event driven language for their syntax,
making the task of writing tests more akin to actual user flows and potential
interactions than just testing x and y. The visual nature of Cypress
differentiates it from other unit testing and e2e suites currently available,
giving the developer the ability to visually check the user experience with DOM
state snapshots and historical contexts against different executions.

This makes Cypress a powerfully easy tool to use and lowers the bar of entry
significantly for developers wanting to drive their development efforts with
tests, catching bugs before they morph into something too big.

\subsection{How to use Cypress with Nx}

Before we proceed, Nx stands for Nrwl Extensions for Angular and is built by a
team of ex-Google employees who were also part of the Angular core team. Nx
isn't a replacement of the Angular CLI and instead extends it.

Nx sits on top of the CLI and works to give your application access to features
and functionality currently not available in the CLI. It has the ability to
create work spaces, along with applications and libraries - thus expanding the
CLI's capabilities.

So how do we use Nx with Cypress? and why should we?

Using Cypress with Angular CLI is possible but there is a lot of manual set up
required. Nrwl Nx solves this problem by creating streamlined experience that
pre-configures all you need for Cypress to work.

To use, Nrwl Nx, you'll need to install it using the following command in your
console:

\begin{verbatim}
npm install -g @nrwl/schematics @angular/cli
\end{verbatim}

After this has installed, you'll need to create a workspace using
create-nx-workspace command:

\begin{verbatim}
create-nx-workspace example-app --e2e-test-runner=cypress
\end{verbatim}

This will give you a series of questions before creating your workspace. If you
select Angular in the question that asks 'what to create in the new workspace',
it will begin to set everything up for you. Nx will create a folder called
example-platform and inside this folder, there will be an apps folder.

In this apps folder, you will find an empty Angular project and a Cypress ready
e2e test unit.

Nx also supports React, so React developers are not left out from this out of
the box set up.

For Angular, to run the testing suit, be sure to navigate into the workspace
folder and use the command below:

\begin{verbatim}
ng e2e --watch
\end{verbatim}

The --watch flag lets you write you tests in the background and Cypress
automatically detects any changes in the test or application code and runs
itself against the changes.

\subsection{ Example usage and cases }

\subsubsection{ Accessing store }
Cypress is able to test a multitude of front end frameworks and libraries,
Redux being one of them. By using if (window.Cypress){window.store = store }
inside your src/index.js, you are exposing the store when the application is
run inside Cypress' browser. Now inside your test file, you can test the store's
state through cy.window(). Here's an example of accessing a list store and
testing if it has a specific item.

\begin{lstlisting}
it('has expected state on load', () => {
   cy.visit('/')
   cy.window()
     .its('store')
     .invoke('getState')
     .should(
       'deep.equal',
       { list: [ { text: 'by apples', } ] }
      )
})
\end{lstlisting}

\subsubsection{ Button Clicked }
Attaching a data-cy attribute to a button exposes the item for testing by
Cypress. While this is not necessary as it can be accessed through classes, it
is not recommended as it can result in the wrong object being targeted. data-cy
therefore makes certain for Cypress that it is testing the right thing.

For example, your HTML button code may look something like this:

\begin{verbatim}
<button id="main" class="btn" data-cy="submit">
    Submit Me!
</button>
\end{verbatim}

To create a test to check if the button is clicked, your Cypress test code may
look something like this:

\begin{verbatim}
cy.get('[data-cy=submit]').click()
// OR
cy.contains('Submit Me!').click()
\end{verbatim}

\subsubsection{ modal should appear when button is clicked }
We all get given business and design rules to help us build our applications.
It may read something like this:
Modal should appear when button clicked, and should have drop down data, but
should not have field available, if x data is not available.

To translate this into a Cypress test unit, your code may look something like this:

\begin{lstlisting}
cy.get('[data-cy=submit]').click()
  cy.window()
    .its('store')
    .invoke('getState')
    .should(
      'deep.equal',
      { list: [ { text: 'by apples', } ] }
    )
cy.get('[data-cy=user-modal-dropdown]').click().contain([{text: 'by apples']);
\end{lstlisting}

\subsection{ Potential issues when using Cypress }
The concept of code coverage refers to the percentage of code that is covered
by my automated tests. Purity of test types may get muddied and percentage of
test coverage may reduce over time when teams begin to mix and math e2e with
unit tests.

The coverage therefore gets thinned over time and may leave pockets untested as
a result. Certain pipelines may only pass a project to the next stage if it
passes a percentage of tests - tests that may may not account for e2e results
or are exclusively e2e.

Another potential issue is that test hooks have the potential to muddy up the
final production code. The idea of hooks is to keep the main code state
untouched and independent from actions that are trying to observe events,
inputs, outputs and outcomes.

\subsection{ Extended Features }
Cypress' test runner is MIT open sourced and free. However, it does have a
commercial side to it and offers a Dashboard service allows you to centralize
your tests and enable continuous integration at a team level. This differs from
the test runner which only runs on your local machine.

There is a free Seed tier that currently allows for up to 3 users and 500 test
recordings. As you move up through the plans, the level of support, users and
number of test recordings increases.

\subsection{ Final words }

Cypress as a testing tool differs from all other testing tools, frameworks and
assertion libraries because it offers front end developers a way to test their
applications that is driven by user experience.

Creating unit tests can be a dry process but Cypress brings the fun and beauty
of creating visuals without having to manually click the buttons yourself every
time to test if something worked. The DOM snapshots makes for fantastic
debugging and the concise, yet information rich, documentation also helps in
the learning process.

Setting up is quick and it doesn't take long at all to get started on Cypress.
