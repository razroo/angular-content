
\section{ Jest }

\subsection{ A Primer. }
Jest is a test runner created by Facebook to allow for the "Delightful
Javascript Testing".

\subsection{ The Benefits of Jest Vs. Karma }
\begin{enumerate}
  \item Fast and sandboxed
  \item Built-in code coverage reports
\end{enumerate}

\subsubsection{ Fast and sandboxed }
Jest parallelizes test runs across workers to maximize performance.
Console messages are buffered and printed together with test results. Sandboxed
test files and automatic global state resets for every test so no two tests
conflict with each other.

\subsubsection{ Built-in code coverage reports }
One has the ability to create code coverage reports using --coverage.  No
additional setup, or libraries are needed. Jest can collect code coverage
information from entire projects, including untested files. The way it moves
focus away from something like istanbul is fantastic. The UI, in my humble
opinion, for what it's worth, is not the greatest.

\subsubsection{ Does not require starting a Browser }
This one in particular helps with regards to CI/CD. Not requiring a browser,
does not require a browser to be built into the CI/CD. In addition, a large
part of performance issues with regards to unit tests, is having to start the
browser every time.

\subsection{ Using Jest within an Nx Setting }
\begin{verbatim}
  ng generate jest
\end{verbatim}

\begin{verbatim}
  ng generate jest
\end{verbatim}

After running the above generator, one can now run jest within your app. When
generating a lib, one can now do:
\begin{verbatim}
  ng generate lib libname --unit-test-runner jest
\end{verbatim}

\footnote{If you are coming from an existing nrwl workspace, please use the
following blog post to find out how to upgrade to Jest https://blog.nrwl.io/nrwl-nx-6-3-faster-testing-with-jest-20a8ddb5064}

\subsection{ Primer on Jest Syntax }
Real quick, I would like to go through a couple of things that Jest offers over
Karma.

\subsection{Switching over from Karma to Jest}
At this point, this is the main selling point, of why Netanel Basel's
library, Spectator, is so valuable, is that it allows for tests to be converted
over to Jest automatically, by simply switching the imports. For instance, let's
say that we are unit testing a service:
\begin{lstlisting}
import { createService } from '@netbasal/spectator';
import { AuthService } from './auth.service';
import { DateService } from './date.service';

const spectator = createService({
  service: AuthService,
  mocks: [DateService]
});

it('should be logged in', () => {
  const dateService = spectator.get<DateService>(DateService);
  dateService.isExpired.mockReturnValue(false);
  expect(spectator.service.isLoggedIn()).toBeTruthy();
});
\end{lstlisting}

The above test is currently using karma. However, if we wanted to switch it over
to use Jest, all we would need to do is change the import path:
\begin{verbatim}
- import { createService } from '@netbasal/spectator';
+ import { createService } from '@netbasal/spectator/jest';
\end{verbatim}

Just like that magic, we can have our entire app using Jest. 
