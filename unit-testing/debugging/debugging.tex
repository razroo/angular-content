\maketitle{}
\section{ Debugging }

Debugging in any development environment is a neccesity. When it comes to unit
testing, sometimes a unit test will not pass, and will be a bit difficult to
decipher why it is that it isn't passing. I would like to present the following
strategy to greatly expedite the process of debugging, when you find yourself
going against a unit test that is taking longer than expected.

\subsection{ A winning Combo }
There are two pieces to debugging Unit Tests, that will allow you to debug unit
tests with ease. One of them, is to create smaller modules whenever possible.
This will allow you to unit test specific modules using:
\begin{verbatim}
  ng test --project user --watch
\end{verbatim}

Now you will have a window that is open, with the ability to debug, that only
runs unit tests for the specific module.

\subsection{ Opening Source }
The simplest solution to start debugging, would be to open up the inspector.
Click on source, add a debugger for the particular function that you are looking
for, and re-run the page. This will cause the page to then pause on the function
you are trying to test.

\subsection{ Turning Off Source Map }
The specifics with regards to how Angular works, it can be advantageous to turn
off the source map. The following is the syntax on how to turn off the source
map for angular:

\begin{verbatim}
ng test --project=px-illustrator --source-maps=false
\end{verbatim}
