
\chapter{ Unit Testing Modules }
I think this is an important chapter, because unit testing modules are a good
little piece of architecture in Angular. Simply put, it allows you to put a
number of providers into the module for unit testing. Then you can just go
ahead and import the module. In fact, even if there is only provider being
added to the module, it is still keeping your app DRY'er(AKA Don't Repeat
Yourself).

\section{Comparison of Using a Module Vs a Provider}
I would just like to illustrate the point of how simple it is to use a module
vs a provider. The reccomended use of a unit testing module, is for a facade.
If you will recall, our use of a facade, for our data-access architecture. This
means, that there is only one file that is required for us to provide in our
unit tests. Nonetheless, let's say that we are attempting to mock our UserFacade
that we use for user-settings. Our file would look like the following:

\begin{lstlisting}
mport { NgModule } from '@angular/core';

import { UserFacade } from './+state/user.facade';
import { MockUserFacade } from './+state/user.facade.mock';

@NgModule({
  providers: [
    {
      provide: UserFacade,
      useClass: MockUserFacade,
    },
  ],
})
export class PxDataAccessUserTestingModule {}
\end{lstlisting}

Now, if we want to mock the data that we are using within our app,

\begin{lstlisting}
describe('UserSettingsComponent', () => {
  let facade: UserFacade;

  beforeEach(() => {
    TestBed.configureTestingModule({
      imports: [PxDataAccessUserTestingModule],
    });

    facade = TestBed.get(UserFacade);
  });
\end{lstlisting}

As components are wont to do, we will most likely be using the UserFacade in a
number of different locations.

\section{Pitfall of Using a Module}
A module is a bit of a blackbox. We don't neccesarily know everything that is
contained within the module. So for any developer other than the one who
created it, it might be a bit confusing when looking at the file for the first
time.

\section{Moving past the Pitfall of using a Module}
It is important to put in place a convention for all places wherein a module
should be used. When running through our SMAG architecture, the only place that
deserves a module, is for our data-access folder. Primarily, because our
data-services are usually only going to be used within our data-access folder.
So really a testing module only makes sense for our facade. So as a convention,
your team should always use a testing module for facades within the context for
your data-access folder.
