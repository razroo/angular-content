\maketitle{}
\section{ Unit Testing Modules }
I think this is an important chapter, because unit testing modules are a good
little piece of architecture in Angular. Simply put, it allows you to put a
number of providers into the module for unit testing. Then you can just go
ahead and import the module. In fact, even if there is only provider being
added to the module, it is still keeping your app DRY'er(AKA Don't Repeat
Yourself).

\subsection{Comparison of Using a Module Vs a Provider}
I would just like to illustrate the point of how simple it is to use a module
vs a provider. The reccomended use of a unit testing module, is for a facade.
If you will recall, our use of a facade, for our data-access architecture. This
means, that there is only one file that is required for us to provide in our
unit tests. Nonetheless, let's say that we are attempting to mock our UserFacade
that we use for user-settings. Our file would look like the following:

\begin{lstlisting}
mport { NgModule } from '@angular/core';

import { UserFacade } from './+state/user.facade';
import { MockUserFacade } from './+state/user.facade.mock';

@NgModule({
  providers: [
    {
      provide: UserFacade,
      useClass: MockUserFacade,
    },
  ],
})
export class PxDataAccessUserTestingModule {}
\end{lstlisting}
