\maketitle{}
\section{ Unit Testing }

Unit testing has always been a hot topic. It's because it really isn't so well
understood amongst many software engineers. In addition, I know of many managers
who look as unit testing as icing on the cake, and perhaps they are right.

However, what unit testing allows us to do, is assure the code we are implementing
is correct. That it is self documenting, and that we have the confidence moving
forward knowing that this will work the way it needs.

Unit testing really is a beast, and is really hard to manage. It's because it
isn't directly related to quality assurance, and can easily go under the radar,
as the app will go on without it. However, everytime that I have communicated
with management the bottom line. Research has been done on unit testing, and that
the application will be harder to manage long term, without it, they understand.

Bring the above point up to them, and hopefully you can have unit testing as a
part of your application.

At this point in our application, the only thing that we have created that
deserves to be unit tested, are the reducers. Let's go ahead and unit test those:

\begin{lstlisting}
describe('Functionality for the ChooseSizeUpdated reducer', () => {
  const chooseSizeData = {
    columns: 20,
    rows: 20,
    pixelSize: 20
  }
  it('should update the chooseSize store as is approprate', () => {
    const action: ChooseSizeUpdated = new ChooseSizeUpdated(chooseSizeData);
    const actual = chooseSizeReducer(initialState, action);
    expect(actual).toEqual(chooseSizeData);
  });
});
\end{lstlisting}

The above is an example of what a sample unit test for a reducer would look like
in the chooseSizeUpdated reducer.

Note: Add in section going over best conventions with regards to unit testing.
