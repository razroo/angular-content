
\section{ Benefits of Unit Testing, TDD, and BDD }
It truly makes sense for a definitive guide to deep dive into what the benefits
of unit testing would be. From a developer perspective:
\begin{enumerate}
  \item Give insight as to what unit testing should accomplish, so that a
  developer, can intuitively decide when appropriate to write a unit test.
  \item Give confidence to developer as to why they should write unit test.
\end{enumerate}

From a management perspective, to help introduce unit testing, for those that
might be less inclined towards unit testing.

\subsection{What is Unit Testing?}
A unit in a best case scenario, is a function that always gives you the same
result for a given input(pure function) Testing that unit, is to make sure that
the expected result happens when running that function.

\subsection{Benefits of Unit Testing}
\begin{enumerate}
  \item Refactoring. Change code once, and see everything else is working.
  \item Focus(See TDD item \#2)
  \item Helps understand design of code working with(See TDD item \#3)
  \item Instant visual feedback that code works as expected.
  \item Documentation (See TDD \#4)
  \item Helps with code-reuse. Ability to re-use code and tests. Tweak tests
  accordingly.
  \item Testable code
    \begin{enumerate}
      \item Modular
      \item Maintainable
      \item Readable
    \end{enumerate}
\end{enumerate}

\subsection{What is TDD (Test Driven Development)}
\begin{enumerate}
  \item Start by writing a test.
  \item Run the test, and any other tests. At this point, your newly added test
  should fail. If it doesn't fail here, it might not be testing the right thing,
  and has a bug in it.
  \item Write the minimum amount of code required to make the test pass.
  \item Run the test to check the new test passes.
  \item Optionally re-factor your code.
  \item Repeat from 1.
\end{enumerate}

\subsection{ What is BDD? }
Typcially when unit testing, a particular function at a later date can change
it's impementation. For instance, if we have a counter function, the counter
can be changed to start at 5 instead of 0, breaking the expect statement of 1.
In BDD we focus on the intended behavior, instead of the expected result. The
following is a great explanation:

\subsection{ The Benefits of TDD }
\begin{enumerate}
  \item Higher Test Coverage
  \item Focus
    \begin{enumerate}
      \item Focus one part of an issue at a time.
      \item Allows one to realize when to stop coding.
    \end{enumerate}
  \item Interfaces. Allows you to think more organically about what should be
  put into interface. Allows for interface to be written bottom
  up(implementation, behavior) instead of top down(behavior, implementation).
\end{enumerate}

\subsection{What is BDD - Code Example}
\begin{lstlisting}
// Non BDD
describe('Counter', () => {
  it('should increase count to 1', () => {
    const counter = new Counter();

    counter.tick();

    expect(counter.count).toEqual(1);
  });
})

// BDD
describe('Counter', () => {
  it('should increase count by 1 after calling tick', () => {
    const counter = new Counter();
    const expectedCount = counter.count + 1;

    counter.tick();

    expect(counter.count).toEqual(expectedCount);
  });
})
\end{lstlisting}

\subsection{ The Benefits of BDD? }
If at a later time, the counter(as seen above), for instance, has to change
based on requirements(starting at 5, instead of 1), it will not affect the unit
test.

\subsection{What, When, and How}
Unit testing, is the what. TDD is the when, and BDD is the how.

\mybox{Convincing the skeptic might be difficult. The following are some steps
to convince the skeptic.
\begin{enumerate}
  \item Too Simple to Fail? If this is used regularly in our application,
  doesn't it make sense to spend 5 - 30 minutes testing it?
  \item Too difficult to Unit Test, won't it be time consuming? Good indication
  that it needs to be re-factored, proving point of benefit of unit test.
  \item Too Time Consuming
    \begin{enumerate}
      \item App will produce more bugs
      \item Lower Quality Code
      \item Lack of confidence. Perhaps bugs under the radar.
      \item Lower design.
    \end{enumerate}
\end{enumerate}
}
