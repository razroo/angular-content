\maketitle{}
\section{ Benefits of Unit Testing - A Primer }
It truly makes sense for a definitive guide to deep dive into what the benefits
of unit testing would be. From a developer perspective:
\begin{enumerate}
  \item Give insight as to what unit testing should accomplish, so that a
  developer, can intuitively decide when appropriate to write a unit test.
  \item Give confidence to developer as to why they should write unit test.
\end{enumerate}

From a management perspective, to help introduce unit testing, for those that
might be less inclined towards unit testing.

\subsection{What is Unit Testing?}
A unit in a best case scenario, is a function that always gives you the same
result for a given input(pure function) Testing that unit, is to make sure that
the expected result happens when running that function.

\subsection{Benefits of Unit Testing}
\begin{enumerate}
  \item Refactoring. Change code once, and see everything else is working.
  \item Focus(See TDD item \#2)
  \item Helps understand design of code working with(See TDD item \#3)
  \item Instant visual feedback that code works as expected.
  \item Documentation (See TDD \#4)
  \item Helps with code-reuse. Ability to re-use code and tests. Tweak tests
  accordingly.
  \item Testable code
    \begin{enumerate}
      \item Modular
      \item Maintainable
      \item Readable
    \end{enumerate}
\end{enumerate}

\subsection{What is TDD (Test Driven Development)}
\begin{enumerate}
  \item Start by writing a test.
  \item Run the test, and any other tests. At this point, your newly added test
  should fail. If it doesn't fail here, it might not be testing the right thing,
  and has a bug in it.
  \item Write the minimum amount of code required to make the test pass.
  \item Run the test to check the new test passes.
  \item Optionally re-factor your code.
  \item Repeat from 1.
\end{enumerate}

\subsection{ The Benefits of TDD }
\begin{enumerate}
  \item Higher Test Coverage
  \item Focus
    \begin{enumerate}
      \item Focus one part of an issue at a time.
      \item Allows one to realize when to stop coding.
    \end{enumerate}
  \item Interfaces. Allows you to think more organically about what should be
  put into interface. Allows for interface to be written bottom
  up(implementation, behavior) instead of top down(behavior, implementation).
\end{enumerate}
