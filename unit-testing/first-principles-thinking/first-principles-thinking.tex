
\chapter{ Unit Testing TDD - First Princples Discovery }

Unit testing as a discipline is something which is very hard to write. Writing
a good unit test is as much as a discipline as writing good software. In
addition, if someone is following TDD standards, then knowing of all the unit
tests ahead of time can be very difficult. Leading many developers to leave the
TDD environment. If they do write unit tests, it will be towards the end. The
following is a great way, and as far as I am concerned the best way to discover
unit tests.

\section{ What is the First Princples Thinking? }
In Physics\footnote{Not a physicist, but I heard of this concept the first time
from Elon Musk. I am, however, a Talmudist, and in talmud we have a similar
concept called Pilpul}, there is a concept called, first principles thinking.
This means boiling down a priciple to it's essnetial truth and then building up
from there. With regards to Unit Testing, and specifically test driven
development, this principle will help to create fantastic unit tests.

\section{ First Princples Thinking - Rubber Ducking - An example }

\begin{lstlisting}
  Q: What would we like to get out of the choose size form?
  A: We would like to specify number of rows, number of columns, and pixel size.

  Q: Being that these are numbers, do we have any way of preventing
  the user from entering in any value other than an number?
  A: The input will only allow numbers.
  Q: Ok, do you see any value in setting up logic, so that if it is not a
  number, then it will throw an error?
  A: No, because there will never be a situation wherein they can not put in a
  number.
  Q: In that case, should we take a snapshot and make sure input fields are
  indeed of the type number. If not then it should error.

  Q: Now that we have established that these are always numbers, is there any
  limit on the size of this grid.
  A: Not really, I can see it as just being what the window size.
  Q: Hmm, I find that interesting, so you are saying that it can be any size.
  A: Yes for this iteration.
  Q: Okay, but it will be specific to window size.
  A: Yes.
  Q: Ok, so let's go ahead and create unit test, that it should have an
  error, that based on window size, if value is greater, it should automatically
  have it go to the height window size. In addition, there should be test that
  specifies as such.

  Q: Now that we have established that these we constrained window, is there any
  limit on the ratio between column size, and row size?
  A: No.
  Q: Is there any way that we can create a unit test, to just say that it
  passes, if there is a difference in ratio, between, then should not error out?
  A: Yes, need to figure this out though. Not sure how to do off hand.

  Q: Now that we have established that there are no constraints, what else is
  there for us to test. We have covered window size, ratio, input type, what
  about a max pixel Size?
  A: Hmm, why would we care about that?
  Q: Not sure, good point.

  Q: Hmm, that being said, should we only allow pixel size that is within the
  frame of the row and column size? For instance, let's say we have 20 rows, and
  20 columns, should the pixel size be something that perfectly divides into
  400?
  A: Not sure, it would be a lot of overhead at this point.
  Q: Almost feel like we should automatically control pixel size based on column
  and row size.
  A: Ok, out of scope, let's not.

  Q: Anything else?
  A: Not that I can think of.

  //End of Scene - dev takes a bow
\end{lstlisting}

To re-iterate in this book, we will not go into specifics of what is going on
at a particular time. However, these unit tests can be found in the Angular
Pixel Illustrator repo.
