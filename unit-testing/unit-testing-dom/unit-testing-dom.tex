\maketitle{}
\section{ Unit Testing the DOM }
The DOM is the one variable with front end, that complicates the unit testing
process. While it is in the process of becoming more stable, it can be difficult
to unit test, and can make TDD more difficult. It is very much important to do
and can also be expensive from a unit testing perspective. A unit test will
run much slower if it has to interact with the DOM. So, this is something to
take into consideration as well. \footnote{Look into chapter on Cypress for
visual unit testing.}

DOM unit testing refers to a couple of scenarios:
\begin{enumerate}
  \item Event Handling.
  \item Element is visible, or hidden.
  \item Element contains certain text, or that text properly transformed.
\end{enumerate}

These three are the major types of DOM unit testing that occurs. There is, of
course, much more to unit test from an E2E perspective. Determining whether, or
not an element is visible, or hidden, and whether it contains text, is better
handled by Cypress. For this please do reference the chapter on visual unit
testing. The one piece, however, that is handled better by unit testing, is
testing that a function did indeed run, when an element has been clicked. There
might also be other unique situations, including the above, if your team does
not have the capacity to use Cypress.

\subsection{ Selecting element }
In an Angular setting, the most important of all testing utilities, is the
Angular TestBed. The TestBed creates a dynamically-constructed Angular test
module, that emulates an \@NgModule \footnote{https://angular.io/guide/testing\#testing-services-with-the-testbed}. In laymans terms, it allows you to swap out any
piece that was included in the component, for testing purposes, and then to
reference that swapped out piece.

Using the testBed, we are also able to create a fixture, which we can reference
to target the nativeElement. We can then use the querySelector on the
nativeElement, to target our element. Let's go back to the one use case we
would like to target. That is, when an event is triggered, we would like to
make sure a particular function is called. In each scenario, we would like to
make sure that once a filter is clicked on, the appropriate filtering function
is called.

\begin{lstlisting}
it('should call the appopriate function when filterToggle element' +
`is clicked on', () => {
  spyOn(component, 'filterUsers')
  const filterToggle = fixture.nativeElement.querySelector(
    '.filter-toggle'
  );
  filterToggle.click();
  expect(component.filterUsers).toHaveBeenCalledWith(component.id);
});
\end{lstlisting}

First, we are spying on the filterUsers method for our component. Next we are
using the querySelector to target the .filter-toggle html class (assuming there
is only one on the page). Moving on, when clicked on again, we want to make sure
that the appropriate function is called.

We have now completely through the power of unit testing, determined whether, or
not an element is going to show up.

\subsection{ Unit Testing - Determining Text }
With regards to text, let's say that we want to test the entire component at a
specific time period, and want to make sure it contains three different words:
\begin{lstlisting}
it('should show buyer company names', () => {
  expect(fixture.nativeElement.innerText).toContain('Apple');
  expect(fixture.nativeElement.innerText).toContain('Microsoft');
  expect(fixture.nativeElement.innerText).toContain('Google');
});
\end{lstlisting}

Text might seem intuitive. However, there is the option to target text at
different areas of time, and to make sure what one is looking is the correct
format at a given time. Doing something like this takes experience to get it
right. However, assuming you didn't know beforehand, you now know that you have
the option to target text at a specific time.

As we mentioned, text and whether, or not an element is hidden, might be better
handled by Cypress. However, just in case you want to see it for yourself, the
above is a great example.
