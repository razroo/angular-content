\maketitle{}
\section{ Angular Observables }
It's important to point out, that there are observables that are unique to 
Angular. It is important to be aware of the fact that internally they are 
using an observable. 

\subsection{ Event Emitter }

\begin{lstlisting}
<zippy (open)="onOpen($event)" (close)="onClose($event)"></zippy>
\end{lstlisting}

\begin{lstlisting}
@Component({
  selector: 'zippy',
  template: `
  <div class="zippy">
    <div (click)="toggle()">Toggle</div>
    <div [hidden]="!visible">
      <ng-content></ng-content>
    </div>
  </div>`})

export class ZippyComponent {
  visible = true;
  @Output() open = new EventEmitter<any>();
  @Output() close = new EventEmitter<any>();

  toggle() {
    this.visible = !this.visible;
    if (this.visible) {
      this.open.emit(null);
    } else {
      this.close.emit(null);
    }
  }
}
\end{lstlisting}

\subsection{ Async Pipe }
The async pipe, which we will discuss more due to it's high usage within the
app. 

\begin{lstlisting}
@Component({
  selector: 'async-observable-pipe',
  template: `<div><code>observable|async</code>:
       Time: {{ time | async }}</div>`
})
export class AsyncObservablePipeComponent {
  time = new Observable(observer =>
    setInterval(() => observer.next(new Date().toString()), 1000)
  );
}
\end{lstlisting}

The async pipe will subscribe to an observable(,or promise) and returns the 
latest value it has emitted. When new value has been emitted, the pipe marks 
the component to be checked for changes. 