\maketitle{}
\section{ Angular Observables }
It's important to point out, that there are observables that are unique to 
Angular. It is important to be aware of the fact that internally they are 
using an observable. 

\subsection{ Event Emitter }

\begin{lstlisting}
<zippy (open)="onOpen($event)" (close)="onClose($event)"></zippy>
\end{lstlisting}

\begin{lstlisting}
@Component({
  selector: 'zippy',
  template: `
  <div class="zippy">
    <div (click)="toggle()">Toggle</div>
    <div [hidden]="!visible">
      <ng-content></ng-content>
    </div>
  </div>`})

export class ZippyComponent {
  visible = true;
  @Output() open = new EventEmitter<any>();
  @Output() close = new EventEmitter<any>();

  toggle() {
    this.visible = !this.visible;
    if (this.visible) {
      this.open.emit(null);
    } else {
      this.close.emit(null);
    }
  }
}
\end{lstlisting}

\subsection{ Async Pipe }
The async pipe, which we will discuss more due to it's high usage within the
app. 

\begin{lstlisting}
@Component({
  selector: 'async-observable-pipe',
  template: `<div><code>observable|async</code>:
       Time: {{ time | async }}</div>`
})
export class AsyncObservablePipeComponent {
  time = new Observable(observer =>
    setInterval(() => observer.next(new Date().toString()), 1000)
  );
}
\end{lstlisting}

The async pipe will subscribe to an observable(,or promise) and returns the 
latest value it has emitted. When new value has been emitted, the pipe marks 
the component to be checked for changes. 

\subsection{ Router }
\subsubsection{Events}
Router events are supplied as an observable. So let's say we want to listen 
in, into when a router event has reached a certain point we would be able to 
do that. 
\begin{lstlisting}
import { Router, NavigationStart } from '@angular/router';
import { filter } from 'rxjs/operators';

@Component({
  selector: 'app-routable',
  templateUrl: './routable.component.html',
  styleUrls: ['./routable.component.css']
})
export class Routable1Component implements OnInit {

  navStart: Observable<NavigationStart>;

  constructor(private router: Router) {
    // Create a new Observable that publishes only the NavigationStart event
    this.navStart = router.events.pipe(
      filter(evt => evt instanceof NavigationStart)
    ) as Observable<NavigationStart>;
  }

  ngOnInit() {
    this.navStart.subscribe(evt => console.log('Navigation Started!'));
  }
}
\end{lstlisting}

\subsubsection{ ActivatedRoute }
\lstinline{ActivatedRoute} "contains the information about a route associated
with a component loaded in an outlet." Specifically, one of the pieces of 
information that the \lstinline{ActivatedRoute} injected router service 
provides is \lstinline{ActivatedRoute.url} which is provided as an observable. 

\begin{lstlisting}
import { ActivatedRoute } from '@angular/router';

@Component({
  selector: 'app-routable',
  templateUrl: './routable.component.html',
  styleUrls: ['./routable.component.css']
})
export class Routable2Component implements OnInit {
  constructor(private activatedRoute: ActivatedRoute) {}

  ngOnInit() {
    this.activatedRoute.url
      .subscribe(url => console.log('The URL changed to: ' + url));
  }
}
\end{lstlisting}
Using the above observable, we are able to determine what the url is at any
given time. 

\subsection{ Reactive Forms }
