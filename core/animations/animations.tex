\maketitle{}
\section{ Animations }

Animations have come a long way. It's one of the most jaw dropping effects
that any application can have. Angular has made it, so that animations can be
baked into the framework. I would like to run through how one would go ahead
and implement a simple animation within your component, so that you can get an 
idea of how you can do it. 

\subsection{ Include Animations Module }
Angular has a \lstinline{BrowserAnimationsModule}, which is based off of the 
\lstinline{BrowserModule}.

\begin{lstlisting}[caption=app.module.ts]
import { NgModule } from '@angular/core';
import { BrowserModule } from '@angular/platform-browser';
import { BrowserAnimationsModule } from '@angular/platform-browser/animations';

@NgModule({
  imports: [
    BrowserModule,
    BrowserAnimationsModule
  ],
  declarations: [ ],
  bootstrap: [ ]
})
export class AppModule { }  
\end{lstlisting}

\subsection{ Importing animation functions into component files }
Just something to be aware of. Angular has many animation functions that can
be used out of the box. We are not going to go into detail on them all now. 
However, I just wanted to bring some up here, for you to be aware of: 
\begin{lstlisting}[caption=app.component.ts]
import { Component, HostBinding } from '@angular/core';
import {
  trigger,
  state,
  style,
  animate,
  transition,
  // ...
} from '@angular/animations';  
\end{lstlisting}

\footnote{For a summary of available animation functions, feel free to navigate 
\href{https://angular.io/guide/animations\#animation-api-summary}{here}}

\subsection{Add the animation metadata property to component}
The next step in the adding an animation to a component process, would be 
adding an animations metadata \footnote{Clarify what a metadata property is}
property.

\begin{lstlisting}
@Component({
  selector: 'app-root',
  templateUrl: 'app.component.html',
  styleUrls: ['app.component.css'],
  animations: [
    // animation triggers go here
  ]
})  
\end{lstlisting}

\subsection{ Animation State, Styles, and Transitions }
With regards to animations, there are three main functions to keep in mind, 
\begin{enumerate}
  \item \lstinline{state()} - Function to define different states to call at 
  the end of each transition. It takes two arguments: 
    \begin{enumerate}
      \item A unique name like \lstinline{open}, or \lstinline{closed}
      \item A \lstinline{style()} function 
    \end{enumerate}
    \begin{lstlisting}  
state('open', style({
  height: '200px',
  opacity: 1,
  backgroundColor: 'yellow'
})),
    \end{lstlisting}  
  \item \lstinline{style()} - A function used to assign a set of styles for 
  a given state name. Style attributes must be camelCase.
    \begin{lstlisting}
style({
  height: '200px',
  opacity: 1,
  backgroundColor: 'yellow'
})),
    \end{lstlisting}
  \item \lstinline{transition} - Used to specify the change that occurs 
  between one state and another over a period of time. Accepts two 
\end{enumerate}

let's say that we wanted to create a simple animation. 

\begin{lstlisting}
@Component({
  selector: 'app-open-close',
  animations: [
    trigger('openClose', [
      // ...
      state('open', style({
        height: '200px',
        opacity: 1,
        backgroundColor: 'yellow'
      })),
      state('closed', style({
        height: '100px',
        opacity: 0.5,
        backgroundColor: 'green'
      })),
      transition('open => closed', [
        animate('1s')
      ]),
      transition('closed => open', [
        animate('0.5s')
      ]),
    ]),
  ],
  templateUrl: 'open-close.component.html',
  styleUrls: ['open-close.component.css']
})
export class OpenCloseComponent {
  isOpen = true;

  toggle() {
    this.isOpen = !this.isOpen;
  }
}
\end{lstlisting}