\maketitle{}
\section{ Transitions and Triggers }

It is worth noting that there are a number of transition states. We discussed 
from open to close, and from close to open. There are, however, other states 
that are equally as important to discuss. 

\subsection{ Wildcard Matching }
Using an asterisk in a transition state, will represent any sort of situation. 

For instance, let's say were to use: 
\begin{verbatim}
  open => closed 
  open => *
  * => closed
  * => *
\end{verbatim}
All of the above four, will match when an element's state changes from open to 
anything else. A good rule of thumb, is that similar to routing, wherein the 
asterisk will be a wild card meant to match any remaining use cases, animation
transitions also follow in the same vein. 

\subsection{ Situations where a Wildcard can be Used }
\subsubsection{ Using Wildcard with Styles }
\begin{lstlisting}
transition ('* => open', [
  animate ('1s',
    style ({ opacity: '*' }),
  ),
]),  
\end{lstlisting}

In the above, the transition of animate of 1s, will match whatever it is that
the current stule is. 

\subsubsection{ Combining Wildcard and Void States }
A void state, is a way of causing an animation to occur whenever an element is 
entering, or leaving a page.
\begin{lstlisting}
animations: [
  trigger('flyInOut', [
    state('in', style({ transform: 'translateX(0)' })),
    transition('void => *', [
      style({ transform: 'translateX(-100%)' }),
      animate(100)
    ]),
    transition('* => void', [
      animate(100, style({ transform: 'translateX(100%)' }))
    ])
  ])
]
\end{lstlisting}

In the above code, when an \lstinline{HTML} element isn't attached to the view, 
we apply a transition. When entering the page, the element will fly in. When 
leaving the page, the element will fly out. 