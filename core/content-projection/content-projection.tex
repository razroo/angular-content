\section{ Content Projection }
\maketitle{}

Any component can technically be re-usable. However, what makes content
projection so great, is that it allows the content inside of a component to
change based on the need of the application. In addition, it allows us to
separate concerns. We can build a component for display, and another component
built for handling user actions.

\subsection{ Single Slot Projection }
If we wanted to create a component wherein we can use content projection, it is
as simple as adding ng-content inside of our component:
\begin{lstlisting}
<!-- inside reusable component -->
<ng-content></ng-content>

<!-- inside component, consuming the re-usable component -->
<reusable-component> <p>Content goes here</p> </reusable-component>
\end{lstlisting}

As we can see, using our <reusable-component>, we have the ability to put it
whereever we want, and change the content based on the parent component
consuming it. However, what if we have two separate places withing our
component that we would like to inject content. For instance, let's say we have
a card component, and we want there to be different content inside of the header
and main body of the component?

\subsection{ Multiple Slot Projection }
This is my preferred method of mutliple content projection, by creating binding
content projection to class. In particular, because I feel it's a great of
making sure content projection is transparent across the entire lifecycle. We
can now do the following:

\begin{lstlisting}
<div class="header">
<ng-content select=".header"></ng-content>
</header>
<div class="body">
<ng-content select=".body"></ng-content>
</div>
\end{lstlisting}

In our parent component consuming the re-usable component:
\begin{lstlisting}
<reusable-component>
<div class="header">CSS</div>
<div class="body">{{css-data}}</div>
</reusable-component>
\end{lstlisting}

Just like that, we can have our content project in multiple places, into the
re-usable component.

\subsection{Styling Projected Content}
One of the scenarios that comes up alot with regards to projected content, is
attempting to style it. For instance, you might want the content in one
component to have a top border, and in others for the text color to be of a
different style. So how would we do this, being that we are projecting the
content into a separate component?
\mybox{It should be mentioned that styling in this matter isn't the right
approach. However, because there are times wherein styling in this fashion is
neccesary for the particular use case, this is most definitely useful
mentioning.}
\begin{verbatim}
:host ::ng-deep .header {
  color: blue;
}

:host ::ng-deep .body {
  margin-top: pxl-space-multiplier(1);
}
\end{verbatim}

Just like that, we are able to style the content within our project.
