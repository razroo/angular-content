\maketitle{}
\section{ Input }

The Input decorator is one of two integral Angular Decorators, instrumental
for passing into a component, and out of a component into a another one. It's
name sounds like what it does. Similar to an html input, wherein a user inputs
a value, and that's the value that the form now has. Similarly, for an Angular
component, you can put a name on what you would like your input is. Use that
anticipated input value/name within your component. Now, you have the ability to
re-use your component in multiple places, and pass in the data of your choosing
based on the Input value you used.

It is important to realize that @Input and creating a re-usable component go
hand in hand. If you would like to create a re-usable component, 9/10 times
(rough estimate), you will need to use an @Input decorator. Ok, so let's get
down to how to use it.

\subsection{ Input - An Example }

Here is a great example. Let's say we have a bank account component, that we
would like to create. This bank account component will be used on multiple
pages for our web application. It will be displayed in multiple places.
Sometimes in the header, sometimes in a modal, or to display the bank used by
someone else, within your network.

\begin{lstlisting}
// bank account component - bank-acount.component.ts
@Component({
  selector: 'bank-account',
  template: './bank-account.component.html',
})
class BankAccount {
  // This property is bound using its original name.
  @Input() bankName: string;
  // this property value is bound to a different property name
  // when this component is instantiated in a template.
  @Input() id: string;
}
\end{lstlisting}

In the html of your bank-account component:
\begin{lstlisting}
<div> Bank Name: {{ bankName }} </div>
<div> Account Id: {{ id }} </div>
\end{lstlisting}

As we can see in the above code, our @Input's are considered as if they are
a value contained directly on our components.

\mybox{Just for the sake of clarity, our re-usable component is going to be put
into it's own module. This module is then going to be imported by the module,
containing the component we are going to use. This code is not included at this
time.}

\subsubsection{ Including Component in another Component }
Now we have the ability to include this component with an Input in another
component:
\begin{lstlisting}
<bank-account bankName="RBC" id="4747"></bank-account>
\end{lstlisting}

This would be @Input() in a nutshell.

\subsection{ bindPropertyName }
Input() does allow for an optional bindingPropertyName. This would mean that
the re-usable component would internally refer to the Input value as one way,
and the consuming component would refer to it, in another.

\begin{lstlisting}
// bank account component - bank-acount.component.ts
@Component({
  selector: 'bank-account',
  template: './bank-account.component.html',
})
class BankAccount {
  // This property is bound using its original name.
  @Input() bankName: string;
  // this property value is bound to a different property name
  // when this component is instantiated in a template.
  @Input('bank-account') id: string;
}
\end{lstlisting}

In the html of your bank-account component:
\begin{lstlisting}
<div> Bank Name: {{ bankName }} </div>
<div> Account Id: {{ id }} </div>
\end{lstlisting}

In the html of the component consuming the bank-account component.
\begin{lstlisting}
<bank-account bankName="RBC" bank-account="4747"></bank-account>
\end{lstlisting}
