\maketitle{}
\section{ @Output }

Output is another decorator native to Angular. On it's own it doesn't do
anything. However, it does allow for the three following things to happen:
\begin{enumerate}
  \item Marks a class field as an output property.
  \item Supplies configuration Metadata (we will get around to what that means
  soon)
  \item DOM property bound to the output property is automatically updated
  during change detection.
\end{enumerate}

The Output decorator is always used in tandem with an event listener. What this
means, is that we can have a re-usable component, with a button for instance.
By using Output, we can have a function within the dumb component, that whenever
it get's called, it triggers the parent function.

\subsection{ Example of @Output }

Here is a great example of what an Output is, and what it accomplishes. In
our scenario, we want to build a re-usable pxl-color-changer component:
\begin{lstlisting}
// pxl-code-changer.component.html

import { Component, EventEmitter, Input, Output } from '@angular/core';

@Component({
  selector: 'pxl-color-changer',
  template: './pxl-code-changer.component.html',
  styleUrls: ['./pxl-code-changer.component.scss'],
})
export class PxlCodeChanger implements OnInit {
  @Output() colorChanged = new EventEmitter<any>();
  constructor() {}

  changeColor(data: string) {
    this.colorChanged.emit(data);
  }
}
\end{lstlisting}

\begin{lstlisting}

\end{lstlisting}
