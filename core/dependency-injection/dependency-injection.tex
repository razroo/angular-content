\chapter{ Dependency Injection }

Dependency Injection from an Angular perspective, pragmatically only solves 
one real issue. That one issue is Unit Testing. That is, because in an 
Angular/Typescript setting a developer will have the ability to import and 
export a file. So there already is a way of decoupling different services/
classes from each other. 
\begin{lstlisting}[caption=passing in without dependency injection]
import { PostsFacade } from '@razroo/razroo/data-access/posts';  

@Component({
  selector: 'razroo-blog',
  templateUrl: './blog.component.html',
  styleUrls: ['./blog.component.scss']
})
export class BlogComponent implements OnInit {
  posts: any[];
  allPosts$: Observable<Post[]> = new PostsFacade().allPosts$;

  constructor() {}

  ngOnInit() {}

}
\end{lstlisting}
In the above, our class doesn't control how this value in injected. It 
goes straight from the import into our class. However, if we were to do 
something like this: 

\begin{lstlisting}[caption=passing in with dependency injection]
import { PostsFacade } from '@razroo/razroo/data-access/posts';  

@Component({
  selector: 'razroo-blog',
  templateUrl: './blog.component.html',
  styleUrls: ['./blog.component.scss']
})
export class BlogComponent implements OnInit {
  posts: any[];
  allPosts$: Observable<Post[]> = this.postFacade.allPosts$;

  constructor(private postsFacade: PostsFacade) {}

  ngOnInit() {}

}  
\end{lstlisting}

Now it is our \lstinline{BlogComponent} class that is controlling how the 
\lstinline{postsFacade} is getting passed through. This makes things 
particularly easy when it comes to unit testing. It allows us to override the 
value in our unit test, and create mocks for all services. \textit{We will discuss more
on this in the chapters in the chapters involving unit testing, but I just 
wanted to bring up the main reason behind unit testing here.}

Less obvious, and specifically if the \lstinline{providedIn} is used, is 
dependency injection also helps keep bundle sizes compact. This is done by tree
shaking, which refers to the compiler removing code from final app, if it is not
actually referenced by the app. If we do not use \lstinline{providedIn}, tree 
shaking will not be done. 

In addition, as we discussed above, it allows us to keep our configurations 
separate. While the ability to export/import is an option within Typescript, 
dependency injection, allows the framework to be completely aware of everything
being used within the framework. So, besides unit testing, this architecture 
can be useful for using tokens which contain a particular value, and then 
being overridden depending on environment of application(e.g. development vs.
production).

\section{ Real World Example }

\subsection{ Creating Injectable Service }
The DI(Dependency Injection) Framework, allows for an injectable service class
to be passed to a component. For instance, let's say that we want to create an
injectable service, we would do the following:
\begin{lstlisting}
import { Injectable } from '@angular/core';

@Injectable({
  providedIn: 'root',
})
export class PxCodeService {
  constructor() { }
}
\end{lstlisting}

This code right here is doing two things:
\begin{enumerate}
  \item It is saying that this service is an Injectable.
  \item It is saying that this injectable should be provided in the root(aka the
  AppModule).
\end{enumerate}

\mybox{Now would be a good time to discuss what the \lstinline{providedIn}
property does. It accomplishes two things:
\begin{enumerate}
  \item Angular creates a single, shared instance of the service and injects it
  into any class that asks for it. So, there is no need to insert it as a
  provider for your module, and you can simply pull it into your class whenever
  you want. \footnote{Don't worry if you are not familiar with how to use a
  service in your class, we will get to that soon.}
  \item It also allows Angular to optimize an app, by removing the service from
  the compiled app, if it isn't used. \footnote{Taken from documentation for
  Angular.}
\end{enumerate}
}

\subsection{ Including Injectable Service in Component }
Now if we would like to include this service in our component, we would do the
following:
\begin{lstlisting}
// code-box.component.html
<div *ngFor="let codeBox of codeBoxes">
  {{codeBox.data}}
</div>
\end{lstlisting}

\begin{lstlisting}[caption=Include Injectable Service in Component]
// code-box.component.ts
import { Component }   from '@angular/core';
import { CodeBox }        from './code-box.interfaces';
import { PxCodeFacade } from './px-code.facade';

@Component({
  selector: 'px-code-box',
  template: './code-box.component.html',
  styles: ['./code-box.component.scss'],
})
export class HeroListComponent {
  codeBoxes: CodeBox[];

  constructor(pxCodeFacade: PxCodeFacade) {
    this.codeBoxes = pxCodeFacade.getCodeBoxes();
  }
}
\end{lstlisting}

\mybox{Now would technically be a good time on discussing how to test these
mocked dependencies, however it get's a little bit complicated. How to mock 
dependencies, will be discussed in the section on unit testing.}

\section{Services that need other services}
Services can have their own dependencies. If we wanted to inject a service, into
our service, it would be as simple as doing the following:
\begin{lstlisting}
import { Injectable } from '@angular/core';
import { Logger } from '../logger.service';

@Injectable({
  providedIn: 'root',
})
export class PxCodeService {
  constructor(private logger: Logger) { }

  getLog() {
    this.logger.log('getting codeboxes');
  }
}
\end{lstlisting}

As we can see in the above, our injected service, is taking another injected
service. By simply passing it into the constructor, similar to how we do for our
components, we can use it within our application.

\section{ Dependency Injection Token }
Internally Angular uses dependency injection tokens for injectable services, to
reference what injectable it is using. As an Angular developer, we also have the 
option to use these tokens directly within our app. There are two different 
ways of providing tokens in Angular:
\begin{enumerate}
  \item Strings 
\begin{lstlisting}  
{provide: 'EmailService', useClass: MandrillService}
\end{lstlisting}  
  \item Type Tokens
\begin{lstlisting}    
{provide: EmailService, useClass: MandrillService}
\end{lstlisting}  
  \item Injection Tokens
\begin{lstlisting}   
new InjectionToken{provide: EmailService, useClass:} 
\end{lstlisting}   
Benefit of this approach, is that it avoids name clashes.
\end{enumerate}

Tokens can be a useful way for providing a default value to be used across the 
app, hijacking the value, and providing something else, in the 
scenarios where it is not of use. An example of such, is within the Angular 
Material Components. By default, the \lstinline{MAT_DATE_LOCALE} will use 
the existing \lstinline{LOCALE_ID}. However, if you would like to override the 
date locale across the app, all that needs to be done, is to overrride the 
\lstinline{MAT_DATE_LOCALE} token. 
\begin{lstlisting}
@NgModule({
  providers: [
    {provide: MAT_DATE_LOCALE, useValue: 'en-GB'},
  ],
})
export class MyApp {}
\end{lstlisting}

This allows for numerous components across the app, using the same token to 
be overridden. 

\section{Wrapping Up}
The intent of this chapter, is to introduce the concept of dependency injection 
in Angular. In addition, present the scenarios in which it is used in a regular 
enterprise application. Dependency injection as a pattern can be very complex, 
and I feel looking at the documentation, the majority of use cases tend not 
to be used. So I wanted to keep this simple. Unit testing, the other widely 
used part of dependency injection, will be discussed in a separate chapter on
unit testing. 
