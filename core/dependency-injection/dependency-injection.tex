\section{ Dependency Injection }
\maketitle{}

Dependency Injection on it's own a important design pattern. It solves the
following problems in any scenario:
\begin{enumerate}
  \item How can a class be independent of how it's objects are created.
  \item How can way objects are create be specified in different configuration
  files.
  \item How can an application support different configurations?
\end{enumerate}

Angular has adopted dependency injection as part of it's framework since the
inception of AngularJs, and has remained throughout Angular. Immediately, it
becomes obvious that it is extremely useful for unit testing. Less obvious, but
something somewhat intuitive as well, is that it helps keep bundle sizes
compact. In addition, as we discussed above, it allows us to keep our
configurations separate. While the ability to export/important is an option
within Typescript, dependency injection, allows the framework to be completely
aware of everything being used within the framework.
