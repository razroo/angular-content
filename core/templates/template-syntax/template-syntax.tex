\maketitle{}
\section{ Template Syntax }

\subsection{ Difference between Property and Attribute in HTML }

First, and foremost, it is important to understand the difference between HTML 
attributes and properties. I know myself, when writing this chapter, I realized
that I did not fully understand the difference between an HTML attribute, and
an HTML property. So I thought, why no re-iterate here, and hopefully it will
make this subject a bit easier to understand. First let's dive into what the 
computer science definition of a property, and an attribute would be. 

Property - Something that can be read and written. Within a typescript setting, 
this would be something that would be translated into a \lstinline{get} and  
\lstinline{set} within Typescript. 

Attribute - More correctly should be considered as a metadata. Something that
is a property of a property, describing what the parent property is doing. 

In HTML, the above definitions are a bit obscured. For instance, let's say
that we are defining the type of input field, as well as it's value. 
\begin{lstlisting}
<input type="text" value="Name:">  
\end{lstlisting}

The \lstinline{type} and \lstinline{value} are attributes, as they are 
explaining what the \lstinline{input} property is doing (aka metadata).
However, once the browser parses the code, it will turn it into an 
\lstinline{HTMLInputElement} object. This contains dozens of properties, like 
\lstinline{className}, \lstinline{clientHeight}, and methods, such as 
\lstinline{click()}. The browser will create a new sort of property based on 
the type of native html element it is creating. 

\subsection{ Property Binding (In Angular) }
In Angular, property binding is a way to set properties of a particular 
element. In addition, it is a way to set \lstinline{@Input()} decorators set 
on the actual directive. 

\begin{lstlisting}
<img [src]="itemImageUrl">
\end{lstlisting}

You might be wondering why this is called property binding instead of 
attribute binding. \lstinline{src} is actually an attribute. The reason behind
this, is that Angular's engine will actually first initialize the component, 
and then change the property set by the browser. In addition, Angular will 
compile the component, as an object, and therefore internally is setting 
objects, i.e. properties. I just thought it was a fun fact, and is actually 
what led me to research the above re: attributes and properties. 


\mybox{\lstinline{<script>} tags in Angular, unlike modern html are forbidden!
It will result in an error. This is done to prevent from script injections.}

