\maketitle{}
\section{ Displaying Data }

In Angular, one of the nicer things about the migration from AngularJS to 
Angular, was that many html bindings felt at home. If you are someone who like
me, had the ability to experience Angular, after working with AngularJS, then 
templates feel entirely intuitive. In fact, I actually had a difficult time 
writing this chapter, because I almost overlooked the fact, that to many,
displaying data in an Angular setting can be counter intuitive.

However, I still think that it rings true, that templates in Angular, are one 
of the easier topics to grasp in Angular. Granted, and this is a very important 
point, granted that the syntax is properly explained. Sometimes, when learning 
syntax for the first time, there's so much there, that it's easy to overlook 
reasons behind syntax. Perhaps we assume the reason there are different symbols,
letters, etc. for the different ways of operating within a framework, is
because we just need to represent the different ways of doing a particular 
action within a framework. I think it goes without saying, that it would be 
valuable as a matter of documentation to mention how component and Templates 
work within Angular. However, this book will also make sure to go into the 
syntax, and offer why it looks like the way it does. 

\subsection{ Interpolation }

Interpolation in the dictionary, means inserting something of a different 
nature into something else. In the context of Angular, interpolation means
being able to place a javascript expression, in your html. This is signified 
by the double curly brace. 
\begin{lstlisting}[caption=interpolation-example.component.html]
This is 2 + 2
<!-- This is 4 --> 
\end{lstlisting}

\subsection{ Angular Components }
In an Angular setting, based on best practices, and those set into place by 
the Angular CLI, the generation of the component, will consist of four files: 
\begin{enumerate}
  \item *.component.ts
  \item *.component.spec.ts
  \item *.component.html
  \item *.component.scss
\end{enumerate}

In the above files, html and component files will tend to interact with each 
other the most. For any component file, there is atleast going to be one, or 
more properties placed in the respective component file. 
\begin{lstlisting}[caption=header.component.ts]
import { Component, OnInit } from '@angular/core';

@Component({
  selector: 'pxl-header',
  templateUrl: './header.component.html',
  styleUrls: ['./header.component.scss']
})
export class HeaderComponent implements OnInit {
  @Input() title: string;
  constructor() { }

  ngOnInit() {}
} 
\end{lstlisting}

\begin{lstlisting}[caption=header.component.html]
<h1>{{title}}</h1>
\end{lstlisting}

Here, we are interpolating the \lstinline{title} property into our html, by 
using double curly braces around the value. This is classic syntax in an 
Angular setting. 
\mybox{ The first usage of curly braces in programming languages was by a 
language called BCPL. It was created by Martin Richards. Instead of the use of
\lstinline{BEGIN} and {END} implemented by other languages, he used 
\lstinline{\$( \$)}. It was a natural extension of parenthesis, which is all
that keyboards offered at that time. In 1967, curly braces, started making their 
way into mainstream keyboards. 

The B Language developed by Ken Thompson, and later adopted the C language, was 
officially made avialable in 1969. Brackets were already used for arrays. Curly
 braces was a natural extension to \lstinline{\$( \$)}, as it was easier to 
 write, and now available in keyboards. So C, which was a natural extension of 
 the B language, and the gateway to most modern programming languages, used 
 curly braces to group statements(i.e. actions to carried out by programming 
 language). 
  
Angular, in a very similar way uses curly braces to represent statement 
in it's html. Anything in the double curly braces \lstinline{\{\{\}\}}, can be
regular Javascript code. 
The reason that angular uses a double curly brace, is because:
\begin{enumerate}
  \item A singular curly brace in Javascript represents an object
  \item We would come across the awkward situation, wherein a curly brace in 
  regular html as a string, would require the need for code to be escaped. 
  (Something I've experienced with other programming languages, and it is 
  indeed really awkward.)
\end{enumerate} 
}

\subsection{ Displaying an Array within an HTML Template }
Quite a common occurence within any application, is thay the data, will return
an array of objects(AKA collection), to be consumed by the app. Many times
there will be a need to iterate over the array with the html, so that the data
objects can be accessed. 
\begin{lstlisting}[caption=px-code-box.component.ts]
import { Component, OnInit } from '@angular/core';

@Component({
  selector: 'px-code-box',
  templateUrl: './code-box.component.html',
  styleUrls: ['./code-box.component.scss']
})
export class CodeBoxComponent implements OnInit {
  @Input() cssCode: string[];
  constructor() { }

  ngOnInit() {}
} 
\end{lstlisting}

\begin{lstlisting}[caption=px-code-box.component.ts]
  import { Component, OnInit } from '@angular/core';
  
  @Component({
    selector: 'px-code-box',
    templateUrl: './code-box.component.html',
    styleUrls: ['./code-box.component.scss']
  })
  export class CodeBoxComponent implements OnInit {
    @Input() cssCode: string[];
    constructor() { }
  
    ngOnInit() {}
  } 
  \end{lstlisting}