\maketitle{}
\section{ Displaying Data }

In Angular, one of the nicer things about the migration from AngularJS to 
Angular, was that many html bindings felt at home. If you are someone who like
me, had the ability to experience Angular, after working with AngularJS, then 
templates feel entirely intuitive. In fact, I actually had a difficult time 
writing this chapter, because I almost overlooked the fact, that to many,
displaying data in an Angular setting can be counter intuitive.

However, I still think that it rings true, that templates in Angular, are one 
of the easier topics to grasp in Angular. Granted, and this is a very important 
point, granted that the syntax is properly explained. Sometimes, when learning 
syntax for the first time, there's so much there, that it's easy to overlook 
reasons behind syntax. Perhaps we assume the reason there are different symbols,
letters, etc. for the different ways of operating within a framework, is
because we just need to represent the different ways of doing a particular 
action within a framework. I think it goes without saying, that it would be 
valuable as a matter of documentation to mention how component and Templates 
work within Angular. However, this book will also make sure to go into the 
syntax, and offer why it looks like the way it does. 

\subsection{ Interpolation }

Interpolation in the dictionary, means inserting something of a different 
nature into something else. In the context of Angular, interpolation means
being able to place a javascript expression, in your html. This is signified 
by the double curly brace. 
\begin{lstlisting}[caption=interpolation-example.component.html]
This is 2 + 2
<!-- This is 4 --> 
\end{lstlisting}
