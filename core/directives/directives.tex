\section{ Attribute Directives }
\maketitle{}

An attribute directive changes the appearance, or behavior of a DOM Element.
It is tagged on of an html element to change the way it works. For instance,
while it is probably better to use CSS in this situation, let's create a
really low level directive, to introduce how it works:
\begin{lstlisting}
import { Directive, ElementRef } from '@angular/core';

@Directive({
  selector: '[appHighlight]'
})
export class HighlightDirective {
  constructor(el: ElementRef) {
    el.nativeElement.style.backgroundColor = 'yellow';
  }
}
\end{lstlisting}

Now we have the ability to apply this directive to out html element:
\begin{lstlisting}
<p appHighlight>Highlight me!</p>
\end{lstlisting}

As a result of the directive we have applied on this p element, the background
for this p element will now yellow.

It, of course, can be incredible useful.

\mybox{
 A @Directive is not to be confused with a @Component. A @Component requires
 html, wherein a @Directive does not use html. A @Directive is meant to be
 tagged onto html, and modify it's behavior.
}

\subsubsection{A Great Example}
A great example of this, is adding drag and drop functionality to a component.
That in it's self is not component worthy. However, adding a dropzone directive
for instance, to the element, would make any potential component drag and
droppable worthy. A

A great example of a component, would be a data table. A large part of it's
functionality is strickly tied to actual UI element.

\mybox{
  A side note with regards to our architecture, directives will be very used. If
  you find yourself writing, with what feels like too many components, don't
  worry, it's to be expected.
}
