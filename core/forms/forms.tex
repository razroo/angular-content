\maketitle{}
\section{ Forms }

Forms to this day, I think is still one of the most complicated pieces of UI,
you will ever come across. Each input in a form, has a unique piece of 
functionality to it. It can be one of the most fustrating pieces of UI, 
because there is no way to make it truly re-usable as a result. In 
addition, it can be very fustrating to explain to business as to why it is 
taking so long, as a form seems truly simple to make. Over the course of this 
book we will discuss why it is that, that is the case, as well. In addition,
we will discuss ways that we can alleviate the pain of forms. In the meantime 
let's discuss the fundementals of forms. 

Angular offers two ways of handling forms: 
\begin{enumerate}
  \item Reactive Forms 
  \item Template-driven Forms
\end{enumerate}

\subsection{ Common Foundation of Forms }
\begin{enumerate}
  \item \lstinline{FormControl} - Tracks the value and validation state of an 
  individual form control.
  \item \lstinline{FormGroup} - Tracks the same values and status for a 
  collection of form controls.
  \item \lstinline{FormArray} - Tracks the same values and status for an array
  of form controls. 
  \item \lstinline{ ControlValueAccessor } - Creates a bridge between Angular
  FormControl instances, and native DOM elements.
\end{enumerate}

\subsection{ Reactive Forms }
First and foremost, it's important to get a really handle on what Reactive 
means. Reactive means that your code is:
\begin{enumerate}
  \item Asynchronous - An action happens, and something happens right after. 
  Even though this can happen in regular javascript code, ideally if 
  something is asyncrhonous, it is baked into the framework, making it 
  more of a default.
  \item Deterministic -  Given action a happens, always action b, will 
  happen. Reactive forms do this, wherein a user can be sure a certain 
  thing will happen granted an input is affected.
\end{enumerate}


