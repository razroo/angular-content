\maketitle{}
\section{ Reactive Forms }

Reactive forms are extremely under-rated. As we have discussed before, there are 
many reasons as to why forms are very complicated. Reactive forms are no 
exception to that rule. However, we will discuss them all now. In additionm we
are going to run through all the steps to get from point a to point b, so that
you can constantly reference this, as you create a new reactive form component.

\subsection{ Registering Reactive Forms }

Importing a reactive module, is no different than your regular module, however,
this is the module to use to when importing reactive forms. 

\begin{lstlisting}
import { ReactiveFormsModule } from '@angular/forms';

@NgModule({
  imports: [
    // other imports ...
    ReactiveFormsModule
  ],
})
export class AppModule { }
\end{lstlisting}

\subsection{Generating a component, and adding FormControl}
No different than any other scenario: 
\begin{verbatim}
ng generate component grid-form 
\end{verbatim}

In your component simply add a new \lstinline{FormControl}: 

\begin{lstlisting}
import { Component } from '@angular/core';
import { FormControl } from '@angular/forms';

@Component({
  selector: 'px-grid-form',
  templateUrl: './grid-form.component.html',
  styleUrls: ['./grid-form.component.css']
})
export class NameEditorComponent {
  size = new FormControl('');
}
\end{lstlisting}

\subsection{Registering Control in Template}
\begin{lstlisting}
<label>
  Name:
  <input type="text" [formControl]="name">
</label>
\end{lstlisting}

As mentioned in the previous chapter, FormControl, will allow you to access 
value of form within component, and view. Most importantly, update in view 
or component, and have it affect the other. 

\subsection{ Displaying Component }
We can now include the component in any other component. E.g. 
\begin{lstlisting}[caption=app.component.html]
<px-grid-form></px-grid-form>  
\end{lstlisting}

\subsection{ Grouping Form Controls }
A FormControl on it's own has value. Primarily being able set a value, and
accesing it within the template. However, a \lstinline{formControl} is 
incomplete without a \lstinline{formGroup}. A formGroup, will give us 
access to all of the formControl values, so we can use them all, when 
submitting a form. 

