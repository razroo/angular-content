\section{ Routing }
\maketitle{}

Routing is an integral part of any single page application:
\begin{displayquote}
\say{It allows a user to navigate from one view to the next, as a user performs
application tasks.} \footnote{Angular Documentation - Routing \& Navigation https://angular.io/guide/router}
\end{displayquote}
The idea is that routing is that it is it's own internal state machine. There
are two things that are unique to state with regards to routing:
\begin{enumerate}
  \item Data to be pulled in based on page.
  \item UI to be shown based on page.
\end{enumerate}

\subsection{The Basics}
\subsubsection{Base Href}
In any Angular application, there is going to be an initial point of entry for
routing. First and foremost, in your src/index.html you will need to add an
\lstinline{<base href="/">}. This is added by the CLI by default, and not
something you have to worry about.

\subsubsection{ RouterModule }
Routes in Angular, are singleton instance.

\begin{lstlisting}
import { NgModule } from '@angular/core';
import { RouterModule } from '@angular/router';
import { AppComponent } from './app.component';

@NgModule({
  imports: [
    RouterModule.forRoot( [
      {
        path: '',
        component: AppComponent
      }
      {
        path: 'draw',
        component: GridComponent
      }
    ], { initialNavigation: 'enabled' })
  ],
})
export class PixelAngstAppRoutingModule {}
\end{lstlisting}
