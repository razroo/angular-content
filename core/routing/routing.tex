\section{ Routing }
\maketitle{}

Routing is an integral part of any single page application:
\begin{displayquote}
\say{It allows a user to navigate from one view to the next, as a user performs
application tasks.} \footnote{Angular Documentation - Routing \& Navigation https://angular.io/guide/router}
\end{displayquote}
The idea is that routing is that it is it's own internal state machine. There
are two things that are unique to state with regards to routing:
\begin{enumerate}
  \item Data to be pulled in based on page.
  \item UI to be shown based on page.
\end{enumerate}

\subsection{Base Href}
In any Angular application, there is going to be an initial point of entry for
routing. First and foremost, in your src/index.html you will need to add an
\lstinline{<base href="/">}. This is added by the CLI by default, and not
something you have to worry about.

\subsection{ RouterModule }
Routes in Angular, are singleton instance.

\begin{lstlisting}[caption=app.module.ts file]
import { NgModule } from '@angular/core';
import { RouterModule } from '@angular/router';
import { AppComponent } from './app.component';

@NgModule({
  imports: [
    RouterModule.forRoot( [
      {
        path: '',
        component: AppComponent
      },
      {
        path: 'draw',
        component: PxGridComponent
      },
      {
        path: '**',
        component: PageNotFoundComponent
      }
    ], { initialNavigation: 'enabled' })
  ],
})
export class PixelAngstAppRoutingModule {}
\end{lstlisting}

As we can see in the above, we are supplying all routes within the app routing
module.

\subsection{ RouterModule Options }
Let's break down all the possible options that can be passed to the router:
\begin{enumerate}
\item url
\begin{lstlisting}
{
  path: 'draw',
  component: GridComponent
},
\end{lstlisting}
Here we are using the url for draw. So, for instance, let's say the url
of our application is razroo.com, then razroo.com/draw, will display
the grid component.
\item id
Many times within our backend, we are going to retrieve data, based on the 
user's id. Alternatively, it might also be the id for a specific api. Being
able to tie in the id for that particular api, into the route is very 
powerful. Angular routing allows for this to happen: 
\begin{lstlisting}
{
  path: 'hero/:id', 
  component: HeroDetailComponent 
},
\end{lstlisting}
The syntax of colon, following by text(does not have to be id), means that 
if we were to navigate to \lstinline{razroo.com/draw/123}, it would 
register within our app, that we want to call the id of '123', within
the draw route. 

Within our app, we are going to use this, so that we can use the custom 
pixel illustrator settings, that our user opted into. 
\item data - Allows us to place static data, that we can retrieve that is 
specific to the route. E.g. page titles, breadcrumb text, other read-only 
static data.
\begin{lstlisting}
{
  path: 'css', 
  component: cssComponent.
  data: { title: 'CSS' }
},
\end{lstlisting}
\item empty path - An empty path is our default route. I.e. when the app 
loads for the first time. 
\begin{lstlisting}
{
  path: '', 
  component: HeroDetailComponent 
},
\end{lstlisting}
\item \lstinline{**} path
Two asterisks means that the route is a wildcard. It is particularly 
advantageous for error reporting: 
\begin{lstlisting}
{
  path: '**',
  component: PageNotFoundComponent 
}
\end{lstlisting}
\end{enumerate}

\mybox{ The order of the routes in the configuration matters. This is intentional
so that specific routes can be matched first. More generic routes, such as the 
wild card (\lstinline{**}) can therefore be matched, without obstructing 
other routes.}

\subsection{ Router Outlet }
\begin{displayquote}
\say{Router outlet is a directive from the router library that acts like a 
component. It marks the spot in the template where the router should 
display the components for that outlet. } 
\footnote{Angular Documentation - Routing \& Navigation https://angular.io/guide/router}
\end{displayquote}

\begin{verbatim}
<router-outlet></router-outlet>
<!-- Routed components go here -->
\end{verbatim}

Let's say now we were to go to razroo/com/draw, the component for the 
\lstinline{draw} route will be placed as a sibling component i.e. 
\begin{lstlisting}
<router-outlet></router-outlet>
<px-grid></px-grid>  
\end{lstlisting}

\subsection{ Router Links }
In order to actually navigate from one route to the next, you will 
need to use the Angular equivalent of href. However, instead of 
the classic functionality of href, routerLink, will instead 
reload the component, based on the new url. 
\begin{lstlisting}
<h1>Px Illustrator</h1>
<nav>
  <a routerLink="/draw">Draw</a>
</nav>
<router-outlet></router-outlet>
\end{lstlisting}

\subsubsection{Active Router Links}
In addition, Angular offers a way for determining what is the current 
active link. Something that is very valuable from a UX perspective 
when the app needs to show to the user, what menu item is 
currently selected. 

\begin{lstlisting}
<h1>Px Illustrator</h1>
<nav>
  <a routerLink="/draw" routerLinkActive="active">Draw</a>
</nav>
<router-outlet></router-outlet>
\end{lstlisting}

Now if the draw route is triggered, the class active will 
be added to the <a> tag. The \lstinline{.active} class 
can obviously be styled.

Multiple active router link classes can be added for a 
particular active route as well: 

\begin{lstlisting}
<h1>Px Illustrator</h1>
<nav>
  <a routerLink="/draw" routerLinkActive="'active '">Draw</a>
</nav>
<router-outlet></router-outlet>
\end{lstlisting}

\subsection{ Router State }
After each successful navigation lifecycle, Angular's internal system 
updates what's called the \lstinline{ActivateRoute} object. This can be
accessed by using the Router service. Inside of the router service, by 
accessing the routerState property, we can get to the plethora of 
properties.

\mybox{As this is a chapter on fundamentals, we will not delve 
into all the details now. However, it is important to know that it does
exist.}

\subsection{ Router Events }
When a route get's triggered, Angular will internally trigger a series 
of events, from when the navigation starts to where it ends. Angular 
also exposes these series of events by using the \lstinline{Router.events}
property. Once again, there is no need to go into all of the events, but 
at this time, they total 17. We will discuss in future chapters situations 
where the need for accessing the routing events can be beneficial. 
