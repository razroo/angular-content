\chapter{ Pipes }

Angular offers the ability to use Pipes out of the box. The idea behind a pipe is to get data, transform it, and show new transformed data to users. 

Just as a primer, using the native angular date pipe, a pipe would do something such as the following:
\begin{lstlisting}
The chained hero's birthday is
{{ birthday | date | uppercase}}
\end{lstlisting}

This would display
\begin{verbatim}
FRIDAY, APRIL 15, 1988
\end{verbatim}

The pipe here is taking in the timestamp of
\begin{verbatim}
  577065600
\end{verbatim}
and converting it to the proper date. Having such dates in one's application makes it very easy to go ahead and transform data throughout one's app. Here we are also chaining pipes, so that the transformed data that comes back in addition to being transformed, is also capitalized.

\section{ Performance Considerations }
It is important off the cuff to be aware of some performance concerns when it comes to pipes.

\subsection{ Understanding Angulars Change Detection }
This is a good time to interject and get into Angular's change detection. Change detection in Angular works from the top down. That is, if a specific set of data changes within a component, then the entire component will update as a result of new data. 

Angular pipes change how that it is done by changing content directly on the object, and only updating that one specific part. By using pipes, it allows us to increase the component's performance. 

\section{ When to Use Pipes }
Pipes cover a lot of ground. Within our architecture, one of the pipes that will be used more so that others in the async pipe. However, now is not the place to put that here. What is important, is that pipes can be used transform data.Should they always be used whenever one is transforming data?

I like to think of pipes as unique to html. They have a way of dealing of performance when working with html templates. However, within a component itself it is questionable. To be honest this can go either way. From a maintainability perspective, whenever data is being transformed, it should be turned into a pipe, instead of a service. That would be the simple rule.

In addition, there are impure pipes that one can potentially do, which ususally has little to do with transforming data. Once again the async pipe is something that would register along these lines. However, I have not seen the need to create an impure pipe with the architecture given. 
