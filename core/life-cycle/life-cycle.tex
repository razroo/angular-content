\maketitle{}
\section{ Life Cycle Hooks }

Any framework these days is going to have a lifecycle hook. All of them are
actually very similar in many ways. There's a component has instantiated,
and in between someplace, as in about to change, and is undergoing change, or
what not. There is also, always a part in the lifecycle where the component
is being destroyed/is destroyed. Angular is no different. I would like to
go through the different part's of Angular's life cycle hooks, as it definitely
is very important when it comes to development. I would also like to discuss the
top 3 most important life cycle hooks when it comes to Angular development.

\subsection{ Lifecycle Example }
Before we get into what the entire lifecycle is, it might be helpful to
visualize a lifecycle hook:

\begin{lstlisting}
export class PeekABoo implements OnInit {
  constructor(private logger: LoggerService) { }

  // implement OnInit's `ngOnInit` method
  ngOnInit() { this.logIt(`OnInit`); }

  logIt(msg: string) {
    this.logger.log(`#${nextId++} ${msg}`);
  }
}
\end{lstlisting}

The above code is hooking into the onInit lifecycle hook. (Hook is exactly what
it sounds like. Angular will hook into that particular part of the lifecycle,
and implement a certain piece of code). When the component is initialized,
it will log out a certain message. OnInit from personal experience is the most
used lifecycle hook, so something to keep in mind.

\subsection{ Angular Lifecycle }
At this time, Angular has eight lifecycle hooks, in this order, more, or less:
\begin{enumerate}
  \item ngOnChanges() - Triggered whenever Angular sets, or resets the data-bound
  input properties. It is called before ngOnInit, and whenever one, or more
  data-bound input properties change.
  \item ngOnInit() - This one was already featured in the code above! It get's
  called after Angular display data-bound properties, and set's the directive,
  or component's input properties.
  \item ngDoCheck() - This is called after an ngOnChanges, or ngOnInit. This
  was created, so that Angular can check on updates it won't check on it's own.
  \footnote{Example of what that would look like should go here.}
  \item ngAfterContentInit() - Triggered after html is populated. It is called
  once after the first ngDoCheck().
  \item ngAfterContentChecked() - After content in html is checked by Angular,
  this will be called. Called after ngAfterContentInit() and every ngDoCheck()
  thereafter.
  \item ngAfterViewInit() - Triggered after not only view for component is
  initialized, but child view is initalized as well. For a directive, will
  trigger once view it is in, will initialize.
  \item ngAfterViewChecked() - Responds after Angular checks the component's
  views and child views / the view that a directive is in. Called after the
  ngAfterViewInit() and every subsequent ngAfterContentChecked().
  \item ngOnDestroy() - Cleanup just before Angular destroys the
  directive/component. Unsubscribe Observables and detach event handlers to
  avoid memory leaks.
\end{enumerate}

\subsection{ Three Lifecycles Used Most Often }
The three lifecycles that are used most often are:
\begin{enumerate}
  \item ngOnChanges()
  \item ngOnInit()
  \item ngOnDestroy()
\end{enumerate}

I would like to explain why. When a component initializes, usually we subscribe
to some data that we have(if not familiar with subscriptions no worries, will
get to that soon.). Sometimes, if we are working with a graphical compoent, for
instance, like a chart, we would like to update the component whenever we get
new data passed into our input. In addition, subscriptions that we pass in from
the outside, mainly using state, will still stay around, and soak up our web
application's memory. So, it is also quite a common occurence to use ngOnDestroy()
to manually destroy subscriptions.
