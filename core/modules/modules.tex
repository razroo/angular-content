\chapter{ Modules }

Modules are an integral part of Angular. It's interesting, because in some other languages, modules are a part of the language(OCaml comes to mind). A module in a language like OCaml, is any code contained within a file. It's similar in TypeScript. The module exposes objects meant to be public by using the export keyword. In Angular, a module, is completely unrelated to the system of modules used by Typescript are complementary. It is simply a decorator attached to a class that ultimately get's bundled together into a single class, for use with the app.

\section{The Module Four, and the Fifth Wheel}
Angular provides five key/values to be used with a module:
\begin{lstlisting}
import { NgModule }      from '@angular/core';
import { BrowserModule } from '@angular/platform-browser';
@NgModule({
  imports:      [ BrowserModule ],
  providers:    [ Logger ],
  declarations: [ AppComponent ],
  exports:      [ AppComponent ],
  bootstrap:    [ AppComponent ]
})
export class AppModule { }
\end{lstlisting}

\begin{enumerate}
  \item imports - This takes in other modules, whose exported classes, are needed by this component. This can include, providers, declarations, exports, and imports, contained within the other module.
  \item providers - This is used to put all services used by components within this module. It should be noted, that as of Angular 6, the option to use \lstinline{forRoot} was introduced. It greatly decreases the need for using providers, but it very much so still has it's place in Angular. Especially for unit tests.
  \item declarations - Components, directives, and pipes that belong to the NgModule are put here.
  \item exports - The components that should be able to be used in templates outside of this module, when this module is imported by other modules.
  \item bootstrap - The main application, i.e, the root component, which hosts all other app views. Only the root NgModule sets the bootstrap property,
  which is usually handled by the CLI/Nx. I consider bootstrap as the fifth wheel when it comes to modules for bootstrap.
\end{enumerate}

Note this is not put in alphabetical order. Rather it is put in the order that the CLI tends to order them. It is this way, because it is the order
in which these items are used within the NgModule declaration.
