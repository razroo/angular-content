\section{ Modules }
\maketitle{}

Modules are an integral part of Angular. It's interesting, because in some
other languages, modules are a part of the language(OCaml comes to mind). A
module in a language like OCaml, is any code contained within a file. In Angular
a module, is a decorator attached to a class. Angular provides five key/values
to be used with a module:
\begin{lstlisting}
import { NgModule }      from '@angular/core';
import { BrowserModule } from '@angular/platform-browser';
@NgModule({
  imports:      [ BrowserModule ],
  providers:    [ Logger ],
  declarations: [ AppComponent ],
  exports:      [ AppComponent ],
  bootstrap:    [ AppComponent ]
})
export class AppModule { }
\end{lstlisting}

\begin{enumerate}
  \item imports - This takes in other modules, whose exported classes, are
  needed by this component. This can include, providers, declarations, exports,
  and imports, contained within the other module.
  \item providers - This is used to put all services used by components within
  this module. \footnote{It should be noted, that as of Angular 6, the option
  to use \lstinline{forRoot} was introduced. It greatly decreases the need for
  using providers, but it very much so still has it's place in Angular.
  Especially for unit tests.}
  \item declarations
  \item exports
  \item bootstrap
\end{enumerate}
\footnote{Note this is not put in alphabetical order. Rather it is put in the
order that the CLI tends to order them. It is this way, because it is the order
in which these items are used within the NgModule declaration.}
