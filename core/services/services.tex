\chapter{ Services }


First and foremost what is a service in Angular? In Angular, a service is a way
to define business logic in a seperate file, and choose it in our appropriate 
component when it makes sense to do so. Services, even though they are a part 
of the core Angular framework, are by definition created in order to alleviate 
the maintainability and scalability of an application. The bulk of discussion 
around services will be discussed in later chapters. However, it is still a 
part of core Angular, and it is important to be aware of couple things. 

\section{ Creating a Service and ProvidedIn: Root }
In order to create a service, navigate to the folder you would like to create 
your service, and run: 
\begin{lstlisting}
ng g service code-box  
\end{lstlisting}

This will create a service that by default includes 
\lstinline{ providedIn: 'root',} It is important to keep in mind, that Angular
provides for services only, the  ability to add to the constructor something 
called \lstinline{ providedIn: 'root',}. In short, it allows for the service 
to injected, without the need to include it in the respective module. It has 
some performance boosts, besides level of convenience. We will get more into 
that in later chapters. 

\section{ What Generally Goes in a Service }
A service generally deals with data. This means making data requests, and 
passing that data into our component. Razroo reccomends using Apollo Client
instead of Angular's internal http client, due to the fact that we strongly 
suggest using GraphQL for our application. This is what a typical service 
would look like: 
\begin{lstlisting}
import { Injectable } from '@angular/core';

import { Observable, from } from 'rxjs';
import { pluck } from 'rxjs/operators';
import { Apollo } from 'apollo-angular';

@Injectable({ providedIn: 'root' })
export class UserService {
  getUser(): Observable<User> {
    const user$ = this.apollo.query({ query: GetCurrentUser });

    return from(user$).pipe(pluck('data', 'getCurrentUser'));
  }
  constructor(private apollo: Apollo) {}
}
\end{lstlisting}

In the above code, you will notice that we have created a \lstinline{getUser()}
method for our \lstinline{UserService}. By doing this, we have separated logic 
from our component, and allow for our code to be more re-usabled. For instance, 
if we want get the user data, we can simply include the service in the 
appropriate component, and use the data as needed. 

\section{Including Service In Our Component}
If we would like to use this service in our component, we can simply inject 
the appropriate service in our constructor. 

\begin{lstlisting}
// code-box.component.ts
import { Component }   from '@angular/core';
import { User }        from './code-box.interfaces';
import { UserFacade } from './user.facade';

@Component({
  selector: 'px-code-box',
  template: './code-box.component.html',
  styles: ['./code-box.component.scss'],
})
export class HeroListComponent {
  user: User;

  constructor(userFacade: UserFacade) {
    this.user = userFacade.getUser();
  }
}
\end{lstlisting}

You will notice in the above we included something called the UserFacade. We will 
get to this in later chapters. Per our architecture, services should always be 
fed through the facade file. Regardless, for now, what should be kept in mind, 
is that including service in the contructor, is all that is needed to consume 
the service. 

\section{Ending Off}
I would just like to end off saying that actually creating a service is 
actually quite simple. It's a piece of architecture, so arguably from an 
architectural perspective, it is one of the more complex pieces of Angular. 
However, to create and start using, is a relatively seamless process.
