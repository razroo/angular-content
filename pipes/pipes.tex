\section{ Pipes }
\maketitle{}

Angular offers the ability to use Pipes out of the box. The idea behind a pipe
is to get data, transform it, and show new transformed data to users. Pipes are
something that I am going to assume the reader is already familiar with. Just as
a primer, using the native angular date pipe, a pipe would do something such
as the following:
\begin{lstlisting}
The chained hero's birthday is
{{ birthday | date | uppercase}}
\end{lstlisting}

This would display
\begin{verbatim}
FRIDAY, APRIL 15, 1988
\end{verbatim}

The pipe here is taking in the timestamp of
\begin{verbatim}
  577065600
\end{verbatim}
and converting it to the proper date. Having such dates in one's application
obviously makes it very easy to go ahead and transform data throughout one's
app. Here we are also chaining pipes, so that the transformed data that comes
back in addition to being transformed, is also capitalized.

\subsection{ Performance Considerations }
It is important off the cuff to be aware of some performance concerns when it
comes to pipes.

\subsubsection{ Understanding Angulars Change Detection }
This is a good time to interject and get into the nitty gritty of angular's
change detection. As we discussed in the earlier chapter on change-detection,
change detection in Angular works from the top down. That is, if a specific set
of data changes within a component, then the entire component will update as a
result of new data. Angular pipes, however, change how that it is done, by
changing content directly on the object, and only updating that one specific
part. By using pipes, it allows one's component to be more performant.

\subsection{ When to Use Pipes }
Pipes cover alot of ground. Within our architecture, one of the pipes that will
be used more so that others in the async pipe. However, now is not the place
to put that here. What is important, is that pipes can be used transform data.
Should they always be used whenever one is transforming data? Personally,
I like to think of pipes as unique to html. They have a way of dealing of
performance when working with html templates. However, within a component
itself it is questionable. Here's a great example of what I mean:
\begin{lstlisting}

\end{lstlisting}
