\section{ Pipes }
\maketitle{}

Angular offers the ability to use Pipes out of the box. The idea behind a pipe
is to get data, transform it, and show new transformed data to users. Pipes are
something that I am going to assume the reader is already familiar with. Just as
a primer, using the native angular date pipe, a pipe would do something such
as the following:
\begin{lstlisting}
The chained hero's birthday is
{{ birthday | date | uppercase}}
\end{lstlisting}

This would display
\begin{verbatim}
FRIDAY, APRIL 15, 1988
\end{verbatim}

The pipe here is taking in the timestamp of
\begin{verbatim}
  577065600
\end{verbatim}
and converting it to the proper date. Having such dates in one's application
obviously makes it very easy to go ahead and transform data throughout one's
app. Here we are also chaining pipes, so that the transformed data that comes
back in addition to being transformed, is also capitalized.

\section{ Performance Considerations }
It is important off the cuff to be aware of some performance concerns when it
comes to pipes.
