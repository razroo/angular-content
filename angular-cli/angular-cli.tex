\maketitle{}
\section{ Angular CLI }

In any Angular setting the Angular CLI(Command Line Interface) is going to be
a good friend. The following are things which you are able to do, in the Angular
CLI:

\begin{enumerate}
  \item Create an application out of the box
  \item Generate a module
  \item Generate a component
  \item Generate a route
  \item Generate a service
  \item Serve application
  \item Test application
  \item Lint application
\end{enumerate}

As of now, we will be needing to create a new application. So, let's install
angular cli:
\begin{verbatim}
  npm install -g @angular/cli
\end{verbatim}
Next, inside the directory that you would like to genrate your application, run:
\begin{verbatim}
  ng new pixelIllustrator
\end{verbatim}
This will create a pixelIllustrator folder \footnote{Angular CLI version as of
this writing is: 1.6.1}, with all of our new angular code
inside of it. The directory structure looks as follows:

\begin{forest}
  for tree={
    font=\ttfamily,
    grow'=0,
    child anchor=west,
    parent anchor=south,
    anchor=west,
    calign=first,
    inner xsep=7pt,
    edge path={
      \noexpand\path [draw, \forestoption{edge}]
      (!u.south west) +(7.5pt,0) |- (.child anchor) pic {folder} \forestoption{edge label};
    },
    before typesetting nodes={
      if n=1
        {insert before={[,phantom]}}
        {}
    },
    fit=band,
    before computing xy={l=15pt},
  }
[system
  [e2e
  ]
  [Node Modules
  ]
  [src
    [app]
    [assets]
    [environments]
  ]
  [tests
  ]
]
\end{forest}
