
\maketitle{}
\section{ Angular CLI }

The purpose of the Angular CLI(Command Line Interface) is:

"The CLI is responsible for automating away many of the challenges and headaches
that come with being a developer in 2017. By starting with a configuration that
works out of the box, baking in best practices that have been discovered by the
community over time, and by helping you build more maintainable code, the CLI
exposes some of the most powerful capabilities of Angular in a way that is
easier to get started, and easier to get moving quickly when compared to rolling
everything together yourself." \footnote{https://blog.angular.io/the-past-present-and-future-of-the-angular-cli-13cf55e455f8}

The following are a couple of things which you are able to do with the Angular
CLI:

\begin{enumerate}
  \item Create an application out of the box
  \item Generate a module
  \item Generate a component
  \item Generate a route
  \item Generate a service
  \item Serve application
  \item Test application
  \item Lint application
\end{enumerate}

As of now, we will be needing to create a new application. So, let's install
angular cli:
\begin{verbatim}
  npm install -g @angular/cli
\end{verbatim}
Next, inside the directory that you would like to genrate your application, run:
\begin{verbatim}
  ng new pixelIllustrator
\end{verbatim}
This will create a pixelIllustrator folder \footnote{Angular CLI version as of
this writing is: 1.6.1}, with all of our new angular code
inside of it. The directory structure looks as follows:

\begin{forest}
  for tree={
    font=\ttfamily,
    grow'=0,
    child anchor=west,
    parent anchor=south,
    anchor=west,
    calign=first,
    inner xsep=7pt,
    edge path={
      \noexpand\path [draw, \forestoption{edge}]
      (!u.south west) +(7.5pt,0) |- (.child anchor) pic {folder} \forestoption{edge label};
    },
    before typesetting nodes={
      if n=1
        {insert before={[,phantom]}}
        {}
    },
    fit=band,
    before computing xy={l=15pt},
  }
[system
  [e2e
  ]
  [Node Modules
  ]
  [src
    [app]
    [assets]
    [environments]
  ]
  [tests
  ]
]
\end{forest}

\subsection{Ng New Notable Mentions}

\paragraph{End to End Integration}

Created for us is an e2e folder. We will be using this for integration tests.

\paragraph{Linting}

A tslint.json has been created. We can add linting rules that we so choose here,
and run linting using ng lint \footnote{Add note with regards to having a watcher
set up}.

\paragraph{App folder}

Inside of the src folder, a folder has been created for styling, components,
modules, and unit testing. We will add onto this file directory, moving forward.

\paragraph{Assets Folder}
Any images, fonts, or otherwise go in this folder. When the angular cli command for
pushing to prod is made, it will transfer over the assets folder to the prod
folder. This will make it so that filepaths, as long as relative, will still
work as expected.

\subsection{What we will be adding}

The Angular CLI is fantastic, and has removed a tremendous amount of effort from
getting our project started.


% Folder has been created for using integration tests.
