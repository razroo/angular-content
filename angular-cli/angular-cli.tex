\maketitle{}
\section{ Angular CLI }

In any Angular setting the Angular CLI(Command Line Interface) is going to be
a good friend. The following are things which you are able to do, in the Angular
CLI:

\begin{enumerate}
  \item Create an application out of the box
  \item Generate a module
  \item Generate a component
  \item Generate a route
  \item Generate a service
  \item Serve application
  \item Test application
  \item Lint application
\end{enumerate}

As of now, we will be needing to create a new application. So, let's install
angular cli:
\begin{verbatim}
  npm install -g @angular/cli
\end{verbatim}
Next, inside the directory that you would like to genrate your application, run:
\begin{verbatim}
  ng new pixelIllustrator
\end{verbatim}
This will create a pixelIllustrator folder, with all of our new angular code
inside of it.
