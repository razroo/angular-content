
\maketitle{}
\section{ Angular CLI }

\textit{None of the code being generated in this part of the chapter will be
used in the actual app we are using. Hold off on coding along(if you are, until
the next chapter. This will be the first and last time this will happen)}

The purpose of the Angular CLI(Command Line Interface) is:
"The CLI is responsible for automating away many of the challenges and headaches
that come with being a developer in 2017. By starting with a configuration that
works out of the box, baking in best practices that have been discovered by the
community over time, and by helping you build more maintainable code, the CLI
exposes some of the most powerful capabilities of Angular in a way that is
easier to get started, and easier to get moving quickly when compared to rolling
everything together yourself." \footnote{https://blog.angular.io/the-past-present-and-future-of-the-angular-cli-13cf55e455f8}

The following are a couple of things which you are able to do with the Angular
CLI:

\begin{enumerate}
  \item Create an application out of the box
  \item Generate a module
  \item Generate a component
  \item Generate a route
  \item Generate a service
  \item Serve application
  \item Test application
  \item Lint application
\end{enumerate}

Just to get an idea for yourself, of what the Angular CLI can do, and how to
generate files, let's run through the CLI high level . Let's install Angular
CLI:
\begin{verbatim}
  npm install -g @angular/cli
\end{verbatim}
Next, inside the directory that you would like to genrate your application, run:
\begin{verbatim}
  ng new pixelIllustrator --service-worker
\end{verbatim}
\footnote{We will be using service workers, and using the above will save us
time later on.}
This will create a pixelIllustrator folder \footnote{Angular CLI version as of
this writing is: 1.6.1}, with all of our new angular code
inside of it. The directory structure looks as follows:

\begin{forest}
  [system
    [e2e
    ]
    [Node Modules
    ]
    [src
      [app]
      [assets]
      [environments]
    ]
    [tests
    ]
  ]
\end{forest}

\subsection{Ng New Notable Mentions}

\paragraph{End to End Integration}

Created for us is an e2e folder. We will be using this for integration tests.

\paragraph{Linting}

A tslint.json has been created. We can add linting rules that we so choose here,
and run linting using ng lint \footnote{Add note with regards to having a watcher
set up}.

\paragraph{App folder}

Inside of the src folder, a folder has been created for styling, components,
modules, and unit testing. We will add onto this file directory, moving forward.

\paragraph{Assets Folder}
Any images, fonts, or otherwise go in this folder. When the angular cli command for
pushing to prod is made, it will transfer over the assets folder to the prod
folder. This will make it so that filepaths, as long as relative, will still
work as expected.

\subsection{Hold on - It's not going to be that easy}

The Angular CLI is fantastic, and has removed a tremendous amount of effort from
getting our project started. However, the idea as mentioned in the Angular CLI,
is that the Angular CLI is meant to expose the most powerful parts of Angular,
but it isn't the full package. This issue arises twofold. It doesn't impart to
us how to use some of the more complicated parts of the created application, in
particular Typescript, Observables, routing, and styling. In addition,
there are things which we will want to add to our application, which the Angular
CLI has not provided:
\begin{enumerate}
  \item State Management
  \item Sass(The CLI will give us the option to do so)
  \item Workspace(Creating a mono repo using Nrwl Nx)
  \item Library(Also going to be created using Nrwl Nx)
\end{enumerate}

That being said, let's come back to the Angular CLI shortly when we are ready
to be build our application. First, however, let's take one step back. Let's
talk about Nrwl Nx and turn our app into a workspace.
