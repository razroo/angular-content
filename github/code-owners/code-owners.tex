
\section{ Creating Code Owners }

As we have discussed previously, Github is our preferred client for pull
requests. As the team grows larger it becomes imperative, that default code
reviewers are setup. Creating a \codeowners{} file is the easiest way to do this.

\subsection{ What is a \codeowners{} file? }
A \codeowners{} file is a config file, used with Github, that will allow you to
specify which github users are considered as \codeowners{} for a specific project.
In addition, it will allow you to target \codeowners{} for a specific file type.
So just to go a bit more in detail, let's say you have a dynamic team, and a
dynamic codebase. Part of the team develops in Python, and part of the team
develops in Javascript. You would be able to set \codeowners{} specifically for
javascript, so that if only js files have been affected, only these specific
\codeowners{} should be brought up.

\subsection{ How to create a \codeowners{} file? }
There are two places wherein you can create a \codeowners{} file. One would be
in the root of your app. The other would be in a .github folder in the root of
your repo. We recommend the .github folder, as we will also be creating webhooks
within our repo. So:
\begin{verbatim}
  mkdir .github; cd .github; touch \codeowners{}
\end{verbatim}
When you make a pull request within your github app, github will automatically
pick up on this file.

\subsection{ Creating \codeowners{} }
As we mentioned before, there are two parts to a \codeowners{} repo. One would
be specifying specific github users who would be considered as \codeowners{}.
The other would be, if the app is more complex, who is considered as
\codeowners{} if a certain file is edited.

\begin{lstlisting}
# Lines starting with '#' are comments.
# Each line is a file pattern followed by one or more owners.

# These owners will be the default owners for everything in the repo.
*        @CharlieGreenman

\end{lstlisting}

\subsection{ Creating \codeowners{} Based on File Type }
\begin{lstlisting}
.js @CharlieGreenman
\end{lstlisting}

Now Charlie Greenman will be a code owner whenever a js file is a part of a
pull request.

\mybox{
\subsection{ Final Note }
When using a \codeowners{} file for the first time, it can be a bit confusing.
The \codeowners{} will only be created once an actual pr is made. However,
during the pr process, the \codeowners{} will not be made. There is no fix to
this to the best of my knowledge.
}
