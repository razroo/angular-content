\maketitle{}
\section{ Creating Code Owners }

As we have discussed previously, Github is our preferred client for pull
requests. As the team grows larger it becomes imperative, that default code
reviewers are setup. Creating a \codeowners{} file is the easiest way to do this.

\section{ What is a \codeowners{} file? }
A \codeowners{} file is a config file, used with Github, that will allow you to
specify which github users are considered as \codeowners{} for a specific project.
In addition, it will allow you to target \codeowners{} for a specific file type.
So just to go a bit more in detail, let's say you have a dynamic team, and a
dynamic codebase. Part of the team develops in Python, and part of the team
develops in Javascript. You would be able to set \codeowners{} specifically for
javascript, so that if only js files have been affected, only these specific
\codeowners{} should be brought up.

\section{ How to create a \codeowners{} file? }
There are two places wherein you can create a \codeowners{} file. One would be
in the root of your app. The other would be in a .github folder in the root of
your repo. We recommend the .github folder, as we will also be creating webhooks
within our repo. So:
\begin{verbatim}
  mkdir .github; cd .github; touch \codeowners{}
\end{verbatim}
When you make a pull request within your github app, github will automatically
pick up on this file.
