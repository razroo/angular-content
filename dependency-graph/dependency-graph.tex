\maketitle{}
\section{ Dependency Graph }

A dependency graph is a very simple way of seeing which components are
dependent on which components. It goes hand in hand with a parent and child
component architecture. By looking a dependency graph, one can see that a
singular component is reliant on another. The following are the benefits of
using a dependency graph:
\begin{enumerate}
  \item Make sure app is following Parent/Child Component architecture.
  \item Allows us to see what components are dependent on the one we are working
  on, so that we may run linting, or unit testing only, based on these
  components.
  \item Visualize all components in use across app.
\end{enumerate}

\subsection{Create a Dependency Graph}
My personal favorite when it comes to using a dependency graph is compodoc.
However, when I am in development mode, and not demo mode, I will tend to use
the dependcy graph provided by nrwl, granted you are using nx.
\begin{verbatim}
  npm run affected:dep-graph;
  npm run dep-graph;
\end{verbatim}

This will show you something like:
(Image of depedency graph goes here)

\subsection{ Using Dependency Graph }
