\maketitle{}
\section{ E2E Testing in a TDD/BDD Setting }

One of the tricky things with regards to E2E tests, is how it fits into a
TDD/BDD environment. Writing unit tests before we can see anything in our UI
already takes quite a bit of discipline. Adding in an E2E test to the workflow
seems like a bit much? Ok, so let's get into the thick of it. I think we'll all
have a good time!

\subsection{ Where does E2E Testing fit in a TDD/BDD Setting? }

\subsubsection{ Write an E2E test and Watch it Fail }
Begin with E2E test, and watch it fail.

\subsubsection{ Write Unit tests and watch them fail }
Write a unit test, that works towards satisfying the goal of your E2E test, and
then watch it fail. One note, is that spectrum of what you will write, will
be wider, being that a single E2E will have a wider scope. This is alright.

\subsubsection{ Code Until All Unit Tests are Satisfied }
Implement code to satisfy your unit tests.

\subsubsection{ Optional - Tuck in any untucked corners }
There might be some unit tests that will need to be written, that might not be
directly correlated with the Protractor tests written for the E2E test. Writing
these additional tests at this time, in between the next E2E test, would make
sense.

\subsubsection{ Check to see if E2E Test Passes }
When all unit tests pass, you should now run the command for running E2E
commands.

\subsubsection{ Repeat the Process }
Now that you have run your E2E tests, and unit tests, and have now code
the appropriate HTML to run unit tests, you should now repeat process with
appropriate unit tests.

\textit{Note: We will not be writing any E2E tests yet, as we do not have any
two components to integrate yet}
