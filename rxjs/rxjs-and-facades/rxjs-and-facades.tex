\chapter{ RxJS and Facades }
\section{The point of this Chapter}
This entire chapter, is going to discuss how you can create your own "light" internal state, using RxJS and Facades. However, the WHOLE point of this, is to make you strongly consider, using Ngrx/store in the first place, without using Facades + RxJS. In addition, proving to the developer, create an internal service for a component, that interacts with an http service, likewise should use ngrx/store in the first place. 

I think this is important to read through, because there is going to be a time in your career already, where you have decided, /lstinline{ngrx/store} might be too much for what you are working on. Likewise the point of this article is to push back on that urge, and how ngrx/store will actually the decrease the time it takes to build the application long term. 

\section{Recap on the Facade Pattern}
I want this piece of information to be as authentic as possible. Therefore, I would like to present the facade pattern in the way that I first learnt how to do so. There is a fundemental book on computer science called "Design Patterns: Elements of Re-usable Object-Oriented Software" otherwise known as the GoF(Gang of Four) book due to it's four authors. It's a bit of classic in software, and I would say analogous to, "How to Win Friends and Influence People" by Dale Carnegie, if you are familiar. In the GoF book, it discusses the idea of the facade pattern. Paraphrasing it: 
\begin{enumerate}
  \item Implements a singular interface that contains multiple interfaces. Those interfaces work through the singular interface.
  \begin{enumerate}
    \item This helps readability by making the name more straight to the point. For instance, in our scenario, \lstinline{UpdateTodo}, vs. \lstinline{this.store.dispatch(new TodoUpdated(TodoPayload))}. 
    \item Usability by removing need of dependencies, and truncated form of functionality. For instance, in our scenario, we are using \lstinline{UpdateTodo}, there is no need to include dependency for store. In addition, developer can now just type in \lstinline{UpdateTodo} instead 
  \end{enumerate}
  \item Provide context specific interface. This is non relevant within an ngrx setting, as front end services by default, without state, is generally very, very specific.
  \item Serve as a launching point for a broader re-factor, or a tightly coupled system, in favor a more loosely coupled code. For instance, in our scenario, by tying all of our state underneath a singular facade, if at a later date we want to swap out the tech needed to manage state, we can do that by simple changing the logic within a particular location. 
\end{enumerate}

\section{Recap: Ngrx/store + Facades}
Just to re-iterate, let's say that we have a todo app, and we create a \lstinline{todo.facade.ts} file:

\begin{lstlisting}[caption=todo.facade.ts]
@Injectable({})
class TodoFacade {
  constructor(private store: Store<any>) {}

  todos$: Observable<Todo[]> = this.store.pipe(select(getTodos));
  idOfTodos$: Observable<Todo[]> = this.store.pipe(select(getTodosIds));
  loaded$: Observable<boolean> = this.store.pipe(select(getTodosLoading));

  UpdateTodo(TodoPayload): void {
    this.store.dispatch(new TodoUpdated(TodoPayload));
  }
}  
\end{lstlisting}

Our facade within an ngrx/store setting is doing three unique things.
\begin{enumerate}
  \item Handling the store constructor for us. (With regards to unit testing this makes life alot easier.)
  \item Allows us to have to only put selector in one location(our facade), and trickle it down to all other areas. 
  \item Create a simpler to use interface for our actions, that can be re-used time and time again. Especially if we decide to change the logic, or dynamics of how action works, we only have to update in one particular area. 
\end{enumerate}

\section{Progressing Idea of Facades over to RxJS}
Similarly, the idea of facades is incredibly valuable within RxJs. Let's imagine the most heavily used RxJS use case(atleast the most heavily used one I've come across). Perhaps, because there isn't a single application that doesn't use it, "Search".

Some of the business use cases with our particular company search includes: 
\begin{enumerate}
  \item Search by typing text into search bar(let's say here searching for companies)
  \item Route between search, company, or company details view 
  \item Company details
\end{enumerate}

Our particular RxJS code 
\begin{lstlisting}[caption=search-companies.component.ts]
@Component({
  selector: 'search-companies',
  templateUrl: './search-companies.component.html',
  styleUrls: ['./search-companies.component.scss']
})
export class SearchCompaniesComponent implements OnInit {
  companies$: Observable<Company[]>;
  searchCriteria = new FormControl();

  constructor(private companyService: CompanyService) {}

  ngOnInit() {
    // Observable stream to input control values
    const searchBy$ = this.searchTerm.valueChanges;
    
    this.companies$ = searchBy$.pipe(
      debounceTime(300),
      distinctUntilChanged(),
      startWith(''),      
      switchMap((criteria:string) => {
        const request$ = this.companyService.searchCompanies(criteria);
        return !criteria.length ? of([]) : request$          
      })
    );
  }
}  
\end{lstlisting}

Let's dissect our RxJS code: 
\begin{enumerate}
  \item \lstinline{debounceTime(300)} - Only after 300 milliseconds, will the request go through. If the user decides to type once again, within 300 milliseconds, the 300 millisecond count time, restarts again. 
  \item \lstinline{distinctUntilChanged()} - Let's say the string the user inputs is "Apple". Then they decide to delete the e "Appl", but then add it again, "Apple", this will not trigger the http request. This is because, this is the exact same word that it was initially. 
  \item \lstinline{startWith('')} - Not completely necessary, but will make sure that is an empty string everytime we initialize this page again. 
  \item \lstinline{switchMap} - Powerful, because it has an internal project function, which allows us to break from the observables and use our "transformed" value. Here we are passing it to the service, which calls our compnayService, based on search criteria user passed in. 
\end{enumerate}

\section{Addressing Architectural Concerns in Above Code}
\subsection{Concern One - Business Logic in View Layer}
Obviously the logic here is contained within the view layer of the actual component. Best practices dictate, logic pertaining to view itself should be contained within component. Business logic, such as here, pertaining to when the service should fire should be moved to a different service. 

\subsection{Concern Two - State }
Primarily around caching. In addition, updating the search criteria. For instance, if we want to search our application by companies, use one service. If we want to search by CEO's, then another function for CEOs. If we want to use another search criteria, let's say a particular category, then we will need another facade for controlling that. This ends up having alot of internal logic, and it greatly simplifies our app, by having the state be external. Especially, if would like to separate search from the data-table. Having the logic be re-usable, and be able to be used in multiple places, is useful.

\section{Creating State Using RxJS}
I would like to jump straight to this one. I don't want to jot down here the interface, rather the internal logic that we can create withing our SearchFacade. It would look something like this:  

\begin{lstlisting}[caption=company-search.facade.ts]
export enum CompanySearchActionTypes {
  UPDATE_CRITERIA = 'Update Search Criteria',
  UPDATE_RESULTS = 'Update Search Results'
}

export interface SearchAction {
  readonly type: SearchActionTypes;
  readonly payload: SearchCriteria | SearchResult[];
}

export class CompanySearchFacade {
  searchResults$: Observable<SearchResult[]> = this.dispatch
    .asObserable() 
    .pipe(map(state => state.companies), startWith([] as SearchResult[]));

  searchCriteria$ = Observable<SearchCriteria> = this.dispatch.asObservable().pipe(map(state => state.criteria)
  ) 

  constructor(private companyService: CompanyService, private userService: UserService)
    
  searchCompanies(companyName: string): Observable<SearchResult[]> {
    // more code potentially goes here
    pendingCompanies$.subscribe(companies => {
      connst type = SearchActionTypes.UPDATE_RERULTS;
      const action = { payload: tickets, type};
      this.dispatch.next(
        (this.state = reduce(this.state, action))
      );
    });
  }

  updateCompanyCriteria(companyName: string, categoryName: string) {
    const type = SearchActionTypes.UPDATE_CRITERIA;
    const payload = { user, ticket };
    const state = reduce(this,state, {type, payload});

    this.dispatch.next((this.state = state));
  }
}  

function reduce(state: SearchState, action: SearchAction): SearchState {
  switch (action.type) {
    case SearchActionTypes.UPDATE_CRITERIA: return {
      ...state,
      criteria: action.payload as SearchCriteria
    };
    case SearchActionTypes.UPDATE_RESULTS: return {
      ...state,
      companies: action.payload as SearchResult[]
    }
  }
}
\end{lstlisting} 

Don't think too much into the above code. However, what we've proven is that we can maintain a relatively simple store within our Facade simply using RxJS. That being said, I would like to make the following case for not doing the above and sticking with ngrx/store.

\section{Why you should stick to ngrx/store}
The assumption with doing the above, is that the code is light enough, to where we don't need to use ngrx/store. However, let's say within our code, we want to create a general company. Here we sort of move companies over to a cart of sorts, where we can then move these companies over an analytics platform. Here, we would be able to combine search, and our product cart under one store. Using the above facade, it would fracture our store between many different things. We might: 
\begin{enumerate}
  \item Want to create an ngrx/effect, so that whenever a user searches, it resets the product cart(who knows, maybe product wants it).
  \item We want to modify our data to use ngrx/entity. 
  \item Maybe we want a history of all searches, to be used for internal analytics periodically. 
  \item Our reducers and actions bloat out of control, wherein it makes sense to put them into two separate files. While we are at it, we might as well be using state. 
  \item We have similar state logic elsewhere(let's say for our influencer search), and want to use that our company logic as well. It might be easier to keep them all in the same sort of structure.
  \item It might make our code more brittle. Who's to say now that we have ngrx/store and rxjs/facades, that the team won't break up into more patterns. For instance, a one-off internal service? 
\end{enumerate}

For this reason, and many others, once you are building an enterprise application, it makes sense to use ngrx/store. 

\texttt{I've worked with many startups, and I've seen them work really quickly with ngrx/store as well. When done correctly, I would argue that ngrx/store speeds up team development, due to it's cookie cutter style of front end architecture}

