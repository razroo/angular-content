\chapter{ Introduction to Rxjs }

\section{ What is Reactive Programming}

Rxjs is a library based on the concept of Reactive programming. To re-iterate 
what we mentioned in the chapter on ngrx. A great founding paper discussing 
reactive programming for concurrent programming can be found here. 
\footnote{http://www-sop.inria.fr/members/Gerard.Berry/Papers/Berry-IFIP-89.pdf}

It discusses the benefit of Real Time Progamming a.k.a., reactive programming.
In it, it discusses the two main benefits of Reactive Programming:
\begin{enumerate}
  \item Asynchronous
  \item Detemrinistic
\end{enumerate}

That would be reactive programming in a nutshell. It makes sure, that one 
function happens after another. In addition, by it's definition of being 
a set of pre-made functions, it gives stability around the way we are 
transforming our data in Angular. However, really the most important part 
about Rxjs, still goes to Andre Staltz: 

\begin{quote}
  \say{Reactive Programming raises the level of abstraction of your code so you
   can focus on the interdependence of events that define the business logic...”.}
\end{quote} 
\footnote{This quote can be found in his excellent article on why you should 
consider adopting Reactive programming principles in your app 
https://gist.github.com/staltz/868e7e9bc2a7b8c1f754\#why-should-i-consider-adopting-rp}

The next question becomes how Reactive programming accomplishes the above. I 
could put a diagram here and explain what it does in detail, but for the sake 
of brevity, I am just going to jump into Rxjs code. But first, let's talk 
about the importance of Rxjs in Angular!

\section{Rxjs's Importance in Angular }
Rxjs has a pretty big place in Angular. I would say that in your average app, 
I would consider it as one of the big three, alongside Typescript, and Ngrx. 
Even though these are independent libaries, when working on enterprise Angular
applications, I personally come across them on a day to day basis. So, while 
they are independent entities to Angular, I very much so consider them as part 
of the Angular framework. 

\section{The Rxjs Observable}
The core of Rxjs, is the ability to program reactively using observables. Rxjs 
offers the ability to create observables, as well as manipulate them. Assuming 
we were using Apollo client to retrieve data from our GraphQL requests, rxjs 
gives us the ability to transfer data we have retrieved into an observable: 
\begin{lstlisting}
import { from } from 'rxjs';

// Create an Observable out of a promise
const data = from(fetch('/api/endpoint'));
// Subscribe to begin listening for async result
data.subscribe((data) => {
  console.log(data);
});  
\end{lstlisting}