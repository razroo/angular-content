\chapter{Transformation}
Transforming in RxJS is the process of modifying a value. It is quite 
a common occurence within RxJS.

\section{bufferTime}
\lstinline{bufferTime} will collect a value until the provided time elapses. 
Once time has elapsed it will emit values collected during that time as an 
array. (It will not unsubscribe from the observable). A good example of when we
would want to use bufferTime, is if we want to give the user notifications 
within their application. However, we don't want to flood them with 
notifications. In addition, the updates within our application are quite 
frequent, and are updated quite frequently. Think something along the lines of
a stock market application.

\begin{lstlisting}
this.notificationService.updates$.pipe(
  bufferTime(5000)
).subscribe(batch => {
  this.notifications = notifications;
});  
\end{lstlisting}

To be honest, within an enterprise Angular application, I haven't used 
\lstinline{bufferTime} very often. 

\section{RxJs Higher Order Mapping}
Before we dive into these three values:
\begin{enumerate}
  \item \lstinline{concatMap}
  \item \lstinline{mergeMap}
  \item \lstinline{switchMap}
\end{enumerate}

it's important to note that they are all similar in one regard. It will 
subscribe to an inner observable. This means that it allows for access to the 
value, and will create an inner observable. This inner observable can then be 
modified. Why create an inner observable? Well this get's into something called
higher order observables. Think about the way classic promises work. Let's say
we wanted to modify the data nested twice within our promise:
\begin{lstlisting}
this.postService.getPosts().then(data => {
  const blogIds = data.map(data => data.id);
  this.analyticsService.analytics(blogIds).then(analyticsData => {
    this.analyticsData = analyticsData;; 
  });
});  
\end{lstlisting}

In the above code, we have one promise inside of another promise. RxJS allows us
to do all of these within the same stream. It allows for the observable stream
to be more manageable and concise(i.e. avoiding nested subscribes). These are
what would be called higher order observables. 

\section{concatMap}
\lstinline{concatMap} will map values to the inner observable, subscribe and 
emit in order. Emphasis is emit in order, that is what \lstinline{concatMap} 
specializes in. A great way to visualize this, is let's say we wanted to 
create a cascading effect within our application. 

\begin{lstlisting}[caption=data-table.component.ts]
getItems(ids: number[]): Observable<Item> {
  return from(ids).pipe(
     concatMap(id => <Observable<Item>> this.httpClient.get(`item/${id}`))
  );
}
\end{lstlisting}

In the above, the previous observable will emit first, causing a visual 
waterfall effect within our app. 

\section{mergeMap}
\lstinline{switchMap} (similar to \lstinline{mergeMap} and \lstinline{concatMap}),
will create an inner observable. The main difference, is that \lstinline{mergeMap},
will merge all observables into one(, as opposed to \lstinline{switchMap}, 
which will cancel all prior observables). 

\begin{lstlisting}[caption=mergeMap example]
  @Effect()
  loadAllBlogPosts$: Observable<any> = this.actions$.pipe(
    ofType(PokemonActions.loadPokemon),
    mergeMap(() =>
      this.postsService.getAll().pipe(
        map(posts => PokemonActions.loadPokemonSuccess({ posts })),
        catchError(message => of(PokemonActions.loadPostsFailed({ message })))
      )
    )
  );  
\end{lstlisting}
\mybox{You might notice that we are using mergeMap for what might be otherwise
thought of using a switchMap for. The reason I like using mergeMap is that 
switchMap doesn't actually do anything in this scenario. Even if we were to 
cancel the inner observable the action and http request has still made it's 
way through. I feel like using mergeMap makes this scenario more transparent 
and is therefore the right thing to use in this scenario.}

\section{switchMap}
\lstinline{switchMap} (similar to \lstinline{concatMap} and \lstinline{mergeMap}),
will create an inner observable. The main difference is that it will complete 
the previous inner observable, so that only the latest observable is re-used. 

\begin{lstlisting}[caption=search-bar.component.ts]
@Effect()
findAddresses: Observable<any> = this.actions.pipe(
  ofType(LocationActionTypes.FindAddresses),
  map(action => action.partialAddress),
  debounceTime(400),
  distinctUntilChanged(),
  switchMap(partialAddress => this.backend
    .findAddresses(partialAddress)
    .pipe(
      map(results => new FindAddressesFulfilled(results)),
      catchError(error => of(new FindAddressesRejected(error)))
    )
  )
);  
\end{lstlisting}
 
In the example above, we have set up our effect to handle search. It has a 
\begin{enumerate}
  \item \lstinline{debounceTime} so that if user types multiple times within
  a 400 milliseconds, it will only trigger once
  \item \lstinline{distinctUntilChanged} to handle the use case wherein user 
  deletes letters after typing, but returns to same word after deleting
  \item ...and then the magic! \lstinline{switchMap} is used within our app,
  because once a new search is made, prior observables are no longer needed 
  and can be removed. 
\end{enumerate}

It's interesting. While this has value, an alternative option could have been 
used, i.e. \lstinline{concatMap}. Due to the finicky nature of maps, and 
complexity behind it, maps can also be considered as a way of documentation.

\mybox{When choosing a flattening operator for an effect/epic, if subsequent 
actions of the same type will render pending results stale, switchMap is 
unequivocally the best choice.}

\section{scan}
\lstinline{scan} actually works exactly like \lstinline{reduce} does in 
regular Javascript for arrays, but for observables. So, why did the RxJS team 
call it \lstinline{scan} instead of \lstinline{reduce}? Well, it turns out there 
is one little difference. \lstinline{reduce} actually is an operator in RxJS as
well. However, \lstinline{scan} will emit the value for every iteration, whereas 
\lstinline{reduce} will emit only the final value. 

\lstinline{scan} therefore as a combinator has more use than reduce does, 
because it allows us to tap into the history of our state. (In fact, while we
won't go into that here, we can create a really low level state management 
using \lstinline{scan})

So in an enterprise setting, where we already have use of 
\lstinline{ngrx/store}, why would scan be considered as a common operator? The 
truth is, that within an enterprise Angular application, using \lstinline{ngrx/store}
there are some really one of cases that use it. It is not an operator that I would keep 
in mind, unless you are trying to introduce state to legacy Angular application, 
that does not have state. 

\section{map} 
\lstinline{map} will apply a projection to each value in source. For instance: 
\begin{lstlisting}[caption=rxjs map example]
// RxJS v6+
import { from } from 'rxjs';
import { map } from 'rxjs/operators';

//emit (1,2,3,4,5)
const source = from([1, 2, 3, 4, 5]);
//add 10 to each value
const example = source.pipe(map(val => val + 10));
//output: 11,12,13,14,15
const subscribe = example.subscribe(val => console.log(val));
\end{lstlisting}

A common occurence within actual applications is to use map within a 
switchMap, when using an \lstinline{ngrx/effect}. That way, it returns the map
to the \lstinline{switchMap}, and kills any other observables from happening.

\begin{lstlisting}[caption=map example]
@Effect()
loadAllBlogPosts$: Observable<any> = this.actions$.pipe(
  ofType(PokemonActions.loadPokemon),
  mergeMap(() =>
    this.postsService.getAll().pipe(
      map(posts => PokemonActions.loadPokemonSuccess({ posts })),
      catchError(message => of(PokemonActions.loadPostsFailed({ message })))
    )
  )
);  
\end{lstlisting}

Here we are mapping and returning an action to our effect. This allows us to 
hook in an effect to our general \lstinline{ngrx/effect} ecosystem.