\chapter{Transformation}
Transforming in RxJS is the process of modifying a value. It is quite 
a common occurence within RxJS.

\section{bufferTime}
\lstinline{bufferTime} will collect a value until the provided time elapses. 
Once time has ellapsed it will emit values collected during that time as an 
array. (It will not unsubscribe from the observable). A good example of when we
would want to use bufferTime, is if we want to give the user notifications 
within their application. However, we don't want to flood them with 
notifications. In addition, the updates within our application are quite 
frequent, and are updated quite frequently. Think a stock market application.

\begin{lstlisting}
this.notificationService.updates$.pipe(
  bufferTime(5000)
).subscribe(batch => {
  this.notifications = notifications;
});  
\end{lstlisting}

\section{RxJs Higher Order Mapping}
Before we dive into these four values it's important to note that they are 
all similar in one regard. It will subscribe to an inner observable. This 
means that it allows for access to the value, and will create an inner 
observable. This inner observable can then be modified. Why create an inner
observable? Well this get's into something called higher order observables. 
Think about the way classic promises work. Let's say we wanted to modify the 
data nested twice within our observable.
\begin{lstlisting}
this.postService.getPosts().then(data => {
  const blogIds = data.map(data => data.id);
  this.analyticsService.analytics(blogIds).then(analyticsData => {
    this.analyticsData = analyticsData;; 
  });
});  
\end{lstlisting}

In the above code, we have one promise inside of another promise. RxJS allows us
to do all of these within the same stream. It allows for the observable stream
to be more manageable and concise(i.e. avoiding nested subscribes). These are
what would be called higher order observables. 

\section{concatMap}
\lstinline{concatMap} will map values to the inner observable, subscribe and 
emit in order. Emphasis is emit in order, that is what \lstinline{concatMap} 
specializes in. 

\section{map} 

\section{mergeMap}

\section{scan}

\section{switchMap}