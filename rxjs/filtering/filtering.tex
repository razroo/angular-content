\chapter{Filtering}

\section{debounceTime}
\lstinline{debounceTime} means that if during the buffer time passed to 
\lstinline{debounceTime} method, another event is fired, it will cancel 
previous stream. This is a very common use case that is used within search.
It can also be used it regular event handlers, to prevent from clicking a 
submit button multiple times.(However, Angular's internal event handler 
is usually pretty good at this.)

\mybox{It's important to note, that adding a debounce method, while will 
greatly help backend with amount of requests needed to be made, will also
help with the users CPU.}

\begin{lstlisting}[caption=blog.effects.ts]
@Effect() searchImage$ = this.actions$.pipe(
  ofType(fromBlog.SEARCH_BLOGS),
  map((action: fromBlogActions.SearchBlog) => action.query),
  debounceTime(300),
  switchMap(query: string) => this.blogService.getBlogBySearching(query))
\end{lstlisting}

The above is a common re-occuring pattern within the \lstinline{ngrx/effects}
library. We will be debouncing our search, so that if a user searches again 
within a given time, then it will cancel the previous observable. 

\mybox{It should be noted, that the service will be called again nonetheless.
This will only help from the CPU side of things. }

\section{distinctUntilChanged} 
don't send a request if value stays the same 

\section{filter} 

\section{take} 

\section{takeUntil}