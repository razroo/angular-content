\chapter{Filtering}

\section{debounceTime}
\lstinline{debounceTime} means that if during the buffer time passed to 
\lstinline{debounceTime} method, another event is fired, it will cancel 
previous stream. This is a very common use case that is used within search.
It can also be used it regular event handlers, to prevent from clicking a 
submit button multiple times.(However, Angular's internal event handler 
is usually pretty good at this.)

\mybox{It's important to note, that adding a debounce method, while will 
greatly help backend with amount of requests needed to be made, will also
help with the users CPU.}

\begin{lstlisting}[caption=blog.effects.ts]
@Effect() searchImage$ = this.actions$.pipe(
  ofType(fromBlog.SEARCH_BLOGS),
  map((action: fromBlogActions.SearchBlog) => action.query),
  debounceTime(300),
  switchMap(query: string) => this.blogService.getBlogBySearching(query))
\end{lstlisting}

The above is a common re-occuring pattern within the \lstinline{ngrx/effects}
library. We will be debouncing our search, so that if a user searches again 
within a given time, then it will cancel the previous observable. 

\section{distinctUntilChanged} 
Adding a \lstinline{distinctUntilChanged()} operator to your pipe, will make 
it so that an observable is not changed if current value emitted, is the same
as the prior value. A great use case for this, is within the effect we 
already used within our app.

\begin{lstlisting}[caption=blog.effects.ts]
  @Effect() searchImage$ = this.actions$.pipe(
    ofType(fromBlog.SEARCH_BLOGS),
    map((action: fromBlogActions.SearchBlog) => action.query),
    debounceTime(300),
    distinctUntilChanged(),
    switchMap(query: string) => this.blogService.getBlogBySearching(query))
\end{lstlisting}

The reason we are adding this, is that there is one use case we have not
handled yet within our stream. Let's say that a user types part of their 
name \texttt{char}, proceeds to write \texttt{charlee}, and then shortly
thereafter deletes \texttt{lee}, because they decide for this site they 
would like keep with their nickname \texttt{char}. The letters l, e, and
e were typed within the 500 \lstinline{debounceTime} time limit. However, 
the deletion of the final letter of \texttt{l} after the deletion of 
\text{ee} did happen after the \lstinline{debounceTime}. Adding 
\lstinline{distinctUntilChanged} into the mix, makes it so that in this one 
particular use case, the event is not fired again. 

\section{filter} 

\section{take} 

\section{takeUntil}