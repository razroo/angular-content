\section{ Change Detection }
\maketitle{}

Change detection is one of the found principles behind Angular. It is a large
part of what makes the frmaework what it is. Of course, understanding the
minute details of Angular's change detection, can very such help us understand
all other parts of the framework. In addition, help us increase boosts when it
comes to performance.

\subsection{Understanding Change Detection as a Concept}
Without a framework in place, we would change a particular piece of text
using Javascript. For instance, let's say we have a promise which returns data,
with that new data, we would do something like the following:

\begin{lstlisting}
yellowBoxText = newData;
\end{lstlisting}

In a framework, doing something like this isn't neccesary and one has the option
to go lean into the framework to this for you. As you are mostly likely a UI
Engineer familiar with Angular, I won't bore you with the details. However, here
is a quick primer:
\begin{lstlisting}
<!-- In our html file -->
{{ newData.name}}
// In our Typescript file
this.facade.user$.subscribe((userData) => {
    this.newData = userData;
  });
});
\end{lstlisting}

Very simply, using the above, any time that our newData changes, all of the
relevant data inside of out html file will be updated.

\subsection{ Understanding Change Detection Performance }
First and foremost, every component in Angular has it's own change detection.
This means that if data changes for the component, only that particular
component will be updated. In addition, as is probably intuitive at this point,
change detection in Angular is top down. If a parent component is changed, then
all child components will be re-rendered.

In addition, in angular, there is the ability to emit data. Even if a component
is a parent component, it will still be updated accordingly.

\subsubsection{ Sibling Components under Same Parent Component }
Sibling components under the same parent component will not update unless they
both feed from the same component. Keeping change detection at this level
from an architectural perspective is really all you need. To be read up more
on change detection, feel free to look here: [Todo].
