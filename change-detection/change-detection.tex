\section{ Change Detection }
\maketitle{}

Change detection is one of the found principles behind Angular. It is a large
part of what makes the frmaework what it is. Of course, understanding the
minute details of Angular's change detection, can very such help us understand
all other parts of the framework. In addition, help us increase boosts when it
comes to performance.

Without a frameowork in place, we would change a particular piece of text
using Javascript. For instance, let's say we have a promise which returns data,
with that new data, we would do something like the following:

\begin{lstlisting}
yellowBoxText = newData;
\end{lstlisting}

In a framework, doing something like this isn't neccesary and one has the option
to go lean into the framework to this for you. As you are mostly likely a UI
Engineer familiar with Angular, I won't bore you with the details. However, here
is a quick primer:
\begin{lstlisting}
<!-- In our html file -->
{{ newData.name}}
// In our Typescript file 
\end{lstlisting}
