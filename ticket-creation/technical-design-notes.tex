
\section{ Technical Design Notes }

When creating a ticket, it is important that technical design notes be written
as a part of actual JIRA ticket.

\subsection{ What are Technical Design Notes? }
Technical Design Notes are a way to abstract the decisions one will make
towards architecting an app.

\subsection{ Benefits of Creating Technical Design Notes }
It allows the app to be thought through before app is built. Saving time on
re-factoring code, ensure code quality, and retain confidence that tickets will
be done the way they should be done.

\subsection{ When to Create Technical Design Notes }
Technical Design Notes can be cumbersome, and writing them does not make sense
in all instances. In one of two situations, technical design notes should be
created, when one, or more of the following is true:

\begin{enumerate}
  \item When Unit Testing is involved. For instance, let's say we have a
  filtering component, and we need to test what will happen if a user inputs a
  word with a space in it, or with a uppercase character.
  \item When strategy architecture is involved. For instance, we need to create
  a strategy for routing, or how we will end up pulling in data. Sometimes, it
  is something which will be an unknown, and saying this is what I am trying to
  figure out, and it is a unknown is more than perfect.
\end{enumerate}

\subsection{ What Goes into Technical Design Notes }
It should mention at a very high level, what should go into the component. For
instance, if I am building a filter, it should mention:
\begin{enumerate}
  \item That I plan on using ngrx/store in order to store filters.
    \begin{enumerate}
      \item Specifically as strings
    \end{enumerate}
  \item Will be using <md-input> material design component for filters.
  \item Will be writing integration test for filters individually, and how they
  will interact with each other.
  \item Will be creating filters for the following test scenarios.
    \begin{enumerate}
      \item Camel case
      \item Space in filter
      \item Pure text
      \item Dates
    \end{enumerate}
\end{enumerate}
