
\chapter{ Ticket Creation - Component Design }
We have discussed the two initial steps with regards to creating a ticket.
Technical Design Notes, and acceptance criteria. The third and final step
with regards to any good ticket is design. The chapter regarding talking to
UI/UX is not here, but you should be using JIRA as your ticketing system.

Within a JIRA setting, two things should happen so you have a clear idea of
how a component should be designed within a PWA setting:

\begin{enumerate}
  \item Description
  \item Invision Link within JIRA
\end{enumerate}

\section{ Component Design Quirk }

When creating a ticket in a PWA environment, there is a need to create a specify
the specific functionality around mobile, and desktop. Many times, the
functionality is not the same. In addition, while we develop from a mobile first
perpective, it is not the case for business and product. For engineers, there
is an understanding that whatever is not used for mobile, is used for desktop.
This is not the case for business. They look at the two as two separate
entities.

In addition, understandably for design, they also look at mobile and desktop as
two different entities.

It is therefore reccomended that you will have to create two separate tickets.

\mybox{
\section{ Development Corner }
Having two tickets for mobile and development will cause conflict with regards
to pull requests. In order to solve this concern, make git commit's against
the web ticket. In addition, in your JIRA ticket, make mobile dependent on the
web ticket.
}
