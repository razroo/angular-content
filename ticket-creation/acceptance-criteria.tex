\maketitle{}
\section{ Acceptance Criteria }
Acceptance Criteria should generally not be in the court of the software
engineer. However, as is quite common in software engineering, product will
need a bit of prodding from engineering, in order to discuss what it is that
they are looking for.

\subsection{ Real Quick - What are Acceptance Criteria? }
They are the conditions that a software product must satisfy to be accepted by
a user, customer, or in the case of system level functionality, the consuming
system.

\subsection{ Ideal Syntax for Acceptance Criteria? }
After being in a number of settings, the ideal way to create acceptance criteria
is to use Cucumber/Gherkin syntax.

\subsection{ What Gherkin Syntax? }
Gherkin is a syntax which supports BDD. It is aimed at making executable
specifications written in plain language\footnote{We will get to how we will
integrate this with our QA efforts in a moment}:
\begin{verbatim}
  Scenario: eat 5 out of 12
  Given there are 12 cucumbers
  When I eat 5 cucumbers
  Then I should have 7 cucumbers
\end{verbatim}

\subsection{ Why is it important that we use Gherkin Syntax? }
Gherkin syntax is designed to be succint, and easily understandable which it is.
In addition, it is a syntax that the entire company can rally around, being that
it will be used by Automation Engineers as well\footnote{Surpise! They will be
using Gherkin as well, unless you knew that one already. In which case it is
not a suprise.}. It is also in my experience, the only way to convince product
to write acceptance criteria that actually stays the same from ticket from
ticket, but don't tell them I said that!
