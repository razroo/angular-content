\chapter{ Data GraphQL }


Assuming your app is using /href{https://graphql.org/}{GraphQL}, you will have
to choose a client to use with GraphQL. Without going to into detail, we will be
using Apollo Client as we truly believe it is best client for GraphQL. Within
Apollo, there are numerous different files that one can create to interact
with GraphQL.

\section{Four Types of Apollo Client Files}
Most notably there are four different types:
\begin{enumerate}
  \item \href{https://www.apollographql.com/docs/angular/features/fragments}{fragments}
  \item \href{https://www.apollographql.com/docs/angular/basics/queries/}{queries}
  \item \href{https://www.apollographql.com/docs/angular/basics/mutations}{mutations}
  \item \href{https://www.apollographql.com/docs/angular/features/subscriptions}{subscriptions}
\end{enumerate}

This chapter is not a time for us to go into detail for each. However, the
question is, where within our Angular Directory Structure should we go ahead and
put it within our app?

\section{Dissecting the Purpose of Apollo Client Files}
It's important to understand that Apollo Client queries, mutations, and
subscriptions will not only be used with their respective data-service. It is
quite possible that multiple services will be using the same query, for
instance, in numerous ways. In addition, the fragments used for a particular
graphql query, mutation, or subscription, will be used within multiple apollo
client files. It therefore make sense for the all apollo client files to be
placed in their own disticnt folder within the libs folder(assuming we are using
a mono repo), for the particular app. So now that we have decided it should
warrant it's own folder let's take a quick look at how the Directory Structure
might look like.

\section{Data GraphQL Data Structure}
\begin{forest}
  [libs
    [px-illustrator
      [data-models
        [src
          [lib
            [user
              [\_user.fragments.ts,file]
              [\_user.mutations.ts,file]
              [\_user.queries.ts,file]
              [\_user.subscriptions.ts,file]
            ]
          ]
          [\_index.ts,file]
          [\_test.ts,file]
        ]
        [\_README.md,file]
        [\_tsconfig.lib,file]
        [\_tsconfig.lib.json,file]
        [\_tsconfig.spec.json,file]
        [\_tslint.json,file]
      ]
    ]
  ]
\end{forest}


As we proposed all files related to graphql are exclusively put into a single
data-graphql folder. This alleviates the potential issues we mentioned above.
