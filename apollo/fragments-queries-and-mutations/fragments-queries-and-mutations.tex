\maketitle{}
\section{ Fragments Queries and Mutations }

An important part of architecture with regards to using GraphQL, is splitting
up data requests into fragments + queries/mutations. It greatly simpifies
the architecture. The closest resemblance of how to describe using fragments
in GraphQL, is functional composition.

\mybox{
\subsection{ Primer on Queries and Mutations }
query = SELECT
mutation = INSERT, UPDATE, DELETE
}

\maketitle{}
\subsection{ Example Fragment }
\begin{verbatim}
export const GridFragment = `
  fragment gridFields on GridSize {
    rowSize
    columnSize
    pixelSize
  }
`;
\end{verbatim}

\subsection{ Using Fragment in Query }
Now in our query, we have the ability to include our fragment.
\begin{verbatim}
export const BuyerQuery = gql`
  query gridSize($projectId: ID!, $id: ID!) {
    gridSize(projectId: $projectId, id: $id) {
      ...gridFields
    }
  }
  ${GridFragment}
`;
\end{verbatim}

We can re-use this fragment elsewhere if we would like. It also helps when
creating an interface. We have the ability to keep a one to one relationship
between our fragments and interfaces as well. For instance:
\begin{lstlisting}
export interface GridSize {
  rowSize: string;
  columnSize: string;
  pixelSize: string;
}
\end{lstlisting}
